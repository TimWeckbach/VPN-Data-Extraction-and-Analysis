\section{Classification Results}
\label{sec:classification_results}

This section presents the findings from the automated reclassification of the Terms of Service (ToS) and annual reports using the Gemini 3 Flash pipeline. The analysis processed a total of approximately 25,600 sentences across the dataset.

\subsection{Distribution of Enforcement Categories}
The classification showed that \textbf{91\%} of the sentences were "General Terms" (legal boilerplate). This is expected, as most corporate reporting is about general operations, not specific geo-arbitrage rules.

After filtering out the general terms, we found a clear structure of enforcement. Table \ref{tab:category_dist} shows how different providers approach this.

\begin{table}[ht]
    \centering
    \begin{tabularx}{\textwidth}{l X c}
        \toprule
        \textbf{Category} & \textbf{Description} & \textbf{Frequency} \\
        \midrule
        Content Licensing & Geographic restrictions based on IP rights. & 857 \\
        Regulatory Compliance & Compliance with local laws and regulations. & 743 \\
        Account Action & Punitive measures (suspension, termination) for violations. & 280 \\
        Price Discrimination & Explicit mentions of regional pricing differences. & 197 \\
        Technical Blocking & Measures to detect/block VPNs, proxies, and IP masking. & 106 \\
        \bottomrule
    \end{tabularx}
    \caption{Distribution of Strategic Categories in ToS Documents}
    \label{tab:category_dist}
\end{table}

\begin{table}[ht]
    \centering
    \renewcommand{\arraystretch}{1.2}
    \begin{tabularx}{\textwidth}{l X c}
        \toprule
        \textbf{Category} & \textbf{Representative Quote (Extracted from ToS)} & \textbf{Confidence} \\
        \midrule
        Content Licensing & "We grant you a limited, non-exclusive license to access the service... only within geographic locations where we offer our service." & 0.98 \\
        \addlinespace
        Technical Blocking & "You may not use any technology to obscure or disguise your location." & 0.95 \\
        \addlinespace
        Account Action & "We reserve the right to terminate or restrict your use of our service, without notice, if we suspect... violation of these Terms." & 0.92 \\
        \addlinespace
        Price Discrimination & "Prices may vary by country... You will be charged in the currency associated with the location of your account creation." & 0.89 \\
        \bottomrule
    \end{tabularx}
    \caption{Representative Clauses for Detected Enforcement Strategies}
    \label{tab:quotes}
\end{table}

\subsection{Service-Specific Analysis}
The enforcement strategies vary significantly across different service providers, reflecting their distinct business models and regional licensing constraints. Figure \ref{fig:service_dist} illustrates the proportional distribution of categories for each service. 

Notably, services with heavy reliance on third-party content licensing exhibit a higher proportion of \textit{Content Licensing} clauses. For instance, \textbf{Disney+} (8.5\%) and \textbf{Netflix} (6.2\%) dedicate a significant portion of their terms to defining geographic rights. In contrast, VPN providers like \textbf{NordVPN} show a distinct focus on \textbf{Account Action} (10.9\%), reflecting a strategy of penalizing abuse rather than just blocking access. \textbf{YouTube Premium} stands out with a relatively high frequency of \textbf{Technical Blocking} clauses (2.2\%), indicating an active technological countermeasures approach.

Global platforms like \textbf{Amazon} and \textbf{Apple Music} also show notable spikes in \textbf{Regulatory Compliance} (4.4\% and 5.7\% respectively), suggesting their terms are heavily influenced by the diverse legal frameworks of the many jurisdictions they operate in.

\begin{figure}[ht]
    \centering
    \includegraphics[width=\textwidth]{figures/service_distribution_ratios.pdf}
    \caption{Proportional Distribution of Enforcement Categories by Service}
    \label{fig:service_dist}
\end{figure}

\subsection{Temporal Evolution of Enforcement}
To understand how these strategies have evolved over time, we analyzed the frequency of category specific clauses across the dataset's years. Figure \ref{fig:timeline_all} shows the aggregate trend, while Figure \ref{fig:timeline_service} breaks this down by service provider.

The data shows an increase in specific enforcement clauses, especially from 2022. This correlates with more global awareness of price arbitrage and better VPN detection tools. The cycle of "action and reaction" is visible: new restrictive clauses often follow periods of increased user workaround activity.

\begin{figure}[ht]
    \centering
    \includegraphics[width=\textwidth]{figures/category_timeline_all_normalized.pdf}
    \caption{Temporal Evolution of Category Frequencies (Aggregate, Excluding General Terms)}
    \label{fig:timeline_all}
\end{figure}

\begin{figure}[ht]
    \centering
    \includegraphics[width=\textwidth]{figures/category_timeline_per_service_normalized.pdf}
    \caption{Temporal Evolution of Category Frequencies by Service (Normalized)}
    \label{fig:timeline_service}
\end{figure}

\subsection{High-Confidence Findings: The Core Clauses}
The Gemini 3 Flash model identified specific, high-confidence clauses that are central to the coercive strategy. For example, clauses stating "You may not use any technology to obscure or disguise your location" were consistently categorized as \textit{Technical Blocking} with $>0.95$ confidence. This confirms that firms have made technical countermeasures a formal part of their legal rules.

\section{The Digital Services Price Index (DSPI)}
\label{sec:dspi_results}

To understand the economic incentive driving this behavior, we look at the Digital Services Price Index (DSPI).

\subsection{Magnitude of the Arbitrage Incentive}
The data (derived from Phase 1 sampling) indicates a massive disparity between markets. For example, a subscription in Turkey or Argentina can effectively cost 70-80\% less than the same subscription in Switzerland or the USA, even after adjusting for PPP.

\begin{figure}[ht]
    \centering
    \includegraphics[width=0.9\textwidth]{figures/dspi_heatmap.pdf}
    \caption{Global Heatmap of Digital Service Pricing (The DSPI). Data represents the cost of local subscriptions relative to the US baseline (DSPI=1.0). Lower values indicate stronger arbitrage incentives.}
    \label{fig:dspi_map}
\end{figure}

This disparity creates a "super-normal" profit margin for the consumer-arbitrageur, which explains the high persistence of the behavior despite the technical barriers described in Section \ref{sec:classification_results}.

\subsection{The Affordability Paradox: Nominal vs. Real Cost}
While the nominal price differences create an arbitrage incentive for Western users, it is crucial to understand the "Real Cost" for local users. Figure \ref{fig:affordability} maps the cost of digital services as a percentage of the average local monthly salary.

\begin{figure}[ht]
    \centering
    \includegraphics[width=0.9\textwidth]{figures/affordability_heatmap.pdf}
    \caption{The Affordability Gap: Digital Service Cost as Percentage of Local Monthly Income. Darker red indicates higher relative cost for local citizens.}
    \label{fig:affordability}
\end{figure}

The data reveals a paradox: while Turkey and Argentina offer the cheapest nominal prices for international arbitrageurs ($<$\$4/month), these services are significantly \textit{more expensive} for locals in real terms. For instance, a Standard Netflix subscription in Turkey consumes a higher percentage of the average monthly wage ($\approx 0.6\%$) compared to the USA ($\approx 0.3\%$). This suggests that the low nominal prices are not "discounts" but necessary adjustments to local purchasing power, which are then exploited by external actors.

\section{Correlation Analysis: The Strategic Trade-off}
\label{sec:correlation}
To empirically validate the relationship between pricing strategy and enforcement intensity, we utilized the cleaned Risk Factor dataset (filtering out general corporate noise) and calculated the correlation coefficient.

\begin{figure}[ht]
    \centering
    \includegraphics[width=0.9\textwidth]{figures/protection_vs_pricing.pdf}
    \caption{Strategic Alignment: firms with higher Price Discrimination scores (Standard Deviation of DSPI) generally exhibit higher Enforcement intensities.}
    \label{fig:correlation}
\end{figure}

The refined analysis ($N=11$) reveals a distinct \textbf{Adversarial Alignment} rather than a simple linear correlation ($R_{global} \approx 0.24$). By categorizing services into their strategic roles—Content Providers (Targets), Utility Software, and VPNs (Adversaries)—we observe clear clustering (see Figure \ref{fig:correlation}).

\begin{itemize}
    \item \textbf{Content Providers (Netflix, Disney+, YouTube, Xbox, etc.):} This group effectively forms a "High Enforcement Cluster," but successfully illustrates the enforcement trade-off ($R_{sector} \approx 0.45$). 
    \begin{itemize}
        \item \textbf{High Variance / High Enforcement:} Services like \textbf{Disney+} and \textbf{YouTube} have large global price gaps (DSPI StdDev $>0.37$) and rely on aggressive "Technical Blocking" (6\%--8\%) to maintain them.
        \item \textbf{Low Variance / Low Enforcement (The Xbox Case):} \textbf{Xbox Game Pass} serves as a crucial control. Governed by the Microsoft ecosystem, it has relatively harmonized global pricing (DSPI StdDev $\approx 0.25$) and correspondingly low enforcement intensity ($\approx 1.9\%$). This suggests that when a content provider harmonizes prices (reducing the arbitrage incentive), the need for a "Fortress" strategy diminishes.
    \end{itemize}
    
    \item \textbf{Utility Software (The Strategic Split):}
    \begin{itemize}
        \item \textbf{Adobe Creative Cloud} is a significant anomaly. It rivals Content Providers in price discrimination (DSPI StdDev $\approx 0.59$) yet maintains very low ToS enforcement ($\approx 0.9\%$). This confirms the \textbf{"Utility Paradox"}: downloadable software relies on cryptographic license keys ("Hard" barriers) rather than the "Soft" IP-blocking threats required by streaming services.
    \end{itemize}

    \item \textbf{VPN Enablers (NordVPN, ExpressVPN):} As expected, these "Adversaries" show minimal "Technical Blocking" enforcement, as their business model depends on circumventing the very barriers erected by the Content Providers.
\end{itemize}

This data suggests that \textbf{Business Model} (Streaming vs. Download vs. Access) is a stronger predictor of enforcement strategy than \textbf{Price Opportunity} alone.
