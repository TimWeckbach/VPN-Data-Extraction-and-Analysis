\section{The Landscape of International Pricing: Findings from the DSPI}
\label{sec:dspi_results}

To understand the economic incentive driving the strategic behavior of firms, we first analyze the quantitative pricing landscape using the Digital Services Price Index (DSPI).

\subsection{Magnitude of the Arbitrage Incentive}
The data (derived from Phase 1 sampling) indicates a massive disparity between markets. For example, a subscription in Turkey or Argentina can effectively cost 70-80\% less than the same subscription in Switzerland or the USA, even when measured in nominal USD.

\begin{figure}[ht]
    \centering
    \includegraphics[width=0.9\textwidth]{figures/dspi_heatmap.pdf}
    \caption{Global Heatmap of Digital Service Pricing (The DSPI). Data represents the cost of local subscriptions relative to the US baseline (DSPI=1.0). Lower values indicate stronger arbitrage incentives.}
    \label{fig:dspi_map}
\end{figure}

This disparity creates a "super-normal" profit margin for the consumer-arbitrageur, which explains the high persistence of the behavior despite the technical barriers described in the following sections.

\subsection{The Affordability Paradox: Nominal vs. Real Cost}
While the nominal price differences create an arbitrage incentive for Western users, it is crucial to understand the "Real Cost" for local users. Figure \ref{fig:affordability} maps the cost of digital services as a percentage of the \textbf{Median National Monthly Wage}.

\begin{figure}[ht]
    \centering
    \includegraphics[width=0.9\textwidth]{figures/affordability_heatmap.pdf}
    \caption{The Affordability Gap: Digital Service Cost as Percentage of Local Monthly Income. Darker red indicates higher relative cost for local citizens.}
    \label{fig:affordability}
\end{figure}

The data reveals a paradox: while Turkey and Argentina offer the cheapest nominal prices for international arbitrageurs ($<$\$4/month), these services are significantly \textit{more expensive} for locals in real terms. For instance, a Standard Netflix subscription in Turkey consumes a higher percentage of the median monthly wage ($\approx 0.6\%$) compared to the USA ($\approx 0.3\%$). This suggests that the low nominal prices are not "discounts" but necessary adjustments to local purchasing power, which are then exploited by external actors.

\section{Classification Results: Strategic Framing}
\label{sec:classification_results}

This section presents the findings from the automated reclassification of the Terms of Service (ToS) and annual reports using the Gemini 3 Flash pipeline. The analysis processed a total of 14,357 sentences (excluding general boilerplate) across the dataset.

\subsection{Distribution of Enforcement Categories}
The classification showed that \textbf{91\%} of the sentences were "General Terms" (legal boilerplate). After filtering these out, we found a distinct structure of enforcement frames. Table \ref{tab:category_dist} shows how different providers approach this.

\begin{table}[ht]
    \centering
    \begin{tabularx}{\textwidth}{l X c c}
        \toprule
        \textbf{Category} & \textbf{Description} & \textbf{Freq (N)} & \textbf{Freq (\%)} \\
        \midrule
        Content Licensing & Geographic restrictions based on rights. & 524 & 30.39\% \\
        Regulatory Compliance & Local laws/tax compliance. & 522 & 30.28\% \\
        Legal Threat & Explicit threats of termination/legal action. & 87 & 5.05\% \\
        Account Action & General punitive measures against accounts. & 110 & 6.38\% \\
        Price Discrimination & Explicit regional pricing rules. & 121 & 7.02\% \\
        Technical Blocking & Active detection/blocking technology. & 46 & 2.67\% \\
        Security Risk & Risks of VPN usage (Service Prov. Frame). & 14 & 0.81\% \\
        Privacy/Security & Encryption/Anonymity (VPN Frame). & 15 & 0.87\% \\
        Legitimate Portability & EU Portability Regulation clauses. & 35 & 2.03\% \\
        User Workaround & References to circumventing blocks. & 7 & 0.41\% \\
        \bottomrule
    \end{tabularx}
    \caption{Distribution of Strategic Categories in ToS Documents}
    \label{tab:category_dist}
\end{table}

\begin{table}[ht]
    \centering
    \renewcommand{\arraystretch}{1.2}
    \small
    \begin{tabularx}{\textwidth}{l X l l l c}
        \toprule
        \textbf{Category} & \textbf{Quote} & \textbf{Service} & \textbf{Year} & \textbf{Doc} & \textbf{Conf} \\
        \midrule
        Content Licensing & "We grant you a limited... license... only within geographic locations..." & Netflix & 2023 & ToS & 0.98 \\
        \addlinespace
        Technical Blocking & "You may not use any technology to obscure or disguise your location." & Disney+ & 2024 & ToS & 0.95 \\
        \addlinespace
        Legal Threat & "We reserve the right to terminate... without notice, if we suspect violation." & Spotify & 2022 & ToS & 0.92 \\
        \addlinespace
        Price Discrimination & "Prices may vary by country... charged in currency of location." & Steam & 2024 & ToS & 0.89 \\
        \bottomrule
    \end{tabularx}
    \caption{Representative Clauses for Detected Enforcement Strategies}
    \label{tab:quotes}
\end{table}

\subsection{Service-Specific Analysis}
The enforcement strategies vary significantly across different service providers, reflecting their distinct business models and regional licensing constraints. Figure \ref{fig:service_dist} illustrates the proportional distribution of categories for each service. 

\subsubsection{Strategic Framing by Digital Service Providers}
The qualitative analysis highlights a distinct "Coercive" framing strategy employed by digital service providers. The dominant rhetorical frame, appearing in over \textbf{30.4\%} of non-boilerplate sentences (see Table \ref{tab:category_dist}), is \textbf{Content Licensing}. Firms consistently position their geographic restrictions not as business decisions, but as external mandates using language like "compliance with local laws," "licensing restrictions," and "obligations to content owners."
The second most dominant frame is \textbf{Regulatory Compliance} (\textbf{30.3\%}), reinforcing this narrative of external obligation.

Content licensing services like \textbf{Disney+} and \textbf{Netflix} exhibit this most strongly, dedicating significant portions of their terms to defining geographic rights. In contrast, global platforms like \textbf{Amazon} show notable spikes in Regulatory Compliance.

\begin{figure}[ht]
    \centering
    \includegraphics[width=\textwidth]{figures/service_distribution_ratios.pdf}
    \caption{Proportional Distribution of Enforcement Categories by Service}
    \label{fig:service_dist}
\end{figure}

\subsubsection{Strategic Framing by VPN Providers}
In sharp contrast, VPN companies adopt a "Liberation" and "Privacy" frame. The analysis reveals a consistent narrative that reframes circumvention as \textbf{User Freedom}. 
A secondary dominant frame is \textbf{Privacy/Security} (\textbf{XX\%}). While many users purchase VPNs for streaming arbitrage, providers legitimize the service by emphasizing "military-grade encryption" and "anonymity." \textbf{NordVPN}, for example, shows a distinct focus on "Account Action" categories (10.9\%) in our data, but frames this in marketing as empowering users against tracking.

\subsection{Temporal Evolution of Enforcement}
To understand how these strategies have evolved over time, we analyzed the frequency of category specific clauses across the dataset's years (see Figures \ref{fig:timeline_all} and \ref{fig:timeline_service}).
The data shows an increase in specific enforcement clauses, especially from 2022. This suggests that restrictive clauses have become more prevalent over the analyzed period.

\begin{figure}[ht]
    \centering
    \includegraphics[width=\textwidth]{figures/category_timeline_all_normalized.pdf}
    \caption{Temporal Evolution of Category Frequencies (Aggregate, Excluding General Terms)}
    \label{fig:timeline_all}
\end{figure}

\begin{figure}[ht]
    \centering
    \includegraphics[width=\textwidth]{figures/category_timeline_per_service_normalized.pdf}
    \caption{Temporal Evolution of Category Frequencies by Service (Normalized)}
    \label{fig:timeline_service}
\end{figure}

\subsection{High-Confidence Findings: The Core Clauses}
The Gemini 3 Flash model identified specific, high-confidence clauses that are central to the coercive strategy. For example, clauses stating "You may not use any technology to obscure or disguise your location" were consistently categorized as \textit{Technical Blocking} with $>0.95$ confidence. This confirms that firms have made technical countermeasures a formal part of their legal rules.

\section{The Digital Services Price Index (DSPI)}
\label{sec:dspi_results}

To understand the economic incentive driving this behavior, we look at the Digital Services Price Index (DSPI).

\subsection{Magnitude of the Arbitrage Incentive}
The data (derived from Phase 1 sampling) indicates a massive disparity between markets. For example, a subscription in Turkey or Argentina can effectively cost 70-80\% less than the same subscription in Switzerland or the USA, even after adjusting for PPP.

\begin{figure}[ht]
    \centering
    \includegraphics[width=0.9\textwidth]{figures/dspi_heatmap.pdf}
    \caption{Global Heatmap of Digital Service Pricing (The DSPI). Data represents the cost of local subscriptions relative to the US baseline (DSPI=1.0). Lower values indicate stronger arbitrage incentives.}
    \label{fig:dspi_map}
\end{figure}

This disparity creates a "super-normal" profit margin for the consumer-arbitrageur, which explains the high persistence of the behavior despite the technical barriers described in Section \ref{sec:classification_results}.

\subsection{The Affordability Paradox: Nominal vs. Real Cost}
While the nominal price differences create an arbitrage incentive for Western users, it is crucial to understand the "Real Cost" for local users. Figure \ref{fig:affordability} maps the cost of digital services as a percentage of the average local monthly salary.

\begin{figure}[ht]
    \centering
    \includegraphics[width=0.9\textwidth]{figures/affordability_heatmap.pdf}
    \caption{The Affordability Gap: Digital Service Cost as Percentage of Local Monthly Income. Darker red indicates higher relative cost for local citizens.}
    \label{fig:affordability}
\end{figure}

The data reveals a paradox: while Turkey and Argentina offer the cheapest nominal prices for international arbitrageurs ($<$\$4/month), these services are significantly \textit{more expensive} for locals in real terms. For instance, a Standard Netflix subscription in Turkey consumes a higher percentage of the average monthly wage ($\approx 0.6\%$) compared to the USA ($\approx 0.3\%$). This suggests that the low nominal prices are not "discounts" but necessary adjustments to local purchasing power, which are then exploited by external actors.

\section{Correlation Analysis: The Strategic Trade-off}
\label{sec:correlation}
To empirically validate the relationship between pricing strategy and enforcement intensity, we utilized the cleaned Risk Factor dataset (filtering out general corporate noise) and calculated the correlation coefficient.

\begin{figure}[ht]
    \centering
    \includegraphics[width=0.9\textwidth]{figures/protection_vs_pricing.pdf}
    \caption{Strategic Alignment: firms with higher Price Discrimination scores (Standard Deviation of DSPI) generally exhibit higher Enforcement intensities.}
    \label{fig:correlation}
\end{figure}

The refined analysis ($N=8$) reveals a \textbf{Negative Correlation} ($R_{global} \approx -0.36$). This suggests that services with higher price variance (Price Discrimination Score) tend to have \textit{lower} observed enforcement intensity (frequency of blocking clauses). This counter-intuitive finding may indicate that firms with established global pricing power (like Amazon) rely less on aggressive legal threats than smaller localized services.

\begin{itemize}
    \item \textbf{Content Providers (Netflix, Disney+, YouTube, Xbox, etc.):} This group effectively forms a "High Enforcement Cluster," but successfully illustrates the enforcement trade-off ($R_{sector} \approx 0.45$). 
    \begin{itemize}
        \item \textbf{High Variance / High Enforcement:} Services like \textbf{Disney+} and \textbf{YouTube} have large global price gaps (DSPI StdDev $>0.37$) and rely on aggressive "Technical Blocking" (6\%--8\%) to maintain them.
        \item \textbf{Low Variance / Low Enforcement (The Xbox Case):} \textbf{Xbox Game Pass} serves as a crucial control. Governed by the Microsoft ecosystem, it has relatively harmonized global pricing (DSPI StdDev $\approx 0.25$) and correspondingly low enforcement intensity ($\approx 1.9\%$). This suggests that when a content provider harmonizes prices (reducing the arbitrage incentive), the need for a "Fortress" strategy diminishes.
    \end{itemize}
    
    \item \textbf{Utility Software (The Strategic Split):}
    \begin{itemize}
        \item \textbf{Adobe Creative Cloud} is a significant anomaly. It rivals Content Providers in price discrimination (DSPI StdDev $\approx 0.59$) yet maintains very low ToS enforcement ($\approx 0.9\%$). This confirms the \textbf{"Utility Paradox"}: downloadable software relies on cryptographic license keys ("Hard" barriers) rather than the "Soft" IP-blocking threats required by streaming services.
    \end{itemize}

    \item \textbf{VPN Enablers (NordVPN, ExpressVPN):} As expected, these "Adversaries" show minimal "Technical Blocking" enforcement, as their business model depends on circumventing the very barriers erected by the Content Providers.
\end{itemize}

This data suggests that \textbf{Business Model} (Streaming vs. Download vs. Access) is a stronger predictor of enforcement strategy than \textbf{Price Opportunity} alone.
