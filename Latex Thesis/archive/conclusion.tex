\label{chap:conclusion}

\section{Summary of Key Findings}
This thesis looked at the strategic conflict between firms' price discrimination practices and consumer-driven geographic arbitrage in digital subscription markets.
\begin{itemize}
    \item \textbf{RQ1 (Economic Incentive):} The Digital Services Price Index (DSPI) reveals substantial price disparities across markets, with discounts exceeding 70\% in some regions, creating strong economic incentives for consumers to employ VPN-based arbitrage.
    \item \textbf{RQ2 (Strategic Response):} Enforcement strategies are primarily determined by \textbf{Business Model Architecture}. Content streaming firms enforce strict geographic blocking due to licensing requirements. Software firms rely on cryptographic activation rather than network-based blocking. VPN providers, by their nature, implement no geographic restrictions.
    \item \textbf{Enforcement Evolution:} Firms have intensified technical countermeasures since 2022, but the persistent prominence of circumvention discussions suggests these barriers increase friction rather than eliminating arbitrage entirely.
\end{itemize}

\section{Contribution to Research}
This study contributes both in methods and theory to the digital economics literature.

\textbf{Methodological Contributions:}
\begin{itemize}
    \item \textbf{Digital Services Price Index (DSPI):} This thesis introduces the DSPI as a novel metric for quantifying international price discrimination in digital subscription markets. Unlike traditional purchasing power indices that focus on physical goods, the DSPI is specifically designed to capture the unique dynamics of digital services, including the distinction between nominal and affordability-adjusted pricing.
    \item \textbf{LLM-Based Legal Text Analysis:} The study shows the effectiveness of Large Language Models (specifically Gemini 3 Flash) in automating legal text analysis at scale. The classification pipeline processed over 25,000 sentences with high confidence scores, setting a precedent for future research in computational legal studies and information systems.
\end{itemize}

\textbf{Theoretical Contributions:}
\begin{itemize}
    \item \textbf{Extension of Business Model Innovation Theory:} The study extends BMI theory by showing that consumer bypassing behaviors work as a disruptive force similar to technological innovation, forcing firms to fundamentally change their value capture mechanisms.
    \item \textbf{Strategic Archetype Framework:} The identification of four distinct archetypes (Content Fortress, Ecosystem Fortress, Enterprise Fortress, Utility Paradox) provides a typology for understanding how different business models respond to the same external threat.
    \item \textbf{Piracy-Arbitrage Parallel:} By drawing explicit parallels between the digital piracy wave of the 2000s and contemporary geo-arbitrage, this thesis contributes to a longer-term understanding of how digital disruption forces business model adaptation.
\end{itemize}


\section{Future Outlook}
The landscape of digital geo-blocking is likely to evolve significantly in the coming years, driven by regulatory, technological, and market forces.

\textbf{Regulatory Developments:}
As frameworks such as the EU's Digital Single Market \parencite{eu2018geoblocking} evolve, the legal status of geo-blocking may undergo fundamental transformation. The EU has already prohibited unjustified geo-blocking for certain categories of goods and services, and similar regulations may emerge in other jurisdictions. Future research should examine how legal interventions influence pricing strategies and enforcement mechanisms, particularly whether regulatory pressure leads firms toward adaptive (price harmonization) rather than coercive (blocking) strategies.

\textbf{Technological Evolution:}
The arms race between blocking technology and bypassing tools shows no signs of slowing. Advanced techniques such as residential IP proxies, browser fingerprinting, and machine learning-based detection create an increasingly complex technical battlefield. Researchers should look at whether the rising cost of this arms race eventually makes geographic price discrimination economically unsustainable.

\textbf{Market Convergence:}
Ultimately, the adversarial cycle of blocking and circumvention may reach resolution not through superior technical countermeasures, but through market forces that drive global price convergence. As digital services become commoditized and competition intensifies, firms may find that the administrative costs of maintaining complex regional pricing structures outweigh the benefits. In such a scenario, geographic arbitrage would become economically obsolete, not because consumers are blocked, but because there is no longer a meaningful price differential to exploit.

\textbf{Directions for Future Research:}
Future studies could extend this work by: (1) expanding the sample size to include a broader range of digital services; (2) conducting longitudinal tracking of DSPI changes in response to specific enforcement events; (3) surveying consumers to understand the psychological and ethical dimensions of geo-arbitrage behavior; and (4) comparative analysis across regulatory regimes to assess the impact of legal interventions on firm strategy.
