\label{chap:introduction}

\section{Background and Context}
The global digital economy has a basic contradiction: while digital services naturally cross geographical borders, companies in this space commonly use geographically segmented pricing strategies. Major platforms such as Netflix, Spotify, Steam, and Microsoft Office divide the world into distinct pricing regions, charging vastly different prices for identical digital products depending on the customer's geographic location. This practice, known as third-degree price discrimination, allows firms to extract maximum revenue from each market by aligning prices with local willingness to pay and purchasing power. In markets with lower average incomes, services are offered at significantly reduced prices to maximize market penetration, while wealthier regions face premium pricing to capture consumer surplus.

However, the technical infrastructure that enables global digital distribution also allows customers to challenge this geographical segmentation. Many users have discovered that Virtual Private Networks (VPNs) can be used not only for privacy protection but also to bypass regional pricing barriers. By hiding their true location and routing their internet traffic through servers in lower-priced regions, users can purchase subscriptions at prices meant for markets with very different economic conditions. This practice, often called ``\gls{digitalgeoarbitrage}'' or ``VPN price hopping,'' is a form of consumer-driven market arbitrage that takes advantage of the price differences created by firms' own segmentation strategies.

This phenomenon has similarities to the digital piracy wave of the early 2000s, where users used technical tools to bypass distribution barriers \parencite{oberholzer2007effect}. Just as file-sharing technologies let users bypass payment systems entirely, VPN-based geo-arbitrage allows consumers to bypass pricing structures, though through purchase rather than outright piracy. In both cases, users use readily available technology to challenge business models designed under the assumption of enforceable geographical or technical barriers.

\section{Problem Statement}
The core problem examined in this thesis is the fundamental conflict between firms' strategic need to maintain market segmentation and users' technical ability to circumvent that segmentation.
Firms operating in the digital subscription economy face a strategic dilemma with two competing pressures:
\begin{enumerate}
	\item \textbf{Economic Necessity:} To maximize global market penetration and revenue, firms must offer substantially lower prices in emerging markets while maintaining premium pricing in wealthier regions.
	\item \textbf{Technical Reality:} The internet's architecture enables users to bypass geographic restrictions using commercially available VPN services, which have become increasingly accessible to non-technical users.
\end{enumerate}
This tension forces companies to rethink their business models. They must choose between "Coercive" strategies that try to keep market segmentation through technical blocking and legal threats, or "Adaptive" strategies that reduce arbitrage incentives through price harmonization or ecosystem integration.

\section{Research Questions (RQs)}
To analyze this strategic conflict, this thesis pursues two complementary research questions that address both the economic drivers of geo-arbitrage and the organizational responses to it.

The first research question focuses on quantifying the economic incentive structure:
\begin{description}
	\item[RQ1 (The Economic Incentive):] To what extent does international price differentiation for digital services deviate from local affordability (measured by the Median National Wage), thereby creating a ``super-normal'' incentive for geographic arbitrage?
\end{description}

This question is operationalized through the construction of the Digital Services Price Index (DSPI), which compares nominal prices across regions while also calculating the ``Real Affordability'' of subscriptions relative to local wages. The distinction between nominal and real pricing is crucial: a service may appear cheap to foreign arbitrageurs while simultaneously being expensive for local consumers.

The second research question investigates how firms strategically respond to the arbitrage threat:
\begin{description}
	\item[RQ2 (The Strategic Response):] \textbf{How do digital subscription providers modify their business models in response to regional pricing circumvention, and how has the balance between coercive and adaptive strategies evolved over time as reflected in their corporate disclosures?}
\end{description}

This question is addressed through systematic text analysis of Terms of Service documents and annual reports, classifying corporate language into strategic categories (e.g., ``Technical Blocking,'' ``Legal Threat,'' ``Price Discrimination''). The longitudinal dimension allows us to trace the evolution of enforcement strategies from 2016 to 2024.

\section{Structure of the Thesis}
The thesis follows a sequential structure designed to build from theoretical foundations through empirical analysis to strategic interpretation.

\textbf{Chapter \ref{chap:theory}} establishes the theoretical foundations by synthesizing three streams of literature: (1) economic theory of price discrimination, particularly Varian's framework for third-degree discrimination; (2) consumer behavior research on digital piracy and circumvention, drawing parallels between geo-arbitrage and historical patterns of file-sharing; and (3) Business Model Innovation theory, which provides the analytical lens for interpreting firm responses.

\textbf{Chapter \ref{chap:methodology}} details the mixed-methods research design. The quantitative component introduces the novel ``Digital Services Price Index'' (DSPI), constructed from pricing data across 30+ countries and 8 major digital services. The qualitative component describes the LLM-based classification pipeline used to analyze over 25,000 sentences from Terms of Service documents and annual reports.

\textbf{Chapter \ref{chap:results}} presents the empirical findings in three parts: (1) the pricing landscape revealed by the DSPI, including the ``Affordability Paradox''; (2) the distribution and evolution of strategic enforcement categories; and (3) the correlation analysis linking pricing strategy to enforcement intensity.

\textbf{Chapter \ref{chap:discussion}} interprets these results through the lens of Business Model Innovation, identifying four strategic archetypes (Content Fortress, Ecosystem Fortress, Enterprise Fortress, and the Utility Paradox) and discussing their implications for digital platform strategy.

\textbf{Chapter \ref{chap:conclusion}} summarizes the key contributions, acknowledges limitations, and outlines directions for future research, including the potential impact of regulatory developments such as the EU Digital Single Market.
