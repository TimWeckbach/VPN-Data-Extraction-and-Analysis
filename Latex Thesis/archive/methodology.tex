\label{chap:methodology}

This chapter details the methods used in this study.

\section{Research Design}

This study uses a sequential explanatory mixed-methods design \parencite{creswell2017designing}, combining quantitative price analysis with qualitative text classification. The reason for this dual approach is to first show the \textit{size} of the economic problem (the arbitrage incentive) and then look at the \textit{strategic responses} of the actors involved.

The quantitative phase (Phase 1) builds the "Digital Services Price Index" (DSPI) to objectively measure differences in global digital service pricing. The qualitative phase (Phase 2) uses a Large Language Model (LLM) pipeline to classify corporate disclosures and Terms of Service, finding the strategic frameworks firms use to manage or fight this variance. This integration provides a comprehensive understanding of the geo-arbitrage ecosystem based on public documentation, though proprietary information and undisclosed technologies remain hidden.

\section{Phase 1: Quantitative Data Collection (for RQ1)}

\subsection{Data Collection}
To construct the DSPI, a representative basket of 11 digital services was selected: Netflix, YouTube Premium, Disney+, Amazon Prime, Spotify, Apple Music, Microsoft 365, Adobe Creative Cloud, Xbox Game Pass, NordVPN, and ExpressVPN. These cover Video on Demand, Music Streaming, Software/Gaming, and VPN services.

Price data was collected from a sample of 11 countries to capture the full spectrum of purchasing power. The countries included are: Argentina, Brazil, Germany, Pakistan, Philippines, Poland, Switzerland, Turkey, Ukraine, United Kingdom, and the United States. Only countries with high-confidence official wage data were included.

Data collection was performed using a \textbf{Digital Audit} design, adapting the methodology established by \textcite{hannak2014measuring} for detecting online price discrimination. A virtual presence was established in each target country using a commercial VPN service to simulate local access, a technique now standard in information systems research for "mystery shopping" in digital markets. For each service and country, the monthly "Standard" subscription price was recorded in local currency. This approach mirrors the methodology of the "Billion Prices Project" \parencite{cavallo2017are}, which demonstrated the validity of using high-frequency online scraping to construct robust price indices that track real-time economic disparities more effectively than traditional CPI baskets.

\subsection{Data Analysis}
The raw price data was processed in two stages. First, all local prices were converted to a common currency (USD) using market exchange rates (recorded in December 2025) to find the ``Nominal Price Inequality.'' Second, to measure ``Real Affordability,'' these prices were calculated as a percentage of the \textit{Median National Monthly Wage} (sourced from OECD and World Bank data), giving a direct measure of the economic burden on the local consumer (`Price-to-Wage Ratio`). This study proposes this approach as a new alternative to standard PPP adjustment, arguing it better reflects subscription goods' affordability relative to disposable income in the specific context of digital services.

It is important to note that a DSPI of 1.0 (Nominal Parity) does not imply equal affordability. Due to vast differences in median wages (e.g., Switzerland vs. India), a service priced identically in USD would be significantly more expensive for the Indian consumer in real terms (requiring a larger percentage of their income). Thus, the arbitrage incentive persists even at nominal parity if the local price is structured to be affordable for the local median earner.

The DSPI was calculated as the ratio of the local price to the US baseline price. A DSPI of 1.0 indicates price parity with the US market; a DSPI < 1.0 indicates a cheaper market (potential arbitrage source), and a DSPI > 1.0 indicates a more expensive market. Statistical variance analysis was performed to identify which service categories exhibit the highest degree of price discrimination.

\section{Phase 2: Qualitative Data Collection \& Analysis (for RQ2)}

\subsection{Coding Procedure}
The analysis follows a systematic coding approach inspired by the \textbf{Gioia Methodology} \parencite{gioia2013seeking}, which organizes qualitative data into layers: 1st-order concepts (raw terms found in text), 2nd-order themes (theoretical categories such as "Technical Blocking"), and aggregate dimensions (Strategic Responses). While first designed for manual coding \parencite{duriau2007content}, this layered structure provided the conceptual framework for the automated classification pipeline described below.

\section{Automated Text Classification}
\label{sec:llm_methodology}

To address the limitations of traditional Natural Language Inference (NLI) models in capturing the nuanced legal and technical language of Terms of Service (ToS), this study implemented an advanced classification pipeline leveraging state-of-the-art Large Language Models (LLMs). Specifically, the pipeline was upgraded from a BERT-based architecture (DeBERTa-v3-large) to the \textit{Gemini 3 Flash} model, accessed via the Google Generative AI API.

\subsection{Model Selection and Rationale}
The selection of \textit{Gemini 3 Flash} was driven by the need for deeper reasoning capabilities and context awareness. Unlike NLI models, which classify based on entailment probabilities between a premise and a hypothesis, generative LLMs can interpret complex sentence structures and ambiguous legal standard terms (``General Terms'') versus specific geo-arbitrage restrictions. 

Key advantages observed during the model transition included:
\begin{itemize}
    \item \textbf{Contextual Understanding}: The ability to distinguish between benign references to ``account suspension'' (e.g., for fraud) and strategic ``Legal Threats'' tailored to prevent cross-border usage.
    \item \textbf{Zero-Shot Performance}: The model demonstrated high accuracy without extensive fine-tuning, utilizing a robust system prompt to align with the theoretical categories defined in Section 2.
    \item \textbf{Efficiency}: The ``Flash'' architecture provided a high throughput, enabling the processing of the entire dataset (approx. 25,000 sentences) within a reasonable timeframe.
\end{itemize}

\subsection{Operationalization of Constructs (The Coding Scheme)}
Based on the theoretical framework, the following coding scheme was enforced via the LLM system prompt. This scheme maps the abstract concept of "Strategic Response" into measurable data points.

\subsubsection{Strategic Frames}
The model was tasked to identify the underlying justification provided by the firm:
\begin{description}
    \item[Frame: Legal Compliance] Justifying geo-blocking as a non-negotiable legal or contractual necessity (e.g., "Due to licensing agreements...").
    \item[Frame: Security Risks] (Service Provider Frame) Arguments that VPNs/Proxies are unsafe, malicious, or compromise user data.
    \item[Frame: Privacy/Security] (VPN Provider Frame) Arguments focusing on encryption, anonymity, and protection from surveillance.
\end{description}

\subsubsection{Firm Actions}
The model categorized specific enforcement clauses into:
\begin{description}
    \item[Action: Technical Blocking] Active technological measures to detect or block the specific use of VPNs/Proxies (e.g., "We use geo-blocking technology", "Error 403").
    \item[Action: Legal Threat] Explicit threats of account termination, suspension, or legal action specifically for using circumvention tools.
    \item[Action: Account Action] General punitive measures against accounts (termination, suspension) for comprehensive violations.
    \item[Action: Price Discrimination] Explicit differences in pricing based on region, currency, or purchasing power.
    \item[Action: Legitimate Portability] Rules allowing temporary access while traveling (e.g., EU Portability Regulation).
\end{description}

\subsection{Pipeline Architecture and Implementation}
The reclassification process was automated using a customized Python script.

\subsubsection{System Prompt Engineering}
To ensure deterministic and theoretically grounded outputs, the system prompt was engineered with strict constraints. The exact prompt structure is provided below:

\begin{figure}[ht]
\begin{verbatim}
SYSTEM_PROMPT = """You are a scientific classifier for a Thesis on 'Digital Geo-Arbitrage'.
Classify a list of sentences into the provided categories. 

CATEGORIES:
1. Technical Blocking: Measures/Technologies used to detect or block...
2. Legal Threat: Explicit threats of account termination...
3. Security Risk: (Service Provider Frame) Arguments that VPNs are unsafe...
4. Privacy/Security: (VPN Provider Frame) Arguments focusing on encryption...
... [Full List of 10 Categories] ...

INSTRUCTIONS:
- Analyze sentences independently.
- Return a JSON array of objects...
- Format: [ { "category": "Category", "confidence": 0.9 }, ... ]
"""
\end{verbatim}
\caption{System Prompt used for Gemini 3 Flash Classification}
\label{fig:system_prompt}
\end{figure}

\subsubsection{Batch Processing and Error Handling}
To optimize for the API's rate limits and ensure data integrity, the pipeline utilized a batch processing approach. Sentences were grouped into batches of 25 and processed in a single API call. This method significantly reduced network overhead and total processing time.
A robust error-handling mechanism was implemented to manage API timeouts or rate limits (HTTP 429). The script included a ``circuit breaker'' to halt execution upon repeated failures and a resume function to continue processing from the last saved state.

\subsection{Methodological Validation: Gemini vs. Zero-Shot BERT}
To validate the choice of the Gemini 3 Flash model, a comparative analysis was conducted against a traditional Zero-Shot classification approach using a BERT-based model. The results demonstrated a massive divergence between the two models, reinforcing the necessity of using a modern LLM with large context windows for this specific task, consistent with recent findings on LLM performance in text annotation \parencite{gilardi2023chatgpt}.

\subsubsection{Agreement Analysis}
The comparison revealed an exceedingly poor agreement rate of \textbf{26.8\% (Accuracy)} between the two models. The Cohen's Kappa score was \textbf{0.032}, suggesting that the agreement was effectively equivalent to random chance. This discrepancy indicated a fundamental difference in how each model interpreted the classification tasks.

\subsubsection{The Core Conflict: Sensitivity vs. Context}
The analysis highlighted two distinct behaviors:
\begin{enumerate}
    \item \textbf{Gemini Performance:} The Gemini model correctly identified that approximately \textbf{91\%} of the dataset consisted of legal boilerplate, categorized as "General Terms." It successfully distinguished specific enforcement clauses from general legal language.
    \item \textbf{BERT Performance:} The BERT model exhibited "Over-Sensitivity," frequently assigning specific strategic tags based on the presence of individual keywords rather than semantic context.
\end{enumerate}

Specific examples of BERT's misclassification included:
\begin{itemize}
    \item \textbf{Legitimate Portability:} BERT flagged 7,853 sentences as "Legitimate Portability" that were merely "General Terms."
    \item \textbf{Account Action:} BERT flagged 6,134 "General Terms" sentences as "Account Action."
\end{itemize}

\textit{Interpretation:} BERT operates on keyword associations; for example, flagging a sentence like "You must have an account" as an "Account Action." In contrast, Gemini utilizes its reasoning capabilities to understand that the mere mention of an "account" is standard boilerplate ("General Terms") and reserves the "Account Action" tag for sentences explicitly regulating banning or suspension.

\subsubsection{Conclusion on Model Selection}
The validation proves that Zero-Shot BERT is insufficient for complex legal text analysis without extensive fine-tuning. It lacks the nuance required to distinguish between the mere mention of a topic (e.g., "portability") and the active regulation of it. Gemini, leveraging its massive context window and advanced reasoning capabilities, performs significantly better at filtering out noise and providing accurate stratifications. Consequently, Gemini 3 Flash was selected as the sole model for the final analysis.


\subsection{Standard Qualitative Coding}
While the automated LLM pipeline provides scalable classification across the full dataset, manual qualitative coding complements this approach by capturing nuances that escape rigid categorization. A sub-sample of 200 sentences was selected for manual review, stratified across service providers and document years to ensure representativeness.

The manual coding process addressed three objectives:
\begin{enumerate}
    \item \textbf{Validation:} Verifying the LLM classifications against human judgment to assess reliability and identify systematic errors or edge cases.
    \item \textbf{Tone Analysis:} Capturing the rhetorical ``tone'' of enforcement language that categorical classification cannot capture. For example, distinguishing between neutral legal boilerplate (``We may terminate your account...'') and threatening language (``Violations will result in immediate termination without refund...'').
    \item \textbf{Emergent Themes:} Identifying themes not captured by the predefined categories, such as references to ``fair use,'' ``educational purposes,'' or ``legitimate business needs'' that may signal adaptive rather than purely coercive approaches.
\end{enumerate}

This triangulation between automated and manual coding strengthens the validity of the overall classification, ensuring that the strategic patterns identified in Chapter 4 reflect genuine corporate positioning rather than artifacts of the LLM classification process.


\section{Data Analysis Procedures}
\label{sec:analysis_procedures}

The final analytical step involved synthesizing the quantitative and qualitative data streams. 

\subsection{Statistical Analysis of the DSPI}
The pricing data was analyzed using Python. Descriptive statistics (mean, median, standard deviation) were calculated for the DSPI across all services and regions. Correlation matrices were generated to examine the relationship between a country's income level and subscription pricing, testing whether price discrimination correlates strictly with national wealth or follows more complex patterns.

\subsection{Interpretation of Qualitative Classifications}
For the qualitative data, the JSON outputs from the Gemini 3 Flash pipeline were parsed and aggregated. The frequency of each "Strategic Frame" (e.g., \textit{Legal Compliance} vs. \textit{User Freedom}) and "Firm Action" (e.g., \textit{Technical Blocking}) was calculated per company and per year. 

To visualize the evolution of enforcement strategies, these frequencies were normalized against the total number of sentences per year to account for the documented growth in ToS document length over time \parencite{reidenberg2021privacy}. This allowed for the generation of longitudinal trend lines (see Chapter 4). Finally, a comparative analysis was conducted to contrast the rhetoric of "Fortress" strategy firms (high blocking) against "Globalist" strategy firms (price harmonization), identifying the key markers of each business model archetype.
