\label{chap:conclusion}

\section{Summary of Key Findings}
This thesis investigated the conflict between international price discrimination and consumer-driven geo-arbitrage.
\begin{itemize}
    \item \textbf{RQ1 (Incentive):} The Digital Services Price Index (DSPI) confirmed significant deviations from Purchasing Power Parity, creating massive economic incentives (often >70\% discounts) for consumers to engage in geo-arbitrage.
    \item \textbf{RQ2 (Response):} Through an advanced LLM-based classification of corporate texts, we found that firms predominantly employ "Coercive" rhetoric, framing circumvention as a violation of third-party licensing rather than a pricing dispute. However, technical enforcement varies significantly by industry.
    \item \textbf{RQ3 (Viability):} While firms have increased the technical sophistication of their barriers (Account Actions), the persistence of "User Workaround" discussions suggests that enforcement creates friction but does not eliminate the practice.
\end{itemize}

\section{Contribution to Research}
The study contributes a standardized metric (DSPI) for measuring digital price dispersion and demonstrates the utility of Large Language Models (Gemini 3 Flash) in automating the analysis of complex legal-strategic texts. Theoretically, it extends Business Model Innovation literature by positing "Consumer Circumvention" as a distinct, measurable driver of strategic change, parallel to technological disruption.

\section{Future Outlook}
As regulatory frameworks like the EU's Digital Single Market evolve, the legality of geo-blocking will face further challenges. Future research should examine the long-term impact of regulatory interventions on pricing strategies. Ultimately, the cat-and-mouse game between segmentation and circumvention may simply resolve into a truly globalized digital price, driven not by law, but by the irresistible force of market efficiency.
