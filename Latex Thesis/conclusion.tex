\label{chap:conclusion}

\section{Summary of Key Findings}
This thesis investigated the conflict between price discrimination and consumer geo-arbitrage.
\begin{itemize}
    \item \textbf{RQ1 (Incentive):} The Digital Services Price Index (DSPI) shows large price differences between countries, creating strong economic incentives (>70\% discounts) for consumers to use VPNs.
    \item \textbf{RQ2 (Response):} Strategies depend on the \textbf{Business Model}. "Content" firms enforce strict blocking because of licensing. "Software" firms use license keys. VPN providers, by definition, do not block users.
    \item \textbf{Effectiveness:} Firms have added more technical barriers since 2022, but the continued discussion of workarounds suggests these barriers only add friction, not a complete stop.
\end{itemize}

\section{Contribution to Research}
The study provides a new metric (DSPI) for measuring price differences and shows how Large Language Models can automate legal text analysis. Theoretically, it shows that "Consumer Circumvention" drives business model changes, just like technological innovation does.

\section{Future Outlook}
As regulations like the EU's Digital Single Market change, geo-blocking may become harder. Future research should track how laws affect pricing. Ultimately, the cycle of blocking and circumventing may end not because of better technology, but because global markets force prices to converge.
