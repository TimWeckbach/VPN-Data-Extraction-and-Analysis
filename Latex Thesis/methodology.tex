This chapter details how the research will be conducted.

To operationalize the 'Coercive' and 'Technical Barrier' constructs, this study employs the 'Litigious' and 'Constraining' word lists developed by Loughran and McDonald (2011), which are the established standard for detecting legal risk and restriction in financial texts.

\section{Research Design}
\begin{itemize}
    \item A mixed-methods approach is used.
    \item Combines quantitative analysis (to build and test the DSPI) with qualitative analysis (to understand strategic responses).
\end{itemize}

\section{Phase 1: Quantitative Data Collection \& Analysis (for RQ1)}

\subsection{Data Collection}
\begin{itemize}
    \item Development of the Digital Services Price Index (DSPI).
    \item Service Selection: Streaming, SaaS, gaming.
    \item Country Selection: High-, middle-, and low-income countries.
    \item Method: VPN-controlled price sampling.
    \item Data Points: Price, currency, and Purchasing Power Parity (PPP) data.
\end{itemize}

\subsection{Data Analysis}
\begin{itemize}
    \item Normalization of prices across different currencies and markets.
    \item Calculation of the DSPI.
    \item Statistical analysis (e.g., ANOVA, regression) to identify patterns and drivers of price differentiation.
\end{itemize}

\section{Phase 2: Qualitative Data Collection \& Analysis (for RQ2)}

\subsection{Qualitative Data Collection \& Analysis}
The analysis follows a systematic coding approach inspired by the \textbf{Gioia Methodology} \parencite{gioia2013seeking}. This involves structuring data into 1st-order concepts (raw terms found in text), 2nd-order themes (theoretical categories like "Technical Blocking"), and aggregate dimensions (Strategic Responses).
While initially conceptualized for manual coding \parencite{duriau2007content}, this hierarchical structure provided the logic for the automated classification pipeline described below.

\section{Automated Text Classification}
\label{sec:llm_methodology}

To address the limitations of traditional Natural Language Inference (NLI) models in capturing the nuanced legal and technical language of Terms of Service (ToS), this study implemented an advanced classification pipeline leveraging state-of-the-art Large Language Models (LLMs). Specifically, the pipeline was upgraded from a BERT-based architecture (DeBERTa-v3-large) to the \textit{Gemini 3 Flash} model, accessed via the Google Generative AI API.

\subsection{Model Selection and Rationale}
The selection of \textit{Gemini 3 Flash} was driven by the need for deeper reasoning capabilities and context awareness. Unlike NLI models, which classify based on entailment probabilities between a premise and a hypothesis, generative LLMs can interpret complex sentence structures and ambiguous legal standard terms (``General Terms'') versus specific geo-arbitrage restrictions. 

Key advantages observed during the model transition included:
\begin{itemize}
    \item \textbf{Contextual Understanding}: The ability to distinguish between benign references to ``account suspension'' (e.g., for fraud) and strategic ``Legal Threats'' tailored to prevent cross-border usage.
    \item \textbf{Zero-Shot Performance}: The model demonstrated high accuracy without extensive fine-tuning, utilizing a robust system prompt to align with the theoretical categories defined in Section 2.
    \item \textbf{Efficiency}: The ``Flash'' architecture provided a high throughput, enabling the processing of the entire dataset (approx. 25,000 sentences) within a reasonable timeframe.
\end{itemize}

\subsection{Operationalization of Constructs (The Coding Scheme)}
Based on the theoretical framework, the following coding scheme was enforced via the LLM system prompt. This scheme maps the abstract concept of "Strategic Response" into measurable data points.

\subsubsection{Strategic Frames}
The model was tasked to identify the underlying justification provided by the firm:
\begin{description}
    \item[Frame: Legal Compliance] Justifying geo-blocking as a non-negotiable legal or contractual necessity (e.g., "Due to licensing agreements...").
    \item[Frame: Security Risks] (Service Provider Frame) Arguments that VPNs/Proxies are unsafe, malicious, or compromise user data.
    \item[Frame: Privacy/Security] (VPN Provider Frame) Arguments focusing on encryption, anonymity, and protection from surveillance.
\end{description}

\subsubsection{Firm Actions}
The model categorized specific enforcement clauses into:
\begin{description}
    \item[Action: Technical Blocking] Active technological measures to detect or block the specific use of VPNs/Proxies (e.g., "We use geo-blocking technology", "Error 403").
    \item[Action: Legal Threat] Explicit threats of account termination, suspension, or legal action specifically for using circumvention tools.
    \item[Action: Account Action] General punitive measures against accounts (termination, suspension) for comprehensive violations.
    \item[Action: Price Discrimination] Explicit differences in pricing based on region, currency, or purchasing power.
    \item[Action: Legitimate Portability] Rules allowing temporary access while traveling (e.g., EU Portability Regulation).
\end{description}

\subsection{Pipeline Architecture and Implementation}
The reclassification process was automated using a customized Python script.

\subsubsection{System Prompt Engineering}
To ensure deterministic and theoretically grounded outputs, the system prompt was engineered with strict constraints. The exact prompt structure is provided below:

\begin{figure}[ht]
\begin{verbatim}
SYSTEM_PROMPT = """You are a scientific classifier for a Thesis on 'Digital Geo-Arbitrage'.
Classify a list of sentences into the provided categories. 

CATEGORIES:
1. Technical Blocking: Measures/Technologies used to detect or block...
2. Legal Threat: Explicit threats of account termination...
3. Security Risk: (Service Provider Frame) Arguments that VPNs are unsafe...
4. Privacy/Security: (VPN Provider Frame) Arguments focusing on encryption...
... [Full List of 10 Categories] ...

INSTRUCTIONS:
- Analyze sentences independently.
- Return a JSON array of objects...
- Format: [ { "category": "Category", "confidence": 0.9 }, ... ]
"""
\end{verbatim}
\caption{System Prompt used for Gemini 3 Flash Classification}
\label{fig:system_prompt}
\end{figure}

\subsubsection{Batch Processing and Error Handling}
To optimize for the API's rate limits and ensure data integrity, the pipeline utilized a batch processing approach. Sentences were grouped into batches of 25 and processed in a single API call. This method significantly reduced network overhead and total processing time.
A robust error-handling mechanism was implemented to manage API timeouts or rate limits (HTTP 429). The script included a ``circuit breaker'' to halt execution upon repeated failures and a resume function to continue processing from the last saved state.

\subsection{Methodological Validation: Gemini vs. Zero-Shot BERT}
To validate the choice of the Gemini 3 Flash model, a comparative analysis was conducted against a traditional Zero-Shot classification approach using a BERT-based model. The results demonstrated a massive divergence between the two models, reinforcing the necessity of using a modern LLM with large context windows for this specific task, consistent with recent findings on LLM performance in text annotation \parencite{gilardi2023chatgpt}.

\subsubsection{Agreement Analysis}
The comparison revealed an exceedingly poor agreement rate of \textbf{26.8\% (Accuracy)} between the two models. The Cohen's Kappa score was \textbf{0.032}, suggesting that the agreement was effectively equivalent to random chance. This discrepancy indicated a fundamental difference in how each model interpreted the classification tasks.

\subsubsection{The Core Conflict: Sensitivity vs. Context}
The analysis highlighted two distinct behaviors:
\begin{enumerate}
    \item \textbf{Gemini Performance:} The Gemini model correctly identified that approximately \textbf{91\%} of the dataset consisted of legal boilerplate, categorized as "General Terms." It successfully distinguished specific enforcement clauses from general legal language.
    \item \textbf{BERT Performance:} The BERT model exhibited "Over-Sensitivity," frequently assigning specific strategic tags based on the presence of individual keywords rather than semantic context.
\end{enumerate}

Specific examples of BERT's misclassification included:
\begin{itemize}
    \item \textbf{Legitimate Portability:} BERT flagged 7,853 sentences as "Legitimate Portability" that were merely "General Terms."
    \item \textbf{Account Action:} BERT flagged 6,134 "General Terms" sentences as "Account Action."
\end{itemize}

\textit{Interpretation:} BERT operates on keyword associations; for example, flagging a sentence like "You must have an account" as an "Account Action." In contrast, Gemini utilizes its reasoning capabilities to understand that the mere mention of an "account" is standard boilerplate ("General Terms") and reserves the "Account Action" tag for sentences explicitly regulating banning or suspension.

\subsubsection{Conclusion on Model Selection}
The validation proves that Zero-Shot BERT is insufficient for complex legal text analysis without extensive fine-tuning. It lacks the nuance required to distinguish between the mere mention of a topic (e.g., "portability") and the active regulation of it. Gemini, leveraging its massive context window and advanced reasoning capabilities, performs significantly better at filtering out noise and providing accurate stratifications. Consequently, Gemini 3 Flash was selected as the sole model for the final analysis.


\subsection{Standard Qualitative Coding}
In addition to the automated pipeline, manual qualitative coding is applied to a sub-sample to capture themes that escape rigid categorization, such as the specific "tone" of VPN provider marketing (e.g., empowering vs. technical).

\section{Analysis}
Description of the analytical process and data handling procedures, excluding the final results.
