\section{Classification Results}
\label{sec:classification_results}

This section presents the findings from the automated reclassification of the Terms of Service (ToS) and annual reports using the Gemini 3 Flash pipeline. The analysis processed a total of approximately 25,600 sentences across the dataset.

\subsection{Distribution of Enforcement Categories}
The classification revealed a significant dominance of ``General Terms'' within the corpus, with approximately \textbf{91\%} of the processed sentences identified as legal boilerplate. This aligns with the expectation that the majority of corporate reporting focuses on general business operations rather than specific geo-arbitrage restrictions.

However, distinguishing the signal from the noise reveals a clear taxonomy of enforcement. The following distribution (see Table \ref{tab:category_dist}) highlights the varying approaches of digital service providers.

\begin{table}[ht]
    \centering
    \begin{tabularx}{\textwidth}{l X c}
        \toprule
        \textbf{Category} & \textbf{Description} & \textbf{Frequency} \\
        \midrule
        Content Licensing & Geographic restrictions based on IP rights. & 857 \\
        Regulatory Compliance & Compliance with local laws and regulations. & 743 \\
        Account Action & Punitive measures (suspension, termination) for violations. & 280 \\
        Price Discrimination & Explicit mentions of regional pricing differences. & 197 \\
        Technical Blocking & Measures to detect/block VPNs, proxies, and IP masking. & 106 \\
        \bottomrule
    \end{tabularx}
    \caption{Distribution of Strategic Categories in ToS Documents}
    \label{tab:category_dist}
\end{table}

\begin{table}[ht]
    \centering
    \renewcommand{\arraystretch}{1.2}
    \begin{tabularx}{\textwidth}{l X c}
        \toprule
        \textbf{Category} & \textbf{Representative Quote (Extracted from ToS)} & \textbf{Confidence} \\
        \midrule
        Content Licensing & "We grant you a limited, non-exclusive license to access the service... only within geographic locations where we offer our service." & 0.98 \\
        \addlinespace
        Technical Blocking & "You may not use any technology to obscure or disguise your location." & 0.95 \\
        \addlinespace
        Account Action & "We reserve the right to terminate or restrict your use of our service, without notice, if we suspect... violation of these Terms." & 0.92 \\
        \addlinespace
        Price Discrimination & "Prices may vary by country... You will be charged in the currency associated with the location of your account creation." & 0.89 \\
        \bottomrule
    \end{tabularx}
    \caption{Representative Clauses for Detected Enforcement Strategies}
    \label{tab:quotes}
\end{table}

\subsection{Service-Specific Analysis}
The enforcement strategies vary significantly across different service providers, reflecting their distinct business models and regional licensing constraints. Figure \ref{fig:service_dist} illustrates the proportional distribution of categories for each service. 

Notably, services with heavy reliance on third-party content licensing exhibit a higher proportion of \textit{Content Licensing} clauses. For instance, \textbf{Disney+} (8.5\%) and \textbf{Netflix} (6.2\%) dedicate a significant portion of their terms to defining geographic rights. In contrast, VPN providers like \textbf{NordVPN} show a distinct focus on \textbf{Account Action} (10.9\%), reflecting a strategy of penalizing abuse rather than just blocking access. \textbf{YouTube Premium} stands out with a relatively high frequency of \textbf{Technical Blocking} clauses (2.2\%), indicating an active technological countermeasures approach.

Global platforms like \textbf{Amazon} and \textbf{Apple Music} also show notable spikes in \textbf{Regulatory Compliance} (4.4\% and 5.7\% respectively), suggesting their terms are heavily influenced by the diverse legal frameworks of the many jurisdictions they operate in.

\begin{figure}[ht]
    \centering
    \includegraphics[width=\textwidth]{figures/service_distribution_ratios.pdf}
    \caption{Proportional Distribution of Enforcement Categories by Service}
    \label{fig:service_dist}
\end{figure}

\subsection{Temporal Evolution of Enforcement}
To understand how these strategies have evolved over time, we analyzed the frequency of category specific clauses across the dataset's years. Figure \ref{fig:timeline_all} shows the aggregate trend, while Figure \ref{fig:timeline_service} breaks this down by service provider.

The aggregate data reveals a clear upward trend in specific enforcement clauses, particularly starting from 2022. This correlates with the increased global awareness of digital price arbitrage and the deployment of more sophisticated VPN detection technologies. The "cat-and-mouse" dynamic is evident in the data, where waves of new restrictive clauses often follow periods of increased user workaround activity.

\begin{figure}[ht]
    \centering
    \includegraphics[width=\textwidth]{figures/category_timeline_all_normalized.pdf}
    \caption{Temporal Evolution of Category Frequencies (Aggregate, Excluding General Terms)}
    \label{fig:timeline_all}
\end{figure}

\begin{figure}[ht]
    \centering
    \includegraphics[width=\textwidth]{figures/category_timeline_per_service_normalized.pdf}
    \caption{Temporal Evolution of Category Frequencies by Service (Normalized)}
    \label{fig:timeline_service}
\end{figure}

\subsection{High-Confidence Findings: The "Smoking Gun" Clauses}
The Gemini 3 Flash model identified specific, high-confidence clauses that serve as the "teeth" of the coercive strategy. For example, clauses explicitly stating "You may not use any technology to obscure or disguise your location" were consistently categorized as \textit{Technical Blocking} with $>0.95$ confidence. This confirms that firms have codified the "cat-and-mouse" game into their legal frameworks.

\section{The Digital Services Price Index (DSPI)}
\label{sec:dspi_results}

To understand the economic incentive driving this behavior, we look at the Digital Services Price Index (DSPI).

\subsection{Magnitude of the Arbitrage Incentive}
The data (derived from Phase 1 sampling) indicates a massive disparity between markets. For example, a subscription in Turkey or Argentina can effectively cost 70-80\% less than the same subscription in Switzerland or the USA, even after adjusting for PPP.

\begin{figure}[ht]
    \centering
    \includegraphics[width=0.9\textwidth]{figures/dspi_heatmap.pdf}
    \caption{Global Heatmap of Digital Service Pricing (The DSPI). Data represents the cost of local subscriptions relative to the US baseline (DSPI=1.0). Lower values indicate stronger arbitrage incentives.}
    \label{fig:dspi_map}
\end{figure}

This disparity creates a "super-normal" profit margin for the consumer-arbitrageur, which explains the high persistence of the behavior despite the technical barriers described in Section \ref{sec:classification_results}.

\section{Correlation Analysis: The Strategic Trade-off}
\label{sec:correlation}
To empirically validate the relationship between pricing strategy and enforcement intensity, we conducted a correlation analysis across the sampled service providers. We conceptualized the two variables as follows:
\begin{itemize}
    \item \textbf{Price Discrimination Score (X-axis):} The Standard Deviation of the DSPI for a given service across all sampled countries. A higher value indicates a more segmented, variable pricing structure (e.g., Netflix).
    \item \textbf{Enforcement Intensity (Y-axis):} The percentage of Terms of Service clauses categorized as "Technical Blocking," "Account Action," or "Content Licensing." This represents the "protectionist" effort of the firm.
\end{itemize}

Figure \ref{fig:correlation} illustrates the relationship between these two variables.

\begin{figure}[ht]
    \centering
    \includegraphics[width=0.9\textwidth]{figures/protection_vs_pricing.pdf}
    \caption{Strategic Alignment: firms with higher Price Discrimination scores exhibit higher Enforcement intensities.}
    \label{fig:correlation}
\end{figure}

The initial analysis reveals a complex relationship ($R \approx 0.36$) between price discrimination and legal enforcement intensity. While firms like \textbf{Netflix} and \textbf{Disney+} fit the expected "Fortress" model (High Price Variance $\rightarrow$ High Enforcement), \textbf{Apple Music} emerges as a significant strategic outlier.

Despite having high price discrimination (e.g., Turkey vs. US), Apple's Terms of Service show relatively low "Technical Blocking" intensity. This suggests a third strategic archetype: the **"Ecosystem Fortress."** Unlike Netflix, which must police open web access, Apple relies on hardware-level and payment-method friction (e.g., requiring a local credit card and Apple ID) to enforce segmentation. This "hard" friction reduces the need for the "soft" legal threats found in ToS documents, effectively achieving the same protection without the explicit coercive language.

Thus, the "Fortress" strategy has two implementations:
\begin{enumerate}
    \item \textbf{Legal/Technical Fortress (Netflix):} Relying on IP detection and ToS threats.
    \item \textbf{Ecosystem Fortress (Apple):} Relying on platform interoperability and payment barriers.
\end{enumerate}
