\label{chap:discussion}

This chapter synthesizes the quantitative findings from the DSPI and the qualitative insights from the automated classifcation pipeline to answer the research questions. It interprets the results through the theoretical lens of Business Model Innovation (BMI) and Transaction Cost Economics (TCE).

\section{The Strategic Archetypes of Geo-Arbitrage}
Based on the analysis of Terms of Service and corporate enforcement actions, we can categorize digital service providers into two distinct strategic archetypes:


\subsection{The Fortress Strategy (Coercive)}
Firms adhering to this strategy, typified by streaming giants like \textbf{Netflix} and \textbf{Disney+}, prioritize the maintenance of regional licensing agreements over user convenience. Our quantitative analysis confirms this focus: \textbf{Disney+} and \textbf{Netflix} allocate approximately \textbf{8.5\%} and \textbf{6.2\%} of their enforcement clauses to \textbf{Content Licensing} issues, respectively.

This aligns with the "Fortress" strategy described by \textcite{schmidt2020transnational}, where incumbent firms build digital walls to protect legacy revenue streams. However, as noted by \textcite{lobato2019geoblocking}, such strategies often suffer from a "legibility" problem—users do not see the legal contracts, only the "This content is not available" error. By relying on "Technical Blocking" and "Account Actions," these firms risk alienating users who perceive these barriers as artificial and unfair \parencite{anderson2020price}. The reliance on IP-based filtering is a direct attempt to enforce \textcite{bechtold2016price}'s concept of "perfect price discrimination," yet it paradoxically increases the "moral intensity" of the circumvention act by framing the user as an adversary.

\subsection{The Globalist Strategy (Adaptive)}
In contrast, platforms like \textbf{Apple Music} exemplify a "Globalist" or adaptive approach. With negligible focus on \textbf{Technical Blocking} (0.15\%) and a strong emphasis on \textbf{Price Discrimination} (5.7\%), Apple appears to accept the reality of the "Splinternet" \parencite{masnick2019splinternet}. Rather than fighting a futile technological war against VPNs \parencite{hohn2021vpn}, they focus on minimizing transaction costs.

This reflects \textcite{aguiar2016digital}'s argument that in the presence of low-cost piracy (or arbitrage), the optimal strategy is often to compete on service quality rather than enforcement. By offering seamless, portable accounts, Apple reduces the "non-monetary" utility of circumvention. A user \textit{could} save \$2 by using a Turkish account, but the friction of maintaining a foreign payment method outweighs the benefit when the service works flawlessly across borders anyway. This supports \textcite{goldfarb2019digital}'s assertion that "friction" is the defining variable in digital trade; Apple monetizes convenience, while Netflix monetizes exclusivity.

\section{Implications for Business Model Innovation}
The persistence of geo-arbitrage suggests that the "Regionally Segmented" business model is facing a "Napster Moment."
\begin{itemize}
    \item \textbf{Consumer Circumvention as Market Signal:} As \textcite{oberholzer2007effect} observed with music piracy, widespread circumvention is often a market signal of unmet demand or pricing inefficiency. The 70\%+ discounts identified by our DSPI for countries like Turkey create a massive "arbitrage wedge" that no amount of technical blocking can fully suppress \parencite{kwon2017geo}.
    \item \textbf{Platform Envelopment:} \textcite{eisenmann2011platform} describes how platforms can be "enveloped" by adjacent providers. Here, VPNs are acting as "parasitic envelopers," layering a global access service on top of regionalized content platforms. To survive, streaming services must innovate their business models to internalize this global access—perhaps by offering "Global Roaming" tiers at a premium, as suggested by \textcite{belleflamme2014price}.
    \item \textbf{The End of the "Law of One Price":} Our findings challenge the static view of PPP. In the digital age, \textcite{brynjolfsson2019measurement} argues we must account for the value of free/cheap digital goods. Geo-arbitrage effectively redistributes this welfare, allowing users in high-income countries to access the "consumer surplus" meant for low-income markets.
\end{itemize}

\section{Managerial Implications}
For managers, the findings suggest a pivot in strategy:
\begin{enumerate}
    \item \textbf{Move beyond IP Blocking:} Simple IP filtering is ineffective against modern residential VPNs.
    \item \textbf{Price Harmonization:} Where possible, reducing the PPP-adjusted spread between markets reduces the incentive for arbitrage (The Economic Solution).
    \item \textbf{Value Differentiation:} Instead of just blocking access, firms should differentiate the \textit{product} by region (e.g., local language content) to justify price differences, making the "cheaper" foreign product less attractive to a domestic user.
\end{enumerate}

\section{Limitations}
This study relies on \textit{ex-ante} enforcement signals (ToS, initial technical blocks) and does not fully capture \textit{ex-post} actions like shadow-banning. Furthermore, the DSPI is a snapshot in time; currency volatility (e.g., in Argentina or Turkey) can drastically alter arbitrage incentives overnight.
