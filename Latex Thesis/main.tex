%%
%\documentclass{}[
	%english,
	%ruledheaders=section,           %Ebene bis zu der die Überschriften mit Linien abgetrennt werden, vgl. DEMO-TUDaPub
	%class=report ,                   % Basisdokumentenklasse. Wählt die Korrespondierende KOMA-Script Klasse
	%thesis={type=bachelor},         % Dokumententyp Thesis, für Dissertationen siehe die Demo-Datei DEMO-TUDaPhd
	%accentcolor=9c,% Auswahl der Akzentfarbe
	%custommargins=true,% Ränder werden mithilfe von typearea automatisch berechnet
	%marginpar=false,% Kopfzeile und Fußzeile erstrecken sich nicht über die Randnotizspalte
	%BCOR=5mm,%Bindekorrektur, falls notwendig
	%parskip=half-,%Absatzkennzeichnung durch Abstand vgl. KOMA-Script
	%fontsize=11pt,%Basisschriftgröße laut Corporate Design ist mit 9pt häufig zu klein
	%logofile={figures/tuda_logo.pdf},
    %pdfa=false
%]%{tudapub}

\documentclass[
	english,
	ruledheaders=section,
	class=report,
	thesis={type=master},
	accentcolor=9c,
	custommargins=geometry,
	marginpar=false,
	parskip=half-,
	fontsize=12pt,
	logofile={figures/tuda_logo.pdf},
    pdfa=true
]{tudapub}

%\usepackage{showframe}
%\usepackage[reset, left=2.5cm, right=2.5cm, top=2.5cm, bottom=2.5cm, includefoot]{geometry}
%\geometry{a4paper, left=2.5cm, right=2.5cm, top=2.5cm, bottom=2.5cm, includefoot}
%\geometry{reset, bottom=2.5cm, includefoot} 


\usepackage[main=english, ngerman]{babel}
\usepackage[autostyle]{csquotes}
\usepackage{microtype}
\usepackage[table]{xcolor}

\usepackage{graphicx}
\usepackage{amsmath}
\usepackage{tabularx}
\usepackage{booktabs}
\usepackage{threeparttable}
\renewcommand{\TPTnoteSettings}{\fontsize{10}{12}\selectfont}
\usepackage{pifont}
\usepackage{tikz}
\usetikzlibrary{shapes,arrows,positioning,calc}

\usepackage{setspace}
\onehalfspacing

\usepackage{listings}
\lstset{
  basicstyle=\ttfamily\small,
  breaklines=true,
  frame=single,
  columns=fullflexible
}

\usepackage{pgfplots}
\pgfplotsset{compat=1.18}
\usepgfplotslibrary{colormaps}
\definecolor{tudared}{HTML}{B90E21}
\definecolor{tudablue}{HTML}{004E8C}
\definecolor{tudagreen}{HTML}{7DA62D}
\definecolor{tudaorange}{HTML}{E65C00}
\definecolor{tudagray}{HTML}{9C9C9C}
\pgfplotsset{
    scaled ticks=false,
    tick label style={/pgf/number format/fixed},
    /pgf/number format/precision=3,
    /pgf/number format/fixed zerofill=false
}

% Use KOMA-Script recommended method for section formatting
\RedeclareSectionCommand[
    beforeskip=0pt,
    afterskip=10pt
]{chapter}


\usepackage{caption}
\usepackage{subcaption}
\DeclareCaptionFont{tenpoint}{\fontsize{10}{12}\selectfont}
\captionsetup{font=tenpoint}


\usepackage[backend=biber, style=apa, sorting=nyt]{biblatex}
\addbibresource{Bibliography.bib}

\usepackage[acronym, section]{glossaries}
%glossaries here
\newacronym{vpn}{VPN}{Virtual Private Network}
\newacronym{ppp}{PPP}{Purchasing Power Parity}
\newacronym{dspi}{DSPI}{Digital Services Price Index}
\newacronym{tos}{ToS}{Terms of Service}
\newacronym{llm}{LLM}{Large Language Model}
\newacronym{ptw}{PTW}{Price-to-Wage}
\newacronym{bmi}{BMI}{Business Model Innovation}
\newacronym{drm}{DRM}{Digital Rights Management}
\newacronym{dpi}{DPI}{Deep Packet Inspection}
\newacronym{rq}{RQ}{Research Question}
\newacronym{eu}{EU}{European Union}
\newacronym{nli}{NLI}{Natural Language Inference}
\newacronym{cdn}{CDN}{Content Delivery Network}
\newacronym{saas}{SaaS}{Software as a Service}
\newacronym{gdpr}{GDPR}{General Data Protection Regulation}
\newacronym{dvpn}{dVPN}{Decentralized Virtual Private Network}
\newacronym{bma}{BMA}{Business Model Adaptation \& Pricing}
\newacronym{crl}{CRL}{Coercive Restriction \& Legal Threat}
\newacronym{gco}{GCO}{General Corporate Operations}

\newglossaryentry{pricehopping}{
	name={price hopping},
	description={The practice of comparing and exploiting different regional prices for digital services by using technology such as VPNs or other tools to access cheaper pricing in foreign markets}
}

\newglossaryentry{discriminationdilemma}{
	name={discrimination dilemma},
	description={The paradox that the more aggressively a firm discriminates on price across regions, the greater the arbitrage incentive it creates and the higher the enforcement costs required to maintain market segmentation}
}

%  Strategic Archetypes 

\newglossaryentry{contentfortress}{
	name={Content Fortress},
	description={Strategic archetype in which content streaming firms (e.g., Netflix, YouTube, Disney+) defend geographic price segmentation primarily through technical blocking and content-licensing restrictions}
}

\newglossaryentry{enterprisefortress}{
	name={Enterprise Fortress},
	description={Strategic archetype in which utility software firms (e.g., Microsoft) protect value capture through identity verification and legal threats at the point of purchase rather than network-level blocking during use}
}

\newglossaryentry{ecosystemfortress}{
	name={Ecosystem Fortress},
	description={Strategic archetype in which platform firms (e.g., Apple, Amazon) make geo-arbitrage impractical by embedding digital services within broader product ecosystems tied to a user's physical identity, creating high switching costs that serve as implicit enforcement}
}

\newglossaryentry{utilityparadox}{
	name={Utility Paradox},
	description={Strategic archetype describing firms like Adobe whose moderate price discrimination and low technical blocking enforcement stem from reliance on on-device software activation (cryptographic license keys) rather than on-network IP filtering, revealing that available enforcement mechanisms depend on value-delivery architecture rather than pricing strategy}
}

\newglossaryentry{affordabilityparadox}{
	name={Affordability Paradox},
	description={The finding that nominally ``cheap'' digital subscription prices in low-income markets are often significantly more expensive for local consumers in real terms (as a share of local wages), challenging the assumption that geographic price discrimination primarily benefits developing-economy consumers}
}

\newglossaryentry{bureaucraticfortress}{
	name={Bureaucratic Fortress},
	description={A variant of the Ecosystem Fortress strategy (used for Amazon Prime) where the barrier to geo-arbitrage is not an IP filter but a valid residential address and local credit card, making arbitrage logistically difficult rather than technically impossible}
}

%  Key Metrics 

\newglossaryentry{fortressindex}{
	name={Fortress Index},
	description={A metric calculating the percentage of enforcement-related clauses (Technical Blocking + Legal Threat) relative to the total number of strategic (non-General Terms) sentences in a firm's documentation, measuring enforcement focus independently of document length}
}

\newglossaryentry{pricediscriminationscore}{
	name={Price Discrimination Score},
	description={The standard deviation of a service's \gls{dspi} values across countries, used as a measure of cross-country price variance and thus the strength of the arbitrage incentive for that service}
}

\newglossaryentry{enforcementintensity}{
	name={Enforcement Intensity},
	description={The percentage of Technical Blocking and Legal Threat clauses relative to total sentences for a given service, measuring the density of coercive enforcement language in corporate documentation}
}

%  Analytical Concepts 

\newglossaryentry{geoarbitrage}{
	name={geo-arbitrage},
	description={The practice of exploiting geographic price differences for digital services by using VPNs or similar tools to purchase subscriptions at prices intended for a different (typically lower-income) market; used synonymously with ``price hopping'' and ``subscription hopping'' in this thesis}
}

\newglossaryentry{digitalgeoarbitrage}{
	name={digital geo-arbitrage},
	description={A subset of \gls{geoarbitrage} specific to digitally delivered subscription services (e.g., streaming, cloud storage, SaaS), where the intangible nature of the product eliminates shipping costs and physical border controls, making price discrimination both easier to implement and easier to circumvent via \gls{vpn} or proxy tools}
}

\newglossaryentry{forwardfill}{
	name={forward-fill strategy},
	description={A gap-filling method used in the longitudinal \gls{tos} analysis: when no new document was published in a given year, the most recent version is carried forward on the assumption that its clauses remain in effect until explicitly replaced, distinguishing between ``missing data'' and ``persistent rules''}
}

\newglossaryentry{coerciveinnovation}{
	name={coercive innovation},
	description={A category of business model response focused on Value Capture, in which firms reinforce barriers to protect the existing pricing model through technical blocking, legal threats, and account restrictions}
}

\newglossaryentry{adaptiveinnovation}{
	name={adaptive innovation},
	description={A category of business model response focused on Value Proposition, in which firms change their product offering to make arbitrage irrelevant (e.g., global pricing, ecosystem lock-in, original content production)}
}

\newglossaryentry{enforcementsurge}{
	name={Enforcement Surge},
	description={The sharp escalation in Technical Blocking and Legal Threat clauses observed in corporate documents beginning in 2022, peaking in 2023, driven primarily by YouTube's crackdown on VPN-based subscription arbitrage}
}

\newglossaryentry{adversarialcycle}{
	name={Adversarial Cycle},
	description={The recurring action-reaction dynamic between platform enforcement measures and VPN circumvention technologies, decomposed into four phases: The Wild West, The First Wall, The Escalation, and The New Equilibrium}
}

\newglossaryentry{arbitragewindow}{
	name={Arbitrage Window},
	description={The dynamic and potentially short-lived period during which a specific country offers a favorable price for geo-arbitrage, subject to change through currency fluctuations, local price adjustments, or regulatory intervention}
}

%  Strategic Frame Categories 

\newglossaryentry{technicalblocking}{
	name={Technical Blocking},
	description={A firm action category in the coding scheme denoting active technological measures to detect or block the use of VPNs or proxies (e.g., IP blacklisting, geo-blocking technology, \gls{drm} enforcement)}
}

\newglossaryentry{legalthreat}{
	name={Legal Threat},
	description={A firm action category in the coding scheme denoting explicit threats of account termination, suspension, or legal action specifically for using circumvention tools}
}

\newglossaryentry{contentlicensing}{
	name={Content Licensing},
	description={A firm action category in the coding scheme denoting geographic restriction of content availability due to territorial rights agreements (e.g., ``not available in your region'')}
}

%  Methodological Concepts 

\newglossaryentry{threelevel}{
	name={Three-Level Mechanism},
	description={A conceptual model proposed in this thesis (drawing from self-control theory) decomposing the consumer decision to engage in geo-arbitrage into three levels: Individual (rational cost-benefit), Inter-personal (social influence from online communities), and Societal (moral intensity and neutralization techniques)}
}

\newglossaryentry{digitalaudit}{
	name={digital audit},
	description={The data collection methodology adapted from Hannak et al.\ (2014) in which a virtual presence is established in each target country using a \gls{vpn} to simulate local access, recording subscription prices as a form of ``mystery shopping'' in digital markets}
}

\newglossaryentry{gioiamethodology}{
	name={Gioia Methodology},
	description={A systematic qualitative coding approach organizing data into layered abstractions (1st-order concepts, 2nd-order themes, aggregate dimensions), used as the conceptual framework for the automated \gls{llm} classification pipeline in this thesis}
}

\makeglossaries

% change count of figures and tables to not include chapter number
\usepackage{chngcntr}
\counterwithout{figure}{chapter}
\counterwithout{table}{chapter}

\title{Business Model Responses to Consumer Circumvention: Lessons from Piracy Applied to
VPN-Enabled Geo-Arbitrage}
\author{Tim Weckbach}
\reviewer{William Schütte \and Prof. Dr. Alexander Kock}




\begin{document}

%\setlength{\textheight}{24cm}
%\setlength{\topmargin}{-1.2cm} % Muss oft angepasst werden
%\setlength{\headsep}{1cm}
%\setlength{\footskip}{1.5cm}
%\setlength{\voffset}{5cm}


\Metadata{
	\title{Business Model Responses to Consumer Circumvention: Lessons from Piracy Applied to
		VPN-Enabled Geo-Arbitrage}
	\author[Tim Weckbach]{Tim Weckbach}
	\keywords{Price Discrimination, Geo-Arbitrage, Business Model Innovation, VPN, Digital
		Services Price Index}
	\publisher{TU Darmstadt}
	\copyright{Tim Weckbach}
	\birthplace{Frankfurt am Main}
	\reviewer{William Schütte \and Prof. Dr. Alexander Kock}
}
\department{Business Informatics}
\institute{Technology and Innovation Management}
\submissiondate{17.02.2026}
\maketitle
\newgeometry{bottom=2.5cm, includefoot}
\raggedbottom

\begin{abstract}
	This thesis examines the strategic conflict between corporate price discrimination and
	consumer-driven geo-arbitrage using \glspl{vpn} in digital subscription markets. Using a
	mixed-methods design, the study first measures the economic motivation for \gls{digitalgeoarbitrage}
	through the \gls{dspi}, a price ratio index benchmarked against the US market. Analysis of
	up to 11 services across 11 countries reveals discounts of up to 90\% in markets like Turkey
	and Pakistan, yet a \gls{ptw} ratio shows an ``Affordability Paradox'': these services remain
	more expensive for local consumers relative to their income.

	An \gls{llm}-based analysis of over 25,000 sentences from corporate documents and \gls{tos}
	then identifies coercive and adaptive \gls{bmi} responses. The findings show that enforcement
	strategies are shaped more by business model architecture than by the scale of price
	differences. A sharp spike in technical countermeasures beginning in 2022, driven primarily
	by YouTube, signals an escalating arms race between platform control and technical bypass.
	The thesis concludes that adaptive strategies such as pricing or platform innovation may prove more
	sustainable than coercive enforcement, mirroring the music industry's shift from
	\gls{drm} enforcement to convenient streaming.
\end{abstract}

\noindent\textbf{Keywords:} Price Discrimination, Geo-Arbitrage, \gls{bmi},
\gls{vpn}, \gls{dspi}
\tableofcontents


\chapter{Introduction}
\label{chap:introduction}
This chapter provides an overview of the research and highlights its significance.
It introduces the central research problem: digital services naturally cross borders,
but platforms attempt to maintain geographic pricing boundaries. By defining the problem
and research questions, this chapter lays out the structure of the thesis.

\section{Background and Context}
The global digital economy has a basic contradiction: while digital services are globally
accessible and deliver everywhere similar value,
companies in this space commonly use, among other price discrimination strategies, geographically segmented pricing
strategies \parencite{goldfarb2019digital}. Major platforms such as Netflix, Spotify, and Microsoft 365 divide the
world into different pricing regions, charging very different prices for the same digital
products depending on where the customer is. This practice, known as third-degree
price discrimination \parencite{belleflamme2015piracy,odlyzko2003privacy}, lets firms get the most revenue from each market by matching
prices with what local users can pay. In markets with lower incomes, services are offered at much
lower prices to reach more customers, while wealthier regions face higher prices.
This splitting of the digital economy into regional pricing zones reflects broader
tensions in digital capitalism \parencite{srnicek2017platform}.

However, the technical setup that enables global digital distribution also lets
customers challenge these boundaries. The global \gls{vpn} market has grown rapidly,
reaching an estimated \$44.6 billion in 2022 and projected to exceed
\$137 billion by 2030 \parencite{grandviewresearch2023vpn}. While privacy protection
remains the primary stated reason for \gls{vpn} adoption, surveys consistently show
that accessing geo-restricted content and obtaining better prices rank among the top
motivations for consumer \gls{vpn} use \parencite{security2023vpn}.
An estimated 31\% of internet users worldwide have used a \gls{vpn} at least
once \parencite{globalwebindex2023vpn}, with adoption rates highest in markets
with significant content restrictions or price differentials, precisely
the markets where geo-arbitrage is most profitable.

Many users have discovered that \glspl{vpn} can be used not only for privacy but also to bypass
regional price blocks. By hiding their true location and routing their internet traffic through
servers in other
regions, users can change their location of subscription to access regionally restricted content
libraries or purchase subscriptions at prices meant for markets with very different economic and
regulatory conditions. This practice, called ``\gls{digitalgeoarbitrage}'' or ``\gls{vpn} price or licensing
hopping'' (used synonymously in this thesis), is a form of consumer driven market arbitrage that takes
advantage of the price and library differences created by firms \parencite{ariyarathna2022circumvention}.

The economic impact of this is large.
While precise figures on revenue losses from geo-arbitrage remain unavailable
in the academic literature, companies treat detection rates and revenue impact
assessments as commercially sensitive, the scale of the driver is clear
from the pricing data itself. For individual consumers, the savings can be
substantial: a user subscribing to five digital services through a low-price
region could save over \$100 per month compared to their home-country prices,
easily justifying the \$5-12 monthly cost of a commercial \gls{vpn} subscription.
This asymmetry between modest circumvention costs and significant potential
savings creates a powerful economic incentive that purely technical barriers
struggle to overcome.

This pattern parallels the digital piracy wave of the early 2000s, where users
used technical tools to bypass blocks \parencite{oberholzer2007effect}. Just as
file-sharing enabled users to bypass payment systems, \gls{vpn}-enabled geo-arbitrage
allows users to bypass pricing, though they still pay for the service. In both cases, consumers employ accessible
technology to circumvent barriers. This forces companies to innovate on their revenue models and develop
new business strategies, just as the rise of piracy ultimately led to the
emergence of new streaming business models such as Spotify and Netflix \parencite{aguiar2018effect}.
However, as discussed in Chapter \ref{chap:discussion}, this cycle of
bypassing is met by more advanced, often secret, detection technologies that try to unmask \gls{vpn}
users. These methods remain difficult to study, as companies treat their detection
mechanisms as proprietary, making them a
``black box'' for researchers.

\section{Problem Statement}
The main problem this thesis looks at is the tension between firms' efforts to maintain segmented markets
and consumers' ability to circumvent geographic restrictions.
Subscription companies face a fundamental trade-off between two competing imperatives.
On the one hand, the \textbf{economic imperative} demands that firms offer substantially lower prices in emerging markets while maintaining premium pricing in wealthier regions, in order to maximize market penetration and revenue. On the other hand, the \textbf{technical reality} is that the internet enables users to circumvent geographic restrictions using readily accessible \gls{vpn} services.
This conflict forces companies to rethink their strategies. They must choose between
``Coercive'' strategies that try to keep markets separate through technical blocking,
``Adaptive'' strategies that reduce the reason for arbitrage by making prices more similar,
or entirely new types of business models that operate fundamentally differently.

The challenge gets worse because these two goals are not independent.
Raising prices in low-income markets to reduce the arbitrage motivation undermines the
economic reasoning for price discrimination in the first place, potentially losing
those markets to competitors or to piracy. Conversely, investing in
ever more sophisticated blocking technologies generates recurring costs,
creates friction for legitimate travelers, and risks alienating the very
users firms seek to keep. This creates what can be described as a
\gls{discriminationdilemma}: the more aggressively a firm discriminates on price,
the greater the arbitrage incentive and the higher the enforcement costs required
to maintain market segmentation \parencite{tirole1988theory}. The dilemma is
further complicated by the asymmetric nature of the contest: while firms must
invest in detection systems that work across all users, circumvention tools need
only succeed for some users some of the time to remain commercially viable.

\section[Research Questions (\glspl{rq})]{Research Questions (\glspl{rq})}
To analyze this conflict, this thesis addresses two main research questions
that address both why users engage in geo-arbitrage and how companies respond to
it.

The first research question focuses on measuring the economic driver:
\textbf{\gls{rq}1 (The Economic Incentive):} \textit{To what extent does international price
	differentiation for digital services deviate from local purchasing power (based on local wages),
	and what degree of economic incentive does this create for consumer-driven geo-arbitrage?}

To answer this, we built the \gls{dspi}, which compares nominal subscription prices across regions relative to a US baseline. We complemented the \gls{dspi} with a separate \gls{ptw} ratio that measures how affordable these subscriptions are relative to local wages. The difference between nominal price (the sticker price in USD) and real affordability (the cost relative to local income) is key: a service may seem cheap to a foreign user while being expensive for a local.

The second research question examines how companies respond to the arbitrage threat:
\textbf{\gls{rq}2 (The Strategic Response):} \textit{How do digital service providers strategically frame the issue of circumvention, and to what
	extent can a shift towards coercive or adaptive business model responses be measured in their
	corporate disclosures over time?}

To answer this, we used AI to analyze and classify \gls{tos} documents and
annual reports, grouping company language into categories like ``Technical
Blocking,'' ``Legal Threat,'' and ``Price Discrimination.'' Looking at data over time lets us
see how strategies have changed from 2020 to 2025.

%\section{Structure of the Thesis}
%The thesis is structured step by step, going from theory through findings to the discussion.

%\textbf{Chapter \ref{chap:theory}} reviews three theoretical domains: (1) the economics of
%price discrimination; (2) consumer behavior and digital piracy; and (3) business model
%innovation theory, which provides the conceptual framework for understanding company responses.

%\textbf{Chapter \ref{chap:methodology}} explains the research methods. We describe how we built
%the \gls{dspi} and the complementary \gls{ptw} ratio using data from many countries and services. We also
%describe the \gls{llm} pipeline used to analyze over 25,000 sentences from company documents.

%\textbf{Chapter \ref{chap:results}} shows the findings in three parts: (1) the price
%picture; (2) how enforcement strategies have changed; and (3) the link between pricing and
%blocking.

%\textbf{Chapter \ref{chap:discussion}} interprets these results through the lens of business theory,
%identifying four strategic archetypes (Content Fortress, Ecosystem Fortress, Enterprise Fortress,
%and the Utility Paradox) and their implications for digital platform strategy.

%\textbf{Chapter \ref{chap:conclusion}} summarizes the study, acknowledges limitations, and outlines directions for
%future research, including the potential implications of emerging regulation such as the \gls{eu} Digital Single Market.

\chapter{Theoretical Foundations \& Literature Review}
\label{chap:theory}


This chapter builds the theoretical framework by looking at three areas: economic
pricing, consumer behavior, and business model innovation. By connecting Hal Varian's work on
price discrimination with the history of digital piracy, we create a way to understand modern
geo-arbitrage as a major challenge for digital companies.

\section{Economic Foundations of International Price Setting}
\label{sec:theory_pricing}

To understand why consumers engage in geo-arbitrage, we first need to understand why firms create
the price gaps that make such arbitrage worthwhile. Geographic price differences are
not random, they follow well-known economic principles.

\subsection{Third-Degree Price Discrimination}
According to \textcite{varian1989price}, third-degree price discrimination happens when a firm divides the
market based on visible traits—in this case, geographic location—and charges different prices
to each group. For digital goods, where the cost of copying is near zero ($MC \approx 0$)
\parencite{shapiro1998information,amit2001value}, this strategy allows firms to get the most
consumer surplus from both high-income (e.g., Switzerland) and low-income (e.g., Turkey)
markets at the same time.

Two conditions must hold for successful price discrimination.
First, \textit{market segmentation} requires that the firm be able to distinguish between consumer groups based on observable characteristics (such as IP address or billing location). Second, the \textit{no-arbitrage condition} requires that the firm be able to prevent the resale or transfer of the good between segments, a condition rooted in the economics of intellectual property rights \parencite{stiglitz2008economic}.
\gls{vpn}-enabled geo-arbitrage directly breaks \textit{Condition 1}, effectively merging the
different regions into a single global market. \textcite{duchbrown2016geoblocking} demonstrate
that geo-blocking has become the primary mechanism for maintaining this segmentation in digital content markets.

This study also suggests that companies accidentally segment their customers into sub-groups
within high-income regions. By maintaining detection mechanisms (such as \gls{vpn} blocking), companies
effectively segment the market based on user willingness to exert effort. This creates a division
between ``active'' users—those who spend time and effort to bypass blocks, and ``passive'' users
who prefer convenience. In this way, technical friction functions as a self-selection mechanism \parencite{shapiro1998information}:
companies retain high revenue from the majority of users while tolerating marginal losses from a small
group of technically proficient users.

\subsection{Purchasing Power Parity (\gls{ppp}) as a Benchmark}
The ``Law of One Price'' suggests that in an efficient market, identical goods should sell at
the same price when shown in a common currency \parencite{pakko2003burgernomics}.
However, differences from this law are common. Indeed, \textcite{engel2004european}
demonstrate that price convergence remains incomplete even under economic integration,
as firms actively maintain price differentials through market segmentation.
\textcite{rogoff1996ppp} argues that for physical goods, shipping costs and trade barriers
justify price differences. In the digital world, \textcite{clemons2002price} note that while
transaction costs are much lower, price differences continue because firms can set up detailed
customer segmentation. \textcite{blum2006gravity} provide important evidence that digital goods
do not fully ``defy the law of gravity'': even in online commerce, geographic distance and
national borders and regulations continue to affect trade patterns, suggesting that physical-world frictions
partially persist in digital markets. The Billion Prices Project \parencite{cavallo2016billion}
has further demonstrated that online price data can serve as a reliable tool for measuring
cross-country price differences, providing methodological precedent for the \gls{dspi} developed in this thesis.

The concept of \gls{ppp} has a long history in economics, originating from Gustav Cassel's
work in the early twentieth century \parencite{cassel1918abnormal} and later formalized through
the Balassa-Samuelson hypothesis, which predicts that price levels tend to be higher in
countries with higher productivity \parencite{balassa1964purchasing}. In its absolute form,
\gls{ppp} posits that a basket of goods should cost the same everywhere after currency conversion.
In its relative form, it predicts that exchange rate changes should offset inflation differentials over time.
For digital goods, however, neither form fully applies: there are no shipping costs, no tariffs,
and no physical constraints on distribution. The marginal cost of serving an additional user in
any country is effectively zero. This means that price differences for digital services are
almost entirely a function of deliberate corporate strategy rather than cost-driven economic forces \parencite{shapiro1998information, goldfarb2019digital}.

We use \gls{ppp} as a benchmark for ``economically justified'' pricing. If a
Netflix subscription in Turkey is cheaper than in the US only because of currency value and
local purchasing power, this fits standard economic theory. However, if the price difference
is bigger than what \gls{ppp} adjustments would predict, it creates a ``above-normal'' arbitrage
motivation, a price gap that motivates bypassing beyond simple purchasing power factors. We
measure this through the \gls{dspi}.

It is important to distinguish between two types of price dispersion \parencite{varian1989price}.
First, ``welfare-enhancing'' price discrimination occurs when firms lower prices in
poorer markets to expand access, effectively subsidizing digital inclusion.
Second, ``rent-extracting'' price discrimination occurs when firms charge higher prices in
wealthy markets not because costs are higher, but because consumers' willingness to pay is greater \parencite{odlyzko2003privacy, stiglitz2008economic}.
The combination of both creates the asymmetric incentive structure at the heart of geo-arbitrage:
the welfare-enhancing discount in one market becomes the rent-extraction loophole exploited by consumers from another.

\section{Consumer Circumvention and the Piracy Parallel}
\label{sec:theory_piracy}

Consumer-driven arbitrage is not new \parencite{anson2019arbitrage}. The digital ``geo-arbitrage''
pattern can be understood through the history of digital piracy.

\subsection{The Piracy Analogue}
\textcite{oberholzer2007effect} showed that file-sharing forced the music and film industry to change
its business model, eventually leading to legitimate digital distribution platforms like
iTunes, Spotify and Netflix. Subsequent research confirmed that while piracy displaced some sales,
it also drove innovation toward legitimate distribution platforms \parencite{waldfogel2010music}.
This historical precedent provides a framework for understanding \gls{vpn}-based geo-arbitrage.
Similarly, geo-arbitrage works as a market signal, showing a basic mismatch between rigid
regional pricing structures and the borderless reality of the global internet.
This division has been described as a ``Splinternet'' \parencite{masnick2019splinternet},
where the same service exists in different versions across different regions.

The parallel is useful: just as Napster and BitTorrent exposed the music industry's
failure to meet consumer demand for convenient digital access, \gls{vpn}-enabled price hopping
highlighted the sustainability challenges of global price discrimination. In both cases,
users innovated before companies adapted. The sudden shutdown of Megaupload in 2012 demonstrated
both the scale of piracy and the limits of enforcement-only approaches \parencite{danaher2014gone}.
France's ``graduated response'' law (HADOPI) provided further evidence:
while \textcite{danaher2014graduated} found a modest positive effect on iTunes sales,
the law did not eliminate piracy but rather displaced it to harder-to-track channels.
Similarly, \textcite{reimers2016effect} found that copyright protection measures have
measurable effects on legitimate sales, confirming that piracy and legitimate consumption
are connected rather than independent phenomena.

The transition from piracy to legitimate consumption was not automatic. It required significant
business model innovation. \textcite{aguiar2018effect} found that the growth of music streaming
displaced both piracy and legitimate purchases, suggesting that convenient,
affordable access can serve as a substitute for unauthorized consumption. This finding is directly relevant to
geo-arbitrage: if the price gap between regions were smaller, or if services offered globally
uniform access, the motivation for \gls{vpn}-based circumvention would diminish,
just as affordable streaming reduced the motivation for file-sharing.

As we discuss later (Chapter \ref{chap:discussion}), companies are likely improving their ability to
accurately geolocate users. However, empirical verification remains difficult because the technical details
are proprietary, similar to the piracy case of \gls{drm} protection.

The broader literature on digital piracy confirms that enforcement alone rarely eliminates
unauthorized access \parencite{peukert2017piracy}. \textcite{dejean2009piracy} argue that
piracy can even function as a ``sampling mechanism,'' where unauthorized access leads to
subsequent legitimate purchases, a dynamic that may also apply to geo-arbitrage, where
users who discover a service at a lower price may later convert to full-price subscribers
when enforcement tightens. However, key differences emerge between piracy and geo-arbitrage.
In terms of \textit{access versus price}, piracy was often about getting content that was not available at all, while geo-arbitrage concerns obtaining a more favorable price; piracy involved no payment, whereas geo-arbitrage involves payment at a reduced rate. Regarding \textit{legal status}, piracy is clearly illegal, while geo-arbitrage occupies a gray zone \parencite{ariyarathna2022circumvention}, as users pay for real accounts but misrepresent their geographic location. Finally, in terms of \textit{industry response}, the music industry eventually adapted with streaming services, and whether companies will adopt similar adaptive responses to geo-arbitrage is a central question of this thesis.

A closely related form of circumvention that bridges piracy and geo-arbitrage is
\textit{password sharing}, where subscribers share their account credentials with
non-paying users. \textcite{chen2026password} model the economics of password sharing
in digital subscription markets, demonstrating that it creates a structurally similar
tension between platforms' revenue protection and consumer surplus maximization.
Like geo-arbitrage, password sharing involves a genuine subscriber account (unlike piracy),
but undermines the platform's value capture by extending access beyond the intended
paying user base. Netflix's high-profile crackdown on password sharing beginning in
2023, which coincides precisely with the broader enforcement surge documented in our
\gls{tos} analysis (Table~\ref{tab:qual_timeline_complete}), illustrates how platforms
increasingly treat all forms of access circumvention, whether geographic or credential-based,
as strategic threats requiring coordinated enforcement responses.
This suggests that geo-arbitrage should not be studied in isolation but as part of a
broader spectrum of consumer-driven circumvention behaviors that collectively challenge
digital subscription business models.

\subsection{The Three-Level Mechanism of Circumvention}
Drawing from the self-control framework applied by \textcite{higgins2008piracy} to digital
piracy, we propose that the decision to engage in geo-arbitrage can be modeled as a three-level mechanism.
This way of looking at it helps explain why otherwise law-abiding consumers engage in
``digital smuggling'':

\subsubsection{Individual Level: Rational Choice and Personal Risk}
At the individual level, the consumer performs a cost-benefit analysis. The financial gain
(e.g., an 83\% discount on YouTube Premium from Turkey) is weighed against the perceived probability
of detection and the severity of punishment \parencite{ransbotham2009choice}
(e.g., account termination). When enforcement is perceived as inconsistent, the perceived
risk may be low. Crucially, the expected utility calculation is heavily skewed
in favor of circumvention: even if one assumes a moderate probability of detection
(say, 20\% to 30\%), the worst-case outcome (account termination with refund
of remaining subscription) is far less severe than the punishment for comparable
physical-world arbitrage (e.g., customs seizures, import duties, legal penalties).
This asymmetry between potential gain and potential loss creates a rational incentive
for risk-neutral consumers to attempt geo-arbitrage, consistent with expected utility
theory \parencite{becker1968crime}.

\subsubsection{Inter-personal Level: Social Influence}
The behavior may be reinforced by online communities (e.g., Reddit, Discord, specialized forums).
Observing others successfully using \glspl{vpn} can lower the psychological barrier to
entry, consistent with social influence research \parencite{kastanakis2012between}.
The role of ``how-to'' guides and community knowledge-sharing is particularly
significant: detailed tutorials with step-by-step instructions for subscribing
to services through foreign accounts effectively reduce the technical skill required
to near zero, democratizing what was once a practice limited to technically
sophisticated users. This ``knowledge diffusion'' effect \parencite{wasko2005share} means that as communities
grow, the effective barrier to entry falls, creating a positive feedback loop between
adoption and accessibility.

\subsubsection{Societal Level: Moral Intensity}
The perception of the act is key. Unlike shoplifting a physical good or pirating digital content outright, digital
arbitrage may be framed by users not as theft, but as a response to pricing perceived
as unfair. This framing aligns with neutralization theory
\parencite{thongmak2017neutralization}, and reflects broader concerns about
perceived price fairness in digital markets \parencite{xia2004price,poort2019price}.
Users employ several ``neutralization techniques'' to justify their behavior
\parencite{sykes1957techniques}: (1) denial of injury
(``The company still gets paid, just less''), (2) denial of victim
(``These are billion-dollar corporations''), (3) condemnation of the condemner
(``They charge unfair prices''), and (4) appeal to higher loyalties
(``I'm supporting the service by subscribing instead of pirating'').
The moral ambiguity is heightened by the fact that geo-arbitrage,
unlike piracy, involves a genuine financial transaction, where the user
pays for the service, just at a price not intended for their market.

\section{Strategic Management and Business Model Innovation}
\label{sec:theory_strategy}

Faced with this disruption, firms must adapt. We analyze their responses using \gls{bmi}.
As defined by \textcite{wirtz2016business}, theorized by \textcite{teece2010business}, and grouped by
\textcite{foss2017fifteen}, \gls{bmi} means rethinking the value offer and delivery methods in
response to outside shocks.

\subsection{Dimensions of Business Model Innovation}
To carefully analyze how firms adapt, we divide their business models into three parts,
following the widely used three-part framework \parencite{teece2010business, amit2012value}:

The first dimension is the \textbf{Value Proposition} (what is offered):
the core product or service and the bundle of benefits it provides to the customer.
In digital streaming, this is the content library and the convenience of ``watch anywhere'' access.
As \textcite{osterwalder2010business} emphasize, the value proposition is the fundamental
reason why customers choose one provider over another, and it must be continuously adapted
to changing market conditions.

The second dimension is \textbf{Value Delivery} (how it is reached):
the channels and technical infrastructure used to deliver the value. This includes the streaming platform,
the \gls{cdn} (where latency and infrastructure variability play
a significant role), and the user interface.
Crucially, it also includes the \textit{geographic segmentation} logic that determines
who can access what. The delivery mechanism is particularly relevant because digital
services, unlike physical goods, are consumed at the point of delivery, making the
delivery infrastructure itself a point of control \parencite{goldfarb2019digital}.

The third dimension is \textbf{Value Capture} (how money is made):
the revenue model and the mechanisms to sustain profitability \parencite{amit2012value}.
This includes the pricing strategy (e.g., price discrimination) and
the enforcement mechanisms used to prevent revenue leakage (e.g., blocking arbitrage).
\textcite{zott2011business} note that value capture is often the most contested dimension
in platform-based business models, as multiple stakeholders compete
over the distribution of value.

\gls{vpn}-enabled arbitrage fundamentally attacks the \textbf{Value Capture} dimension by breaking the
link between location and price. It also exploits the \textbf{Value Delivery} infrastructure (the
open internet). However, the impact on the \textbf{Value Proposition} is indirect but significant:
if enough users circumvent regional restrictions, firms may be forced to reevaluate what they offer in each market
(e.g., by shifting to globally available original content rather than regionally licensed catalogs).

Based on this, we expect firm responses to fall into two types of innovation.
This dual-response framework aligns with research on strategic agility, which shows that firms
capable of both defensive and adaptive innovation outperform those relying on a single approach \parencite{karimi2020strategic}.
The first is \textbf{coercive innovation} (Value Capture focus), which reinforces the barriers to protect the existing model through technical blocking and legal threats. The second is \textbf{adaptive innovation} (Value Proposition focus), which changes the product offer to make arbitrage irrelevant, for example through global pricing or ecosystem lock-in.

\subsection{Theoretical Framework: Protection vs. Pricing}
Digital strategy and arbitrage have been widely discussed.
\textcite{johnson2008reinventing} define the necessity of business model reinvention when
facing disruptive shifts, while \textcite{granados2006transparency} show how e-commerce
increases market efficiency by making arbitrage easier. However,
\textcite{anson2019arbitrage} note that cross-border arbitrage responds to exchange rate differentials
in ways that create strategic challenges for firms, leading to complex responses such as those described by \textcite{thongmak2017neutralization}
in the context of digital piracy. \textcite{beunza2004price} also argue that price is
ultimately a social construct, heavily influenced by the ``material sociology'' of the market—in
this case, the \gls{vpn} technology that alters the visibility of the consumer.

To categorize firm responses, we adopt the framework established by
\textcite{sundararajan2004managing} on managing digital piracy, mapping it to our \gls{bmi}
dimensions.
The first category is \textbf{protection} (coercive / Value Capture), which increases the technological or legal costs of circumvention, attempting to \textit{repair} the broken Value Capture mechanism. The second is \textbf{pricing} (adaptive / Value Proposition), which adjusts the business model through pricing and versioning to lower the economic motivation for arbitrage, effectively \textit{innovating} the Value Proposition to be less sensitive to location.
Firms face a fundamental trade-off: Is the cost of enforcing market segmentation (repairing
Value Capture through blocking technology and legal resources) lower than the revenue lost to
arbitrage?

This trade-off can be formalized through the lens of game theory
\parencite{tirole1988theory}. The interaction between a platform and its
users constitutes a sequential game: the firm chooses an enforcement
level $e$ (at cost $C(e)$), and users observe this enforcement level
and decide whether to attempt arbitrage based on their perceived probability of detection $p(e)$.
The firm's optimization problem is to choose $e^*$ that maximizes profit:

\begin{equation}
	\label{eq:enforcement_optimization}
	\max_e \left[ R_{\text{legitimate}}(e) - R_{\text{lost}}(e) - C(e) \right]
\end{equation}

\noindent where $R_{\text{legitimate}}$ is revenue from customers paying full price,
$R_{\text{lost}}$ is revenue lost to successful arbitrage,
and $C(e)$ is the cost of enforcement. The key insight is that $C(e)$ is convex
(each marginal unit of enforcement is more expensive than the last),
while $R_{\text{lost}}(e)$ is concave (diminishing returns from blocking additional arbitrageurs).
This implies an interior optimum: perfect enforcement is never optimal,
and some level of ``tolerated arbitrage'' is economically rational \parencite{becker1968crime}.
This theoretical prediction aligns with our empirical finding that no firm in our sample achieves
(or apparently attempts) 100\% blocking effectiveness.

The game-theoretic framing also explains why different firms choose such different
enforcement levels. A firm's optimal $e^*$ depends on: (1) the price gap across markets
(larger gaps increase $R_{\text{lost}}$ and shift $e^*$ upward),
(2) the marginal cost of enforcement technology (streaming services face lower
marginal costs for IP-blocking than software firms do for license verification), and
(3) the elasticity of user circumvention behavior to enforcement intensity
(if users quickly adapt to new blocking measures, the effective $p(e)$ declines
over time, reducing the return on enforcement investment).

\subsection{Platforms and Ecosystem Control}
Digital platforms operate within a fundamental tension between growth and control. To attract
users and content creators, platforms must maintain a degree of openness that facilitates
participation and innovation. However, to protect revenue streams and maintain quality,
platforms must also exercise control over who accesses what content and at what price point.
This tension is central to platform economics, as described by \textcite{rochet2003platform},
who model two-sided markets where platforms must balance the interests of multiple user groups simultaneously.

\gls{vpn} providers exploit this tension. They use the platform's content (e.g.,
Netflix's streaming library) while bypassing its payment rules (regional pricing). This
creates a technical and strategic cycle of countermeasures and counter-countermeasures.
\textbf{Coercive strategies} include legal threats embedded in \gls{tos}, IP address blocking, payment verification requirements, and strict geographic checks on billing addresses; these aim to restore the segmentation condition required for price discrimination, but they impose costs on both the platform (technology investment, user friction) and legitimate users (false positives, reduced portability). \textbf{Adaptive strategies}, by contrast, involve standardizing global prices to eliminate the arbitrage motivation, creating ecosystem lock-in through hardware integration (e.g., Apple's approach), or developing content exclusive to specific regions rather than restricting access to a global catalog. These strategies accept some loss of price discrimination power in exchange for reduced enforcement costs and improved user experience.

The broader trend toward sharing and platform-based economies \parencite{srnicek2017platform}
further complicates geographic price boundaries, as users increasingly expect borderless
access to digital services. This expectation is reinforced by the ``born global''
nature of digital platforms \parencite{brouthers2016explaining}, which initially
attract users with the promise of universal access before later imposing geographic
restrictions as licensing and pricing complexities emerge.

Critically, \textcite{parker2018innovation} demonstrate that platforms face an inherent
tension between openness (which drives innovation and user growth) and control (which protects
revenue and quality). Their framework suggests that the optimal balance point shifts depending
on platform maturity and competitive dynamics. \gls{vpn} arbitrage directly exploits this
fundamental trade-off, forcing platforms to reevaluate where that balance lies.
The concept of ``platform envelopment'' \parencite{eisenmann2011platform} is also
relevant here: as platforms like Amazon and Apple expand into multiple service categories
(shopping, streaming, cloud storage), they create ecosystem dependencies that make
\gls{vpn}-based arbitrage increasingly impractical, as users would need to relocate their
entire digital identity, not just a single subscription.

The strategic implications are significant: platforms that choose aggressive blocking may
sacrifice user experience and brand perception, while those that tolerate arbitrage may face
revenue leakage. Neither approach is without cost, and the optimal strategy likely depends on
the specific business model and competitive context of each platform.
As \textcite{cusumano2019platform} argue, the most successful platforms are those that
find ways to create value for all participants while maintaining enough control to sustain
their revenue models, a balance that geo-arbitrage directly threatens.


\section{Research Gap}
\label{sec:theory_gap}

While price discrimination theory \parencite{varian1989price} and platform strategy \parencite{parker2018innovation} have
been extensively researched independently, there is still a clear gap where they overlap.
Although several research agendas have been proposed for \gls{bmi} \parencite{foss2017fifteen}, there is a lack of empirical work
connecting the \textit{size} of pricing drivers (as measured by indices like the \gls{dspi})
with the \textit{specific strategic responses} of firms.

The existing research has three main gaps.

The first is \textbf{theoretical isolation}.
Most studies focus either exclusively on the economics of pricing
(e.g., optimal price discrimination strategies as modeled by
\textcite{varian1989price} and \textcite{tirole1988theory}) or on the legal
aspects of copyright enforcement and digital rights management
\parencite{lindsay2006copyright}, but rarely examine the strategic interaction
between these domains as mediated by consumer-side technology such as
\glspl{vpn}. \textcite{goldfarb2019digital} identify this disciplinary
divide as a broader challenge in digital economics, noting that the field
lacks integrative frameworks capable of connecting pricing decisions,
consumer behavior, and technological countermeasures. The platform economics literature
\parencite{rochet2003platform, parker2018innovation} has examined pricing in
two-sided markets, but has not addressed the specific challenge of geographic
arbitrage enabled by circumvention technology.

The second gap is a \textbf{lack of quantification}.
While anecdotal evidence of geo-arbitrage is abundant in consumer forums and
technology journalism, systematic quantification of the arbitrage driver
across services and regions is lacking. Existing price comparison indices
(such as the Big Mac Index \parencite{pakko2003burgernomics} or the Billion
Prices Project \parencite{cavallo2016billion}) have focused on physical goods or
aggregate online price levels, without developing specialized metrics for digital
subscription services. The \gls{dspi} developed in this thesis addresses this gap by providing a standardized
nominal price index, complemented by a separate \gls{ptw} ratio that captures real
affordability, together enabling cross-service and cross-country comparisons that have not
been available in prior research.

The third gap is \textbf{limited strategic analysis}.
Previous research on digital piracy has examined how firms respond to
unauthorized copying \parencite{oberholzer2007effect, peukert2017piracy, danaher2014graduated},
but the distinct characteristics of geo-arbitrage
(payment rather than piracy, location rather than access) warrant specialized investigation.
The piracy literature has established that firms respond through a combination of legal,
technical, and market-based strategies \parencite{sundararajan2004managing},
but it remains unclear how these response categories map onto the geo-arbitrage context,
where the ``infringing'' behavior involves a genuine commercial transaction rather than theft.
The coercive-adaptive split proposed in this thesis provides a framework for categorizing
these responses that is specifically designed for the geo-arbitrage phenomenon.

This thesis addresses all three gaps through a mixed-methods design that combines quantitative price measurement (\gls{dspi}) with qualitative analysis of corporate enforcement documents, bridging the economic and strategic dimensions that prior work has treated in isolation.



\chapter{Methodology}
\label{chap:methodology}

This chapter describes the research design, how data was collected, and the analysis methods
used to investigate both the economic rationale for geo-arbitrage (\gls{rq}1) and the strategic
responses of digital service providers (\gls{rq}2). The methods were chosen based on both
research questions: quantifying price gaps requires standardized measurement,
while understanding corporate strategy requires interpretive analysis
of textual documents. The chapter starts with the research design, goes through both
research phases, and ends with how we combined both data sets.

\section{Research Design}

This study used a sequential explanatory mixed-methods design \parencite{creswell2017designing},
combining quantitative price analysis with qualitative text classification.
The reason for this dual approach was to first show the \textit{size} of the economic
problem (the arbitrage driver) and then look at the \textit{strategic responses} of
the actors involved. The sequential design was appropriate because the quantitative findings
(price disparities) provided necessary context for interpreting the qualitative findings
(enforcement strategies): understanding \textit{how much} firms discriminate on price was
essential before analyzing \textit{how} they protect those price differences.

The quantitative phase (Phase 1) built the ``Digital Services Price Index''
(\gls{dspi}) to objectively measure nominal price differences in global digital service pricing,
complemented by a \gls{ptw} ratio to assess real affordability relative to local wages.
This phase established the dependent variable (the arbitrage driver) by collecting
and normalizing subscription prices across 11 countries.

The qualitative phase (Phase 2)
used a Large Language Model (\gls{llm}) pipeline to classify corporate disclosures and
\gls{tos}, finding the strategic frameworks firms used to manage or fight this variance.
This phase identified the independent variable (the enforcement strategy)
by systematically coding over 25,000 sentences from corporate documents into theoretically grounded categories.

The integration of both phases occurs in the Discussion (Chapter \ref{chap:discussion}),
where pricing patterns are mapped against enforcement intensity to identify strategic archetypes.
This integration provides a full understanding of the geo-arbitrage ecosystem based on
public documentation, though proprietary information and undisclosed technologies remain hidden.
Specifically, firms' internal detection algorithms, real-time blocking rates, and revenue impact
assessments are not accessible through public documents and represent an inherent
limitation of document-based analysis.

\section{Phase 1: Quantitative Data Collection (for \gls{rq}1)}

\subsection{Data Collection}
To construct the \gls{dspi}, a representative basket of 11 digital services was selected:
Netflix, YouTube Premium, Disney+, Amazon Prime, Spotify, Apple Music, Microsoft 365,
Adobe Creative Cloud, Xbox Game Pass, NordVPN, and ExpressVPN.
These cover Video on Demand, Music Streaming, Software/Gaming, and \gls{vpn} services.
Note that for the qualitative analysis (Phase 2), Xbox Game Pass data is combined with
Microsoft 365 under the ``Microsoft'' category, as both are governed by the same Microsoft Services Agreement.

Price data was collected from a sample of 11 countries to capture the full spectrum of
purchasing power. The countries included are: Argentina, Brazil, Germany, Pakistan, Philippines,
Poland, Switzerland, Turkey, Ukraine, United Kingdom, and the United States.
Only countries with high-confidence official wage data were included.

The country selection followed a purposive stratified sampling strategy designed to maximize
variance along three dimensions. First, \textbf{income level}: the sample spans from
high-income OECD economies (Switzerland, USA, UK, Germany) through upper-middle-income economies
(Poland, Argentina, Brazil, Turkey) to lower-middle-income economies
(Philippines, Pakistan, Ukraine), ensuring representation across the World Bank's income
classifications. Second, \textbf{geographic diversity}: the sample covers Western Europe,
Eastern Europe, the Americas, South Asia, and Southeast Asia, reducing the risk of
region-specific confounds. Third, \textbf{arbitrage relevance}: several countries
(Turkey, Argentina, Pakistan) were specifically included because they are frequently cited in
consumer forums and technology journalism as popular ``target'' countries for
\gls{vpn}-based price hopping, making them directly relevant to the research questions.
Countries with hyperinflationary conditions (Argentina, Turkey) were deliberately retained
despite their price volatility, as they represent extreme cases that show the boundaries
of price discrimination strategies.

The service selection similarly followed a purposive approach. The 11 services were chosen
to represent four distinct business model categories: Video on Demand
(Netflix, YouTube Premium, Disney+, Amazon Prime), Music Streaming (Spotify, Apple Music),
Software/Gaming (Microsoft 365, Adobe Creative Cloud, Xbox Game Pass), and
\gls{vpn} Services (NordVPN, ExpressVPN). This categorization ensured that the analysis
can compare enforcement strategies across fundamentally different value delivery mechanisms
(streaming vs.\ download vs.\ ecosystem-embedded). The \gls{vpn} providers were included
as a contrasting case: as enablers of arbitrage rather than targets of it, their
strategic framing provides a useful counterpoint to the defensive approaches of content and software providers.
All selected services met three inclusion criteria: (1) global availability in at least 9 of the
11 sample countries, (2) a publicly accessible individual subscription tier, and
(3) publicly available \gls{tos} documents spanning at least three years of the study period (2020--2025).

Data was collected using a \textbf{Digital Audit} approach, adapting the methodology
established by \textcite{hannak2014measuring} for detecting online price discrimination.
A virtual presence was established in each target country using a commercial \gls{vpn} service to
simulate local access, a technique now standard in information systems research for
``mystery shopping'' in digital markets. For each service and country, the monthly ``Standard''
subscription price was recorded in local currency. This approach mirrors the methodology of the
``Billion Prices Project'' \parencite{cavallo2017are}, which demonstrated the validity of
using high-frequency online scraping to construct robust price indices that track
real-time economic disparities more effectively than traditional CPI baskets.

\subsection{Data Analysis}
The raw price data was processed in two stages, yielding two distinct metrics.

\textbf{Stage 1: The \gls{dspi} (Nominal Price Index).} All local prices were
converted to a common currency (USD) using market exchange rates (recorded in December 2025).
The \gls{dspi} was then calculated as the ratio of the local price to the US baseline price.
A \gls{dspi} of 1.0 indicates price parity with the US market, while a \gls{dspi} $<$ 1.0 indicates a
cheaper market (a potential arbitrage source), and a \gls{dspi} $>$ 1.0 indicates a more expensive market.
Statistical variance analysis was performed to identify which service categories
showed the highest degree of nominal price discrimination.

\textbf{Stage 2: The \gls{ptw} Ratio (Real Affordability).} As a separate complementary metric,
each USD-converted price was also calculated as a percentage of the \textit{Median National Monthly Wage}
(sourced from OECD \parencite{oecd2023wages}, national statistical institutes
\parencite{indec2023argentina}, and economic databases \parencite{tradingeconomics2023turkey};
the complete list of wage figures, exchange rates, and sources is documented in
Table~\ref{tab:wage_reference} in Appendix~\ref{app:reference_data}),
giving a direct measure of the economic burden on the local consumer.
This \gls{ptw} ratio was proposed as a new alternative
to standard \gls{ppp} adjustment, arguing it better reflects subscription goods' affordability
relative to disposable income in the specific context of digital services.

Note that a \gls{dspi} of 1.0 (Nominal Parity) does not imply equal affordability.
Due to vast differences in median wages (e.g., Switzerland vs. Pakistan), a service priced identically
in USD would be significantly more expensive for the Pakistani consumer in real terms
(requiring a larger percentage of their income). Thus, the arbitrage motivation persists
even at nominal parity if the local price is structured to be affordable for the local
median earner. This distinction between nominal price (captured by the \gls{dspi}) and real
affordability (captured by the \gls{ptw} ratio) is central to understanding the full impact of digital pricing
that standard economic data might miss \parencite{brynjolfsson2019measurement}.

\section{Phase 2: Qualitative Data Collection \& Analysis (for \gls{rq}2)}

\subsection{Coding Procedure}
The analysis followed a systematic coding approach inspired by the
\textbf{Gioia Methodology} \parencite{gioia2013seeking}, which organizes
qualitative data into layers: 1st-order concepts (raw terms found in text),
2nd-order themes (theoretical categories such as ``Technical Blocking''), and
aggregate dimensions (Strategic Responses). While first designed for manual coding
\parencite{duriau2007content}, this layered structure provided the conceptual framework
for the automated classification pipeline described below.

\subsection{Automated Text Classification}
\label{sec:llm_methodology}

To address the limits of traditional \gls{nli} models in
capturing the complex legal and technical language of \gls{tos},
this study used an advanced classification pipeline with modern
\glspl{llm}. Specifically, the pipeline was upgraded from a
BERT-based architecture (DeBERTa-v3-large) to the \textit{Gemini 3 Flash} model,
accessed via the Google Generative AI API.

\subsection{Model Selection}
The choice of \textit{Gemini 3 Flash} was driven by the need for deeper
reasoning abilities and better context awareness. Unlike \gls{nli} models, which classify based on entailment probabilities between premises
and hypotheses, generative \glspl{llm} can interpret complex sentence structures
and tell the difference between ambiguous legal standard terms
(``General Terms'') and specific geo-arbitrage restrictions. Recent work
has confirmed the viability of \glspl{llm} for automated legal analysis tasks \parencite{carneiro2025automated}.

Key advantages observed during the model transition were threefold. First, the model demonstrated strong \textbf{contextual understanding}, distinguishing between benign references to ``account suspension'' (e.g., for fraud) and strategic ``Legal Threats'' tailored to prevent cross-border usage. Second, its \textbf{zero-shot performance} was notable: the model achieved high accuracy without extensive fine-tuning, utilizing a robust system prompt to align with the theoretical categories defined in Chapter~\ref{chap:theory}. Third, the ``Flash'' architecture provided high \textbf{efficiency} and throughput, enabling the processing of the entire dataset (approx.\ 25,000 sentences) within a reasonable timeframe of only a few days when done in batching.

\subsection{Operationalization of Constructs (The Coding Scheme)}
Based on the theoretical framework, the following coding scheme was enforced via the
\gls{llm} system prompt. This scheme maps the abstract concept of ``Strategic Response''
into measurable data points.

\subsubsection{Strategic Frames}
The model was tasked to identify the underlying justification provided by the firm:
\begin{description}
	\item[Frame: Legal Compliance] Justifying geo-blocking as a non-negotiable legal or
	      contractual necessity (e.g., ``Due to licensing agreements...'').
	\item[Frame: Security Risks] (Service Provider Frame) Arguments that \glspl{vpn}/Proxies
	      are unsafe, malicious, or compromise user data.
	\item[Frame: Privacy/Security] (\gls{vpn} Provider Frame) Arguments focusing on encryption,
	      anonymity, and protection from surveillance.
\end{description}

\subsubsection{Firm Actions}
The model categorized specific enforcement clauses into:
\begin{description}
	\item[Action: Technical Blocking] Active technological measures
	      to detect or block the specific use of \glspl{vpn}/Proxies
	      (e.g., ``We use geo-blocking technology'', ``to provide digital rights management'').
	\item[Action: Legal Threat] Explicit threats of account termination,
	      suspension, or legal action specifically for using circumvention tools.
	\item[Action: Account Action] General punitive measures against accounts
	      (termination, suspension) for broad violations.
	      Note: in the implemented classification prompt, this category was merged into
	      ``Legal Threat'' for cases specifically tied to circumvention tools
	      and under ``General Terms'' for standard account-related boilerplate,
	      resulting in zero standalone classifications (see Table~\ref{tab:app_service_stats}).
	\item[Action: Price Discrimination] Explicit differences in pricing based on
	      region, currency, or purchasing power.
	\item[Action: Legitimate Portability] Rules allowing temporary access
	      while traveling (e.g., \gls{eu} Portability Regulation).
	\item[Action: General Terms] Standard legal boilerplate and general
	      contractual language (e.g., liability clauses, warranty disclaimers,
	      general user obligations) that does not fall into any specific enforcement
	      category. While constituting approximately 94\% of the dataset,
	      General Terms are retained in the master dataset to preserve the full document
	      structure but are excluded from strategic trend analyses to isolate distinct enforcement categories.
\end{description}

\subsection{Pipeline Architecture and Implementation}
The reclassification process was automated using customized Python scripts.

\subsubsection{System Prompt Engineering}
To ensure consistent and theoretically grounded outputs,
the system prompt was engineered with strict constraints.
The exact prompt structure is provided below:

\begin{lstlisting}[caption={System Prompt used for Gemini 3 Flash Classification},
	label={lst:system_prompt},captionpos=b]
SYSTEM_PROMPT = """You are a scientific classifier.
CATEGORIES:
1. Technical Blocking: Measures/Technologies used to detect
   or block the specific use of VPNs/Proxies.
2. Legal Threat: Explicit threats of account termination,
   suspension, or legal action for using circumvention
   tools.
3. Price Discrimination: Differences in pricing based on
   region, currency, or purchasing power.
4. Content Licensing: Geographic restriction of content
   availability (e.g. 'not available in your region')
   due to rights.
5. Legitimate Portability: Rules allowing temporary access
   while traveling (e.g. EU Portability Regulation).
6. Regulatory Compliance: References to local laws, tax/VAT
   compliance, or export controls.
7. User Workaround: Descriptions of users bypassing
   restrictions (using VPNs, changing store region).
8. Security Risk: (Service Provider Frame) Arguments that
   VPNs/Proxies are unsafe, malicious, or compromise
   user data.
9. Privacy/Security: (VPN Provider Frame) Arguments
   focusing on encryption, anonymity, and protection
   from surveillance.
10. General Terms: Standard legal text, general marketing,
    or unrelated content.

INSTRUCTIONS:
- Return a JSON array of objects for the sentences in
  EXACT order.
- Format: [{"category": "Category Name",
  "confidence": 0.9}, ...]
"""
\end{lstlisting}

\subsubsection{Batch Processing and Error Handling}
To optimize for the API's rate limits and ensure data integrity, the pipeline
utilized a batch processing approach. Sentences were grouped into batches of 25
and processed in a single API call. This method significantly reduced network
overhead and total processing time.
A robust error-handling mechanism was implemented to manage API timeouts
or rate limits (HTTP 429). The script included a ``circuit breaker'' to
halt execution upon repeated failures and a resume function to continue
processing from the last saved state.

\subsubsection{Forward-Fill Strategy for Longitudinal Consistency}
A key challenge in analyzing longitudinal \gls{tos} data is
the rarity of document updates, as companies do not release
new \gls{tos} documents every year. To address this, a ``forward-fill''
strategy was applied: whenever no new document was published for a
given year, the most recent version was carried forward on the
assumption that its clauses remain legally in effect until explicitly
replaced. For example, if a document was released in 2020 and the
next update appeared in 2023, the 2020 version was treated as active
for 2021 and 2022. This ensured the dataset accurately reflected the
\textit{active} regulatory environment in every year, distinguishing
between genuinely ``missing data'' and ``persistent rules.''

\subsection{Methodological Validation: Gemini vs. Zero-Shot BERT}
To validate the choice of the Gemini 3 Flash model, a comparison
was done against a traditional Zero-Shot classification approach
using a BERT-based model. The results showed a substantial difference
between the two models, confirming the need for a modern \gls{llm} with
large context windows for this specific task, consistent with recent
findings on \gls{llm} performance in text annotation \parencite{gilardi2023chatgpt,hakimi2023legal}.

\subsubsection{Agreement Analysis}
The comparison revealed a very low agreement rate of
\textbf{26.8\% (Accuracy)} between the two models. Cohen's Kappa
score was \textbf{0.032}, suggesting agreement effectively equivalent
to random chance. This discrepancy indicated a fundamental
difference in how each model interpreted the classification tasks.

\subsubsection{The Core Conflict: Sensitivity vs. Context}
The analysis highlighted two distinct behaviors.
The \textbf{Gemini} model correctly identified that approximately \textbf{94\%} of the dataset consisted of legal boilerplate, categorized as ``General Terms,'' and successfully distinguished specific enforcement clauses from general legal language. The \textbf{BERT} model, by contrast, showed ``Over-Sensitivity,'' frequently assigning specific strategic tags based on the presence of individual keywords rather than semantic context. For example, BERT flagged 7,853 sentences as ``Legitimate Portability'' that were merely ``General Terms,'' and misclassified 6,134 ``General Terms'' sentences as ``Account Action.''

\textit{Interpretation:} BERT operates on keyword connections,
flagging sentences like ``You must have an account'' as ``Account Action.''
In contrast, Gemini uses reasoning abilities to understand that
just mentioning an ``account'' is standard boilerplate (``General Terms'')
and saves the ``Account Action'' tag for sentences that explicitly
regulate account termination or suspension.

\subsubsection{Conclusion on Model Selection}
The comparison showed that Zero-Shot BERT was not sufficient for
complex legal text analysis without extensive fine-tuning,
as it lacked the nuance needed to distinguish between merely
mentioning a topic (e.g., ``portability'') and its active regulation.
Note that this constitutes a model-to-model comparison
rather than a validation against a full human-annotated gold standard.
While a manual review of 200 sentences (Section~\ref{sec:llm_methodology})
confirmed 93\% agreement with Gemini's classifications, a larger-scale
human-coded benchmark would strengthen confidence in the automated coding.
Table \ref{tab:model_comparison} and Figure \ref{fig:model_comparison_viz}
provide a detailed breakdown of the category distribution discrepancies.

\begin{table}[ht]
	\centering
	\small
	\begin{tabularx}{\textwidth}{l c c c}
		\toprule
		\textbf{Category}      & \textbf{Gemini \%} & \textbf{BERT \%} & \textbf{Delta} \\
		\midrule
		Technical Blocking     & 0.41\%             & 0.09\%           & +0.32\%        \\
		Price Discrimination   & 0.48\%             & 0.03\%           & +0.45\%        \\
		Content Licensing      & 2.18\%             & 5.76\%           & -3.58\%        \\
		Regulatory Compliance  & 2.05\%             & 0.41\%           & +1.64\%        \\
		Legal Threat           & 0.47\%             & 0.00\%           & +0.47\%        \\
		Account Action         & 0.00\%             & 25.89\%          & -25.89\%       \\
		Legitimate Portability & 0.01\%             & 31.99\%          & -31.98\%       \\
		General Terms          & 94.12\%            & 26.12\%          & +68.00\%       \\
		\bottomrule
	\end{tabularx}
	\caption{Model Comparison: Gemini 3 Flash vs. Zero-Shot BERT Classification.
		Categories with $<0.2\%$ share in both models (Security Risk, User Workaround)
		are omitted for clarity; column percentages therefore do not sum to exactly 100\%.}
	\label{tab:model_comparison}
\end{table}

\begin{figure}[ht]
	\centering
	\begin{tikzpicture}
		\begin{axis}[
				ybar,
				width=0.95\textwidth,
				height=8cm,
				symbolic x coords={General,Licensing,Regulatory,Price Discr.,Tech.\ Block.,Sec.\ Risk,Privacy,Portability,Workaround},
				xtick=data,
				xticklabel style={rotate=45, anchor=north east, font=\footnotesize},
				ylabel={Share of Dataset (\%)},
				legend style={at={(0.5,-0.35)}, anchor=north, legend columns=-1, font=\small},
				ymin=0, ymax=108,
				clip=false,
				bar width=10pt,
				nodes near coords,
				every node near coord/.append style={font=\tiny, rotate=90, anchor=west},
				/pgf/number format/fixed, /pgf/number format/precision=1
			]
			% Gemini % (sorted by Gemini share, largest to smallest)
			\addplot[fill=tudablue!80] coordinates {
					(General,94.1) (Licensing,2.2) (Regulatory,2.1) (Price Discr.,0.5) (Tech.\ Block.,0.4)
					(Sec.\ Risk,0.2) (Privacy,0.1) (Portability,0.0) (Workaround,0.0)
				};
			% BERT %
			\addplot[fill=tudared!80] coordinates {
					(General,26.1) (Licensing,5.8) (Regulatory,0.4) (Price Discr.,0.0) (Tech.\ Block.,0.1)
					(Sec.\ Risk,0.0) (Privacy,0.0) (Portability,32.0) (Workaround,9.7)
				};
			\legend{Gemini 3 Flash, BERT Baseline}
		\end{axis}
	\end{tikzpicture}
	\caption{Visualizing the Classification Gap: Proportional Category
		Distribution Between Models (sorted by Gemini share).}
	\label{fig:model_comparison_viz}
\end{figure}

Gemini, using its extensive context window and advanced reasoning abilities,
performed much better at filtering noise and provided accurate classifications.
As a result, Gemini 3 Flash was chosen as the only model for the final analysis.

\subsection{Standard Qualitative Coding}
While the automated \gls{llm} pipeline provided scalable classification across
the full dataset, manual qualitative coding complemented this approach
by capturing nuances that escape rigid categorization.
A sub-sample of 200 sentences was selected for manual review,
stratified across service providers and document years to ensure representativeness.

The manual coding process addressed three objectives:
\begin{enumerate}
	\item \textbf{Validation:} Verifying the \gls{llm} classifications against human
	      judgment to assess reliability and identify systematic errors or edge cases.
	\item \textbf{Tone Analysis:} Capturing the ``tone'' of enforcement
	      language that categorical classification cannot capture. For example,
	      distinguishing between neutral legal boilerplate (``We may terminate your account...'')
	      and threatening language (``Violations will result in immediate termination without refund...'').
	\item \textbf{Emergent Themes:} Identifying themes not captured by
	      the predefined categories, such as references to ``fair use'',
	      ``educational purposes'', or ``legitimate business needs''
	      that may signal adaptive rather than purely coercive approaches.
\end{enumerate}

The manual review confirmed the \gls{llm} classifications in the large
majority of cases. Of the 200 manually reviewed sentences, 186 (93\%)
were classified identically by both the human coder and the Gemini model.
The 14 disagreements primarily occurred in edge cases where sentences
contained elements of multiple categories (e.g., a clause referencing
both content licensing and price discrimination). In these ambiguous
cases, the \gls{llm} tended to classify toward the dominant category,
while the human coder noted the secondary category as well.
No systematic bias was detected: the disagreements were distributed across
categories rather than concentrated in a single area.

The tone analysis revealed a clear pattern: enforcement language
has become progressively more specific and assertive over time.
Earlier \gls{tos} documents (2020--2021) tended to use passive, permissive
constructions (``We may suspend your account if...''), while more recent documents
(2023--2025) employ direct, prohibitive language (``You must not use any technology
to disguise your location''). This shift from permissive to prohibitive framing is
consistent with the quantitative finding of increased Technical Blocking clauses post-2022.

This cross-checking between automated and manual coding strengthens
the validity of the overall classification, ensuring that the strategic
patterns identified in Chapter \ref{chap:results} reflect genuine
corporate positioning rather than artifacts of the \gls{llm} classification process.


\section{Data Analysis Procedures}
\label{sec:analysis_procedures}

The last analysis step was combining the quantitative and qualitative
data into one framework. This was the study's main methodological
contribution, as it linked the \textit{scale} of the economic
driver (measured by the \gls{dspi}) with the \textit{nature} of the strategic
response (measured by enforcement category frequencies), making it possible to spot
strategic patterns that neither data source could reveal on its own.

\subsection{Statistical Analysis of the \gls{dspi}}
The pricing data was analyzed using Python (NumPy, Pandas, and SciPy libraries).
Descriptive statistics (mean, median, standard deviation) were calculated for
the \gls{dspi} across all services and regions. The standard deviation of
each service's \gls{dspi} across countries was used as the ``Price Discrimination Score'':
a higher standard deviation indicated greater cross-country price variance and
thus a stronger arbitrage motivation. Correlation matrices were generated
to examine the relationship between a country's income level and subscription
pricing, testing whether price discrimination correlates strictly with
national wealth or follows more complex patterns. Pearson correlation
coefficients were calculated between the Price Discrimination Score
and the Enforcement Intensity (defined as the percentage of Technical
Blocking and Legal Threat clauses relative to total sentences) for each
service. This correlation was calculated both for the full sample
($N=10$) and for the sub-sample excluding \gls{vpn} providers ($N=8$),
as \gls{vpn} providers occupy a structurally different position in
the ecosystem (enablers rather than targets of arbitrage).

\subsection{Interpretation of Qualitative Classifications}
For the qualitative data, the JSON outputs from the Gemini 3 Flash
pipeline were parsed and aggregated using custom Python scripts.
The frequency of each ``Strategic Frame'' (e.g., \textit{Legal Compliance}
vs. \textit{User Freedom}) and ``Firm Action'' (e.g., \textit{Technical Blocking})
was calculated per company and per year.

To visualize the evolution of enforcement strategies, these frequencies
were normalized against the total number of sentences per year to account
for the documented growth in \gls{tos} document length over time
\parencite{reidenberg2015privacy,milne2006readability}.
Without this normalization, an increase in absolute enforcement clause
counts could be an artifact of longer documents rather than a genuine
strategic shift. The normalization produced percentage-based trend lines
that reflect the \textit{relative emphasis} firms place on different
strategic categories over time. Also, to enable cross-company
comparisons, we developed the ``Fortress Index,'' the percentage of
enforcement-related clauses (Technical Blocking and Legal Threat)
relative to the total number of strategic (non-General Terms) sentences.
This metric isolated each firm's enforcement \textit{focus} from the
volume of its documentation, allowing meaningful comparison between
companies with vastly different document lengths (e.g., Microsoft
with 6,516 total sentences vs.\ ExpressVPN with 60).

Finally, a comparative analysis was conducted to contrast the
language of ``Fortress'' strategy firms (high blocking) against
``Globalist'' strategy firms (price harmonization), identifying
the key markers of each business model archetype.
This comparative framework, built up from the data,
ultimately produced the four strategic archetypes presented in
Chapter~\ref{chap:discussion}.






\chapter{Results}
\label{chap:results}

This chapter presents the empirical findings in three stages. First, we characterize
the analyzed corpus to establish the data foundation (Section~\ref{sec:data_overview}).
Second, we present the \gls{dspi} results to quantify the arbitrage driver across services
and countries (Section~\ref{sec:dspi_results}). Third, we analyze the qualitative classification
results to map enforcement strategies and their evolution over time
(Sections~\ref{sec:service_deep_dive}--\ref{sec:correlation}). The integration of these
quantitative and qualitative findings, connecting price disparities to strategic responses,
is presented in the Discussion (Chapter~\ref{chap:discussion}).

\section{Dataset Overview}
\label{sec:data_overview}

Before presenting the findings, we briefly characterize the analyzed corpus.
Figure~\ref{fig:app_company_dist} shows the distribution of total sentences by
company, while Figure~\ref{fig:app_doctype_dist} breaks down the corpus by document type.
Table~\ref{tab:app_service_stats} presents the complete distribution of strategic
frame counts for each of the 10 analyzed services. The master corpus
($N=25{,}570$, Table~\ref{tab:app_service_stats}) includes 700 carry-forward
observations (see Chapter~\ref{chap:methodology} for details on how gaps in document coverage were handled).
The year-by-year analysis in
Table~\ref{tab:qual_timeline_complete} ($N=25{,}345$) excludes 225 sentences
from documents that could not be reliably assigned to a specific publication year.
This minor difference does not affect proportional trends.

\begin{figure}[ht]
	\centering
	\resizebox{\linewidth}{!}{%
		\begin{tikzpicture}
			\begin{axis}[
					xbar,
					width=0.9\textwidth,
					height=8cm,
					xlabel={Total Sentences (N)},
					symbolic y coords={ExpressVPN,NordVPN,Apple Music,Amazon Prime,Disney+,Netflix,Adobe,Spotify,YouTube Premium,Microsoft},
					ytick=data,
					nodes near coords,
					nodes near coords align={horizontal},
					xmin=0, xmax=7500,
					bar width=12pt,
					yticklabel style={font=\footnotesize},
				]
				\addplot[fill=tudablue!60] coordinates {
						(60,ExpressVPN)
						(110,NordVPN)
						(1328,Apple Music)
						(1639,Amazon Prime)
						(2171,Disney+)
						(2872,Netflix)
						(3159,Adobe)
						(3551,Spotify)
						(4164,YouTube Premium)
						(6516,Microsoft)
					};
			\end{axis}
		\end{tikzpicture}}
	\caption{Total Clause Counts by Company. The dataset is weighted towards Microsoft
		and YouTube due to the complexity and length of their multiple policy documents.}
	\label{fig:app_company_dist}
\end{figure}

\begin{figure}[ht]
	\centering
	\resizebox{\linewidth}{!}{%
		\begin{tikzpicture}
			\begin{axis}[
					xbar,
					width=0.9\textwidth,
					height=6cm,
					xlabel={Total Sentences (N)},
					symbolic y coords={Other,Shareholder Letter,Terms of Service,Earnings Call,10-K Annual Report},
					ytick=data,
					nodes near coords,
					nodes near coords align={horizontal},
					xmin=0, xmax=12000,
					xtick={0,3000,6000,9000,12000},
					bar width=15pt,
					yticklabel style={font=\footnotesize},
					xticklabel style={font=\footnotesize},
					/pgf/number format/1000 sep={\,},
				]
				\addplot[fill=tudagreen!60] coordinates {
						(634,Other)
						(1866,Shareholder Letter)
						(6103,Terms of Service)
						(6794,Earnings Call)
						(10173,10-K Annual Report)
					};
			\end{axis}
		\end{tikzpicture}}
	\caption{Distribution of Data by Document Type. Annual Reports (10-K) and Earnings Calls
		provide the bulk of strategic context, while \gls{tos} documents provide the
		specific enforcement clauses.}
	\label{fig:app_doctype_dist}
\end{figure}

\begin{table}[ht]
	\centering
	\scriptsize
	\renewcommand{\arraystretch}{1.1}
	\resizebox{\textwidth}{!}{%
		\begin{tabular}{l *{11}{r} | r}
			\toprule
			\textbf{Service} & \rotatebox{70}{\textbf{Tech.\ Block.}} & \rotatebox{70}{\textbf{Price Discr.}} & \rotatebox{70}{\textbf{Licensing}} & \rotatebox{70}{\textbf{Regulatory}} & \rotatebox{70}{\textbf{Legal Thr.}} & \rotatebox{70}{\textbf{Acc.\ Act.}} & \rotatebox{70}{\textbf{Privacy}} & \rotatebox{70}{\textbf{Sec.\ Risk}} & \rotatebox{70}{\textbf{Portab.}} & \rotatebox{70}{\textbf{Workaro.}} & \rotatebox{70}{\textbf{General}} & \rotatebox{70}{\textbf{Total}} \\
			\midrule
			YouTube          & 94                                     & 2                                     & 99                                 & 135                                 & 34                                  & 0                                   & 16                               & 0                                   & 0                                & 0                                 & 3{,}784                          & 4{,}164                        \\
			Microsoft        & 4                                      & 0                                     & 10                                 & 80                                  & 53                                  & 0                                   & 4                                & 23                                  & 0                                & 0                                 & 6{,}342                          & 6{,}516                        \\
			Netflix          & 1                                      & 36                                    & 127                                & 76                                  & 4                                   & 0                                   & 0                                & 0                                   & 2                                & 0                                 & 2{,}626                          & 2{,}872                        \\
			Disney+          & 0                                      & 41                                    & 114                                & 37                                  & 4                                   & 0                                   & 0                                & 0                                   & 0                                & 0                                 & 1{,}975                          & 2{,}171                        \\
			Spotify          & 1                                      & 22                                    & 132                                & 75                                  & 0                                   & 0                                   & 0                                & 0                                   & 0                                & 1                                 & 3{,}320                          & 3{,}551                        \\
			Adobe            & 0                                      & 18                                    & 4                                  & 45                                  & 4                                   & 0                                   & 1                                & 1                                   & 0                                & 0                                 & 3{,}086                          & 3{,}159                        \\
			Amazon           & 1                                      & 2                                     & 44                                 & 44                                  & 3                                   & 0                                   & 0                                & 9                                   & 0                                & 0                                 & 1{,}536                          & 1{,}639                        \\
			Apple            & 2                                      & 2                                     & 28                                 & 29                                  & 8                                   & 0                                   & 5                                & 6                                   & 0                                & 0                                 & 1{,}248                          & 1{,}328                        \\
			ExpressVPN       & 1                                      & 1                                     & 0                                  & 2                                   & 4                                   & 0                                   & 1                                & 0                                   & 0                                & 0                                 & 51                               & 60                             \\
			NordVPN          & 0                                      & 0                                     & 0                                  & 0                                   & 6                                   & 0                                   & 0                                & 6                                   & 0                                & 0                                 & 98                               & 110                            \\
			\midrule
			\textbf{Total}   & \textbf{104}                           & \textbf{124}                          & \textbf{558}                       & \textbf{523}                        & \textbf{120}                        & \textbf{0}                          & \textbf{27}                      & \textbf{45}                         & \textbf{2}                       & \textbf{1}                        & \textbf{24,066}                  & \textbf{25,570}                \\
			\bottomrule
		\end{tabular}}
	\caption{Absolute Category Counts by Service. YouTube dominates Technical Blocking (N=94), while streaming services rely heavily on Content Licensing.}
	\label{tab:app_service_stats}
\end{table}

\section{The Landscape of International Pricing: Findings from the \gls{dspi}}
\label{sec:dspi_results}

To understand the economic reason driving firm strategic behavior, we first analyzed
the quantitative pricing picture using the \gls{dspi}.

\subsection{Scale of the Arbitrage Incentive}
The data show large and systematic price differences across markets.
For example, subscriptions in Turkey or Pakistan can cost up to 90\% less than
the same subscriptions in Switzerland or the USA when measured in nominal USD.
These differences are not uniform across service categories: content streaming
services showed the widest price spreads
(consistent with territorial licensing constraints), while software utilities
and \gls{vpn} services maintained more harmonized global pricing.
The size of these differences creates a ``above-normal'' profit margin
for consumers engaging in arbitrage, savings that far exceed the cost of a
commercial \gls{vpn} subscription (\$5--12/month), explaining the persistence
and growth of this behavior despite the technical barriers analyzed in later sections.
Critically, the arbitrage motivation is not static: currency fluctuations,
local price adjustments, and regulatory changes continuously reshape the
pricing picture, meaning that the ``cheapest'' target country for
arbitrage can shift over relatively short periods.

Table~\ref{tab:app_dspi} presents the complete \gls{dspi} and affordability
metrics for all 11 countries in the dataset. Notably, while
\textbf{Pakistan} appears cheapest ($\text{DSPI}=0.45$), it is relatively
expensive for locals ($1.13\%$ of wage). In contrast, \textbf{Turkey} ($\text{DSPI}=0.65$)
and \textbf{Argentina} ($\text{DSPI}=0.76$) show high nominal discounts,
but Argentina's high \gls{ptw} ($3.28\%$) suggested pricing there was
effectively ``dollarized'' for elites or external arbitrageurs.

\begin{table}[ht]
	\centering
	\small
	\begin{tabular}{l c c c}
		\toprule
		\textbf{Country} & \textbf{Avg.\ \gls{dspi}} & \textbf{N Services} & \textbf{Netflix \gls{ptw} (\%)} \\
		\midrule
		Switzerland      & 1.24                      & 11                  & 0.33                            \\
		United Kingdom   & 1.04                      & 11                  & 0.24                            \\
		Germany          & 1.01                      & 11                  & 0.29                            \\
		United States    & 1.00                      & 11                  & 0.27                            \\
		Poland           & 0.77                      & 11                  & 0.73                            \\
		Argentina        & 0.76                      & 11                  & 3.28                            \\
		Turkey           & 0.65                      & 11                  & 1.22                            \\
		Ukraine          & 0.62                      & 10                  & 1.60                            \\
		Brazil           & 0.59                      & 11                  & 1.50                            \\
		Philippines      & 0.54                      & 11                  & 2.18                            \\
		Pakistan         & 0.45                      & 9                   & 1.13                            \\
		\bottomrule
	\end{tabular}
	\caption{Complete \gls{dspi} Summary by Country. \gls{ptw} = Netflix Standard subscription cost as \% of median monthly wage. Lower \gls{dspi} indicates cheaper markets relative to the US baseline.}
	\label{tab:app_dspi}
\end{table}

Table~\ref{tab:dspi_matrix} presents the complete per-service \gls{dspi} values across
all 11 countries in the dataset (recall that the \gls{dspi} scale is defined in
Chapter~\ref{chap:methodology}). Cells marked with ``--''
indicate that the service is not available in that country.

\begin{table}[ht]
	\centering
	\caption{Complete \gls{dspi} Matrix: Per-Service Price Index by Country (US = 1.00).
		Values $< 0.50$ are highlighted as high-arbitrage opportunities.}
	\label{tab:dspi_matrix}
	\scriptsize
	\renewcommand{\arraystretch}{1.1}
	\resizebox{\textwidth}{!}{%
		\begin{tabular}{l *{11}{r}}
			\toprule
			\textbf{Country} & \rotatebox{70}{\textbf{Netflix}} & \rotatebox{70}{\textbf{YouTube}} & \rotatebox{70}{\textbf{Disney+}} & \rotatebox{70}{\textbf{Amazon}} & \rotatebox{70}{\textbf{Spotify}} & \rotatebox{70}{\textbf{Apple M.}} & \rotatebox{70}{\textbf{MS 365}} & \rotatebox{70}{\textbf{Adobe CC}} & \rotatebox{70}{\textbf{Xbox GP}} & \rotatebox{70}{\textbf{NordVPN}} & \rotatebox{70}{\textbf{ExpressVPN}} \\
			\midrule
			Switzerland      & 1.44                             & 1.45                             & 1.47                             & 0.75                            & 1.50                             & 1.43                              & 1.13                            & 1.24                              & 1.13                             & 1.09                             & 0.99                                \\
			United Kingdom   & 0.78                             & 1.18                             & 1.07                             & 0.76                            & 1.27                             & 1.27                              & 1.08                            & 1.21                              & 0.89                             & 0.97                             & 1.00                                \\
			Germany          & 0.85                             & 1.01                             & 0.92                             & 0.65                            & 1.18                             & 1.09                              & 1.08                            & 1.21                              & 0.98                             & 1.09                             & 1.02                                \\
			United States    & 1.00                             & 1.00                             & 1.00                             & 1.00                            & 1.00                             & 1.00                              & 1.00                            & 1.00                              & 1.00                             & 1.00                             & 1.00                                \\
			Poland           & 0.68                             & 0.70                             & 0.67                             & 0.18                            & 0.50                             & 0.50                              & 1.08                            & 1.26                              & 0.93                             & 0.85                             & 1.13                                \\
			Argentina        & 1.00                             & 0.74                             & 1.14                             & 0.64                            & 0.33                             & 0.65                              & 0.45                            & 1.01                              & 1.08                             & 0.44                             & 0.92                                \\
			Turkey           & 0.52                             & 0.17                             & 1.11                             & 0.15                            & 0.26                             & 0.17                              & 1.06                            & 0.74                              & 0.86                             & 1.01                             & 1.10                                \\
			Ukraine          & 0.45                             & 0.18                             & --                               & 0.51                            & 0.42                             & 0.45                              & 0.70                            & 0.58                              & 0.75                             & 1.01                             & 1.10                                \\
			Brazil           & 0.50                             & 0.36                             & 0.72                             & 0.27                            & 0.40                             & 0.40                              & 1.02                            & 0.61                              & 0.88                             & 0.44                             & 0.92                                \\
			Philippines      & 0.45                             & 0.24                             & 0.35                             & 0.18                            & 0.25                             & 0.23                              & 0.88                            & 0.98                              & 0.58                             & 0.90                             & 0.92                                \\
			Pakistan         & 0.16                             & 0.12                             & --                               & 0.14                            & 0.10                             & --                                & 0.83                            & 1.00                              & 0.18                             & 0.81                             & 0.73                                \\
			\bottomrule
		\end{tabular}}
\end{table}

The matrix reveals several important patterns. Content streaming services
(Netflix, YouTube, Disney+, Spotify) showed the highest price variance across
countries, with ratios ranging from 0.10 to 1.50. This creates the strongest
arbitrage drivers. In contrast, software services (Microsoft 365, Adobe Creative Cloud)
and \gls{vpn} providers maintained more uniform global pricing. Notably, Turkey
and Pakistan offered extreme discounts for streaming services
(YouTube Premium in Pakistan at \gls{dspi} = 0.12 is just 12\% of the US price),
while \gls{vpn} services like NordVPN and ExpressVPN showed much smaller regional
differences, consistent with their global pricing model.

Table~\ref{tab:dspi_usd} shows the raw monthly subscription prices in USD
for direct comparison across all services and markets.

\begin{table}[ht]
	\centering
	\caption{Monthly Subscription Prices in USD Across All Services and Countries
		(December 2025). Microsoft 365 and NordVPN/ExpressVPN show the equivalent
		monthly cost derived from annual/multi-year plans.}
	\label{tab:dspi_usd}
	\scriptsize
	\renewcommand{\arraystretch}{1.1}
	\resizebox{\textwidth}{!}{%
		\begin{tabular}{l r *{11}{r}}
			\toprule
			\textbf{Country} & \rotatebox{70}{\textbf{Wage (\$)}} & \rotatebox{70}{\textbf{Netflix}} & \rotatebox{70}{\textbf{YouTube}} & \rotatebox{70}{\textbf{Disney+}} & \rotatebox{70}{\textbf{Amazon}} & \rotatebox{70}{\textbf{Spotify}} & \rotatebox{70}{\textbf{Apple M.}} & \rotatebox{70}{\textbf{MS 365}} & \rotatebox{70}{\textbf{Adobe CC}} & \rotatebox{70}{\textbf{Xbox GP}} & \rotatebox{70}{\textbf{NordVPN}} & \rotatebox{70}{\textbf{ExpressVPN}} \\
			\midrule
			Switzerland      & 7{,}800                            & 25.88                            & 20.23                            & 19.10                            & 11.29                           & 18.02                            & 15.71                             & 112.94                          & 86.73                             & 11.29                            & 104.93                           & 105.64                              \\
			United Kingdom   & 5{,}715                            & 13.96                            & 16.50                            & 13.96                            & 11.42                           & 15.23                            & 13.96                             & 107.94                          & 84.44                             & 8.88                             & 93.45                            & 106.25                              \\
			Germany          & 5{,}232                            & 15.25                            & 14.16                            & 11.98                            & 9.80                            & 14.16                            & 11.98                             & 107.91                          & 85.01                             & 9.80                             & 104.72                           & 108.60                              \\
			United States    & 6{,}600                            & 17.99                            & 13.99                            & 12.99                            & 14.99                           & 11.99                            & 10.99                             & 99.99                           & 69.99                             & 9.99                             & 96.07                            & 106.40                              \\
			Poland           & 1{,}675                            & 12.25                            & 9.75                             & 8.75                             & 2.75                            & 6.00                             & 5.50                              & 107.50                          & 87.86                             & 9.25                             & 82.11                            & 120.20                              \\
			Argentina        & 548                                & 18.00                            & 10.32                            & 14.76                            & 9.54                            & 3.96                             & 7.17                              & 44.64                           & 70.84                             & 10.80                            & 42.60                            & 97.72                               \\
			Turkey           & 761                                & 9.28                             & 2.43                             & 14.40                            & 2.24                            & 3.17                             & 1.92                              & 105.60                          & 52.07                             & 8.61                             & 96.88                            & 117.26                              \\
			Ukraine          & 510                                & 8.16                             & 2.57                             & --                               & 7.62                            & 4.99                             & 4.99                              & 70.17                           & 40.79                             & 7.54                             & 96.88                            & 117.26                              \\
			Brazil           & 600                                & 8.98                             & 4.98                             & 9.38                             & 3.98                            & 4.78                             & 4.38                              & 101.80                          & 42.80                             & 8.78                             & 42.66                            & 97.72                               \\
			Philippines      & 370                                & 8.08                             & 3.40                             & 4.48                             & 2.68                            & 3.04                             & 2.50                              & 88.18                           & 68.40                             & 5.76                             & 86.40                            & 97.72                               \\
			Pakistan         & 255                                & 2.88                             & 1.72                             & --                               & 2.16                            & 1.26                             & --                                & 82.80                           & 69.99                             & 1.80                             & 77.72                            & 78.18                               \\
			\bottomrule
		\end{tabular}}
\end{table}

\begin{figure}[ht]
	\centering
	\begin{tikzpicture}
		\begin{axis}[
				xbar,
				width=0.8\textwidth,
				height=8cm,
				xlabel={Share (\%)},
				ytick={0,1,2,3,4,5},
				yticklabels={Security Risk, Technical Blocking, Price Discrimination, Legal Threat, Regulatory Compliance, Content Licensing},
				nodes near coords,
				xmin=0,
				bar width=15pt,
				yticklabel style={font=\footnotesize},
			]
			\addplot[fill=tudared] coordinates {
					(2.8,0) (7.2,1) (8.0,2) (8.0,3) (35.0,4) (37.2,5)
				};
		\end{axis}
	\end{tikzpicture}
	\caption{Proportional Distribution of Enforcement Categories (Aggregate)}
	\label{fig:category_dist_viz}
\end{figure}


To further quantify the intensity of these enforcement regimes, we propose the
\textbf{Fortress Index}, a metric that calculates the percentage of
enforcement-related clauses (Technical Blocking and Legal Threat) relative
to the total number of \textit{strategic} (non-General Terms) sentences in
a firm's documentation. By excluding boilerplate General Terms from the denominator,
the index measures enforcement \textit{focus}: what share of a firm's substantive
policy language is dedicated to coercive measures. The resulting percentage for each
service is referred to as its \textbf{Fortress Index}.
Figure~\ref{fig:fortress_index} visualizes the resulting ranking,
while Table~\ref{tab:fortress_index_complete} (Section~\ref{sec:detailed_analysis})
provides the complete numerical breakdown with strategic archetype assignments.

Note that the Fortress Index measures conceptually
different phenomena depending on the actor's position in the ecosystem.
For digital service providers (Netflix, YouTube, etc.), the index captures
\textit{defensive} enforcement focus: the share of policy language dedicated
to blocking circumvention and threatening violators. For \gls{vpn} providers
(ExpressVPN, NordVPN), the same categories captured \textit{adversarial}
security focus: the share of language dedicated to describing threats
against users and framing their service as protection.
The \gls{vpn} providers' high scores reflect not enforcement of geographic
boundaries but rather the mirror image: language about the security
threats that justify circumvention. We keep both groups in one ranking
to show the industry's split: the most rule-focused players sit at
opposite ends of the value chain.

\begin{figure}[ht]
	\centering
	\resizebox{\linewidth}{!}{%
		\begin{tikzpicture}
			\begin{axis}[
					xbar,
					width=0.9\textwidth,
					height=8cm,
					xlabel={Fortress Index (\%)},
					ytick={0,1,2,3,4,5,6,7,8,9},
					yticklabels={Spotify, Netflix, Disney+, Amazon Prime, Adobe, Apple Music, Microsoft, YouTube Premium, NordVPN, ExpressVPN},
					nodes near coords,
					nodes near coords align={horizontal},
					xmin=0, xmax=70,
					bar width=15pt,
					yticklabel style={font=\footnotesize},
					every node near coord/.append style={font=\footnotesize},
				]
				\addplot[fill=tudared] coordinates {
						(0.43,0) (2.03,1) (2.04,2) (3.88,3) (5.48,4) (12.50,5) (32.76,6) (33.68,7) (50.00,8) (55.56,9)
					};
			\end{axis}
		\end{tikzpicture}}
	\caption{The Fortress Index: Percentage of Enforcement Clauses
		(Technical Blocking + Legal Threat) as a Share of Strategic
		Sentences per Service}
	\label{fig:fortress_index}
\end{figure}

As shown in Figure~\ref{fig:fortress_index}, YouTube and Microsoft scored
highest among digital service providers, while Netflix and Spotify remained
at the low end. The strategic implications of this ranking are discussed in
Section~\ref{sec:detailed_analysis}.

\subsubsection{Strategic Framing by \gls{vpn} Providers}
In sharp contrast, \gls{vpn} companies adopt a ``Liberation'' and ``Privacy'' frame.
The analysis revealed a consistent narrative that reframes circumvention as \textbf{User Freedom}.
A secondary dominant frame identified in our analysis is \textbf{Privacy/Security}.
While many users may purchase \glspl{vpn} for streaming arbitrage, providers legitimize
the service by emphasizing security features. \textbf{NordVPN}, for example,
showed a distinct focus on ``Security Risk'' categories in our dataset,
with marketing materials framing this as empowering users against tracking.
However, research has shown that not all \gls{vpn} applications deliver on their
security promises \parencite{ikram2016analysis}, adding complexity to the
trust dynamic between providers and users.

\subsection{Temporal Evolution of Enforcement}
To understand how these strategies have evolved over time, we analyzed
the frequency of category-specific clauses across the dataset's years.
Table \ref{tab:timeline_count} shows the raw count of enforcement-related
incidents detected per service per year.

\begin{table}[ht]
	\centering
	\small
	\begin{tabular}{l cccccc c}
		\toprule
		\textbf{Service} & \textbf{2020} & \textbf{2021} & \textbf{2022} & \textbf{2023} & \textbf{2024} & \textbf{2025} & \textbf{Total} \\
		\midrule
		Adobe            & 2             & 2             & 0             & 0             & 1             & 1             & 6              \\
		Amazon Prime     & 1             & 0             & 0             & 0             & 1             & 2             & 4              \\
		Apple Music      & 0             & 0             & 1             & 4             & 1             & 3             & 9              \\
		Disney+          & 0             & 0             & 0             & 0             & 4             & 0             & 4              \\
		ExpressVPN       & 0             & 0             & 0             & 0             & 0             & 5             & 5              \\
		Microsoft        & 9             & 7             & 9             & 12            & 6             & 6             & 49             \\
		Netflix          & 1             & 1             & 0             & 0             & 0             & 3             & 5              \\
		NordVPN          & 0             & 0             & 0             & 0             & 3             & 3             & 6              \\
		Spotify          & 0             & 0             & 0             & 0             & 0             & 0             & 0              \\
		YouTube Premium  & 8             & 7             & 16            & 53            & 31            & 20            & 135            \\
		\midrule
		\textbf{Total}   & \textbf{21}   & \textbf{17}   & \textbf{26}   & \textbf{69}   & \textbf{47}   & \textbf{43}   & \textbf{223}   \\
		\bottomrule
	\end{tabular}
	\caption{Raw Count of Enforcement Incidents per Service (2020--2025)}
	\label{tab:timeline_count}
\end{table}

The data show a significant increase in specific enforcement clauses,
especially from 2022 onwards, driven primarily by \textbf{YouTube Premium}
(see also Figure~\ref{fig:timeline_all} for the overall trends over time
and Figure~\ref{fig:strategic_frames_evolution} for the stacked category view).
This suggested that restrictive clauses had become more prevalent and more
specific over the analyzed period, transitioning from general boilerplate
to active regulatory language.

\section{Deep Dive: Service-Specific Strategic Evolution}
\label{sec:service_deep_dive}

To understand the operational realities of geo-arbitrage enforcement,
we analyzed the longitudinal patterns of specific providers.
The absolute sentence counts per service per year are detailed in
Table~\ref{tab:timeline_count}, with absolute-count evolution charts
provided in Appendix~\ref{app:service_evolution} (Figures~\ref{fig:evol_streaming}
and~\ref{fig:evol_software_vpn}). The proportional (percentage-based)
evolution of strategic frames across service categories is discussed
in detail in Chapter~\ref{chap:discussion}.

Among content providers, \textbf{YouTube Premium} stood out with the sharpest
enforcement escalation (Table~\ref{tab:timeline_count}), while \textbf{Spotify}
remained relatively boilerplate-heavy throughout the period.

In contrast, software providers like \textbf{Adobe} maintained a
consistent, lower level of \gls{tos} enforcement language, suggesting
a reliance on technical licensing (cryptographic keys) rather than
after-the-fact legal threats. \gls{vpn} providers (\textbf{NordVPN},
\textbf{ExpressVPN}) showed minimal strategic variation in their
\gls{tos} over time, as their documentation consistently emphasizes
encryption and user protection as primary value propositions
(see Appendix~\ref{app:service_evolution} for \gls{vpn}-specific
evolution data and absolute count charts).

\subsection{High-Confidence Findings: The Core Clauses}
The Gemini 3 Flash model identified specific, high-confidence
clauses that are central to the coercive strategy.
For example, clauses stating ``You may not use any
technology to obscure or disguise your location''
were consistently categorized as \textit{Technical Blocking}
with $>0.95$ confidence. Similarly, clauses referencing
``We may terminate your account if you access the service
from a country different from your registration country''
were classified as \textit{Legal Threat} with comparable confidence levels.

The high-confidence classifications cluster around four distinct linguistic patterns.
The first is \textbf{explicit technology prohibition}, consisting of
direct references to \glspl{vpn}, proxies, or location-disguising tools.
These clauses appeared primarily in YouTube
($N=94$ Technical Blocking clauses) and Microsoft ($N=4$)
documentation, representing the most unambiguous form of
coercive enforcement language.
The second pattern is \textbf{geographic conditionality},
where clauses tie service access, pricing, or content availability to the
user's verified geographic location. These were most prevalent in
Netflix ($N=36$ Price Discrimination clauses) and
Disney+ ($N=41$), reflecting their territorial licensing models.
Third, \textbf{account integrity requirements} encompass
provisions requiring accurate registration information,
particularly regarding country of residence. These appeared
across all service providers but were most concentrated in
Microsoft's documentation ($N=53$ Legal Threat clauses).
Finally, \textbf{compliance framing} refers to
clauses that justify geographic restrictions by referencing
legal obligations, licensing agreements, or regulatory requirements.
These were the most common non-boilerplate category overall
($N=525$ Content Licensing clauses), reflecting the industry's
preferred strategy of attributing restrictions
to external legal constraints rather than corporate pricing decisions.

This confirmed that firms had made technical countermeasures a
formal, documented part of their legal frameworks rather than
relying solely on undisclosed technical measures.


\subsection{The Affordability Paradox: Real vs. Nominal Cost}
While the \gls{dspi} measured the \textit{nominal} price difference
(relevant to arbitrageurs), it is important to analyze the ``Real Cost''
for local residents. Figure \ref{fig:affordability_real} maps the
cost of digital services as a percentage of the \textbf{Median National Monthly Wage},
serving as a digital equivalent to ``Time-to-Earn'' indices used in
purchasing power comparisons (e.g., the Big Mac Index's affordability variant).

\begin{figure}[ht]
	\centering
	\includegraphics[width=0.9\textwidth]{figures/affordability_heatmap.pdf}
	\caption{The Affordability Gap: Digital Service Cost as Percentage of
		Local Monthly Income. Darker red indicates higher relative cost for local citizens.}
	\label{fig:affordability_real}
\end{figure}

The data reveal a critical paradox: while low-income markets such as
Pakistan, the Philippines, and Turkey offered the cheapest nominal prices
for international arbitrageurs (average \gls{dspi} values of 0.45, 0.54, and
0.65, respectively), these same services were significantly
\textit{more expensive} for locals in real terms. For instance, a Standard
Netflix subscription in Turkey consumes approximately 1.22\% of the median
monthly wage compared to approximately 0.27\% in the USA.

This distinction is critical. On the one hand, \textbf{high \gls{dspi} variance} creates incentives for \textit{external} abuse through \gls{vpn} arbitrage. On the other hand, \textbf{low affordability} justifies the \textit{internal} pricing strategy, as low nominal prices are necessary for market penetration rather than being optional discounts.

Thus, low nominal prices observed in the Global South are not ``bargains''
but necessary economic adjustments that accidentally create vulnerabilities
exploited by Global North users. This paradox has broader ethical
implications. When a user in Germany subscribes to YouTube Premium
through a Turkish \gls{vpn} at \$2.43/month instead of \$14.16/month,
they are not simply finding a ``deal'', they are exploiting a pricing
structure designed to make digital services accessible to consumers
earning a median wage of \$761/month. If this behavior becomes widespread,
firms may respond by raising prices in low-income markets
(to reduce the arbitrage incentive) or by withdrawing services entirely
from unprofitable regions, both of which harm the local consumers
the pricing was designed to serve.

The Affordability Paradox also revealed that the commonly used metric of
nominal price comparison (which dominates consumer forums and technology journalism)
fundamentally misrepresents the economic reality. A subscription that
costs 0.27\% of monthly income in the USA but 3.28\% in Argentina is
not ``cheaper'' in Argentina in any meaningful economic sense:
it is over twelve times more burdensome relative to local earning power.
This finding shows the importance of the \gls{ptw} ratio as a
complement to the \gls{dspi} when assessing the true picture of digital pricing.

\section{Correlation Analysis: The Strategic Trade-off}
\label{sec:correlation}
To test the relationship between pricing strategy and enforcement intensity,
we used the cleaned dataset to calculate the correlation between Price Discrimination
(PD) and observed Enforcement Intensity (EI).
Table \ref{tab:correlation_data} summarizes the key metrics.

\begin{table}[ht]
	\centering
	\small
	\begin{tabular}{l c c}
		\toprule
		\textbf{Service} & \textbf{PD Score (\gls{dspi} StdDev)} & \textbf{Enforcement Intensity (\%)}                   \\
		\midrule
		Microsoft        & 0.208                                 & 0.87                                                  \\
		YouTube Premium  & 0.464                                 & 3.07                                                  \\
		Spotify          & 0.486                                 & 0.03                                                  \\
		Adobe            & 0.245                                 & 0.13                                                  \\
		Netflix          & 0.352                                 & 0.17                                                  \\
		Disney+          & 0.324                                 & 0.18                                                  \\
		Amazon Prime     & 0.304                                 & 0.24                                                  \\
		Apple Music      & 0.446                                 & 0.75                                                  \\
		ExpressVPN       & 0.112                                 & 8.33                                                  \\
		NordVPN          & 0.231                                 & 5.45                                                  \\
		\midrule
		\multicolumn{3}{l}{\textbf{Pearson $r$, all $N=10$:} $-0.5506$ ($p = 0.10$, not significant at $\alpha = 0.05$)} \\
		\multicolumn{3}{l}{\textbf{Pearson $r$, excl.\ \glspl{vpn} $N=8$:} $+0.3467$ ($p = 0.40$, not significant)}      \\
		\bottomrule
	\end{tabular}
	\caption{Price Discrimination Score, Enforcement Intensity, and Their Correlation.
		The full-sample Pearson $r \approx -0.55$ ($p = 0.10$) is driven primarily by
		\gls{vpn} providers (low PD, high EI) and is not statistically significant at
		conventional thresholds. Excluding \glspl{vpn} ($N=8$), the correlation reverses to a
		weak positive $r \approx +0.35$ ($p = 0.40$), suggesting a temporary trend
		where higher price discrimination relates with slightly higher enforcement
		among digital service providers, though neither result is statistically significant.
		Given the small sample sizes, these correlations should be interpreted as
		exploratory pattern observations.}
	\label{tab:correlation_data}
\end{table}

\begin{figure}[ht]
	\centering
	\includegraphics[width=0.9\textwidth]{figures/protection_vs_pricing_updated.pdf}
	\caption{Strategic Alignment: Comparison of Price Discrimination scores
		vs. Enforcement Intensities across analyzed services.}
	\label{fig:correlation}
\end{figure}

The refined analysis ($N=10$), visualized in Figure~\ref{fig:correlation},
revealed a complex relationship between price variance and enforcement.
Specific sector clusters emerged that showed distinct strategic behaviors.
This suggests that firms with established global pricing power and providing real local services (like Amazon)
rely less on aggressive legal threats than smaller localized services or those
in highly contested content markets.

Among \textbf{content providers} (Netflix, Disney+, YouTube, Xbox, etc.),
a ``High Enforcement Cluster'' emerged that illustrated
the enforcement trade-off within the streaming sector.
\textbf{YouTube} exhibits the largest combination of
price variance and enforcement intensity
(\gls{dspi} StdDev $= 0.46$, Enforcement Intensity $= 3.07\%$),
with a Fortress Index of 33.68\% reflecting its aggressive
Technical Blocking approach.
By contrast, \textbf{Disney+}, \textbf{Netflix}, and \textbf{Spotify}
maintain substantial price variance
(\gls{dspi} StdDev $0.32$--$0.49$) but rely on Content Licensing
rather than Technical Blocking
(Fortress Index values below 2.1\%), suggesting enforcement
through licensing agreements rather than active
technical measures.
\textbf{Xbox Game Pass} serves as a useful control case within this cluster.
Governed by the Microsoft ecosystem (and combined with Microsoft 365 in the
qualitative analysis, as both services share the same \gls{tos}),
Xbox Game Pass has moderate global price variance
(\gls{dspi} StdDev $= 0.27$), higher than Microsoft 365's own
($0.21$), yet shares Microsoft's low enforcement intensity
(EI $= 0.87\%$).
This suggested that when a firm's enforcement approach is
set at the ecosystem level (via shared \gls{tos}), even a product
with moderate pricing variance can exhibit low enforcement
if the parent entity favors identity-based rather than
location-based controls.

The \textbf{\gls{vpn} enablers} (NordVPN, ExpressVPN), as expected,
showed minimal ``Technical Blocking'' enforcement,
as their business model depends on circumventing the very barriers
erected by the Content Providers. The legal ambiguity surrounding \gls{vpn}
usage for geo-arbitrage has been explored by \textcite{ariyarathna2022circumvention},
who argues that digital location is an increasingly contested legal concept.

These data suggest that \textbf{Business Model} (Streaming vs. Download vs. Access)
was a stronger predictor of enforcement strategy than \textbf{Price Opportunity} alone.

\begin{figure}[ht]
	\centering
	\begin{tikzpicture}
		\begin{axis}[
				width=0.9\textwidth,
				height=7cm,
				xlabel={Year},
				ylabel={Incident Count},
				xmin=2019.5, xmax=2025.5,
				xtick={2020,2021,2022,2023,2024,2025},
				xticklabel style={/pgf/number format/set thousands separator={}},
				grid=major,
				legend style={at={(0.5,-0.22)}, anchor=north, legend columns=3, font=\footnotesize},
				ymin=0
			]
			\addplot[thick, color=tudablue, mark=square*] coordinates { (2020,71) (2021,70) (2022,87) (2023,102) (2024,96) (2025,99) };
			\addlegendentry{Content Licensing}
			\addplot[thick, color=tudagreen, mark=triangle*] coordinates { (2020,55) (2021,74) (2022,72) (2023,117) (2024,114) (2025,87) };
			\addlegendentry{Regulatory Compliance}
			\addplot[thick, color=tudared, mark=x] coordinates { (2020,6) (2021,5) (2022,13) (2023,44) (2024,24) (2025,14) };
			\addlegendentry{Technical Blocking}
			\addplot[thick, color=cyan, mark=diamond*] coordinates { (2020,16) (2021,26) (2022,20) (2023,12) (2024,33) (2025,13) };
			\addlegendentry{Price Discrimination}
			\addplot[thick, color=orange, mark=pentagon*] coordinates { (2020,15) (2021,12) (2022,13) (2023,25) (2024,20) (2025,21) };
			\addlegendentry{Legal Threat}
			\addplot[thick, color=tudagray, mark=otimes] coordinates { (2020,10) (2021,4) (2022,10) (2023,2) (2024,7) (2025,6) };
			\addlegendentry{Security Risk}
		\end{axis}
	\end{tikzpicture}
	\caption{Temporal Evolution of Category Incident Counts (Aggregate, Excluding \gls{vpn} Providers)}
	\label{fig:timeline_all}
\end{figure}

\begin{figure}[ht]
	\centering
	\begin{tikzpicture}
		\begin{axis}[
				ybar stacked,
				width=0.85\textwidth,
				height=7cm,
				xlabel={Year},
				ylabel={Count (Excl.\ General Terms)},
				xmin=2019.5, xmax=2025.5,
				ymin=0,
				xtick={2020,2021,2022,2023,2024,2025},
				xticklabel style={/pgf/number format/set thousands separator={}},
				legend style={at={(0.5,-0.22)}, anchor=north, legend columns=3, font=\footnotesize},
				legend cell align={left},
				bar width=15pt,
				enlarge x limits=0.08,
			]
			\addplot[ybar stacked, fill=tudablue, draw=black!30] coordinates { (2020,71) (2021,70) (2022,87) (2023,102) (2024,96) (2025,99) };
			\addlegendentry{Content Licensing}
			\addplot[ybar stacked, fill=tudagreen, draw=black!30] coordinates { (2020,55) (2021,74) (2022,72) (2023,117) (2024,114) (2025,87) };
			\addlegendentry{Regulatory Compliance}
			\addplot[ybar stacked, fill=tudared, draw=black!30] coordinates { (2020,6) (2021,5) (2022,13) (2023,44) (2024,24) (2025,14) };
			\addlegendentry{Technical Blocking}
			\addplot[ybar stacked, fill=cyan!70, draw=black!30] coordinates { (2020,16) (2021,26) (2022,20) (2023,12) (2024,33) (2025,13) };
			\addlegendentry{Price Discrimination}
			\addplot[ybar stacked, fill=orange!80, draw=black!30] coordinates { (2020,15) (2021,12) (2022,13) (2023,25) (2024,20) (2025,21) };
			\addlegendentry{Legal Threat}
			\addplot[ybar stacked, fill=tudagray, draw=black!30] coordinates { (2020,10) (2021,6) (2022,14) (2023,15) (2024,9) (2025,13) };
			\addlegendentry{Other (Privacy, Sec.~Risk, Portab.)}
		\end{axis}
	\end{tikzpicture}
	\caption{Evolution of Strategic Frames over Time (Excluding General Terms and \gls{vpn} Providers)}
	\label{fig:strategic_frames_evolution}
\end{figure}


In the ``Software and Music'' category (see Figure~\ref{fig:evol_software_main}
for the proportional evolution), \textbf{Microsoft}
stood out with a consistent use of ``Legal Threat'' and ``Regulatory Compliance''
frames, likely due to its enterprise customer base and strict licensing requirements.
\textbf{Adobe} showed a return of ``Price Discrimination'' language in 2024,
possibly linked to new regional pricing structures.
\textbf{Spotify} and \textbf{Apple Music} remained largely focused on
``Licensing'' and passive ``Regulatory Compliance,'' showing less active
technical enforcement than their video counterparts.

\subsubsection{Evolution of Ecosystem Players (Amazon \& Apple)}
Beyond the pure content and \gls{vpn} providers, the ``Ecosystem'' players
(\textbf{Amazon Prime}, \textbf{Apple Music}) showed distinct evolutionary
paths (see \textbf{Appendix \ref{app:service_evolution}} for full charts).
\textbf{Amazon Prime} (Figure \ref{fig:evol_video_main}) displayed a unique,
consistent focus on \textbf{Regulatory Compliance} frames ($\approx 35\%$)
rather than Technical Blocking. This suggested Amazon managed cross-border
access through account-level shipping/billing addresses rather than active network filtering.
In contrast, \textbf{Apple Music} (Figure \ref{fig:evol_music_vs_video}) remained
the most ``static'' of all services, with very low enforcement counts that have
barely changed since 2020. The strategic implications of this stability are
discussed in Section~\ref{sec:detailed_analysis}.

\section{Strategic Shifts and Fortress Index Ranking}
\label{sec:detailed_analysis}

This section provides a granular view of the enforcement picture, detailing the specific category
shifts and the calculated ``Fortress Index values'' for each service.

\begin{table}[ht]
	\centering
	\caption{Complete Fortress Index Ranking. \gls{vpn} providers score highest due
		to their security-focused framing, followed by YouTube and Microsoft
		with active enforcement strategies.}
	\label{tab:fortress_index_complete}
	\begin{tabular}{l r l}
		\toprule
		\textbf{Company} & \textbf{Fortress Index (\%)} & \textbf{Archetype}  \\
		\midrule
		ExpressVPN       & 55.56                        & \gls{vpn} Enabler   \\
		NordVPN          & 50.00                        & \gls{vpn} Enabler   \\
		YouTube Premium  & 33.68                        & Content Fortress    \\
		Microsoft        & 32.76                        & Enterprise Fortress \\
		Apple Music      & 12.50                        & Ecosystem Fortress  \\
		Adobe            & 5.48                         & Utility Paradox     \\
		Amazon Prime     & 3.88                         & Ecosystem Fortress  \\
		Disney+          & 2.04                         & Content Fortress    \\
		Netflix          & 2.03                         & Content Fortress    \\
		Spotify          & 0.43                         & Content Fortress    \\
		\bottomrule
	\end{tabular}
\end{table}



The Fortress Index (Table \ref{tab:fortress_index_complete}) confirms the split in the market.
\textbf{YouTube Premium} (33.68\%) and \textbf{Microsoft} (32.76\%) have effectively built
``digital fortresses,'' distinguishing themselves from the lower-scoring streaming
incumbents like \textbf{Netflix} (2.03\%) and \textbf{Spotify} (0.43\%),
which continue to rely on passive licensing terms.

\begin{table}[ht]
	\centering
	\footnotesize
	\renewcommand{\arraystretch}{1.1}
	\caption{Absolute Category Counts by Year Across All Services (2020--2025).
		Technical Blocking peaks sharply in 2023, coinciding with YouTube's enforcement escalation.}
	\label{tab:qual_timeline_complete}
	\resizebox{\textwidth}{!}{%
		\begin{tabular}{l *{11}{r} | r}
			\toprule
			\textbf{Year}  & \rotatebox{70}{\textbf{Tech. Block.}} & \rotatebox{70}{\textbf{Price Discr.}} & \rotatebox{70}{\textbf{Licensing}} & \rotatebox{70}{\textbf{Regulatory}} & \rotatebox{70}{\textbf{Legal Thr.}} & \rotatebox{70}{\textbf{Acc.\ Act.}} & \rotatebox{70}{\textbf{Privacy}} & \rotatebox{70}{\textbf{Sec.\ Risk}} & \rotatebox{70}{\textbf{Portab.}} & \rotatebox{70}{\textbf{Workaro.}} & \rotatebox{70}{\textbf{General}} & \rotatebox{70}{\textbf{Total}} \\
			\midrule
			2020           & 6                                     & 16                                    & 71                                 & 55                                  & 15                                  & 0                                   & 0                                & 10                                  & 0                                & 0                                 & 3,386                            & 3,559                          \\
			2021           & 5                                     & 26                                    & 70                                 & 74                                  & 12                                  & 0                                   & 2                                & 4                                   & 0                                & 0                                 & 3,645                            & 3,838                          \\
			2022           & 13                                    & 20                                    & 87                                 & 72                                  & 13                                  & 0                                   & 4                                & 10                                  & 0                                & 0                                 & 4,001                            & 4,220                          \\
			2023           & 44                                    & 12                                    & 102                                & 117                                 & 25                                  & 0                                   & 11                               & 2                                   & 2                                & 0                                 & 4,567                            & 4,882                          \\
			2024           & 24                                    & 33                                    & 96                                 & 114                                 & 23                                  & 0                                   & 2                                & 10                                  & 0                                & 0                                 & 4,880                            & 5,182                          \\
			2025           & 15                                    & 14                                    & 99                                 & 89                                  & 28                                  & 0                                   & 8                                & 9                                   & 0                                & 0                                 & 3,402                            & 3,664                          \\
			\midrule
			\textbf{Total} & \textbf{107}                          & \textbf{121}                          & \textbf{525}                       & \textbf{521}                        & \textbf{116}                        & \textbf{0}                          & \textbf{27}                      & \textbf{45}                         & \textbf{2}                       & \textbf{0}                        & \textbf{23,881}                  & \textbf{25,345}                \\
			\bottomrule
		\end{tabular}}
\end{table}

Table \ref{tab:qual_timeline_complete} reveals the year-by-year trends
behind these scores. The aggregate data show a clear spike
in ``Technical Blocking'' clauses in 2023 (44 incidents), a direct response
to the post-pandemic surge in \gls{vpn} usage. ``Regulatory Compliance''
also saw a steady increase, reflecting the growing complexity of global
digital trade laws. Note that the decrease in total counts for 2025 reflects the
natural variation in document publication frequency across services. A consistent
forward-fill strategy ensures that each service's most recent \gls{tos} is
carried into years without an update, providing an accurate picture of the active
regulatory environment at any given time.

The distribution of these categories (Figure \ref{fig:category_dist_viz})
highlighted distinct enforcement styles. Notably, \textbf{Microsoft} has largely
abandoned active enforcement for its productivity suite, relying instead on
ecosystem lock-in, whereas \textbf{YouTube} has pursued aggressive boundary
enforcement followed by a consolidation period, as seen in the fluctuation of
its ``Technical Blocking'' and ``Legal Threat'' clauses over time.

\chapter{Discussion}
\label{chap:discussion}

This chapter brings together the pricing data and the corporate strategy findings
to answer the main question of this thesis:
how do digital service providers strategically respond to \gls{vpn}-enabled geo-arbitrage,
and what determines the nature of that response? The results from Chapter~\ref{chap:results}
established two key findings that require interpretation. First, the \gls{dspi}
demonstrated that price differences across markets are substantial and systematic,
with content streaming services showing the highest variance. Second, the \gls{tos}
analysis revealed that enforcement intensity varies dramatically across providers,
with YouTube and Microsoft exhibiting far higher enforcement density than Netflix
or Spotify. The key question is \textit{why} these differences exist.
The following discussion identifies distinct strategic archetypes that illustrate
how different industry segments respond to arbitrage, arguing that business model
architecture (not the size of the price gap) is the primary determinant of
enforcement strategy. The chapter concludes by exploring the broader implications
of the ongoing technological contest between platforms and consumers and the policy
questions this contest raises.

\section{Strategic Archetypes}
Based on the \gls{tos} analysis ($N=10$, covering 8 digital service providers
and 2 \gls{vpn} enablers), we organize the findings around three high-level
\textbf{Macro-Categories} that aggregate the initial 10 coding categories
to facilitate cross-sectoral comparison:
\begin{enumerate}
	\item \textbf{Business Model Adaptation \& Pricing (\gls{bma}):}
	      Aggregates \textit{Content Licensing}, \textit{Regulatory Compliance},
	      \textit{Price Discrimination}, and \textit{Legitimate Portability}.
	      This represents the ``soft'' strategic layer where firms adjust their offerings
	      to local market conditions.
	\item \textbf{Coercive Restriction \& Legal Threat (\gls{crl}):}
	      Aggregates \textit{Technical Blocking}, \textit{Legal Threat},
	      \textit{Account Action}, and \textit{Security Risk}.
	      This represents the ``hard'' enforcement layer designed to physically or
	      legally prevent circumvention.
	\item \textbf{General Corporate Operations (\gls{gco}):}
	      Aggregates \textit{Privacy/Security} (\gls{vpn} frame) and \textit{User Workaround}.
	      This captures supporting narratives and adversarial user behavior descriptions.
\end{enumerate}

Figure \ref{fig:strategic_framing_pivot} provides a high-level overview of
how these macro-strategies appear across the analyzed companies.

\begin{figure}[ht]
	\centering
	\begin{tikzpicture}
		\begin{axis}[
				xbar stacked,
				width=0.85\textwidth,
				height=10cm,
				symbolic y coords={ExpressVPN,NordVPN,Apple Music,YouTube Premium,Spotify,Netflix,Disney+,Amazon Prime,Adobe,Microsoft},
				ytick=data,
				yticklabel style={font=\small},
				xlabel={Share of Strategic Sentences (\%)},
				xmin=0, xmax=105,
				legend style={at={(0.5,-0.15)}, anchor=north, legend columns=-1, font=\footnotesize},
				bar width=12pt,
				nodes near coords,
				every node near coord/.append style={font=\tiny},
				/pgf/number format/fixed, /pgf/number format/precision=1,
			]
			% Business Model Adaptation & Pricing
			\addplot[fill=tudablue!80] coordinates {
					(80.3,Adobe) (79.6,Amazon Prime) (57.1,Apple Music) (73.2,Disney+) (35.0,ExpressVPN) (82.0,Microsoft) (69.4,Netflix) (34.5,NordVPN) (75.7,Spotify) (60.4,YouTube Premium)
				};
			% Coercive Restriction & Legal Threat
			\addplot[fill=tudared!80] coordinates {
					(18.6,Adobe) (18.0,Amazon Prime) (40.7,Apple Music) (26.6,Disney+) (61.7,ExpressVPN) (17.1,Microsoft) (29.8,Netflix) (60.0,NordVPN) (23.5,Spotify) (39.0,YouTube Premium)
				};
			% General Corporate Operations
			\addplot[fill=tudagreen!80] coordinates {
					(1.1,Adobe) (2.4,Amazon Prime) (2.2,Apple Music) (0.2,Disney+) (3.3,ExpressVPN) (0.9,Microsoft) (0.9,Netflix) (5.5,NordVPN) (0.7,Spotify) (0.6,YouTube Premium)
				};
			\legend{\gls{bma} (Adaptation), \gls{crl} (Enforcement), \gls{gco} (Operations)}
		\end{axis}
	\end{tikzpicture}
	\caption{Strategic Framing by Company: Macro-Category Distribution. \gls{vpn} providers (ExpressVPN, NordVPN) show the highest proportion of coercive/legal framing (\gls{crl} $>$ 60\%), while software companies (Microsoft, Adobe) and streaming platforms overwhelmingly use business model adaptation framing (\gls{bma} $>$ 70\%).}
	\label{fig:strategic_framing_pivot}
\end{figure}

The change over time in these categories, as previously detailed in
Table \ref{tab:qual_timeline_complete}, confirms the shifting focus
toward technical countermeasures within the \gls{crl} macro-category.



\subsection{The Content Fortress: Defensive Value Capture}
Firms like \textbf{Netflix} and \textbf{Disney+} focus on keeping their prices separate rather
than making things easy for the user. Our data shows that in this sector, high price
differences and strict blocking go together. This finding is consistent with
\textcite{cardona2015consumer}, who demonstrate that geo-blocking significantly
affects consumer shopping behavior and willingness to pay. This shows that blocking
is not just a reaction
to price hops, but a standard way these companies work to protect their content in each region.

The ``Content Fortress'' strategy is deeply rooted in the structure of the
entertainment industry's licensing model. Content rights are typically sold on a
territory-by-territory basis \parencite{lobato2019geoblocking}, meaning that a
streaming platform may hold the rights to show a film in Germany but not in the
United Kingdom, where a competing platform may hold the license. This territorial
licensing model is older than digital distribution, as it was designed for
theatrical releases and broadcast television, where geographic boundaries were
enforced by physical infrastructure. The transition to digital streaming has
not eliminated these territorial rights. Instead, it has created a situation
where platforms must enforce artificial geographic boundaries on an inherently
borderless medium.

This aligns with the ``Fortress'' strategy described by \textcite{alaveras2017geoblocking},
where incumbent firms construct digital barriers to protect legacy revenue streams. However,
as noted by \textcite{lobato2019geoblocking}, such strategies often suffer from a ``clarity''
problem: users encounter ``This content is not available'' error messages without understanding
the underlying legal framework. The frustration this creates among consumers should not be
underestimated: when users perceive restrictions as arbitrary or unfair, they are more
likely to seek circumvention tools \parencite{xia2004price}, feeding the very arbitrage
cycle that the fortress strategy aims to prevent.

\textbf{YouTube Premium} stands out as the most aggressive enforcer in this category,
with a Fortress Index of 33.68\%. YouTube's enforcement surge beginning in 2022
(Table~\ref{tab:timeline_count}) coincides with a sharp rise in Technical Blocking
clauses from 13 in 2022 to 44 in 2023 (Table~\ref{tab:qual_timeline_complete}),
reflecting a systematic crackdown on users who had subscribed through Turkish or
Argentine accounts to exploit significantly lower regional prices. Unlike Netflix
or Disney+, YouTube's
enforcement is not primarily driven by third-party licensing constraints
(since much of YouTube's content is user-generated), but rather by direct revenue
protection for its Premium subscription tier. This makes YouTube a ``pure'' case
of the Content Fortress strategy: blocking is motivated entirely by pricing integrity
rather than contractual obligations.

Interestingly, \textbf{Netflix's} low Fortress score (2.03\%) is surprising given its role
as a streaming pioneer. \textcite{aguiar2018netflix} examine whether Netflix acts as
a ``global hegemon'' or a facilitator of frictionless digital trade, finding that its
catalog varies dramatically across countries, a direct consequence of territorial licensing
that gives users an additional motivation to circumvent geo-blocking beyond price alone.
This low score can be explained by a strategic shift in \gls{bmi}: Netflix has moved
heavily toward in-house content production (``Netflix Originals''). By owning its own
content, Netflix faces fewer regional licensing problems, which reduces the need for strict
geographic blocking. Over time, this strategy may also lower the arbitrage motivation, as
Netflix Originals are available globally without regional restrictions.
This represents a transition from defensive innovation (Value Capture) toward
adaptive innovation (Value Proposition): rather than building higher walls, Netflix is
making the walls less necessary by producing content it fully controls.

\subsection{The Ecosystem Fortress: Adaptive Value Proposition}
Platforms like \textbf{Apple Music} and \textbf{Amazon Prime} show a ``Globalist''
approach that innovates on the \textbf{Value Proposition}. Unlike the Content Fortress,
which defends existing revenue models through technical barriers, the Ecosystem
Fortress makes arbitrage impractical by embedding digital services within broader
product ecosystems that are inherently tied to a user's physical identity and location.

\textbf{Apple Music}, with a low focus on \textbf{Technical Blocking}
(only 2 clauses, representing 12.5\% of its strategic focus but only 0.15\% of total policy volume)
and a strong emphasis on \textbf{Price Discrimination} (5.7\%), appears to accept the reality
of international price division \parencite{brouthers2016explaining}. Rather than ``repairing''
the Value Capture mechanism through blocking, they rely on a superior Value Delivery ecosystem
(hardware integration, iCloud) that makes the friction of using a \gls{vpn}-based ``foreign'' account
essentially ``not worth it'' for the user. To use a foreign Apple Music account, a user would
need to change their entire Apple ID region, which affects App Store purchases, iCloud storage,
warranty coverage, and payment methods across all Apple devices. This ``switching cost''
is far higher than the potential savings from a cheaper music subscription, effectively
making the ecosystem itself the enforcement mechanism. This aligns with the concept of
platform lock-in described by \textcite{shapiro1998information}, where compatibility
and integration costs deter users from switching between ecosystems.
\textcite{burnham2003switching} identify three types of switching costs, namely procedural
(learning and setup effort), financial (sunk costs and lost benefits), and relational
(emotional attachment), all of which are high in Apple's ecosystem. Also,
\textcite{zhu2012platform} show that platform-based markets with strong indirect network
effects create substantial entry barriers, which in the context of geo-arbitrage means that
the ``cost'' of maintaining a foreign account extends far beyond the subscription price itself.

\textbf{Amazon Prime} adopts a similar but legally distinct strategy. As shown in our
\textbf{Detailed Evolution Analysis} (Appendix \ref{app:service_evolution},
Figure \ref{fig:evol_video_main}), Amazon relies heavily on \textbf{Regulatory Compliance} language.
Instead of technical cat-and-mouse games, Amazon anchors its digital services to physical
shipping addresses and tax jurisdictions. This creates a ``Bureaucratic Fortress'' where the
barrier to entry is not an IP filter, but a valid residential address and local credit card,
making \gls{vpn}-based arbitrage logistically difficult rather than technically impossible.
The bundle nature of Amazon Prime (combining video streaming, music, e-book lending,
free shipping, and cloud storage) further increases the switching cost. A user seeking
cheaper streaming prices through a foreign Amazon account would lose access to shipping
benefits in their home country, creating a natural deterrent that no technical blocking
system could match.

\subsection{The Enterprise Fortress: Identity-Based Value Capture}
A new archetype identified in this study is the ``Enterprise Fortress,'' shown by
\textbf{Microsoft}. Despite having the lowest global price variance among digital service
providers (\gls{dspi} StdDev $= 0.21$, indicating a relatively harmonized global price for
Microsoft 365), Microsoft exhibits the highest
intensity of \textbf{Legal Threat} clauses (53 total, the highest among all service providers).
This suggests that for utility software, the \textbf{Value Capture} is protected not
by \textit{network} blocking (which targets Location), but by \textit{identity}
verification (which targets the User). The ``Fortress'' is built to keep unauthorized resellers
out, reinforcing the subscription model's integrity without compromising the global
\textbf{Value Delivery} of the software itself.

Microsoft's approach reflects a fundamentally different enforcement philosophy compared
to streaming services. While Netflix and YouTube must verify a user's location in real
time during every streaming session, Microsoft only needs to verify identity at the
point of purchase and during periodic license validation checks. This means that
Microsoft's enforcement is ``front-loaded'' (concentrated at the point of sale)
rather than ``continuous'' (ongoing during use). This distinction has important
implications: a user who successfully purchases a Microsoft 365 license through a
\gls{vpn}-enabled foreign account can use the software without further geographic
verification, whereas a Netflix user must maintain their \gls{vpn} connection
during every viewing session.

Microsoft's enterprise customer base also creates a natural deterrent
against arbitrage. Corporate IT departments purchasing Microsoft 365 licenses
in bulk are unlikely to route their procurement through foreign \gls{vpn} servers,
as this would create compliance, tax, and audit risks that far outweigh any
subscription savings. The ``Enterprise Fortress'' thus works primarily because
the target customer segment has fundamentally different risk tolerances than individual consumers.

\subsection{The Utility Paradox (Adobe)}
\textbf{Adobe} presents a unique case. Despite moderate price discrimination
(\gls{dspi} StdDev $= 0.25$, comparable to Microsoft rather than the higher-variance content
streaming providers), it maintains remarkably low ``Technical Blocking'' enforcement. Based on our analysis of Adobe's Terms
of Service and product documentation, this appears to be because Adobe's enforcement mechanism
is ``on-device'' (software activation keys) rather than ``on-network'' (IP filtering). This
observation suggests that ``Technical Blocking'' is a strategy specific to
\textit{cloud-streamed} content, whereas \textit{downloaded software} may rely on different
protection mechanisms.

This is consistent with the broader literature on digital content protection.
As \textcite{aguiar2016digital} demonstrate, digital consumption patterns are
shaped by access mechanisms, and \gls{drm} plays a role in how consumers interact
with digital products. Interestingly,
\textcite{sinha2010drm} demonstrated that removing \gls{drm} from music files actually
\textit{increased} legitimate purchases by converting pirates into paying customers,
suggesting that overly restrictive protection can be self-defeating.
Adobe's position is best understood through the historical evolution of software protection.
In the era of physical media distribution (1990s--2000s), software piracy was widespread
because protection was limited to serial numbers and basic installation checks that
were trivially circumvented. The transition to subscription-based \gls{saas} models
(Adobe's shift from perpetual licenses to Creative Cloud in 2013 \parencite{adobe2013creative})
fundamentally changed the enforcement situation: instead of protecting a one-time purchase,
Adobe now validates subscriptions continuously through cloud authentication.
This ``always-connected'' model provides a built-in enforcement mechanism that
streaming services lack: while Netflix must verify location during playback,
Adobe verifies identity and license validity during software launch,
making the enforcement architecture fundamentally different.


However, a hybrid future appears to be emerging in the form of \textbf{``Always-Online \gls{drm}''}.
Features like Adobe's cloud-dependent generative tools (Firefly, Neural Filters)
require authenticated connections to
function, effectively merging network-based verification with identity authentication.
As more software features migrate from local computation to cloud processing, the
distinction between ``downloaded software'' and ``streamed content'' blurs.
Adobe's Fortress Index of 5.48\% may therefore understate its future enforcement
trajectory: as cloud-dependent features become the primary value proposition,
Adobe may converge toward the Content Fortress model, where the delivery mechanism
itself enforces geographic restrictions. This convergence raises important questions
about the impact of such mechanisms on user privacy and security \parencite{lindsay2006copyright}.
This reflects the ``Opportunities and Risks'' of \gls{saas} adoption \parencite{benlian2011opportunities},
where control shifts from the client device to the cloud provider.

The Utility Paradox thus reveals a broader pattern in digital economics:
the enforcement mechanism available to a firm depends not on its pricing strategy
or market power, but on the \textit{technical architecture of value delivery}.
Downloaded software can embed cryptographic verification, streamed content requires
network-level control, and ecosystem-embedded services leverage identity dependencies.
As these architectures converge through cloud computing and AI-dependent features,
the strategic archetypes identified in this study may themselves evolve,
potentially collapsing the current diversity of enforcement approaches into a
more uniform ``cloud fortress'' model where all digital services require
continuous authenticated connections. The accelerating integration of AI into
digital services reinforces this trajectory: features such as Adobe's Firefly
generative tools or Spotify's AI-curated playlists are inherently server-side,
requiring real-time authentication that simultaneously enables geographic
enforcement as a by-product of the service architecture itself.

\subsection{The Adversarial Cycle: A View Over Time}
The relationship between providers and consumers is not static. Our historical analysis
reveals a clear ``Action-Reaction'' cycle, visualized in Figure \ref{fig:timeline}.

\begin{figure}[ht]
	\centering
	\resizebox{\textwidth}{!}{%
		\begin{tikzpicture}[x=2.1cm, y=1cm]
			% Draw the timeline line
			\draw[->, thick] (0,0) -- (7.0,0) node[right] {Year};

			% Ticks and Labels
			\foreach \x/\year in {0.5/2020, 1.7/2021, 2.9/2022, 4.1/2023, 5.3/2024, 6.5/2025} {
					\draw[thick] (\x,0.1) -- (\x,-0.1);
					\node[below=0.2cm] at (\x,0) {\textbf{\year}};
				}

			% Events Top (Corporate/Coercive)
			\node[align=center, text width=3.0cm, above=1.8cm, font=\footnotesize] (pandemic) at
			(0.5,0) {\textbf{Pandemic Surge}\\(Arbitrage Wave)\\[-2pt]{\scriptsize\parencite{atlasvpn2021covid}}};
			\draw[thin] (0.5,0.2) -- (pandemic);

			\node[align=center, text width=3.0cm, above=0.5cm, font=\footnotesize] (resip) at
			(1.7,0) {\textbf{Residential IP Crackdown}\\(The ``Hard'' Filter)\\[-2pt]{\scriptsize\parencite{maxmind2021geolocation}}};
			\draw[thin] (1.7,0.2) -- (resip);

			\node[align=center, text width=3.0cm, above=1.8cm, font=\footnotesize] (youtube) at
			(2.9,0) {\textbf{YouTube Enforcement Surge}\\(Technical Blocking)\\[-2pt]{\scriptsize Table~\ref{tab:qual_timeline_complete}}};
			\draw[thin] (2.9,0.2) -- (youtube);

			\node[align=center, text width=3.0cm, above=0.5cm, font=\footnotesize] (peak) at
			(4.1,0) {\textbf{Peak Enforcement}\\(44 Tech.\ Block.\ clauses)\\[-2pt]{\scriptsize Table~\ref{tab:qual_timeline_complete}}};
			\draw[thin] (4.1,0.2) -- (peak);

			\node[align=center, text width=3.0cm, above=1.8cm, font=\footnotesize] (equilib) at
			(5.9,0) {\textbf{New Equilibrium}\\(Blocking as Standard)\\[-2pt]{\scriptsize Table~\ref{tab:qual_timeline_complete}}};
			\draw[thin] (5.9,0.2) -- (equilib);

			% Events Bottom (VPN/Adaptive)
			\node[align=center, text width=3.0cm, below=1.8cm, font=\footnotesize] (obfus) at
			(0.5,0) {\textbf{Obfuscated Servers}\\(Chameleon/XOR)\\[-2pt]{\scriptsize\parencite{winter2013scramblesuit}}};
			\draw[thin] (0.5,-0.2) -- (obfus);

			\node[align=center, text width=3.0cm, below=2.8cm, font=\footnotesize] (wireguard) at (1.7,0) {\textbf{WireGuard \& NordLynx}\\(Speed + Stealth)\\[-2pt]{\scriptsize\parencite{donenfeld2017wireguard}}};
			\draw[thin] (1.7,-0.2) -- (wireguard);

			\node[align=center, text width=3.0cm, below=1.8cm, font=\footnotesize] (residential) at
			(4.1,0) {\textbf{Residential Proxies}\\(Undetectable IPs)\\[-2pt]{\scriptsize\parencite{mi2019residential}}};
			\draw[thin] (4.1,-0.2) -- (residential);

			\node[align=center, text width=3.0cm, below=2.8cm, font=\footnotesize] (dvpn) at (5.9,0) {\textbf{Decentralized VPNs}\\(Blockchain dVPNs)\\[-2pt]{\scriptsize\parencite{mysterium2020dvpn}}};
			\draw[thin] (5.9,-0.2) -- (dvpn);

		\end{tikzpicture}}
	\caption{The Adversarial Timeline: Coercive Barriers vs. Technical Circumvention (2020--2025). Events above the timeline reflect corporate enforcement measures documented in the \gls{tos} analysis; events below the timeline represent circumvention technologies.}
	\label{fig:timeline}
\end{figure}

%  Generated Video Group 
% Note: The following proportional (percentage-based) evolution figures are placed
% in the Discussion rather than Results because they support the interpretive
% analysis of strategic archetypes. Absolute-count charts appear in Appendix A.
\begin{figure}[ht]
	\centering
	\begin{minipage}{0.47\textwidth}
		\centering
		\begin{tikzpicture}
			\begin{axis}[
					width=\linewidth, height=5cm, xlabel={Year}, ylabel={Share (\%)},
					xmin=2020, xmax=2025, xtick={2020,2022,2024},
					xticklabel style={/pgf/number format/set thousands separator={}},
					grid=major,
					title={Netflix}, title style={font=\footnotesize\bfseries},
					ymin=0, ymax=100
				]
				\addplot[thick, color=tudablue, mark=square*] coordinates { (2020,61.3) (2021,44.2) (2022,43.6) (2023,44.1) (2024,57.1) (2025,57.7)  };
				\addplot[thick, color=tudagreen, mark=triangle*] coordinates { (2020,22.6) (2021,25.6) (2022,28.2) (2023,35.3) (2024,39.3) (2025,33.8)  };
				\addplot[thick, color=tudared, mark=x] coordinates { (2020,0.0) (2021,2.3) (2022,0.0) (2023,0.0) (2024,0.0) (2025,0.0)  };
				\addplot[thick, color=cyan, mark=diamond*] coordinates { (2020,12.9) (2021,27.9) (2022,28.2) (2023,14.7) (2024,3.6) (2025,4.2)  };
				\addplot[thick, color=orange, mark=pentagon*] coordinates { (2020,3.2) (2021,0.0) (2022,0.0) (2023,0.0) (2024,0.0) (2025,4.2)  };
			\end{axis}
		\end{tikzpicture}
	\end{minipage}
	\hfill
	\begin{minipage}{0.47\textwidth}
		\centering
		\begin{tikzpicture}
			\begin{axis}[
					width=\linewidth, height=5cm, xlabel={Year}, ylabel={Share (\%)},
					xmin=2020, xmax=2025, xtick={2020,2022,2024},
					xticklabel style={/pgf/number format/set thousands separator={}},
					grid=major,
					title={YouTube}, title style={font=\footnotesize\bfseries},
					ymin=0, ymax=100
				]
				\addplot[thick, color=tudablue, mark=square*] coordinates { (2020,31.8) (2021,25.0) (2022,29.8) (2023,23.0) (2024,20.7) (2025,32.6)  };
				\addplot[thick, color=tudagreen, mark=triangle*] coordinates { (2020,31.8) (2021,40.0) (2022,36.2) (2023,32.4) (2024,40.2) (2025,36.0)  };
				\addplot[thick, color=tudared, mark=x] coordinates { (2020,22.7) (2021,20.0) (2022,25.5) (2023,30.2) (2024,28.0) (2025,13.5)  };
				\addplot[thick, color=cyan, mark=diamond*] coordinates { (2020,0.0) (2021,0.0) (2022,0.0) (2023,0.7) (2024,0.0) (2025,1.1)  };
				\addplot[thick, color=orange, mark=pentagon*] coordinates { (2020,13.6) (2021,15.0) (2022,8.5) (2023,7.9) (2024,9.8) (2025,9.0)  };
			\end{axis}
		\end{tikzpicture}
	\end{minipage}
	\vspace{0.5cm}
	\begin{minipage}{0.47\textwidth}
		\centering
		\begin{tikzpicture}
			\begin{axis}[
					width=\linewidth, height=5cm, xlabel={Year}, ylabel={Share (\%)},
					xmin=2020, xmax=2025, xtick={2020,2022,2024},
					xticklabel style={/pgf/number format/set thousands separator={}},
					grid=major,
					title={Disney+}, title style={font=\footnotesize\bfseries},
					ymin=0, ymax=100
				]
				\addplot[thick, color=tudablue, mark=square*] coordinates { (2020,50.0) (2021,52.6) (2022,60.0) (2023,58.6) (2024,58.1) (2025,76.5)  };
				\addplot[thick, color=tudagreen, mark=triangle*] coordinates { (2020,22.2) (2021,26.3) (2022,5.0) (2023,31.0) (2024,17.6) (2025,0.0)  };
				\addplot[thick, color=tudared, mark=x] coordinates { (2020,0.0) (2021,0.0) (2022,0.0) (2023,0.0) (2024,0.0) (2025,0.0)  };
				\addplot[thick, color=cyan, mark=diamond*] coordinates { (2020,27.8) (2021,21.1) (2022,35.0) (2023,10.3) (2024,18.9) (2025,23.5)  };
				\addplot[thick, color=orange, mark=pentagon*] coordinates { (2020,0.0) (2021,0.0) (2022,0.0) (2023,0.0) (2024,5.4) (2025,0.0)  };
			\end{axis}
		\end{tikzpicture}
	\end{minipage}
	\hfill
	\begin{minipage}{0.47\textwidth}
		\centering
		\begin{tikzpicture}
			\begin{axis}[
					width=\linewidth, height=5cm, xlabel={Year}, ylabel={Share (\%)},
					xmin=2020, xmax=2025, xtick={2020,2022,2024},
					xticklabel style={/pgf/number format/set thousands separator={}},
					grid=major,
					title={Amazon}, title style={font=\footnotesize\bfseries},
					ymin=0, ymax=100
				]
				\addplot[thick, color=tudablue, mark=square*] coordinates { (2020,50.0) (2021,50.0) (2022,46.2) (2023,58.8) (2024,42.1) (2025,28.6)  };
				\addplot[thick, color=tudagreen, mark=triangle*] coordinates { (2020,33.3) (2021,50.0) (2022,53.8) (2023,41.2) (2024,31.6) (2025,39.3)  };
				\addplot[thick, color=tudared, mark=x] coordinates { (2020,0.0) (2021,0.0) (2022,0.0) (2023,0.0) (2024,5.3) (2025,0.0)  };
				\addplot[thick, color=cyan, mark=diamond*] coordinates { (2020,0.0) (2021,0.0) (2022,0.0) (2023,0.0) (2024,0.0) (2025,7.1)  };
				\addplot[thick, color=orange, mark=pentagon*] coordinates { (2020,16.7) (2021,0.0) (2022,0.0) (2023,0.0) (2024,0.0) (2025,7.1)  };
			\end{axis}
		\end{tikzpicture}
	\end{minipage}
	\vspace{0.2cm}
	% Shared legend for all subplots
	{\footnotesize
		\fbox{\begin{tabular}{@{}l@{\qquad}l@{\qquad}l@{}}
				\tikz{\draw[thick, tudablue] plot[mark=square*] coordinates {(0,0) (0.4,0)};} Content Licensing        &
				\tikz{\draw[thick, tudagreen] plot[mark=triangle*] coordinates {(0,0) (0.4,0)};} Regulatory Compliance &
				\tikz{\draw[thick, tudared] plot[mark=x] coordinates {(0,0) (0.4,0)};} Technical Blocking                \\
				\tikz{\draw[thick, cyan] plot[mark=diamond*] coordinates {(0,0) (0.4,0)};} Price Discrimination        &
				\tikz{\draw[thick, orange] plot[mark=pentagon*] coordinates {(0,0) (0.4,0)};} Legal Threat             & \\
			\end{tabular}}
	}
	\caption{Strategic Evolution: Video Streaming Leaders (2020--2025). Grouping Netflix, YouTube, Disney+, and Amazon Prime.}
	\label{fig:evol_video_main}
\end{figure}

%  Music Streaming 
\begin{figure}[ht]
	\centering
	\begin{minipage}{0.47\textwidth}
		\centering
		\begin{tikzpicture}
			\begin{axis}[
					width=\linewidth, height=5cm, xlabel={Year}, ylabel={Share (\%)},
					xmin=2020, xmax=2025, xtick={2020,2022,2024},
					xticklabel style={/pgf/number format/set thousands separator={}},
					grid=major,
					title={Spotify}, title style={font=\footnotesize\bfseries},
					ymin=0, ymax=100
				]
				\addplot[thick, color=tudablue, mark=square*] coordinates { (2020,65.1) (2021,55.6) (2022,71.8) (2023,53.3) (2024,22.6) (2025,33.3)  };
				\addplot[thick, color=tudagreen, mark=triangle*] coordinates { (2020,23.3) (2021,29.6) (2022,23.1) (2023,42.2) (2024,54.8) (2025,66.7)  };
				\addplot[thick, color=tudared, mark=x] coordinates { (2020,0.0) (2021,0.0) (2022,0.0) (2023,0.0) (2024,0.0) (2025,0.0)  };
				\addplot[thick, color=cyan, mark=diamond*] coordinates { (2020,11.6) (2021,14.8) (2022,5.1) (2023,4.4) (2024,22.6) (2025,0.0)  };
				\addplot[thick, color=orange, mark=pentagon*] coordinates { (2020,0.0) (2021,0.0) (2022,0.0) (2023,0.0) (2024,0.0) (2025,0.0)  };
			\end{axis}
		\end{tikzpicture}
	\end{minipage}
	\hfill
	\begin{minipage}{0.47\textwidth}
		\centering
		\begin{tikzpicture}
			\begin{axis}[
					width=\linewidth, height=5cm, xlabel={Year}, ylabel={Share (\%)},
					xmin=2020, xmax=2025, xtick={2020,2022,2024},
					xticklabel style={/pgf/number format/set thousands separator={}},
					grid=major,
					title={Apple Music}, title style={font=\footnotesize\bfseries},
					ymin=0, ymax=100
				]
				\addplot[thick, color=tudablue, mark=square*] coordinates { (2020,60.0) (2021,9.1) (2022,40.0) (2023,27.3) (2024,35.7) (2025,27.3)  };
				\addplot[thick, color=tudagreen, mark=triangle*] coordinates { (2020,40.0) (2021,72.7) (2022,35.0) (2023,18.2) (2024,35.7) (2025,27.3)  };
				\addplot[thick, color=tudared, mark=x] coordinates { (2020,0.0) (2021,0.0) (2022,0.0) (2023,9.1) (2024,0.0) (2025,9.1)  };
				\addplot[thick, color=cyan, mark=diamond*] coordinates { (2020,0.0) (2021,0.0) (2022,0.0) (2023,9.1) (2024,0.0) (2025,9.1)  };
				\addplot[thick, color=orange, mark=pentagon*] coordinates { (2020,0.0) (2021,0.0) (2022,5.0) (2023,27.3) (2024,7.1) (2025,18.2)  };
			\end{axis}
		\end{tikzpicture}
	\end{minipage}
	\vspace{0.2cm}
	% Shared legend for all subplots
	{\footnotesize
		\fbox{\begin{tabular}{@{}l@{\qquad}l@{\qquad}l@{}}
				\tikz{\draw[thick, tudablue] plot[mark=square*] coordinates {(0,0) (0.4,0)};} Content Licensing        &
				\tikz{\draw[thick, tudagreen] plot[mark=triangle*] coordinates {(0,0) (0.4,0)};} Regulatory Compliance &
				\tikz{\draw[thick, tudared] plot[mark=x] coordinates {(0,0) (0.4,0)};} Technical Blocking                \\
				\tikz{\draw[thick, cyan] plot[mark=diamond*] coordinates {(0,0) (0.4,0)};} Price Discrimination        &
				\tikz{\draw[thick, orange] plot[mark=pentagon*] coordinates {(0,0) (0.4,0)};} Legal Threat             & \\
			\end{tabular}}
	}
	\caption{Strategic Evolution: Music Streaming (2020--2025). Compared with the video streaming leaders in Figure~\ref{fig:evol_video_main}, music services show markedly lower Technical Blocking intensity. Spotify shows zero Technical Blocking clauses across the entire period, while Apple Music shows only intermittent spikes.}
	\label{fig:evol_music_vs_video}
\end{figure}

%  Software Utilities 
\begin{figure}[ht]
	\centering
	\begin{minipage}{0.47\textwidth}
		\centering
		\begin{tikzpicture}
			\begin{axis}[
					width=\linewidth, height=5cm, xlabel={Year}, ylabel={Share (\%)},
					xmin=2020, xmax=2025, xtick={2020,2022,2024},
					xticklabel style={/pgf/number format/set thousands separator={}},
					grid=major,
					title={Microsoft}, title style={font=\footnotesize\bfseries},
					ymin=0, ymax=100
				]
				\addplot[thick, color=tudablue, mark=square*] coordinates { (2020,6.2) (2021,4.3) (2022,6.7) (2023,3.2) (2024,0.0) (2025,9.5)  };
				\addplot[thick, color=tudagreen, mark=triangle*] coordinates { (2020,34.4) (2021,47.8) (2022,33.3) (2023,45.2) (2024,77.8) (2025,61.9)  };
				\addplot[thick, color=tudared, mark=x] coordinates { (2020,3.1) (2021,0.0) (2022,3.3) (2023,3.2) (2024,0.0) (2025,4.8)  };
				\addplot[thick, color=cyan, mark=diamond*] coordinates { (2020,0.0) (2021,0.0) (2022,0.0) (2023,0.0) (2024,0.0) (2025,0.0)  };
				\addplot[thick, color=orange, mark=pentagon*] coordinates { (2020,25.0) (2021,30.4) (2022,26.7) (2023,35.5) (2024,22.2) (2025,23.8)  };
			\end{axis}
		\end{tikzpicture}
	\end{minipage}
	\hfill
	\begin{minipage}{0.47\textwidth}
		\centering
		\begin{tikzpicture}
			\begin{axis}[
					width=\linewidth, height=5cm, xlabel={Year}, ylabel={Share (\%)},
					xmin=2020, xmax=2025, xtick={2020,2022,2024},
					xticklabel style={/pgf/number format/set thousands separator={}},
					grid=major,
					title={Adobe}, title style={font=\footnotesize\bfseries},
					ymin=0, ymax=100
				]
				\addplot[thick, color=tudablue, mark=square*] coordinates { (2020,0.0) (2021,0.0) (2022,0.0) (2023,0.0) (2024,0.0) (2025,28.6)  };
				\addplot[thick, color=tudagreen, mark=triangle*] coordinates { (2020,75.0) (2021,69.2) (2022,90.9) (2023,100.0) (2024,38.1) (2025,28.6)  };
				\addplot[thick, color=tudared, mark=x] coordinates { (2020,0.0) (2021,0.0) (2022,0.0) (2023,0.0) (2024,0.0) (2025,0.0)  };
				\addplot[thick, color=cyan, mark=diamond*] coordinates { (2020,12.5) (2021,15.4) (2022,0.0) (2023,0.0) (2024,52.4) (2025,28.6)  };
				\addplot[thick, color=orange, mark=pentagon*] coordinates { (2020,12.5) (2021,15.4) (2022,0.0) (2023,0.0) (2024,4.8) (2025,14.3)  };
			\end{axis}
		\end{tikzpicture}
	\end{minipage}
	\vspace{0.2cm}
	% Shared legend for all subplots
	{\footnotesize
		\fbox{\begin{tabular}{@{}l@{\qquad}l@{\qquad}l@{}}
				\tikz{\draw[thick, tudablue] plot[mark=square*] coordinates {(0,0) (0.4,0)};} Content Licensing        &
				\tikz{\draw[thick, tudagreen] plot[mark=triangle*] coordinates {(0,0) (0.4,0)};} Regulatory Compliance &
				\tikz{\draw[thick, tudared] plot[mark=x] coordinates {(0,0) (0.4,0)};} Technical Blocking                \\
				\tikz{\draw[thick, cyan] plot[mark=diamond*] coordinates {(0,0) (0.4,0)};} Price Discrimination        &
				\tikz{\draw[thick, orange] plot[mark=pentagon*] coordinates {(0,0) (0.4,0)};} Legal Threat             & \\
			\end{tabular}}
	}
	\caption{Strategic Evolution: Software Utilities (2020--2025). Microsoft and Adobe show distinct patterns from streaming services, relying more on Regulatory Compliance and Legal Threats than Technical Blocking.}
	\label{fig:evol_software_main}
\end{figure}




This timeline illustrates the dynamic relationship between corporate enforcement and consumer
behavior. Early enforcement measures by streaming providers led \gls{vpn} providers to develop ``Obfuscated
Servers'' (techniques similar to those used to circumvent China's Great Firewall
\parencite{ensafi2015examining}), which subsequently led to more sophisticated blocking techniques (circa
2021). This pattern suggests an ongoing adaptation process on both sides.

The adversarial cycle can be decomposed into four distinct phases, each characterized by a
shift in the balance of power between platforms and circumvention tools:

\textbf{Phase 1: The Wild West (Pre-2020).}
In the earliest phase of widespread streaming, geographic enforcement was minimal and inconsistent.
Many platforms relied solely on IP geolocation databases that were easily circumvented
by any commercial \gls{vpn}. During this period, geo-arbitrage was effectively risk-free,
as platforms lacked both the technical infrastructure and the institutional motivation
to actively police subscriber locations. Netflix's initial global expansion in
January 2016 \parencite{netflix2016global} and the subsequent splitting of the
streaming market into competing services created the conditions for widespread arbitrage
by multiplying both the number of services and the aggregate savings available to circumventing users.

\textbf{Phase 2: The First Wall (2020--2021).}
The COVID-19 pandemic dramatically accelerated \gls{vpn} adoption as remote work
normalized the use of virtual networking tools \parencite{atlasvpn2021covid}.
Simultaneously, streaming subscriptions surged globally, everyone was stuck at home, increasing the economic
stakes of geo-arbitrage for platforms. In response, services began deploying dedicated IP
blacklisting infrastructure, partnering with specialized geolocation companies to identify
and block known \gls{vpn} exit nodes \parencite{maxmind2021geolocation}.
This period corresponds to the ``Residential IP Crackdown'' noted in our timeline
(Figure \ref{fig:timeline}), where platforms shifted from blocking known datacenter
IP ranges to attempting to identify residential proxy services.

\textbf{Phase 3: The Escalation (2022--2023).}
Our data shows a dramatic turning point in 2022--2023, driven primarily by YouTube's
enforcement surge (Table \ref{tab:timeline_count}: from 16 incidents in 2022 to 53 in 2023).
This phase is characterized by the deployment of multi-signal detection:
platforms moved beyond simple IP-based blocking to incorporate billing address
verification, payment method geolocation, and behavioral analytics.
The shift from passive to active enforcement is clearly visible in the \gls{tos}
language, which evolved from permissive constructions (``We may restrict access...'')
to prohibitive directives (``You must not use any technology to disguise your location'').
This linguistic shift, documented in our tone analysis (Section \ref{sec:llm_methodology}),
reflects a strategic decision to make the consequences of circumvention explicit and unmistakable.

\textbf{Phase 4: The New Equilibrium (2024--Present).} The most recent phase shows a partial
stabilization, with Technical Blocking counts declining from their 2023 peak
(44 to 24 in 2024, Table \ref{tab:qual_timeline_complete}). This does not necessarily
indicate relaxed enforcement. Rather, it likely reflects the integration of blocking measures
into standard operational procedures. Once a firm has established its detection infrastructure
and updated its \gls{tos}, the need for \textit{new} enforcement clauses diminishes.
The blocking technology becomes part of the background architecture rather than a point
of active policy development. Meanwhile, \gls{vpn} providers have responded with increasingly
sophisticated obfuscation technologies, including traffic shaping that mimics regular
HTTPS patterns, distributed residential exit nodes, and protocol-level stealth
features \parencite{winter2013scramblesuit}. Beyond simple IP masking, a new
generation of \gls{vpn} products now aims to provide consumers with complete
anonymous digital identities, bundling region-specific payment methods,
temporary email addresses, and browser fingerprint randomization into
integrated toolkits. Decentralized \gls{vpn} architectures
\parencite{mysterium2020dvpn} further complicate platform enforcement by
routing traffic through peer-to-peer networks of residential nodes rather
than identifiable commercial server infrastructure.
The result is an uneasy equilibrium
where both sides have invested heavily in competing technologies, and the marginal
cost of gaining a further advantage is rising for both platforms and circumvention providers.
\section{The Secret Tech Race}
\label{sec:secret_tech_race}
Returning to the history of piracy, the modern world is defined by a growing tech race. Our
research shows that companies are getting much better at finding where users really are, even
with a high-quality \gls{vpn}. They no longer just block IP addresses. They use advanced tools like
Deep Packet Inspection (\gls{dpi}), AI to read traffic fingerprints, and browser fingerprinting
techniques \parencite{laperdrix2020browser}. Machine learning approaches to \gls{vpn}
traffic classification have become increasingly sophisticated \parencite{dainotti2012issues},
enabling platforms to detect circumvention attempts even when traditional IP-based blocking fails.

Based on the available technical literature and our document analysis, the technical
escalation can be conceptualized as a multi-layered detection stack. At the first layer,
platforms maintain databases of known \gls{vpn} and proxy IP addresses
\parencite{maxmind2021geolocation}, blocking connections from these ranges.
This is the simplest form of detection but also the easiest to circumvent, as \gls{vpn}
providers regularly rotate their IP addresses. At the second layer, platforms analyze
traffic patterns, as \gls{vpn} connections often exhibit distinctive packet sizes,
timing patterns, and protocol signatures that differ from regular browsing traffic,
even when encrypted. At the third layer, platforms employ browser and device fingerprinting
to detect inconsistencies: a user whose browser reports a German timezone and language
setting but connects from a Turkish IP address raises a detection flag. At the most
advanced layer, some platforms reportedly use behavioral analytics, tracking usage
patterns (login times, content preferences, payment history) to identify accounts
that are likely being used from a different region than their registered location.

On the circumvention side, \gls{vpn} providers have responded with equally
sophisticated countermeasures. Obfuscated server protocols
(such as NordVPN's ``NordLynx'' or Surfshark's ``Camouflage Mode'') disguise \gls{vpn}
traffic to look like regular HTTPS browsing. Residential proxy services route traffic
through real home internet connections, making it indistinguishable from genuine
local traffic. Some services even offer dedicated IP addresses that are not shared
with other users, reducing the likelihood of blacklisting.

However, a key challenge for research is that these tools are kept secret. Unlike the open
legal fights over Napster, modern geo-blocking happens in private. Companies keep their
methods secret so that \gls{vpn} providers can't adapt. This means that while we can see an
increase in blocking, it is hard to say exactly how the technology works.
This opacity creates a significant methodological challenge: the most important enforcement
mechanisms are precisely those that are least observable to researchers.
Our \gls{tos} analysis captures the \textit{stated} enforcement approach,
but the \textit{actual} technical capabilities may be significantly more advanced than
what firms publicly disclose.

\section{Aggregate Enforcement Trends}
\label{sec:appendix_detailed_analysis}

This section provides a deeper look at the data, analyzing the specific numbers
that drive the strategic trends.

To understand the macro-trends, we analyze how the total volume of policy
text has shifted. Our analysis shows the overwhelming dominance of General Terms
(legal boilerplate), which consistently make up over 90\% of all sentences
(see Table \ref{tab:qual_timeline_complete}). However, when we filter for
\textit{strategic} categories (Figure \ref{fig:dist_strategic}), a clear
pattern emerges: \textbf{Content Licensing} and \textbf{Regulatory Compliance} are
the baseline ``noise'' of digital business, while \textbf{Technical Blocking}
and \textbf{Account Action} show specific, event-driven spikes.

%  Distribution Strategic Categories Only 
\begin{figure}[ht]
	\centering
	\begin{tikzpicture}
		\begin{axis}[
				ybar stacked,
				width=0.85\textwidth,
				height=8cm,
				xlabel={Year},
				ylabel={Share (\%)},
				xmin=2019.5, xmax=2025.5,
				ymin=0, ymax=100,
				xtick={2020,2021,2022,2023,2024,2025},
				xticklabel style={/pgf/number format/set thousands separator={}},
				legend style={at={(0.5,-0.2)}, anchor=north, legend columns=3, font=\footnotesize},
				bar width=15pt,
				area style,
				enlarge x limits=0.08,
			]
			\addplot[ybar stacked, fill=tudablue, draw=black!30] coordinates { (2020,41.0) (2021,36.3) (2022,39.7) (2023,32.4) (2024,31.8) (2025,37.8) };
			\addlegendentry{Content Licensing}
			\addplot[ybar stacked, fill=tudaorange, draw=black!30] coordinates { (2020,31.8) (2021,38.3) (2022,32.9) (2023,37.1) (2024,37.7) (2025,34.0) };
			\addlegendentry{Regulatory Compliance}
			\addplot[ybar stacked, fill=tudared, draw=black!30] coordinates { (2020,3.5) (2021,2.6) (2022,5.9) (2023,14.0) (2024,7.9) (2025,5.7) };
			\addlegendentry{Technical Blocking}
			\addplot[ybar stacked, fill=tudagreen, draw=black!30] coordinates { (2020,9.2) (2021,13.5) (2022,9.1) (2023,3.8) (2024,10.9) (2025,5.3) };
			\addlegendentry{Price Discrimination}
			\addplot[ybar stacked, fill=tudagray, draw=black!30] coordinates { (2020,14.5) (2021,9.3) (2022,12.3) (2023,12.7) (2024,11.6) (2025,17.2) };
			\addlegendentry{Enforcement Actions}
		\end{axis}
	\end{tikzpicture}
	\caption{Distribution of Policy Text Categories Over Time (Strategic Categories Only).}
	\label{fig:dist_strategic}
\end{figure}

As illustrated in the preceding analysis, the strategic categories are overwhelmingly
overshadowed by general terms, which constitute over 90\% of all sentences
(Table \ref{tab:qual_timeline_complete}). Consequently, the strategic
trends are best visualized by focusing exclusively on non-boilerplate
categories, as shown in Figure \ref{fig:dist_strategic}.

\subsection{Service-Level Enforcement Trends}
At the service level (Table~\ref{tab:timeline_count}), two patterns stand out.
Microsoft maintains a consistent low-level ``background radiation'' of legal checks
without sudden spikes, consistent with the Enterprise Fortress archetype.
YouTube's enforcement surge peaked in 2023 before declining in 2024--2025,
suggesting a ``cooldown'' period where blocking rules became standard practice
(see Section~\ref{sec:service_deep_dive} for the full year-by-year breakdown).




\section{Implications for Policy and Practice}
\label{sec:implications}

The findings have important implications for several groups
in the digital economy: platform operators, regulators, \gls{vpn} providers, and consumers.

\subsection{Implications for Platform Operators}
The strategic archetypes identified in this thesis suggest that firms should align
their enforcement strategy with their underlying value delivery architecture rather
than pursuing a ``one-size-fits-all'' approach to geo-arbitrage.

\textbf{Content streaming platforms}
(Netflix, YouTube, Disney+) should consider whether the escalating costs of
the ``Content Fortress'' approach (both in terms of technology investment and user friction)
justify the revenue protection achieved. Netflix's strategic pivot toward global
original content production represents a potentially more sustainable long-term
response than ever-more-sophisticated blocking technology. By owning content globally,
platforms eliminate the territorial licensing constraints that necessitate
geo-blocking in the first place, effectively making the ``fortress'' unnecessary.

\textbf{Software providers} (Adobe, Microsoft) already benefit from identity-based
enforcement that is more robust than network-based blocking. These firms should
focus on strengthening subscription management and license validation rather
than investing in \gls{vpn} detection technologies that are more relevant to
streaming platforms. The migration toward cloud-dependent features (e.g., Adobe Firefly)
naturally strengthens enforcement without requiring explicit anti-circumvention measures.

\textbf{Ecosystem platforms} (Apple, Amazon) should recognize that their
multi-service bundle strategy provides the most durable form of protection against
geo-arbitrage. The switching costs embedded in ecosystem dependencies create natural
barriers that are far more difficult to circumvent than any technical blocking system.
Strategic investments in deepening ecosystem integration
(e.g., linking subscription pricing to hardware ownership or loyalty programs)
may be more effective than direct enforcement.

\subsection{Implications for Regulators}
The Affordability Paradox documented in this study highlights a tension that
regulators should address. On one hand, geographic price discrimination enables
digital inclusion by making services affordable in lower-income markets.
On the other hand, the enforcement measures required to maintain this discrimination
impose costs on legitimate users (travelers, expatriates, students abroad) and
raise questions about digital sovereignty and consumer rights.

The \gls{eu}'s Geo-Blocking Regulation (2018/302) represents one approach,
prohibiting unjustified geo-blocking within the single market \parencite{eu2018geoblocking}.
As \textcite{trimble2024geoblocking} comprehensively documents, this regulation
exempts audiovisual content, precisely the category where our data shows the highest
enforcement intensity. Regulators should consider whether this exemption remains
justified as content licensing models evolve and as the social costs of aggressive
enforcement (false positives, privacy intrusions through behavioral analytics,
restricted portability) become more apparent. \textcite{zuiderveen2017online} argue
that geographic price discrimination in digital markets raises significant data
privacy concerns under \gls{eu} law, as the detection technologies required to enforce
regional pricing, including IP geolocation, behavioral tracking, and payment verification,
necessarily involve processing personal data.

Our finding that firms with more harmonized global pricing require
less enforcement spending suggests a potential ``regulatory nudge'' approach \parencite{thaler2003libertarian}:
rather than prohibiting geo-blocking directly, regulators could incentivize price
harmonization through transparency requirements
(mandating disclosure of regional price differences) or through competition
policy measures that address the territorial licensing practices underlying
content division \parencite{gomez2016geo}.

\subsection{Implications for Consumers}
The Three-Level Mechanism analysis (Section \ref{sec:theory_piracy}) reveals
that consumers often underestimate the broader consequences of geo-arbitrage.
While individual acts of price hopping may seem victimless, our analysis
suggests three systemic risks.

First, if geo-arbitrage becomes widespread enough to significantly erode revenue
from low-price markets, firms may respond by raising prices in those markets,
harming the local consumers the pricing was designed to serve. There is already
evidence of this dynamic: YouTube's enforcement surge in 2023, documented in
our \gls{tos} analysis (Table~\ref{tab:qual_timeline_complete}), coincided with
reported price increases in markets such as Turkey and Argentina. While our
\gls{dspi} captures only a December 2025 snapshot and does not include longitudinal
price data, this overlap in timing suggests
that platforms may raise prices in ``exploited'' markets partly in response to
arbitrage losses. This creates a harmful outcome where the consumers
least able to afford price increases bear the costs of circumvention behavior by
wealthier users in other countries.

Second, the enforcement technologies deployed to combat arbitrage
(behavioral analytics, device fingerprinting, payment verification) create
surveillance infrastructure that extends well beyond its original purpose,
with implications for digital privacy that affect all users, not just those
engaging in circumvention. As \textcite{zuiderveen2017online} argue, the data
collection required to enforce geographic pricing, including IP tracking,
payment method analysis, and behavioral profiling, raises significant concerns
under data protection frameworks such as the \gls{eu}'s General Data Protection
Regulation (\gls{gdpr}). The irony is that privacy-enhancing technologies (\glspl{vpn})
used for arbitrage are met with privacy-reducing technologies
(fingerprinting, behavioral analytics) deployed for enforcement,
creating an escalating cycle that erodes digital privacy for all users.

Third, consumers who engage in geo-arbitrage face tangible individual risks that
are often downplayed in online communities. Account termination without refund,
loss of accumulated content libraries (e.g., purchased films, saved playlists, cloud-stored files),
and potential legal liability in jurisdictions where \gls{tos} violations carry
contractual penalties represent real costs that may only materialize months
or years after the initial arbitrage decision. While online communities such as
Reddit do share enforcement experiences alongside success stories, the information
environment remains skewed: successful circumvention is an ongoing, visible state
(the subscription continues to work), whereas enforcement actions are discrete,
often embarrassing events that users may attribute to unrelated causes or simply
move on from without detailed reporting. This structural asymmetry, compounded
by the self-selecting nature of active forum participants, means that the
\textit{perceived} probability of detection in these communities likely
understates the actual enforcement rate.


\section{Limits of the study}
While this study gives us a new way to look at geo-arbitrage, there are some limits to our
findings.

\subsection{Sample Size and Generalizability}
The correlation analysis relies on a strategic sample of $N=10$ digital service
providers. While these firms represent a significant majority of the consumer subscription
market by revenue and user base, the sample is small in statistical terms. With 10 observations,
correlation coefficients are inherently unstable: a single outlier can substantially
shift the Pearson $r$ value, as demonstrated by the dramatic reversal from $r = -0.55$ (full sample)
to $r = +0.35$ (excluding \gls{vpn} providers), not merely a reduction in size but a
complete sign change driven by the removal of just two data points. Consequently, the findings
should be interpreted as ``exploratory'' evidence of strategic archetypes rather than a
definitive ``law'' of digital economics.

The country sample ($N=11$) presents a similar limitation. While the purposive
stratified sampling strategy ensures representation across income levels and
geographic regions, the exclusion of major markets such as India
(the world's second-largest internet market by user base), China (where \gls{vpn}
usage is subject to state-level regulation), and several African economies limits the
generalizability of the \gls{dspi} findings. These markets may exhibit pricing patterns
and enforcement dynamics that differ substantially from those observed in our sample,
particularly given the unique regulatory environments in China and the rapidly growing
digital economies of Sub-Saharan Africa.

Future research could expand this dataset to include mid-tier \gls{saas} providers,
gaming platforms (e.g., PlayStation Plus, Nintendo Online), and emerging AI subscription
services (e.g., ChatGPT Plus, Midjourney) to test whether the ``Enterprise Fortress''
model holds for smaller B2B firms and whether new service categories develop their
own distinct strategic archetypes.

\subsection{The ``Average Citizen'' Bias (Socioeconomic Mismatch)}
Our ``Affordability'' metric calculates cost as a percentage of the \textit{Median National Monthly Wage}.
However, in emerging markets like Turkey or Argentina, the target demographic
for services like Netflix or Adobe is likely the urban upper-middle class, whose income is
significantly higher than the national average.

For instance, World Bank data indicate that in Turkey, the top 20\% of
earners capture nearly 48\% of total disposable income \parencite{worldbank2023wdi}.
Similarly, in Argentina, the top 10\%
of earners have average monthly incomes exceeding \$496 USD, well above the national median
\parencite{indec2023argentina}.
This implies that global digital services are aggressively priced to target this specific
``Global Elite'' segment. As \textcite{kastanakis2012between} argue, in markets with high income
inequality, luxury consumption (including premium digital subscriptions) serves as a critical
status signal for the upper class. This ``Elite Targeting'' pricing strategy explains why firms
tolerate some level of piracy from the lower 80\%—they were never the primary customer segment
to begin with.

The Digital Services Price Index (\gls{dspi}) represents a snapshot of pricing data from
December 2025. In hyperinflationary economies such as Argentina and Turkey (both classified as
hyperinflationary under IAS 29 \parencite{ifrs2023ias29}), local currency prices are adjusted
frequently in response to
macroeconomic conditions. A ``cheap'' arbitrage opportunity identified in this thesis could be
eliminated by a price adjustment or currency devaluation. The ``Arbitrage Window'' is
therefore dynamic rather than static.

\subsection{AI Classification Reliability}
The use of \glspl{llm} (Gemini 3 Flash) creates a potential ``Black Box'' validity
risk. To reduce this, we used the model's self-reported confidence scores as a filtering
mechanism. The final dataset achieved an average confidence score of \textbf{0.947}, with
\textbf{80.5\%} of classifications exceeding a confidence threshold of 0.9. This high degree
of certainty suggests that the detection of ``coercive'' vs. ``general'' language is reliable,
even without human verification for every datapoint.

\subsection{Document-Based Analysis Limitations}
A fundamental limitation of this study is its reliance on publicly available documents
(\gls{tos}, annual reports, earnings calls) as proxies for corporate strategy.
The assumption that stated policy reflects actual enforcement behavior is not perfect.
Companies might use strict rules for legal safety but only enforce some of them, or
they may employ undisclosed detection technologies that extend beyond the scope of their stated policies.
The ``Secret Tech Race'' discussed in Section
\ref{sec:secret_tech_race} highlights this gap: the most effective enforcement
technologies are precisely those that firms have the strongest motivation to conceal.
Future research could complement document-based analysis with experimental methods
(e.g., systematically testing \gls{vpn} access across services and regions) to
measure actual enforcement rates rather than stated policies.

The forward-fill strategy (Chapter~\ref{chap:methodology}),
which added 700 carry-forward observations (2.8\% of the time-series dataset),
assumes that policies remain unchanged between documented updates.
While this is reasonable because \gls{tos} typically remain legally active until
explicitly revised, it may overcount the persistence of specific clauses
if internal policies change without a formal document update.

\subsection{Methodological Reflections}
Beyond the substantive findings, this study shows important methodological
lessons for researchers working with corporate legal documents.
As documented in Section~\ref{sec:llm_methodology}, the Gemini 3 Flash model
dramatically outperformed the traditional BERT-based Zero-Shot classifier
(26.8\% agreement, Cohen's Kappa $= 0.032$), confirming that advanced
generative models are essential for legal text analysis where the task
requires distinguishing between the mere \textit{mention} of a concept
and its active \textit{regulation}, a distinction that keyword-sensitive
\gls{nli} models cannot reliably make. This is consistent with
\textcite{gilardi2023chatgpt}, who demonstrated that large language
models outperform traditional approaches in complex text annotation tasks.

This finding suggests that future research on complex legal texts should
employ advanced generative models rather than traditional \gls{nli} approaches,
particularly when the classification task requires distinguishing between
the mere \textit{mention} of a concept and its active \textit{regulation}.
However, the use of \gls{llm}-based classification introduces its own
reproducibility challenges. Model outputs can vary between API versions,
and the specific weights and training data of Gemini 3 Flash may not be available
for future replication. To mitigate this, we have documented the exact
system prompt (Listing \ref{lst:system_prompt}), batch processing parameters,
and confidence thresholds used in our pipeline, enabling methodological
reproducibility even if exact numerical reproduction is not guaranteed.
The complete source code, data processing pipelines, and the master dataset
are publicly available in the accompanying GitHub repository~\parencite{weckbach2025repo}.
We recommend that future studies employing \gls{llm}-based classification
adopt a similar transparency protocol: publishing the complete system prompt,
documenting the model version and API parameters, and reporting confidence
score distributions alongside classification results.

\chapter{Conclusion}
\label{chap:conclusion}

This chapter sums up the findings, discusses what the study adds to the literature,
and notes its limitations. Finally, we look at how new regulations and technologies
may keep changing the digital services market.

\section{Summary of Key Findings}
This thesis examined the strategic conflict between firms' price discrimination practices and
consumer-driven geo-arbitrage in digital subscription markets. The study combined a novel
quantitative index (the \gls{dspi}) with \gls{llm}-based qualitative analysis of corporate documents
to provide the first complete mapping of both the economic forces driving geo-arbitrage
and the strategic responses firms deploy to counter it.

Regarding \textbf{\gls{rq}1 (Economic Incentive)}, the \gls{dspi} reveals substantial and systematic
price differences across markets. The analysis of up to 11 digital services across
11 countries demonstrates that subscriptions in low-income markets can cost up to 90\%
less than identical services in high-income markets
(e.g., YouTube Premium in Pakistan at 12\% of the US price, Spotify in Pakistan at 10\% of the US price).
These differences are not arbitrary. They follow a broadly predictable pattern aligned with national
income levels, consistent with third-degree price discrimination theory \parencite{varian1989price}.
However, the complementary \gls{ptw} analysis reveals an \textbf{Affordability Paradox}:
these nominally ``cheap'' prices are often significantly more expensive for local
consumers in real terms (Section~\ref{sec:dspi_results}), fundamentally challenging
the popular narrative that low-income markets receive ``bargain'' prices and reframing
the arbitrage motivation as an exploitation of welfare-enhancing pricing structures.

Regarding \textbf{\gls{rq}2 (Strategic Response)}, enforcement strategies are primarily determined by
\textbf{Business Model Architecture} rather than by the extent of price differences.
This is the study's most surprising finding. Content streaming firms enforce strict geographic
blocking due to licensing requirements, with YouTube exhibiting the highest Fortress Index (33.68\%)
among content providers. Software firms like Microsoft rely on identity verification and legal threats
rather than network-based blocking, reflecting the ``Enterprise Fortress'' archetype.
Ecosystem platforms (Apple, Amazon) leverage multi-service bundle dependencies as implicit enforcement,
making arbitrage impractical without relocating one's entire digital identity.
\gls{vpn} providers, by their nature, implement no geographic restrictions, instead framing
circumvention as a privacy right.

Beyond the two research questions, the \textbf{enforcement evolution} analysis reveals a clear escalation in
enforcement intensity, peaking in 2023 before stabilizing
(Table~\ref{tab:qual_timeline_complete}). This surge, driven primarily by
YouTube's enforcement campaign, coincides with post-pandemic increases in \gls{vpn}
adoption and geo-arbitrage awareness. The subsequent stabilization suggests that
firms have moved from active enforcement expansion to maintenance of established
blocking infrastructure. However, the constant presence of circumvention
discussions in online communities suggests these barriers increase friction rather
than eliminating arbitrage entirely, consistent with the game-theoretic prediction
(Equation~\ref{eq:enforcement_optimization}) that perfect enforcement is neither
achievable nor economically optimal.

Taken together, these findings paint a picture of a digital economy in transition.
The current regime of territorial price discrimination, inherited from the pre-digital
era of physical media distribution and broadcast licensing \parencite{lobato2019geoblocking}, is under growing pressure
from the inherently borderless nature of digital delivery. Firms' responses to this
pressure, whether through technological fortress-building, ecosystem lock-in, or
adaptive global pricing, reveal fundamentally different theories about how digital
markets should function. The long-term trajectory, informed by the piracy
parallel developed in Chapter~\ref{chap:theory}, suggests that adaptive strategies
(reducing the arbitrage motivation through pricing innovation)
may prove more sustainable than coercive strategies
(raising the cost of circumvention through technology).

\section{Contribution to Research}
This study adds both methods and theory to digital economics.

\subsection{Methodological Contributions}

\subsubsection{The Digital Subscription Price Index}
The \gls{dspi} was created as a new way to measure
price discrimination for digital services. Unlike existing price indices
(such as the Big Mac Index or the iPod Index), which measure a single product,
the \gls{dspi} captures a basket of digital services across multiple categories,
providing a fuller picture of digital pricing patterns.
The complementary \gls{ptw} ratio adds an affordability dimension that purely
nominal indices lack. Together, these metrics offer a replicable framework
that future researchers can apply to track pricing changes over time or extend
to additional service categories.

\subsubsection{AI-Assisted Legal Analysis}
This study demonstrates that generative
\gls{llm} models can successfully analyze thousands of legal sentences with
high reliability (average confidence 0.947), significantly outperforming
traditional \gls{nli} models for complex legal text classification.
The pipeline architecture (combining structured system prompts, batch processing,
and forward-fill gap handling) provides a replicable template for future research
on corporate policy documents at scale. This contribution is particularly
relevant given the rapid growth of \gls{tos} documents, which makes
traditional manual coding increasingly impractical.

\subsection{Theoretical Contributions}

\subsubsection{Extension of Business Model Innovation Theory}
The study extends \gls{bmi} theory by showing that consumer bypassing behaviors
work as a disruptive force similar to technological innovation, forcing firms
to fundamentally change their value capture mechanisms. This finding adds a
new category of external pressure to the \gls{bmi} literature, which has traditionally
focused on competitor-driven or technology-driven disruption \parencite{foss2017fifteen}.
Consumer-driven disruption through circumvention technology represents a distinct
mechanism where the disruption originates from the demand side rather than the supply side.

\subsubsection{Strategic Archetypes}
The identification of four distinct types
(Content Fortress, Ecosystem Fortress, Enterprise Fortress, Utility Paradox) provides a
framework for understanding how different business models respond to the same external threat.
The key finding that business model architecture is a stronger predictor of enforcement
strategy than price gap size challenges the intuitive assumption that firms
with the largest price differences would also have the strongest enforcement.
Instead, the delivery mechanism (streaming vs.\ download vs.\ ecosystem) determines
the available enforcement tools.

\subsubsection{The Piracy-Arbitrage Parallel}
By drawing explicit parallels between
the digital piracy wave of the 2000s and modern geo-arbitrage, this thesis
contributes to a longer-term understanding of how digital disruption forces
business model adaptation. The historical comparison suggests that the current
enforcement-heavy phase may eventually give way to adaptive strategies.
This contribution is distinct from the existing piracy literature
\parencite{oberholzer2007effect, peukert2017piracy} because it identifies geo-arbitrage
as a \textit{qualitatively different} form of consumer circumvention
(one involving payment rather than theft, location rather than access)
that nonetheless triggers similar strategic responses from firms. This suggests
that the dynamics of disruption and adaptation in digital markets may follow
recurring patterns regardless of the specific circumvention mechanism,
a finding with implications for anticipating firm responses to future
forms of consumer-driven boundary-crossing.

\subsubsection{The Affordability Paradox}
The documentation of the Affordability Paradox,
that nominally ``cheap'' prices in low-income markets are often more expensive
in real terms (relative to local wages) than high-income market prices,
challenges a widespread assumption in both the popular press and the academic
literature that price discrimination in digital markets primarily benefits consumers
in developing economies. By demonstrating that a Netflix subscription in
Argentina consumes over twelve times the share of monthly income compared
to the USA, this finding reframes the equity debate around geo-arbitrage
and provides empirical grounding for policy discussions about digital inclusion and fair pricing.

\section{The Future}
Geo-blocking is expected to undergo significant changes in the coming years.

\subsection{Regulatory Evolution}
As rules like the \gls{eu}'s Digital Single Market evolve
(building on regulations such as the \gls{eu} Geo-Blocking Regulation \parencite{eu2018geoblocking}),
geo-blocking practices may change significantly. The current regulation exempts
audiovisual content from its prohibition on unjustified geo-blocking,
but this exemption is subject to periodic review, and the European Commission
has signaled interest in expanding the regulation's scope \parencite{trimble2024geoblocking}.
New laws might move companies away from blocking and toward keeping prices the same everywhere.
\textcite{gomez2016geo} provide evidence that geo-blocking within the \gls{eu} reduces
consumer welfare and digital market integration. Beyond the \gls{eu}, emerging digital
trade agreements (such as the Digital Economy Partnership Agreement between
Singapore, Chile, and New Zealand) may establish new norms around cross-border
digital access that put additional pressure on territorial pricing models.
The tension between national tax regimes (which incentivize location-based pricing)
and consumer expectations of borderless digital access will likely intensify as
digital services become a larger share of household expenditure globally.

\subsection{Technological Evolution}
The arms race between blocking technology and bypassing tools shows no signs of slowing.
Advanced techniques such as residential IP proxies, browser fingerprinting, and machine
learning-based detection create a technically complex environment that demands continuous
investment from both sides. On the enforcement side, the integration of artificial intelligence
into detection systems, including behavioral pattern recognition and cross-signal analysis,
promises to make \gls{vpn} detection more accurate but also more invasive. On the circumvention
side, emerging technologies such as decentralized \gls{vpn} networks (\glspl{dvpn}) built on blockchain
infrastructure \parencite{mysterium2020dvpn} could make traditional IP-based blocking fundamentally
ineffective by distributing exit nodes across millions of individual devices rather than identifiable
data centers. Researchers
should investigate whether the rising cost of this arms race eventually makes geographic price
discrimination economically unsustainable, particularly for smaller platforms that cannot afford
enterprise-grade detection infrastructure.

\subsection{Market Convergence}
Ultimately, the cycle of blocking and bypassing may end not through better technology, but
through market forces that promote global price convergence
\parencite{goldfarb2019digital}. As digital services become increasingly
standardized and competition intensifies, companies may find that the costs of
maintaining differentiated regional pricing, including enforcement technology,
legal compliance, user friction, and reputational damage, outweigh the incremental
revenue gained from price discrimination \parencite{varian1989price}. The music
industry's trajectory offers a precedent: Spotify's pricing has converged
significantly across markets compared to the highly fragmented pricing of the
iTunes store era \parencite{aguiar2018effect, waldfogel2010music}, suggesting
that mature digital markets trend toward simpler, more uniform pricing structures.
In such a scenario, geo-arbitrage would become obsolete, not because users are
blocked, but because there is no longer a meaningful price difference to exploit.
However, this convergence scenario assumes that income inequality between countries
will narrow or that firms will accept lower margins in wealthy markets, assumptions
that current macroeconomic trends do not uniformly support
\parencite{worldbank2023wdi, rogoff1996ppp}.

\subsection{Future Research}
Future studies could build on this work in several directions. First, expanding the service
sample beyond 11 services to include mid-tier \gls{saas} providers, gaming platforms
(e.g., PlayStation Plus and Steam),
and emerging categories (e.g., AI subscription services like ChatGPT Plus or Midjourney)
would test whether the strategic archetypes identified here generalize across the broader
digital economy. Second, longitudinal tracking of the \gls{dspi} over multiple years would
capture how companies adjust prices in response to macroeconomic events
(currency devaluations, inflation spikes) and regulatory changes, transforming
the current snapshot into a dynamic monitoring tool. Third, demand-side research
using surveys or interviews with consumers who engage in geo-arbitrage would complement
this supply-side analysis, particularly regarding ethical perceptions, risk assessment,
and the social dynamics of circumvention communities. Fourth, comparative regulatory
analysis across jurisdictions (e.g., the \gls{eu}'s Geo-Blocking Regulation versus
the absence of equivalent legislation in North America or Asia) would help isolate the
causal effect of legal frameworks on firm enforcement strategies. Finally, the methodological
approach of \gls{llm}-based policy analysis could be extended to other domains where firms manage
digital boundaries, such as data localization requirements, content moderation policies,
or algorithmic pricing disclosures.



\appendix

\chapter{Detailed Service Evolution}
\label{app:service_evolution}

This appendix presents evolution data for \gls{vpn} providers
(Figure~\ref{fig:app_vpn_evolution}), as the detailed evolution charts
for other services have been integrated into Chapter \ref{chap:results} and Chapter \ref{chap:discussion}.

\begin{figure}[ht]
	\centering
	\begin{minipage}{0.47\textwidth}
		\centering
		\begin{tikzpicture}
			\begin{axis}[
					width=\linewidth,
					height=5cm,
					xlabel={Year},
					ylabel={Sentences},
					xmin=2020, xmax=2025,
					xtick={2020,2022,2024},
					xticklabel style={/pgf/number format/set thousands separator={}},
					grid=major,
					legend pos=north west,
					legend style={font=\tiny},
					title={Generic VPN (Aggregated)},
					title style={font=\footnotesize\bfseries}
				]
				\addplot[thick, color=red, mark=pentagon*] coordinates { (2020,0) (2021,0) (2022,0) (2023,0) (2024,3) (2025,7) };
				\addlegendentry{Legal Thr.}
				\addplot[thick, color=brown!60!black, mark=otimes] coordinates { (2020,0) (2021,0) (2022,0) (2023,0) (2024,0) (2025,1) };
				\addlegendentry{Priv./Sec.}
			\end{axis}
		\end{tikzpicture}
	\end{minipage}
	\caption{Evolution of \gls{vpn} Provider Enforcement. Note: Specific \gls{vpn} graphs have been excluded for brevity as they show minimal variation.}
	\label{fig:app_vpn_evolution}
\end{figure}

\begin{figure}[ht]
	\centering
	\begin{minipage}{0.47\textwidth}
		\centering
		\begin{tikzpicture}
			\begin{axis}[
					width=\linewidth, height=5cm, xlabel={Year}, ylabel={Sentences},
					xmin=2020, xmax=2025, xtick={2020,2022,2024},
					xticklabel style={/pgf/number format/set thousands separator={}},
					grid=major,
					title={Netflix}, title style={font=\footnotesize\bfseries},
					ymin=0
				]
				\addplot[thick, color=tudablue, mark=square*] coordinates { (2020,19) (2021,19) (2022,17) (2023,15) (2024,16) (2025,41) };
				\addplot[thick, color=tudagreen, mark=triangle*] coordinates { (2020,7) (2021,11) (2022,11) (2023,12) (2024,11) (2025,24) };
				\addplot[thick, color=tudared, mark=x] coordinates { (2020,0) (2021,1) (2022,0) (2023,0) (2024,0) (2025,0) };
				\addplot[thick, color=cyan, mark=diamond*] coordinates { (2020,4) (2021,12) (2022,11) (2023,5) (2024,1) (2025,3) };
				\addplot[thick, color=orange, mark=pentagon*] coordinates { (2020,1) (2021,0) (2022,0) (2023,0) (2024,0) (2025,3) };
			\end{axis}
		\end{tikzpicture}
	\end{minipage}
	\hfill
	\begin{minipage}{0.47\textwidth}
		\centering
		\begin{tikzpicture}
			\begin{axis}[
					width=\linewidth, height=5cm, xlabel={Year}, ylabel={Sentences},
					xmin=2020, xmax=2025, xtick={2020,2022,2024},
					xticklabel style={/pgf/number format/set thousands separator={}},
					grid=major,
					title={YouTube Premium}, title style={font=\footnotesize\bfseries},
					ymin=0
				]
				\addplot[thick, color=tudablue, mark=square*] coordinates { (2020,7) (2021,5) (2022,14) (2023,32) (2024,17) (2025,29) };
				\addplot[thick, color=tudagreen, mark=triangle*] coordinates { (2020,7) (2021,8) (2022,17) (2023,45) (2024,33) (2025,32) };
				\addplot[thick, color=tudared, mark=x] coordinates { (2020,5) (2021,4) (2022,12) (2023,42) (2024,23) (2025,12) };
				\addplot[thick, color=cyan, mark=diamond*] coordinates { (2020,0) (2021,0) (2022,0) (2023,1) (2024,0) (2025,1) };
				\addplot[thick, color=orange, mark=pentagon*] coordinates { (2020,3) (2021,3) (2022,4) (2023,11) (2024,8) (2025,8) };
			\end{axis}
		\end{tikzpicture}
	\end{minipage}

	\vspace{0.5cm}

	\begin{minipage}{0.47\textwidth}
		\centering
		\begin{tikzpicture}
			\begin{axis}[
					width=\linewidth, height=5cm, xlabel={Year}, ylabel={Sentences},
					xmin=2020, xmax=2025, xtick={2020,2022,2024},
					xticklabel style={/pgf/number format/set thousands separator={}},
					grid=major,
					title={Disney+}, title style={font=\footnotesize\bfseries},
					ymin=0
				]
				\addplot[thick, color=tudablue, mark=square*] coordinates { (2020,9) (2021,20) (2022,12) (2023,17) (2024,43) (2025,13) };
				\addplot[thick, color=tudagreen, mark=triangle*] coordinates { (2020,4) (2021,10) (2022,1) (2023,9) (2024,13) (2025,0) };
				\addplot[thick, color=tudared, mark=x] coordinates { (2020,0) (2021,0) (2022,0) (2023,0) (2024,0) (2025,0) };
				\addplot[thick, color=cyan, mark=diamond*] coordinates { (2020,5) (2021,8) (2022,7) (2023,3) (2024,14) (2025,4) };
				\addplot[thick, color=orange, mark=pentagon*] coordinates { (2020,0) (2021,0) (2022,0) (2023,0) (2024,4) (2025,0) };
			\end{axis}
		\end{tikzpicture}
	\end{minipage}
	\hfill
	\begin{minipage}{0.47\textwidth}
		\centering
		\begin{tikzpicture}
			\begin{axis}[
					width=\linewidth, height=5cm, xlabel={Year}, ylabel={Sentences},
					xmin=2020, xmax=2025, xtick={2020,2022,2024},
					xticklabel style={/pgf/number format/set thousands separator={}},
					grid=major,
					title={Spotify}, title style={font=\footnotesize\bfseries},
					ymin=0
				]
				\addplot[thick, color=tudablue, mark=square*] coordinates { (2020,28) (2021,15) (2022,28) (2023,24) (2024,7) (2025,1) };
				\addplot[thick, color=tudagreen, mark=triangle*] coordinates { (2020,10) (2021,8) (2022,9) (2023,19) (2024,17) (2025,2) };
				\addplot[thick, color=tudared, mark=x] coordinates { (2020,0) (2021,0) (2022,0) (2023,0) (2024,0) (2025,0) };
				\addplot[thick, color=cyan, mark=diamond*] coordinates { (2020,5) (2021,4) (2022,2) (2023,2) (2024,7) (2025,0) };
				\addplot[thick, color=orange, mark=pentagon*] coordinates { (2020,0) (2021,0) (2022,0) (2023,0) (2024,0) (2025,0) };
			\end{axis}
		\end{tikzpicture}
	\end{minipage}
	\vspace{0.2cm}
	\par\centering
	{\footnotesize
		\fbox{\begin{tabular}{@{}l@{\qquad}l@{\qquad}l@{}}
				\tikz{\draw[thick, tudablue] plot[mark=square*] coordinates {(0,0) (0.4,0)};} Content Licensing        &
				\tikz{\draw[thick, tudagreen] plot[mark=triangle*] coordinates {(0,0) (0.4,0)};} Regulatory Compliance &
				\tikz{\draw[thick, tudared] plot[mark=x] coordinates {(0,0) (0.4,0)};} Technical Blocking                \\
				\tikz{\draw[thick, cyan] plot[mark=diamond*] coordinates {(0,0) (0.4,0)};} Price Discrimination        &
				\tikz{\draw[thick, orange] plot[mark=pentagon*] coordinates {(0,0) (0.4,0)};} Legal Threat             & \\
			\end{tabular}}
	}
	\caption{Absolute Sentence Counts: Video and Music Streaming (2020--2025). Unlike the proportional charts in Chapter~\ref{chap:discussion}, this figure shows raw sentence counts to illustrate the volume of enforcement language.}
	\label{fig:evol_streaming}
\end{figure}

\begin{figure}[ht]
	\centering
	\begin{minipage}{0.47\textwidth}
		\centering
		\begin{tikzpicture}
			\begin{axis}[
					width=\linewidth, height=5cm, xlabel={Year}, ylabel={Sentences},
					xmin=2020, xmax=2025, xtick={2020,2022,2024},
					xticklabel style={/pgf/number format/set thousands separator={}},
					grid=major,
					title={Microsoft 365}, title style={font=\footnotesize\bfseries},
					ymin=0
				]
				\addplot[thick, color=tudablue, mark=square*] coordinates { (2020,2) (2021,1) (2022,2) (2023,1) (2024,0) (2025,2) };
				\addplot[thick, color=tudagreen, mark=triangle*] coordinates { (2020,11) (2021,11) (2022,10) (2023,14) (2024,21) (2025,13) };
				\addplot[thick, color=tudared, mark=x] coordinates { (2020,1) (2021,0) (2022,1) (2023,1) (2024,0) (2025,1) };
				\addplot[thick, color=cyan, mark=diamond*] coordinates { (2020,0) (2021,0) (2022,0) (2023,0) (2024,0) (2025,0) };
				\addplot[thick, color=orange, mark=pentagon*] coordinates { (2020,8) (2021,7) (2022,8) (2023,11) (2024,6) (2025,5) };
			\end{axis}
		\end{tikzpicture}
	\end{minipage}
	\hfill
	\begin{minipage}{0.47\textwidth}
		\centering
		\begin{tikzpicture}
			\begin{axis}[
					width=\linewidth, height=5cm, xlabel={Year}, ylabel={Sentences},
					xmin=2020, xmax=2025, xtick={2020,2022,2024},
					xticklabel style={/pgf/number format/set thousands separator={}},
					grid=major,
					title={Adobe Creative Cloud}, title style={font=\footnotesize\bfseries},
					ymin=0
				]
				\addplot[thick, color=tudablue, mark=square*] coordinates { (2020,0) (2021,0) (2022,0) (2023,0) (2024,0) (2025,2) };
				\addplot[thick, color=tudagreen, mark=triangle*] coordinates { (2020,12) (2021,9) (2022,10) (2023,9) (2024,8) (2025,2) };
				\addplot[thick, color=tudared, mark=x] coordinates { (2020,0) (2021,0) (2022,0) (2023,0) (2024,0) (2025,0) };
				\addplot[thick, color=cyan, mark=diamond*] coordinates { (2020,2) (2021,2) (2022,0) (2023,0) (2024,11) (2025,2) };
				\addplot[thick, color=orange, mark=pentagon*] coordinates { (2020,2) (2021,2) (2022,0) (2023,0) (2024,1) (2025,1) };
			\end{axis}
		\end{tikzpicture}
	\end{minipage}
	\vspace{0.2cm}
	\par\centering
	{\footnotesize
		\fbox{\begin{tabular}{@{}l@{\qquad}l@{\qquad}l@{}}
				\tikz{\draw[thick, tudablue] plot[mark=square*] coordinates {(0,0) (0.4,0)};} Content Licensing        &
				\tikz{\draw[thick, tudagreen] plot[mark=triangle*] coordinates {(0,0) (0.4,0)};} Regulatory Compliance &
				\tikz{\draw[thick, tudared] plot[mark=x] coordinates {(0,0) (0.4,0)};} Technical Blocking                \\
				\tikz{\draw[thick, cyan] plot[mark=diamond*] coordinates {(0,0) (0.4,0)};} Price Discrimination        &
				\tikz{\draw[thick, orange] plot[mark=pentagon*] coordinates {(0,0) (0.4,0)};} Legal Threat             & \\
			\end{tabular}}
	}
	\caption{Absolute Sentence Counts: Software Utilities (2020--2025).}
	\label{fig:evol_software_vpn}
\end{figure}


\chapter{Quantitative Reference Data}
\label{app:reference_data}

Table~\ref{tab:wage_reference} documents the median monthly wage figures and
exchange rates used to compute the \gls{dspi} and \gls{ptw} metrics throughout
this thesis. All exchange rates reflect market rates recorded in December 2025
at the time of data collection. The \gls{dspi} is calculated purely as the ratio
of the USD-converted local price to the US baseline price
(\(\text{DSPI} = \text{Price}_{\text{USD}} \;/\; \text{US Baseline Price}\)),
without any wage adjustment. The \gls{ptw} ratio is calculated separately as
the subscription price expressed as a percentage of the median monthly wage
(\(\text{PTW} = \text{Price}_{\text{USD}} \;/\; \text{Monthly Wage}_{\text{USD}}\)).

\begin{table}[ht]
	\centering
	\caption{Median Monthly Wages and Exchange Rates Used for \gls{ptw} Calculations
		(December 2025). Countries are ordered by USD wage (descending).}
	\label{tab:wage_reference}
	\small
	\renewcommand{\arraystretch}{1.15}
	\resizebox{\textwidth}{!}{%
		\begin{tabular}{l l r r r l}
			\toprule
			\textbf{Country} & \textbf{Cur.} & \textbf{Monthly Wage} & \textbf{Rate to USD} & \textbf{Wage (USD)} & \textbf{Source}                                   \\
			\midrule
			Switzerland      & CHF           & 6{,}903               & 1.1300               & 7{,}800             & \parencite{tradingeconomics2023switzerland}       \\
			United States    & USD           & 6{,}600               & 1.0000               & 6{,}600             & \parencite{oecd2023wages}                         \\
			United Kingdom   & GBP           & 4{,}500               & 1.2700               & 5{,}715             & \parencite{oecd2023wages}                         \\
			Germany          & EUR           & 4{,}800               & 1.0900               & 5{,}232             & \parencite{oecd2023wages}                         \\
			Poland           & PLN           & 6{,}700               & 0.2500               & 1{,}675             & \parencite{gus2024poland}\textsuperscript{a}      \\
			Turkey           & TRY           & 23{,}789              & 0.0320               & 761                 & \parencite{tradingeconomics2023turkey}            \\
			Brazil           & BRL           & 3{,}000               & 0.2000               & 600                 & \parencite{oecd2023wages}                         \\
			Argentina        & ARS           & 456{,}813             & 0.0012               & 548                 & \parencite{indec2023argentina}                    \\
			Ukraine          & UAH           & 19{,}600              & 0.0260               & 510                 & \parencite{sssu2024ukraine}\textsuperscript{a}    \\
			Philippines      & PHP           & 20{,}583              & 0.0180               & 370                 & \parencite{psa2024philippines}\textsuperscript{a} \\
			Pakistan         & PKR           & 70{,}700              & 0.0036               & 255                 & \parencite{pbs2024pakistan}\textsuperscript{a}    \\
			\bottomrule
			\multicolumn{6}{l}{\footnotesize \textsuperscript{a} Median estimate derived from national statistical office data (2024).}
		\end{tabular}}
\end{table}


\backmatter
\addcontentsline{toc}{chapter}{Bibliography}
\printbibliography

% List of Figures and Tables 
\cleardoublepage
\phantomsection
\addcontentsline{toc}{chapter}{List of Figures}
\listoffigures

\cleardoublepage
\phantomsection
\addcontentsline{toc}{chapter}{List of Tables}
\listoftables

% Glossary and Acronyms 
\cleardoublepage
\phantomsection
\addcontentsline{toc}{chapter}{Glossary and Acronyms}
\chapter*{Glossary and Acronyms}
\printglossaries

\cleardoublepage
\phantomsection
\addcontentsline{toc}{chapter}{Declaration on the Use of AI-Based Tools}
\chapter*{Declaration on the Use of AI-Based Tools}
In accordance with the guidelines of TU Darmstadt on the use of generative AI in academic work,
the following AI-based tools were used during the preparation of this thesis.
All AI-generated outputs were critically reviewed, verified against primary sources,
and adapted by the author. No content was adopted without independent verification.
The intellectual contribution, argumentation, and all scientific conclusions
remain entirely the author's own work.

\begin{description}
	\item[Google Gemini 3 Pro] Used for ideation and brainstorming during the early conceptual
	      phase of the research design. Outputs served as discussion prompts and were not adopted verbatim.
	\item[Google Gemini 3 Flash] Used as the classification engine in the
	      \gls{llm}-based \gls{tos} analysis pipeline (see Section~\ref{sec:llm_methodology}).
	      The model processed approximately 25,000 sentences from corporate documents.
	      The system prompt, parameters, and validation procedure are fully documented in the methodology chapter.
	\item[Anthropic Claude (Opus 4.5 and Opus 4.6)] Used as a coding assistant for the development
	      of the Python-based data collection and analysis scripts (web scraping, \gls{dspi} calculation, visualization).
	      All code was reviewed, tested, and debugged by the author.
\end{description}

%\noindent No AI tool was used to generate, draft, or paraphrase the prose text of this thesis.
%All written content was authored independently.

\affidavit

\end{document}


