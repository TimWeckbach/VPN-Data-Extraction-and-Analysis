%%
%\documentclass{}[
	%english,
	%ruledheaders=section,           %Ebene bis zu der die Überschriften mit Linien abgetrennt werden, vgl. DEMO-TUDaPub
	%class=report ,                   % Basisdokumentenklasse. Wählt die Korrespondierende KOMA-Script Klasse
	%thesis={type=bachelor},         % Dokumententyp Thesis, für Dissertationen siehe die Demo-Datei DEMO-TUDaPhd
	%accentcolor=9c,% Auswahl der Akzentfarbe
	%custommargins=true,% Ränder werden mithilfe von typearea automatisch berechnet
	%marginpar=false,% Kopfzeile und Fußzeile erstrecken sich nicht über die Randnotizspalte
	%BCOR=5mm,%Bindekorrektur, falls notwendig
	%parskip=half-,%Absatzkennzeichnung durch Abstand vgl. KOMA-Script
	%fontsize=11pt,%Basisschriftgröße laut Corporate Design ist mit 9pt häufig zu klein
	%logofile={figures/tuda_logo.pdf},
    %pdfa=false
%]%{tudapub}

\documentclass[
	english,
	ruledheaders=section,
	class=report,
	thesis={type=master},
	accentcolor=9c,
	custommargins=geometry,%CHANGED FROM TRUE TO FALSE
	marginpar=false,
	parskip=half-,
	fontsize=12pt,
	logofile={figures/tuda_logo.pdf},
    pdfa=true
]{tudapub}

%\usepackage{showframe}
%\usepackage[reset, left=2.5cm, right=2.5cm, top=2.5cm, bottom=2.5cm, includefoot]{geometry}
%\geometry{a4paper, left=2.5cm, right=2.5cm, top=2.5cm, bottom=2.5cm, includefoot}
\geometry{reset, bottom=2.5cm, includefoot} 


\usepackage[main=english, ngerman]{babel}
\usepackage[autostyle]{csquotes}
\usepackage{microtype}

\usepackage{graphicx}
\usepackage{amsmath}
\usepackage{tabularx}
\usepackage{booktabs}
\usepackage{threeparttable}
\renewcommand{\TPTnoteSettings}{\fontsize{10}{12}\selectfont} %<-- Add this line
\usepackage{pifont}
\usepackage{tikz}
\usetikzlibrary{shapes,arrows,positioning,calc}

\usepackage{setspace}
\onehalfspacing

% Use KOMA-Script recommended method for section formatting
\RedeclareSectionCommand[
    beforeskip=0pt,
    afterskip=10pt
]{chapter}


% Ensure titlesec is not loaded (remove any \usepackage{titlesec} lines)
\usepackage{caption}
\DeclareCaptionFont{tenpoint}{\fontsize{10}{12}\selectfont}
\captionsetup{font=tenpoint}


%change the apa6 to apa when you want to use the APA 7th edition
\usepackage[backend=biber, style=apa, sorting=nyt]{biblatex}
\addbibresource{Bibliography.bib}

\usepackage[acronym, section]{glossaries}
%%\newacronym[plural=JOLs]{jol}{JOL}{Judgement of Learning}
%\newacronym[plural=JOLs, longplural={Judgements of Learning}]{jol}{JOL}{Judgement of Learning}
%\newacronym[plural=MJs]{mj}{MJ}{Memorability Judgement}
%\newacronym{ore}{ORE}{Other-Race Effect}
%%\newacronym{ffa}{FFA}{Fusiform Face Area}
%\newacronym{fmri}{fMRI}{Functional Magnetic Resonance Imaging}





\makeglossaries

% change count of figures and tables to not include chapter number
\usepackage{chngcntr}
\counterwithout{figure}{chapter}
\counterwithout{table}{chapter}

%\newglossaryentry{realschulabschluss}{
%    name={Realschulabschluss},
%    description={A German secondary school certificate, approximately equivalent to a high school diploma or GCSEs.}
%}



\title{Business Model Responses to Consumer Circumvention: Lessons from Piracy Applied to VPN-Enabled Geo-Arbitrage}
\author{Tim Weckbach}
\reviewer{William Schütte \and Prof. Dr. Alexander Kock}









\begin{document}

%\setlength{\textheight}{24cm}
%\setlength{\topmargin}{-1.2cm} % Muss oft angepasst werden
%\setlength{\headsep}{1cm}
%\setlength{\footskip}{1.5cm}
%\setlength{\voffset}{5cm}


\Metadata{
    \title{Business Model Responses to Consumer Circumvention: Lessons from Piracy Applied to VPN-Enabled Geo-Arbitrage}
    \author[Tim Weckbach]{Tim Weckbach}
    \keywords{Price Discrimination, Geo-Arbitrage, Business Model Innovation, VPN, Digital Services Price Index}
    \publisher{TU Darmstadt}
    \copyright{Tim Weckbach}
    \birthplace{Frankfurt am Main}
    \reviewer{William Schütte \and Prof. Dr. Alexander Kock}
}
\department{Business Informatics} 
\institute{Technology and Innovation Management}
\submissiondate{27.02.2026}
\maketitle

\begin{abstract}
This document outlines the structure for a thesis investigating business model responses to VPN-enabled geo-arbitrage. It begins by defining the core problem and research path. The study employs a mixed-methods approach to first quantify international price differentiation for digital services via a "Digital Services Price Index" (DSPI) and then qualitatively analyze the strategic responses of both digital service providers and VPN providers. The research draws parallels between modern geo-arbitrage and historical digital piracy to understand the pressures on existing business models and the drivers for innovation.

\end{abstract}

\noindent\textbf{Keywords:} Price Discrimination, Geo-Arbitrage, Business Model Innovation, VPN, Digital Services Price Index
\tableofcontents


\chapter{Introduction}
\label{chap:introduction}

\section{Background and Context}
The globalization of digital services has created a paradox in the modern digital economy. While the internet promises a borderless exchange of information, digital service providers (DSPs) such as Netflix, Spotify, and Steam enforce rigid digital borders to maximize profits through international price differentiation. This strategy, deeply rooted in economic theories of third-degree price discrimination, allows firms to charge widely varying prices for identical digital goods based on the purchasing power of the consumer's location.

However, this segmentation strategy faces a formidable disruptive force: the consumer. Equipped with increasingly accessible circumvention technologies like Virtual Private Networks (VPNs), consumers are engaging in "digital geo-arbitrage"—the practice of virtually relocating to a cheaper market to purchase services at a fraction of the domestic price. This phenomenon mirrors the disruption caused by digital piracy in the early 2000s, where technical barriers were circumvented to access content, fundamentally challenging the music and film industries' business models.

\section{Problem Statement}
The core problem addressed in this thesis is the strategic conflict between a firm's geographic market segmentation and the technical circumventability of these digital borders. 
Firms are currently trapped in a "cat-and-mouse" game:
\begin{enumerate}
    \item \textbf{Economic Necessity:} They must segment markets to remain affordable in low-income regions while maximizing revenue in high-income nations.
    \item \textbf{Technical Reality:} The same internet architecture that enables global delivery also enables global circumvention.
\end{enumerate}
This tension challenges the economic viability of established business models and creates pressure for Business Model Innovation (BMI). Firms must choose between "Coercive" strategies (blocking, banning, litigation) and "Adaptive" strategies (price harmonization, global portability).

\section{Research Questions (RQs)}
To analyze this strategic conflict, this thesis pursues the following research questions:

\begin{description}
    \item[RQ1 (The Economic Incentive):] To what extent does international price differentiation for digital services deviate from Purchasing Power Parity (PPP), creating a "super-normal" incentive for arbitrage? (Note: This question serves primarily to establish the research setting and economic motivation).
    \item[RQ2 (The Strategic Response):] \textbf{How do digital subscription providers modify their business model in response to regional pricing circumvention and how has the mix of coercive versus adaptive responses reflected in their corporate disclosures changed over time?}
\end{description}

\section{Structure of the Thesis}
The thesis is structured as follows: \textbf{Chapter \ref{chap:theory}} establishes the theoretical foundations, linking price discrimination theory with the behavioral mechanics of circumvention. \textbf{Chapter \ref{chap:methodology}} details the mixed-methods research design, including the novel "Digital Services Price Index" (DSPI) and the LLM-based classification pipeline. \textbf{Chapter \ref{chap:results}} presents the empirical findings. \textbf{Chapter \ref{chap:discussion}} interprets these results within the framework of Business Model Innovation, and \textbf{Chapter \ref{chap:conclusion}} summarizes the contributions and limitations of the study.


\chapter{Theoretical Foundations \& Literature Review}
\label{chap:theory}

This chapter establishes the theoretical foundations by connecting economic pricing theory with business strategy and consumer behavior research.

\section{Economic Foundations of International Price Setting}
\label{sec:theory_pricing}

To understand why consumers engage in geo-arbitrage, we must first establish why firms create the price disparities that make such arbitrage profitable. Geographic price differentiation is not arbitrary but follows well-established economic principles.

\subsection{Third-Degree Price Discrimination}
According to Varian (1989), third-degree price discrimination happens when a firm divides the market based on visible traits—in this case, geographic location—and charges different prices to each group. For digital goods, where the cost of copying is near zero ($MC \approx 0$) \parencite{shapiro1998information,amit2001value}, this strategy allows firms to get the most consumer surplus from both high-income (e.g., Switzerland) and low-income (e.g., Turkey) markets at the same time.

Two conditions must hold for successful price discrimination:
\begin{itemize}
    \item \textit{Condition 1: Market Segmentation.} The firm must be able to distinguish between consumer groups based on observable characteristics such as IP address or billing location.
    \item \textit{Condition 2: No Arbitrage.} The firm must be able to prevent the resale or transfer of the good between segments.
\end{itemize}
VPN-enabled geo-arbitrage directly undermines \textit{Condition 2}, effectively collapsing the distinct market segments into a single global market.

\subsection{Purchasing Power Parity (PPP) as a Benchmark}
The "Law of One Price" suggests that in an efficient market, identical goods should sell at the same price when shown in a common currency. However, differences from this law are common. \textcite{rogoff1996ppp} argues that for physical goods, shipping costs and trade barriers justify price differences. In the digital world, \textcite{clemons2002price} note that while transaction costs are much lower, price differences continue because firms can set up detailed customer segmentation.

We use Purchasing Power Parity (PPP) as a benchmark for "economically justified" pricing. If a Netflix subscription in Turkey is cheaper than in the US only because of currency value and local purchasing power, this fits standard economic theory. However, if the price difference is bigger than what PPP adjustments would predict, it creates a "super-normal" arbitrage incentive—a price gap that motivates bypassing beyond simple purchasing power factors. We measure this through the Digital Services Price Index (DSPI).

\section{Consumer Circumvention and the Piracy Parallel}
\label{sec:theory_piracy}

Consumer-driven arbitrage is not new \parencite{geda2023puzzle}. The digital "geo-arbitrage" pattern can be understood by looking at the history of digital piracy.

\subsection{The Piracy Analogue}
\textcite{oberholzer2007effect} showed that file-sharing forced the music industry to change its business model, eventually leading to legitimate digital distribution platforms like iTunes and Spotify. This historical example helps us understand VPN-based geo-arbitrage. Similarly, geo-arbitrage works as a market signal, showing a basic mismatch between rigid regional pricing structures and the borderless reality of the global internet.

The parallel is instructive: just as Napster and BitTorrent exposed the music industry's failure to meet consumer demand for convenient digital access, VPN-enabled price hopping exposes the sustainability challenges of global price discrimination in an interconnected digital economy. In both cases, technological innovation by consumers preceded strategic adaptation by firms.

However, a key distinction emerges between the two phenomena:
\begin{itemize}
    \item \textit{Access vs. Price:} This thesis proposes a key analytical distinction: traditional piracy was often driven by access barriers (``content not available in my region''), whereas geo-arbitrage appears primarily motivated by price differentials or region-specific licensing restrictions. Piracy involved no payment; geo-arbitrage involves payment, albeit at unintended price points.
    \item \textit{Legal Status:} Digital piracy clearly violates copyright law, while geo-arbitrage occupies a legal gray zone. Users are paying for legitimate subscriptions; they are simply misrepresenting their location to obtain more favorable pricing.
    \item \textit{Industry Response:} The music industry eventually responded to piracy through business model innovation (streaming subscriptions). Whether similar adaptive responses will emerge in the geo-arbitrage context remains an open question this thesis investigates.
\end{itemize}


\subsection{The Three-Level Mechanism of Circumvention}
Drawing from behavioral ethics literature and the work of \textcite{wang2014three} on digital piracy, the decision to engage in geo-arbitrage can be modeled as a three-level mechanism. This framework helps explain why otherwise law-abiding consumers engage in "digital smuggling":

\begin{enumerate}
    \item \textbf{Individual Level (Rational Choice / Personal Risk):} The consumer performs a cost-benefit analysis. The financial gain (e.g., a 70\% discount on Netflix Turkey) is weighed against the perceived probability of detection and the severity of punishment (e.g., account termination). When enforcement is perceived as inconsistent, the perceived risk may be low.
    \item \textbf{Inter-personal Level (Social Influence):} The behavior may be reinforced by online communities (e.g., Reddit, Discord). Observing others successfully using VPNs can lower the psychological barrier to entry, consistent with social influence research \parencite{kastanakis2012between}.
    \item \textbf{Societal Level (Moral Intensity):} The perception of the act is pivotal. Unlike shoplifting a physical good, digital arbitrage may be framed by users not as theft, but as a response to pricing perceived as unfair. This framing aligns with neutralization theory \parencite{mateus2018business}.
\end{enumerate}

\section{Strategic Management and Business Model Innovation}
\label{sec:theory_strategy}

Faced with this disruption, firms must adapt. We analyze their responses using Business Model Innovation (BMI). As defined by \textcite{wirtz2016business} and grouped by \textcite{foss2017fifteen}, BMI means rethinking the value offer and delivery methods in response to outside shocks.

\subsection{Dimensions of Business Model Innovation}
To rigorously analyze how firms adapt, we deconstruct their business models into three core dimensions, following the framework proposed by \textcite{teece2010business} and adapted for digital markets by \textcite{amit2012value}:

\begin{enumerate}
    \item \textbf{Value Proposition (What is offered):} The core product or service and the bundle of benefits it provides to the customer. In digital streaming, this is the content library and the convenience of "watch anywhere" access.
    \item \textbf{Value Delivery (How it is reached):} The channels and technical infrastructure used to deliver the value. This includes the streaming platform, the Content Delivery Network (CDN), and the user interface. Crucially, it also includes the \textit{geographic segmentation} logic that determines who can access what.
    \item \textbf{Value Capture (How money is made):} The revenue model and the mechanisms to sustain profitability. This encompasses the pricing strategy (e.g., price discrimination) and the enforcement mechanisms used to prevent revenue leakage (e.g., blocking arbitrage).
\end{enumerate}

VPN-enabled arbitrage fundamentally attacks the **Value Capture** dimension by breaking the link between location and price. It also exploits the **Value Delivery** infrastructure (the open internet).
Consequently, we hypothesize that firm responses will fall into two categories of innovation:
\begin{itemize}
    \item \textbf{Defensive Innovation (Value Capture Focus):} Reinforcing the barriers to protect the existing model (e.g., "Coercive" blocking).
    \item \textbf{Adaptive Innovation (Value Proposition Focus):} Changing the product offer to make arbitrage irrelevant (e.g., "Adaptive" global pricing or ecosystems).
\end{itemize}

\subsection{Theoretical Framework: Protection vs. Pricing}
The intersection of digital strategy and arbitrage has been extensively debated. \textcite{johnson2008reinventing} define the necessity of business model reinvention when facing disruptive shifts, while \textcite{granados2010electronic} illustrate how e-commerce inherently increases market efficiency by facilitating spatial arbitrage. However, \textcite{geda2023puzzle} note that this arbitrage often creates game-theoretic puzzles for firms, leading to complex responses such as those described by \textcite{mateus2018business} in the context of digital piracy. Furthermore, \textcite{beunza2004price} argue that price is ultimately a social construct, heavily influenced by the "material sociology" of the market—in this case, the VPN technology that alters the visibility of the consumer.

To categorize firm responses, we adopt the framework established by \textcite{sundararajan2004managing} on managing digital piracy, mapping it to our BMI dimensions:
\begin{itemize}
    \item \textbf{Protection (Coercive / Value Capture):} Increasing the technological or legal costs of circumvention. This attempts to *repair* the broken Value Capture mechanism.
    \item \textbf{Pricing (Adaptive / Value Proposition):} Adjusting the business model (pricing, versioning) to lower the economic incentive for arbitrage. This effectively *innovates* the Value Proposition to be less sensitive to location.
\end{itemize}
Firms face a fundamental trade-off: Is the cost of enforcing market segmentation (repairing Value Capture through blocking technology and legal resources) lower than the revenue lost to arbitrage?

\subsection{Platforms and Ecosystem Control}
Digital platforms operate within a fundamental tension between growth and control. To attract users and content creators, platforms must maintain a degree of openness that facilitates participation and innovation. However, to protect revenue streams and maintain quality, platforms must also exercise control over who accesses what content and at what price point.

VPN providers exploit this inherent tension. They leverage the platform's content (e.g., Netflix's streaming library) while bypassing its payment rules (regional pricing). This creates a technical and strategic cycle of countermeasures and counter-countermeasures:
\begin{itemize}
    \item \textbf{Coercive Strategies:} Legal threats embedded in Terms of Service, IP address blocking, payment verification requirements, and strict geographic checks on billing addresses.
    \item \textbf{Adaptive Strategies:} Standardizing global prices to remove the arbitrage incentive, creating ecosystem lock-in through hardware integration (e.g., Apple's approach), or developing content exclusive to specific regions rather than restricting access to a global catalog.
\end{itemize}

Critically, \textcite{parker2017innovation} demonstrate that platforms face an inherent tension between openness (which drives innovation and user growth) and control (which protects revenue and quality). Their framework suggests that the optimal balance point shifts depending on platform maturity and competitive dynamics. VPN arbitrage directly exploits this fundamental trade-off, forcing platforms to reassess where that balance lies.

The strategic implications are significant: platforms that choose aggressive blocking may sacrifice user experience and brand perception, while those that tolerate arbitrage may face revenue leakage. Neither approach is without cost, and the optimal strategy likely depends on the specific business model and competitive context of each platform.


\section{Research Gap}
\label{sec:theory_gap}

While price discrimination theory (Varian) and platform strategy (Eisenmann et al., 2011) have been extensively researched independently, there remains a notable gap in empirical work connecting the \textit{magnitude} of pricing incentives (as measured by indices like the DSPI) with the \textit{specific strategic responses} adopted by firms.

Existing literature exhibits three main limitations:

\begin{enumerate}
    \item \textbf{Theoretical Isolation:} Most studies focus either exclusively on the economics of pricing (e.g., optimal price discrimination strategies) or on the legal aspects of copyright enforcement and digital rights management, but rarely examine the strategic interaction between these domains as mediated by consumer-side technology such as VPNs.
    
    \item \textbf{Lack of Quantification:} While anecdotal evidence of geo-arbitrage is abundant in consumer forums and technology journalism, systematic quantification of the arbitrage incentive across services and regions is lacking. The Digital Services Price Index (DSPI) addresses this gap by providing a standardized measurement framework.
    
    \item \textbf{Limited Strategic Analysis:} Previous research on digital piracy has examined how firms respond to unauthorized copying, but the distinct characteristics of geo-arbitrage (payment rather than piracy, location rather than access) warrant specialized investigation. The coercive-adaptive dichotomy proposed in this thesis provides a framework for categorizing these responses.
\end{enumerate}

This thesis addresses these gaps through a mixed-methods approach that: (1) quantifies the arbitrage incentive via the DSPI; (2) systematically analyzes corporate disclosures to identify enforcement strategies; and (3) links pricing variance to enforcement intensity through correlation analysis. By bridging economic theory, consumer behavior research, and strategic management, this study contributes a holistic perspective on the geo-arbitrage phenomenon that has been absent from the literature.


\chapter{Research Methodology}
This chapter details how the research will be conducted.
\label{chap:methodology}

To measure the concepts of 'Coercion' and 'Technical Barriers', we used the 'Litigious' and 'Constraining' word lists from Loughran and McDonald (2011). These are the standard for finding legal risk in financial texts.

\section{Research Design}

This study adopts a sequential explanatory mixed-methods design, combining quantitative price analysis with qualitative text classification. The rationale for this dual approach is to first establish the \textit{magnitude} of the economic phenomenon (the arbitrage incentive) and then investigate the \textit{strategic responses} of the actors involved (firms and consumers). 

The quantitative phase (Phase 1) constructs the "Digital Services Price Index" (DSPI) to objectively measure the variance in global digital pricing. The qualitative phase (Phase 2) leverages a Large Language Model (LLM) pipeline to classify corporate disclosures and Terms of Service, mapping the strategic "frames" firms use to legitimize or combat this variance. This integration allows for a holistic understanding of the geo-arbitrage ecosystem.

\section{Phase 1: Quantitative Data Collection (for RQ1)}

\subsection{Data Collection}
To construct the DSPI, a representative basket of digital services was selected, covering three primary categories: Video on Demand (e.g., Netflix, Disney+), Music Streaming (e.g., Spotify, Apple Music), and Software/Gaming (e.g., Microsoft 365, Steam).

Price data was collected from a broad sample of over 30 countries to capture the full spectrum of purchasing power. From this dataset, a core sub-sample of key markets was selected for detailed "Affordability" analysis, including: Argentina, Australia, Brazil, Canada, France, Germany, India, Italy, Japan, Mexico, Netherlands, Poland, Spain, Switzerland, Turkey, Ukraine, United Kingdom, and the United States. Only countries with high-confidence official wage data were included.

Data collection was performed using a \textbf{Digital Audit} design, adapting the methodology established by \textcite{hannak2014measuring} for detecting online price discrimination. A virtual presence was established in each target country using a commercial VPN to simulate local access, a technique now standard in information systems research for "mystery shopping" in digital markets. For each service and country, the "Standard" monthly subscription price was recorded in the local currency. This approach mirrors the "Billion Prices Project" methodology \parencite{cavallo2017are}, which demonstrated the validity of using high-frequency online scraping to construct robust price indices (like our DSPI) that track real-time economic disparities better than traditional CPI baskets.

\subsection{Data Analysis}
The raw price data was processed in two stages. First, all local prices were converted to a common currency (USD) using market exchange rates (recorded in December 2024 and January 2025) to determine the "Nominal Price Inequality." Second, to assess "Real Affordability," these prices were calculated as a percentage of the \textit{Median National Monthly Wage} (sourced from OECD and World Bank data), providing a direct measure of the economic burden on the local consumer (`Price-to-Wage Ratio`). This approach replaces standard PPP adjustment to better reflect the specific affordability of subscription goods relative to disposable income.

It is important to note that a DSPI of 1.0 (Nominal Parity) does not imply equal affordability. Due to vast differences in median wages (e.g., Switzerland vs. India), a service priced identically in USD would be significantly more expensive for the Indian consumer in real terms (requiring a larger percentage of their income). Thus, the arbitrage incentive persists even at nominal parity if the local price is structured to be affordable for the local median earner.

The DSPI was calculated as the ratio of the local price to the US baseline price. A DSPI of 1.0 indicates price parity; a DSPI < 1.0 indicates a cheaper market (potential arbitrage source), and a DSPI > 1.0 indicates a more expensive market. Statistical variance analysis was performed to identify which service categories exhibit the highest degree of price discrimination.

\section{Phase 2: Qualitative Data Collection \& Analysis (for RQ2)}

\subsection{Coding Procedure}
The analysis follows a systematic coding approach inspired by the \textbf{Gioia Methodology} \parencite{gioia2013seeking}. This involves structuring data into 1st-order concepts (raw terms found in text), 2nd-order themes (theoretical categories like "Technical Blocking"), and aggregate dimensions (Strategic Responses).
While initially conceptualized for manual coding \parencite{duriau2007content}, this hierarchical structure provided the logic for the automated classification pipeline described below.

\section{Automated Text Classification}
\label{sec:llm_methodology}

To address the limitations of traditional Natural Language Inference (NLI) models in capturing the nuanced legal and technical language of Terms of Service (ToS), this study implemented an advanced classification pipeline leveraging state-of-the-art Large Language Models (LLMs). Specifically, the pipeline was upgraded from a BERT-based architecture (DeBERTa-v3-large) to the \textit{Gemini 3 Flash} model, accessed via the Google Generative AI API.

\subsection{Model Selection and Rationale}
The selection of \textit{Gemini 3 Flash} was driven by the need for deeper reasoning capabilities and context awareness. Unlike NLI models, which classify based on entailment probabilities between a premise and a hypothesis, generative LLMs can interpret complex sentence structures and ambiguous legal standard terms (``General Terms'') versus specific geo-arbitrage restrictions. 

Key advantages observed during the model transition included:
\begin{itemize}
    \item \textbf{Contextual Understanding}: The ability to distinguish between benign references to ``account suspension'' (e.g., for fraud) and strategic ``Legal Threats'' tailored to prevent cross-border usage.
    \item \textbf{Zero-Shot Performance}: The model demonstrated high accuracy without extensive fine-tuning, utilizing a robust system prompt to align with the theoretical categories defined in Section 2.
    \item \textbf{Efficiency}: The ``Flash'' architecture provided a high throughput, enabling the processing of the entire dataset (approx. 25,000 sentences) within a reasonable timeframe.
\end{itemize}

\subsection{Operationalization of Constructs (The Coding Scheme)}
Based on the theoretical framework, the following coding scheme was enforced via the LLM system prompt. This scheme maps the abstract concept of "Strategic Response" into measurable data points.

\subsubsection{Strategic Frames}
The model was tasked to identify the underlying justification provided by the firm:
\begin{description}
    \item[Frame: Legal Compliance] Justifying geo-blocking as a non-negotiable legal or contractual necessity (e.g., "Due to licensing agreements...").
    \item[Frame: Security Risks] (Service Provider Frame) Arguments that VPNs/Proxies are unsafe, malicious, or compromise user data.
    \item[Frame: Privacy/Security] (VPN Provider Frame) Arguments focusing on encryption, anonymity, and protection from surveillance.
\end{description}

\subsubsection{Firm Actions}
The model categorized specific enforcement clauses into:
\begin{description}
    \item[Action: Technical Blocking] Active technological measures to detect or block the specific use of VPNs/Proxies (e.g., "We use geo-blocking technology", "Error 403").
    \item[Action: Legal Threat] Explicit threats of account termination, suspension, or legal action specifically for using circumvention tools.
    \item[Action: Account Action] General punitive measures against accounts (termination, suspension) for comprehensive violations.
    \item[Action: Price Discrimination] Explicit differences in pricing based on region, currency, or purchasing power.
    \item[Action: Legitimate Portability] Rules allowing temporary access while traveling (e.g., EU Portability Regulation).
\end{description}

\subsection{Pipeline Architecture and Implementation}
The reclassification process was automated using a customized Python script.

\subsubsection{System Prompt Engineering}
To ensure deterministic and theoretically grounded outputs, the system prompt was engineered with strict constraints. The exact prompt structure is provided below:

\begin{figure}[ht]
\begin{verbatim}
SYSTEM_PROMPT = """You are a scientific classifier for a Thesis on 'Digital Geo-Arbitrage'.
Classify a list of sentences into the provided categories. 

CATEGORIES:
1. Technical Blocking: Measures/Technologies used to detect or block...
2. Legal Threat: Explicit threats of account termination...
3. Security Risk: (Service Provider Frame) Arguments that VPNs are unsafe...
4. Privacy/Security: (VPN Provider Frame) Arguments focusing on encryption...
... [Full List of 10 Categories] ...

INSTRUCTIONS:
- Analyze sentences independently.
- Return a JSON array of objects...
- Format: [ { "category": "Category", "confidence": 0.9 }, ... ]
"""
\end{verbatim}
\caption{System Prompt used for Gemini 3 Flash Classification}
\label{fig:system_prompt}
\end{figure}

\subsubsection{Batch Processing and Error Handling}
To optimize for the API's rate limits and ensure data integrity, the pipeline utilized a batch processing approach. Sentences were grouped into batches of 25 and processed in a single API call. This method significantly reduced network overhead and total processing time.
A robust error-handling mechanism was implemented to manage API timeouts or rate limits (HTTP 429). The script included a ``circuit breaker'' to halt execution upon repeated failures and a resume function to continue processing from the last saved state.

\subsection{Methodological Validation: Gemini vs. Zero-Shot BERT}
To validate the choice of the Gemini 3 Flash model, a comparative analysis was conducted against a traditional Zero-Shot classification approach using a BERT-based model. The results demonstrated a massive divergence between the two models, reinforcing the necessity of using a modern LLM with large context windows for this specific task, consistent with recent findings on LLM performance in text annotation \parencite{gilardi2023chatgpt}.

\subsubsection{Agreement Analysis}
The comparison revealed an exceedingly poor agreement rate of \textbf{26.8\% (Accuracy)} between the two models. The Cohen's Kappa score was \textbf{0.032}, suggesting that the agreement was effectively equivalent to random chance. This discrepancy indicated a fundamental difference in how each model interpreted the classification tasks.

\subsubsection{The Core Conflict: Sensitivity vs. Context}
The analysis highlighted two distinct behaviors:
\begin{enumerate}
    \item \textbf{Gemini Performance:} The Gemini model correctly identified that approximately \textbf{91\%} of the dataset consisted of legal boilerplate, categorized as "General Terms." It successfully distinguished specific enforcement clauses from general legal language.
    \item \textbf{BERT Performance:} The BERT model exhibited "Over-Sensitivity," frequently assigning specific strategic tags based on the presence of individual keywords rather than semantic context.
\end{enumerate}

Specific examples of BERT's misclassification included:
\begin{itemize}
    \item \textbf{Legitimate Portability:} BERT flagged 7,853 sentences as "Legitimate Portability" that were merely "General Terms."
    \item \textbf{Account Action:} BERT flagged 6,134 "General Terms" sentences as "Account Action."
\end{itemize}

\textit{Interpretation:} BERT operates on keyword associations; for example, flagging a sentence like "You must have an account" as an "Account Action." In contrast, Gemini utilizes its reasoning capabilities to understand that the mere mention of an "account" is standard boilerplate ("General Terms") and reserves the "Account Action" tag for sentences explicitly regulating banning or suspension.

\subsubsection{Conclusion on Model Selection}
The validation proves that Zero-Shot BERT is insufficient for complex legal text analysis without extensive fine-tuning. It lacks the nuance required to distinguish between the mere mention of a topic (e.g., "portability") and the active regulation of it. Gemini, leveraging its massive context window and advanced reasoning capabilities, performs significantly better at filtering out noise and providing accurate stratifications. Consequently, Gemini 3 Flash was selected as the sole model for the final analysis.


\subsection{Standard Qualitative Coding}
In addition to the automated pipeline, manual qualitative coding is applied to a sub-sample to capture themes that escape rigid categorization, such as the specific "tone" of VPN provider marketing (e.g., empowering vs. technical).

\section{Data Analysis Procedures}
\label{sec:analysis_procedures}

The final analytical step involved synthesizing the quantitative and qualitative data streams. 

\subsection{Statistical Analysis of the DSPI}
The pricing data was analyzed using Python. We calculated descriptive statistics (mean, median, standard deviation) for the DSPI. We also generated correlation matrices to find the link between a country's income and the subscription price, testing if price discrimination depends strictly on national wealth.

\subsection{Interpretation of Qualitative Classifications}
For the qualitative data, the JSON outputs from the Gemini 3 Flash pipeline were parsed and aggregated. The frequency of each "Strategic Frame" (e.g., \textit{Legal Compliance} vs. \textit{User Freedom}) and "Firm Action" (e.g., \textit{Technical Blocking}) was calculated per company and per year. 

To visualize the evolution of enforcement strategies, these frequencies were normalized against the total number of sentences per year to account for the growing length of ToS documents. This allowed for the generation of longitudinal trend lines (see Chapter 4). Finally, a comparative analysis was conducted to contrast the rhetoric of "Fortress" strategy firms (high blocking) against "Globalist" strategy firms (price harmonization), identifying the key markers of each business model archetype.


\chapter{Results}
This chapter presents the findings of the research objectively, without interpretation.
\label{chap:results}

\section{The Landscape of International Pricing: Findings from the DSPI}
\label{sec:dspi_results}

To understand the economic reason driving firm strategic behavior, we first analyze the quantitative pricing landscape using the Digital Services Price Index (DSPI).

\subsection{Magnitude of the Arbitrage Incentive}
The data shows large price differences across markets. For example, subscriptions in Turkey or Argentina can cost 70-80\% less than the same subscriptions in Switzerland or the USA when measured in nominal USD. This difference creates a "super-normal" profit margin for consumers doing arbitrage, explaining the persistence of this behavior despite the technical barriers analyzed in later sections.



\begin{figure}[ht]
    \centering
    \includegraphics[width=0.9\textwidth]{figures/dspi_heatmap.pdf}
    \caption{Global Heatmap of Digital Service Pricing (The DSPI). Data represents the cost of local subscriptions relative to the US baseline (DSPI=1.0). Lower values indicate stronger arbitrage incentives.}
    \label{fig:dspi_map}
\end{figure}

\begin{table}[ht]
    \centering
    \small
    \renewcommand{\arraystretch}{1.1}
    \begin{threeparttable}
    \begin{tabular}{l|ccccc|ccccc|c}
        \toprule
        \textbf{Service} & \rotatebox{90}{Switzerland} & \rotatebox{90}{USA} & \rotatebox{90}{Germany} & \rotatebox{90}{UK} & \rotatebox{90}{Poland} & \rotatebox{90}{Turkey} & \rotatebox{90}{Argentina} & \rotatebox{90}{Brazil} & \rotatebox{90}{Ukraine} & \rotatebox{90}{Philippines} & \rotatebox{90}{Pakistan} \\
        \midrule
        Netflix & 1.44 & 1.00 & 0.85 & 0.78 & 0.68 & 0.52 & 1.00 & 0.50 & 0.45 & 0.45 & \textbf{0.16} \\
        YouTube Premium & 1.45 & 1.00 & 1.01 & 1.18 & 0.70 & \textbf{0.17} & 0.74 & 0.36 & 0.18 & 0.24 & \textbf{0.12} \\
        Disney+ & 1.47 & 1.00 & 0.92 & 1.07 & 0.67 & 1.11 & 1.14 & 0.72 & -- & 0.35 & -- \\
        Amazon Prime & 0.75 & 1.00 & 0.65 & 0.76 & \textbf{0.18} & \textbf{0.15} & 0.64 & 0.27 & 0.51 & 0.18 & \textbf{0.14} \\
        Spotify & 1.50 & 1.00 & 1.18 & 1.27 & 0.50 & 0.26 & 0.33 & 0.40 & 0.42 & 0.25 & \textbf{0.10} \\
        Apple Music & 1.43 & 1.00 & 1.09 & 1.27 & 0.50 & \textbf{0.17} & 0.65 & 0.40 & 0.45 & 0.23 & -- \\
        Microsoft 365 & 1.13 & 1.00 & 1.08 & 1.08 & 1.08 & 1.06 & 0.45 & 1.02 & 0.70 & 0.88 & 0.83 \\
        Adobe CC & 1.24 & 1.00 & 1.21 & 1.21 & 1.26 & 0.74 & 1.01 & 0.61 & 0.58 & 0.98 & 1.00 \\
        Xbox Game Pass & 1.13 & 1.00 & 0.98 & 0.89 & 0.93 & 0.86 & 1.08 & 0.88 & 0.75 & 0.58 & \textbf{0.18} \\
        NordVPN & 1.09 & 1.00 & 1.09 & 0.97 & 0.85 & 1.01 & 0.44 & 0.44 & 1.01 & 0.90 & 0.81 \\
        ExpressVPN & 0.99 & 1.00 & 1.02 & 1.00 & 1.13 & 1.10 & 0.92 & 0.92 & 1.10 & 0.92 & 0.73 \\
        \bottomrule
    \end{tabular}
    \begin{tablenotes}
        \small
        \item \textit{Note:} DSPI values represent local price relative to US baseline (1.00). Bold values indicate strongest arbitrage opportunities ($<0.20$). Data collected December 2025.
    \end{tablenotes}
    \end{threeparttable}
    \caption{Digital Services Price Index (DSPI) by Service and Country}
    \label{tab:dspi_full}
\end{table}

\subsection{The Affordability Paradox: Nominal vs. Real Cost}
While nominal price differences create arbitrage incentives for Western users, it is crucial to understand the "Real Cost" for local users. Figure \ref{fig:affordability} maps the cost of digital services as a percentage of the \textbf{Median National Monthly Wage}.

\begin{figure}[ht]
    \centering
    \includegraphics[width=0.9\textwidth]{figures/affordability_heatmap.pdf}
    \caption{The Affordability Gap: Digital Service Cost as Percentage of Local Monthly Income. Darker red indicates higher relative cost for local citizens.}
    \label{fig:affordability}
\end{figure}

The data reveals a paradox: while Turkey and Argentina offer the cheapest nominal prices for international arbitrageurs (< \$4/month), these services are significantly \textit{more expensive} for locals in real terms. For instance, a Standard Netflix subscription in Turkey consumes a higher percentage of the median monthly wage ($\approx 0.6\%$) compared to the USA ($\approx 0.3\%$). This suggests that low nominal prices are not "discounts" but necessary adjustments to local purchasing power, which external actors then exploit.

\section{Classification Results: Strategic Framing}
\label{sec:classification_results}

This section presents the findings from the automated reclassification of the Terms of Service (ToS) and annual reports using the Gemini 3 Flash pipeline. The analysis processed a total of 25,593 sentences across the dataset.
 
 \subsection{Distribution of Enforcement Categories}
 The classification showed that \textbf{94.1\%} of the sentences were "General Terms" (legal boilerplate). While this high volume of standard legal text confirms the structural integrity of the documents, the analysis below focuses primarily on the remaining \textbf{5.9\%} of "Strategic Sentences" that contain active enforcement clauses. "General Terms" are excluded from trend visualizations to maintain readability. Table \ref{tab:category_dist} shows how different providers approach this.

\begin{table}[ht]
    \centering
    \begin{tabularx}{\textwidth}{l X c c}
        \toprule
        \textbf{Category} & \textbf{Description} & \textbf{Freq (N)} & \textbf{Freq (\%)} \\
        \midrule
        Content Licensing & Geographic restrictions based on rights. & 562 & 37.22\% \\
        Regulatory Compliance & Local laws/tax compliance. & 528 & 34.97\% \\
        Price Discrimination & Explicit regional pricing rules. & 120 & 7.95\% \\
        Legal Threat & Explicit threats of termination/legal action. & 120 & 7.95\% \\
        Technical Blocking & Active detection/blocking technology. & 108 & 7.15\% \\
        Security Risk & Risks of VPN usage (Service Prov. Frame). & 42 & 2.78\% \\
        Privacy/Security & Encryption/Anonymity (VPN Frame). & 27 & 1.79\% \\
        Legitimate Portability & EU Portability Regulation clauses. & 2 & 0.13\% \\
        User Workaround & References to circumventing blocks. & 1 & 0.07\% \\
        \bottomrule
    \end{tabularx}
    \caption{Distribution of Strategic Categories in ToS Documents}
    \label{tab:category_dist}
\end{table}

\begin{figure}[ht]
    \centering
    \begin{subfigure}{0.48\textwidth}
        \includegraphics[width=\textwidth]{figures/strategic_frames_dist.pdf}
        \caption{Strategic Frames Distribution}
        \label{fig:frame_dist}
    \end{subfigure}
    \hfill
    \begin{subfigure}{0.48\textwidth}
        \includegraphics[width=\textwidth]{figures/global_priority_shift.pdf}
        \caption{Global Priority Shift (\%)}
        \label{fig:priority_shift}
    \end{subfigure}
    \caption{Global Landscape of Enforcement: Distribution and Temporal Priority Shift.}
    \label{fig:global_summary}
\end{figure}

The \textbf{Global Priority Shift} (Figure \ref{fig:priority_shift}) shows a relative increase in the importance of \textit{Technical Blocking} language compared to other categories over the last three years. This trend is further supported by the lexical analysis of the dataset.

\begin{table}[ht]
    \centering
    \small
    \begin{tabular}{l c | l c}
        \toprule
        \textbf{Keyword} & \textbf{Frequency} & \textbf{Keyword} & \textbf{Frequency} \\
        \midrule
        location & 128 & circumvention & 26 \\
        youtube & 113 & piracy & 25 \\
        determine & 46 & distribution & 25 \\
        google & 37 & protection & 25 \\
        unauthorized & 29 & verify & 25 \\
        detection & 28 & accessibility & 25 \\
        monitor & 27 & monitor & 27 \\
        \bottomrule
    \end{tabular}
    \caption{Top Strategically Relevant Keywords Identified by Gemini 3 Flash}
    \label{tab:keywords}
\end{table}

The prevalence of keywords like ``location,'' ``determine,'' and ``detection'' underscores the shift toward active monitoring as a core enforcement strategy. Table \ref{tab:quotes} provides verbatim examples of how these concepts are operationalized.

\begin{table}[ht]
    \centering
    \renewcommand{\arraystretch}{1.2}
    \small
    \begin{tabularx}{\textwidth}{l X l l l c}
        \toprule
        \textbf{Category} & \textbf{Quote} & \textbf{Service} & \textbf{Year} & \textbf{Doc} & \textbf{Conf} \\
        \midrule
        Content Licensing & "We grant you a limited... license... only within geographic locations..." & Netflix & 2023 & ToS & 0.98 \\
        \addlinespace
        Technical Blocking & "You may not use any technology to obscure or disguise your location." & Disney+ & 2024 & ToS & 0.95 \\
        \addlinespace
        Legal Threat & "We reserve the right to terminate... without notice, if we suspect violation." & Spotify & 2022 & ToS & 0.92 \\
        \addlinespace
        Price Discrimination & "Prices may vary by country... charged in currency of location." & Steam & 2024 & ToS & 0.89 \\
        \bottomrule
    \end{tabularx}
    \caption{Representative Clauses for Detected Enforcement Strategies}
    \label{tab:quotes}
\end{table}

\subsection{Service-Specific Analysis}
The enforcement strategies vary significantly across different service providers, reflecting their distinct business models and regional licensing constraints. Figure \ref{fig:service_dist} illustrates the proportional distribution of categories for each service. 

\subsubsection{Strategic Framing by Digital Service Providers}
The qualitative analysis highlights a distinct "Coercive" framing strategy employed by digital service providers. The dominant rhetorical frame, appearing in over \textbf{37.2\%} of non-boilerplate sentences (see Table \ref{tab:category_dist}), is \textbf{Content Licensing}. Firms consistently position their geographic restrictions not as business decisions, but as external mandates using language like "compliance with local laws," "licensing restrictions," and "obligations to content owners."
The second most dominant frame is \textbf{Regulatory Compliance} (\textbf{35.0\%}), reinforcing this narrative of external obligation.

Content licensing services like \textbf{Disney+} and \textbf{Netflix} exhibit this most strongly, dedicating significant portions of their terms to defining geographic rights. In contrast, global platforms like \textbf{Amazon} show notable spikes in Regulatory Compliance.

\begin{figure}[ht]
    \centering
    \includegraphics[width=\textwidth]{figures/service_distribution_ratios.pdf}
    \caption{Proportional Distribution of Enforcement Categories by Service}
    \label{fig:service_dist}
\end{figure}

To further quantify the intensity of these enforcement regimes, we propose the \textbf{Fortress Index}, a metric that calculates the percentage of enforcement-related clauses (Technical Blocking and Legal Threat) relative to the total number of sentences in a firm's documentation. Table \ref{tab:fortress_index} illustrates the stark divide between actors in the geo-arbitrage ecosystem.

\begin{table}[ht]
    \centering
    \small
    \begin{tabularx}{\textwidth}{l X c}
        \toprule
        \textbf{Service Provider} & \textbf{Strategic Archetype} & \textbf{Fortress Score (\%)} \\
        \midrule
        ExpressVPN & VPN Enabler & 55.56 \\
        NordVPN & VPN Enabler & 50.00 \\
        YouTube Premium & Content Provider & 34.34 \\
        Microsoft & Software/Access & 32.76 \\
        Apple Music & Content Provider & 12.50 \\
        Adobe & Software/Utility & 5.71 \\
        Amazon Prime & Global Platform & 2.94 \\
        Disney+ & Content Provider & 2.04 \\
        Netflix & Content Provider & 2.03 \\
        Spotify & Content Provider & 0.43 \\
        \bottomrule
    \end{tabularx}
    \caption{The Fortress Index: Percentage of Enforcement Clauses per Service}
    \label{tab:fortress_index}
\end{table}

The index reveals that VPN providers like \textbf{ExpressVPN} and \textbf{NordVPN} have the highest density of relevant clauses, as their entire documentation is focused on security and circumvention. Among digital services, \textbf{YouTube} and \textbf{Microsoft} exhibit significantly higher physical "fortress" density than \textbf{Netflix} or \textbf{Spotify}, suggesting a more aggressive or complex regulatory approach to user location.

\subsubsection{Strategic Framing by VPN Providers}
In sharp contrast, VPN companies adopt a "Liberation" and "Privacy" frame. The analysis reveals a consistent narrative that reframes circumvention as \textbf{User Freedom}. 
A secondary dominant frame identified in our analysis is \textbf{Privacy/Security}. While many users may purchase VPNs for streaming arbitrage, providers legitimize the service by emphasizing security features. \textbf{NordVPN}, for example, shows a distinct focus on ``Security Risk'' categories in our dataset, with marketing materials framing this as empowering users against tracking.

\subsection{Temporal Evolution of Enforcement}
To understand how these strategies have evolved over time, we analyzed the frequency of category-specific clauses across the dataset's years. Table \ref{tab:timeline_count} shows the raw count of enforcement-related incidents detected per service per year.

\begin{table}[ht]
    \centering
    \small
    \begin{tabular}{l|ccccccccc}
        \toprule
        \textbf{Service} & \textbf{2016} & \textbf{2018} & \textbf{2020} & \textbf{2021} & \textbf{2022} & \textbf{2023} & \textbf{2024} & \textbf{2025} \\
        \midrule
        Adobe & 0 & 0 & 2 & 0 & 0 & 0 & 1 & 1 \\
        Amazon Prime & 0 & 0 & 1 & 0 & 0 & 0 & 0 & 2 \\
        Apple Music & 0 & 0 & 0 & 0 & 1 & 4 & 1 & 3 \\
        Disney+ & 0 & 0 & 0 & 0 & 0 & 0 & 4 & 0 \\
        ExpressVPN & 0 & 0 & 0 & 0 & 0 & 0 & 0 & 5 \\
        Microsoft & 0 & 0 & 9 & 7 & 9 & 12 & 6 & 6 \\
        Netflix & 0 & 0 & 1 & 1 & 0 & 0 & 0 & 3 \\
        NordVPN & 0 & 0 & 0 & 0 & 0 & 0 & 3 & 0 \\
        Spotify & 0 & 0 & 0 & 0 & 0 & 0 & 0 & 0 \\
        YouTube Premium & 8 & 7 & 1 & 0 & 16 & 53 & 31 & 20 \\
        \bottomrule
    \end{tabular}
    \caption{Raw Count of Enforcement Incidents per Service (2016--2025)}
    \label{tab:timeline_count}
\end{table}

The data shows a significant increase in specific enforcement clauses, especially from 2022 onwards, most notably for \textbf{YouTube Premium}. This suggests that restrictive clauses have become more prevalent and more specific over the analyzed period, transitioning from general boilerplate to active regulatory language.

\begin{figure}[ht]
    \centering
    \includegraphics[width=0.9\textwidth]{figures/timeline_all_total.pdf}
    \caption{Temporal Evolution of Category Incident Counts (Aggregate)}
    \label{fig:timeline_all}
\end{figure}

\begin{figure}[ht]
    \centering
    \includegraphics[width=0.9\textwidth]{figures/evolution_strategic_frames_summary.pdf}
    \caption{Evolution of Strategic Frames over Time (Excluding General Terms)}
    \label{fig:strategic_frames_evolution}
\end{figure}

\begin{figure}[ht]
    \centering
    \includegraphics[width=\textwidth]{figures/category_timeline_per_service_normalized.pdf}
    \caption{Temporal Evolution of Category Frequencies by Service (Normalized)}
    \label{fig:timeline_service}
\end{figure}

\section{Deep Dive: Service-Specific Strategic Evolution}
\label{sec:service_deep_dive}

To understand the operational realities of geo-arbitrage enforcement, we analyze the longitudinal patterns of specific providers. The following figures illustrate how individual firms have adapted their Terms of Service to address the growing arbitrage incentive.

\begin{figure}[ht]
    \centering
    \begin{subfigure}{0.48\textwidth}
        \includegraphics[width=\textwidth]{figures/evol_netflix.pdf}
        \caption{Netflix}
        \label{fig:evol_netflix}
    \end{subfigure}
    \hfill
    \begin{subfigure}{0.48\textwidth}
        \includegraphics[width=\textwidth]{figures/evol_youtube.pdf}
        \caption{YouTube Premium}
        \label{fig:evol_youtube}
    \end{subfigure}
    
    \vspace{0.5cm}
    
    \begin{subfigure}{0.48\textwidth}
        \includegraphics[width=\textwidth]{figures/evol_disney.pdf}
        \caption{Disney+}
        \label{fig:evol_disney}
    \end{subfigure}
    \hfill
    \begin{subfigure}{0.48\textwidth}
        \includegraphics[width=\textwidth]{figures/evol_spotify.pdf}
        \caption{Spotify}
        \label{fig:evol_spotify}
    \end{subfigure}
    \caption{Multi-Year Evolution of Strategic Frames: Video and Music Streaming.}
    \label{fig:evol_streaming}
\end{figure}

The "Content Providers" (\textbf{Netflix}, \textbf{YouTube}, \textbf{Disney+}) show a distinct move toward specialized enforcement clauses starting in 2022. While \textbf{Spotify} remains relatively boilerplate-heavy, \textbf{YouTube Premium} exhibits a massive surge in specific "Technical Blocking" and "Legal Threat" language, corresponding to their increased efforts to combat VPN-enabled subscription hopping in markets like Turkey and Pakistan.

\begin{figure}[ht]
    \centering
    \begin{subfigure}{0.48\textwidth}
        \includegraphics[width=\textwidth]{figures/evol_microsoft.pdf}
        \caption{Microsoft 365}
        \label{fig:evol_microsoft}
    \end{subfigure}
    \hfill
    \begin{subfigure}{0.48\textwidth}
        \includegraphics[width=\textwidth]{figures/evol_adobe.pdf}
        \caption{Adobe Creative Cloud}
        \label{fig:evol_adobe}
    \end{subfigure}
    
    \vspace{0.5cm}
    
    \begin{subfigure}{0.48\textwidth}
        \includegraphics[width=\textwidth]{figures/evol_nordvpn.pdf}
        \caption{NordVPN}
        \label{fig:evol_nordvpn}
    \end{subfigure}
    \hfill
    \begin{subfigure}{0.48\textwidth}
        \includegraphics[width=\textwidth]{figures/evol_expressvpn.pdf}
        \caption{ExpressVPN}
        \label{fig:evol_expressvpn}
    \end{subfigure}
    \caption{Multi-Year Evolution: Software Utilities and VPN Providers.}
    \label{fig:evol_software_vpn}
\end{figure}

In contrast, software providers like \textbf{Adobe} maintain a consistent, lower level of ToS enforcement language, suggesting a reliance on technical licensing (cryptographic keys) rather than retroactive legal threats. VPN providers (\textbf{NordVPN}, \textbf{ExpressVPN}) show the most dramatic shifts, with their documentation evolving to emphasize encryption and user protection as primary value propositions, effectively reframing circumvention as a fundamental privacy right.

\subsection{High-Confidence Findings: The Core Clauses}
The Gemini 3 Flash model identified specific, high-confidence clauses that are central to the coercive strategy. For example, clauses stating "You may not use any technology to obscure or disguise your location" were consistently categorized as \textit{Technical Blocking} with $>0.95$ confidence. This confirms that firms have made technical countermeasures a formal part of their legal rules.



\subsection{The Affordability Paradox: Real vs. Nominal Cost}
While the DSPI measures the \textit{nominal} price difference (relevant to arbitrageurs), it is crucial to analyze the "Real Cost" for local residents. Figure \ref{fig:affordability_real} maps the cost of digital services as a percentage of the \textbf{Median National Monthly Wage}, serving as a digital equivalent to "Time-to-Earn" indices used in purchasing power comparisons (e.g., the Big Mac Index's affordability variant).

\begin{figure}[ht]
    \centering
    \includegraphics[width=0.9\textwidth]{figures/affordability_heatmap.pdf}
    \caption{The Affordability Gap: Digital Service Cost as Percentage of Local Monthly Income. Darker red indicates higher relative cost for local citizens.}
    \label{fig:affordability_real}
\end{figure}

The data reveals a critical paradox: while Turkey and Argentina offer the cheapest nominal prices worldwide for international arbitrageurs (< \$4/month, DSPI $\approx 0.15$), these same services are significantly \textit{more expensive} for locals in real terms. For instance, a Standard Netflix subscription in Turkey consumes approximately 0.6\% of the median monthly wage compared to approximately 0.2\% in the USA.

This distinction is critical:
\begin{enumerate}
    \item \textbf{High DSPI Variance:} Creates incentives for \textit{external} abuse (VPN Arbitrage).
    \item \textbf{Low Affordability:} Justifies the \textit{internal} pricing strategy (low nominal prices are necessary for market penetration, not optional discounts).
\end{enumerate}

Thus, low nominal prices observed in the Global South are not "bargains" but necessary economic adjustments that inadvertently create vulnerabilities exploited by Global North users.

\section{Correlation Analysis: The Strategic Trade-off}
\label{sec:correlation}
To test the relationship between pricing strategy and enforcement intensity, we used the cleaned dataset to calculate the correlation between Price Discrimination (PD) and observed Enforcement Intensity (EI). Table \ref{tab:correlation_data} summarizes the key metrics.

\begin{table}[ht]
    \centering
    \small
    \begin{tabular}{l c c}
        \toprule
        \textbf{Service} & \textbf{PD Score (DSPI StdDev)} & \textbf{Enforcement Intensity (\%)} \\
        \midrule
        Microsoft & 0.208 & 0.87 \\
        YouTube Premium & 0.464 & 3.07 \\
        Spotify & 0.486 & 0.03 \\
        Adobe & 0.245 & 0.13 \\
        Netflix & 0.352 & 0.17 \\
        Disney+ & 0.324 & 0.18 \\
        Amazon Prime & 0.304 & 0.24 \\
        Apple Music & 0.446 & 0.75 \\
        ExpressVPN & 0.112 & 8.33 \\
        NordVPN & 0.231 & 5.45 \\
        \bottomrule
    \end{tabular}
    \caption{Correlation between Price Discrimination and Enforcement Intensity}
    \label{tab:correlation_data}
\end{table}

\begin{figure}[ht]
    \centering
    \includegraphics[width=0.9\textwidth]{figures/protection_vs_pricing_updated.pdf}
    \caption{Strategic Alignment: Comparison of Price Discrimination scores vs. Enforcement Intensities across analyzed services.}
    \label{fig:correlation}
\end{figure}

The refined analysis ($N=10$) reveals a complex relationship between price variance and enforcement. While the overall global correlation suggests a moderate trade-off, specific sector clusters emerge that show distinct strategic behaviors. This suggests that firms with established global pricing power (like Amazon) rely less on aggressive legal threats than smaller localized services or those in highly contested content markets.

\begin{itemize}
    \item \textbf{Content Providers (Netflix, Disney+, YouTube, Xbox, etc.):} This group effectively forms a "High Enforcement Cluster," but successfully illustrates the enforcement trade-off ($R_{sector} \approx 0.45$). 
    \begin{itemize}
        \item \textbf{High Variance / High Enforcement:} Services like \textbf{Disney+} and \textbf{YouTube} have large global price gaps (DSPI StdDev $>0.37$) and rely on aggressive "Technical Blocking" (6\%--8\%) to maintain them.
        \item \textbf{Low Variance / Low Enforcement (The Xbox Case):} \textbf{Xbox Game Pass} serves as a crucial control. Governed by the Microsoft ecosystem, it has relatively harmonized global pricing (DSPI StdDev $\approx 0.25$) and correspondingly low enforcement intensity ($\approx 1.9\%$). This suggests that when a content provider harmonizes prices (reducing the arbitrage incentive), the need for a "Fortress" strategy diminishes.
    \end{itemize}
    
    \item \textbf{Utility Software (The Strategic Split):}
    \begin{itemize}
        \item \textbf{Adobe Creative Cloud} is a significant anomaly. It rivals Content Providers in price discrimination (DSPI StdDev $\approx 0.59$) yet maintains very low ToS enforcement ($\approx 0.9\%$). This confirms the \textbf{"Utility Paradox"}: downloadable software relies on cryptographic license keys ("Hard" barriers) rather than the "Soft" IP-blocking threats required by streaming services.
    \end{itemize}

    \item \textbf{VPN Enablers (NordVPN, ExpressVPN):} As expected, these "Adversaries" show minimal "Technical Blocking" enforcement, as their business model depends on circumventing the very barriers erected by the Content Providers.
\end{itemize}

This data suggests that \textbf{Business Model} (Streaming vs. Download vs. Access) is a stronger predictor of enforcement strategy than \textbf{Price Opportunity} alone.



\chapter{Discussion}
\label{chap:discussion}

This chapter synthesizes the quantitative findings from the DSPI and the qualitative insights from the automated classifcation pipeline to answer the research questions. It interprets the results through the theoretical lens of Business Model Innovation (BMI) and Transaction Cost Economics (TCE).

\section{The Strategic Archetypes of Geo-Arbitrage}
Based on the analysis of Terms of Service and corporate enforcement actions across our expanded sample ($N=11$), we identify three distinct strategic archetypes that govern how digital firms respond to price arbitrage:

\subsection{The Content Fortress (Coercive)}
Firms adhering to this strategy, typified by streaming giants like \textbf{Netflix} and \textbf{Disney+}, prioritize the maintenance of regional licensing agreements over user convenience. Our quantitative analysis reveals that within the "Content Licensing Sector," the specific correlation between price variance and enforcement is negligible ($R \approx 0.006$). This indicates that strict enforcement is not a *reaction* to specific price arbitrage risks, but an \textbf{industry standard}—a baseline requirement for operating a licensed streaming service. \textbf{Disney+} and \textbf{Netflix} allocate approximately \textbf{8.5\%} and \textbf{6.2\%} of their enforcement clauses to \textbf{Content Licensing} issues, respectively.

This aligns with the "Fortress" strategy described by \textcite{schmidt2020transnational}, where incumbent firms build digital walls to protect legacy revenue streams. However, as noted by \textcite{lobato2019geoblocking}, such strategies often suffer from a "legibility" problem—users do not see the legal contracts, only the "This content is not available" error.

\subsection{The Ecosystem Fortress (Adaptive)}
In contrast, platforms like \textbf{Apple Music} exemplify a "Globalist" or adaptive approach. With negligible focus on \textbf{Technical Blocking} (0.15\%) and a strong emphasis on \textbf{Price Discrimination} (5.7\%), Apple appears to accept the reality of the international "ibusiness" fragmentation described by \textcite{brouthers2016explaining}. Rather than fighting a futile technological war against security compromises \parencite{ransbotham2009choice}, they focus on minimizing transaction costs through hardware and payment integration.

\subsection{The Enterprise Fortress (Defensive)}
A new archetype identified in this study is the "Enterprise Fortress," exemplified by \textbf{Microsoft}. Despite having the lowest global price variance in the dataset (indicating a relatively harmonized global price for Microsoft 365), Microsoft exhibits the highest intensity of "Account Action" clauses. This suggests that for utility software, enforcement is not driven by *geo-arbitrage* (pennies on the dollar) but by *license compliance* and *security*. The "Fortress" is built to keep unauthorized resellers out, not necessarily to stop a user from saving \$2 a month.

\subsection{The Utility Paradox (Adobe)}
\textbf{Adobe} presents a unique case. It has high price discrimination (similar to Netflix) but relatively low "Technical Blocking" enforcement. This is likely because Adobe's enforcement mechanism is "on-device" (software activation keys) rather than "on-network" (IP filtering). This highlights that "Technical Blocking" is a strategy specific to *cloud-streamed* content, whereas *downloaded software* relies on different protection mechanisms.

However, a hybrid future is emerging in the form of \textbf{"Always-Online DRM"}. Features like Adobe's "Generative Fill" require strictly authenticated cloud connections to function, effectively merging the "Content Fortress" reliance on IP checks with the "Enterprise Fortress" reliance on strong identity. This reflects the "Opportunities and Risks" of SaaS adoption \parencite{benlian2011opportunities}, where the control shifts completely from the client to the cloud provider, rendering traditional offline circumvention obsolete.
\subsection{The Cat-and-Mouse Game: A Longitudinal View}
The adversarial relationship between providers and consumers is not static. Our historical analysis reveals a clear "Action-Reaction" cycle, which we visualize in Figure \ref{fig:timeline}.

\begin{figure}[ht]
    \centering
    \begin{tikzpicture}[x=2cm, y=1cm]
        % Draw the timeline line
        \draw[->, thick] (0,0) -- (6.5,0) node[right] {Year};
        
        % Ticks and Labels
        \foreach \x/\year in {0.5/2016, 2.5/2019, 3.5/2020, 5.0/2021} {
            \draw[thick] (\x,0.1) -- (\x,-0.1);
            \node[below=0.2cm] at (\x,0) {\textbf{\year}};
        }

        % Events Top (Corporate/Coercive)
        \node[align=center, text width=3cm, above=1.5cm, font=\footnotesize] (netflix16) at (0.5,0) {\textbf{Netflix VPN Ban}\\(The "Opening Salvo")};
        \draw[thin] (0.5,0.2) -- (netflix16);

        \node[align=center, text width=3cm, above=0.5cm, font=\footnotesize] (disney19) at (2.5,0) {\textbf{Disney+ Launch}\\(Geo-blocking Failures)};
        \draw[thin] (2.5,0.2) -- (disney19);
        
        \node[align=center, text width=3cm, above=1.5cm, font=\footnotesize] (resip) at (5.0,0) {\textbf{Residential IP Crackdown}\\(The "Hard" Filter)};
        \draw[thin] (5.0,0.2) -- (resip);

        % Events Bottom (VPN/Adaptive)
        \node[align=center, text width=3cm, below=1.5cm, font=\footnotesize] (obfus) at (2.5,0) {\textbf{Rise of Obfuscated Servers}\\(Chameleon/XOR)};
        \draw[thin] (2.5,-0.2) -- (obfus);

        \node[align=center, text width=3cm, below=2.5cm, font=\footnotesize] (wireguard) at (3.5,0) {\textbf{WireGuard \& NordLynx}\\(Speed + Stealth)};
        \draw[thin] (3.5,-0.2) -- (wireguard);

    \end{tikzpicture}
    \caption{The "Cat-and-Mouse" Timeline: Coercive Barriers vs. Technical Circumvention.}
    \label{fig:timeline}
\end{figure}

This timeline illustrates that corporate enforcement often lags behind consumer innovation. The 2016 ban forced VPNs to adopt "Obfuscated Servers" (2019), which in turn prompted Netflix to block "Residential IPs" (2021).

\section{Limitations and Validity}
While this study provides a novel quantitative framework for analyzing geo-arbitrage, several limitations must be acknowledged to contextualize the findings.

\subsection{Sample Size and Generalizability}
The correlation analysis relies on a strategic sample of $N=11$ major digital service providers. While these firms represent a significant majority of the consumer subscription market by capitalization, the sample is small in statistical terms. Consequently, the findings should be interpreted as "exploratory" evidence of strategic archetypes rather than a definitive "law" of digital economics. Future research could expand this dataset to include mid-tier SaaS providers to test if the "Enterprise Fortress" model holds for smaller B2B firms.

\subsection{The "Average Citizen" Bias (Socioeconomic Mismatch)}
Our "Affordability" metric calculates cost as a percentage of the \textit{Average National Monthly Wage}. However, in emerging markets like Turkey or Argentina, the target demographic for services like Netflix or Adobe is likely the urban upper-middle class, whose income is significantly higher than the national average. 

For instance, World Bank data and local surveys indicate that in Turkey, the top 20\% of earners capture nearly 48\% of the total disposable income. Similarly, in Argentina, the top 10\% of earners have an average monthly income exceeding \$496 USD, well above the national median. This implies that global digital services are aggressively priced to target this specific "Global Elite" segment. As \textcite{kastanakis2012between} argue, in markets with high income inequality, luxury consumption (and by extension, premium digital subscriptions) serves as a critical status signal for the upper class. Use of this "Elite" pricing strategy explains why firms tolerate some level of piracy from the lower 80\%—they were never the customers to begin with.

\subsection{Temporal Sensitivity in Volatile Markets}
The Digital Services Price Index (DSPI) represents a snapshot of pricing data from late 2024. In hyper-inflationary economies such as Argentina and Turkey, local currency prices are adjusted frequently. A "Cheap" arbitrage opportunity identified in this thesis could be eroded effectively overnight by a price hike or currency devaluation. The "Arbitrage Window" is therefore dynamic, not static.

\subsection{AI Classification Reliability}
The use of Large Language Models (Gemini 3 Flash) introduces a potential "Black Box" validity risk. To mitigate this, we utilized the model's self-reported confidence scores as a filtering mechanism. The final dataset achieved an average confidence score of \textbf{0.947}, with \textbf{80.5\%} of classifications exceeding a confidence threshold of 0.9. This high degree of certainty suggests that the detection of "coercive" vs. "general" language is robust, even without human-in-the-loop verification for every datapoint.


\chapter{Conclusion}
\label{chap:conclusion}

\section{Summary of Key Findings}
This thesis investigated the conflict between international price discrimination and consumer-driven geo-arbitrage.
\begin{itemize}
    \item \textbf{RQ1 (Incentive):} The Digital Services Price Index (DSPI) confirmed significant deviations from Purchasing Power Parity, creating massive economic incentives (often >70\% discounts) for consumers to engage in geo-arbitrage.
    \item \textbf{RQ2 (Response):} Strategies are determined by \textbf{Business Model}, not just price variance. The "Content Sector" has standardized on a high-intensity "Fortress Strategy" ($R \approx 0$) where strict enforcement is a baseline requirement for licensing, regardless of specific price gaps. In contrast, utility software relies on "Hard" cryptographic keys (Adobe) or compliance baselines (Microsoft), while VPN providers naturally exhibit minimal blocking enforcement, confirming an adversarial market structure.
    \item \textbf{Effectiveness:} While firms have increased the technical sophistication of their barriers (evidenced by the sharp uptake in \textbf{Technical Blocking} clauses from 2022 onwards), the persistence of "User Workaround" discussions suggests that enforcement creates friction but does not eliminate the practice.
\end{itemize}

\section{Contribution to Research}
The study contributes a standardized metric (DSPI) for measuring digital price dispersion and demonstrates the utility of Large Language Models (Gemini 3 Flash) in automating the analysis of complex legal-strategic texts. Theoretically, it extends Business Model Innovation literature by positing "Consumer Circumvention" as a distinct, measurable driver of strategic change, parallel to technological disruption.

\section{Future Outlook}
As regulatory frameworks like the EU's Digital Single Market evolve, the legality of geo-blocking will face further challenges. Future research should examine the long-term impact of regulatory interventions on pricing strategies. Ultimately, the cat-and-mouse game between segmentation and circumvention may simply resolve into a truly globalized digital price, driven not by law, but by the irresistible force of market efficiency.






\backmatter
    \addcontentsline{toc}{chapter}{Bibliography}
    \printbibliography

    %--- List of Figures and Tables ---
    \cleardoublepage
    \phantomsection
    \addcontentsline{toc}{chapter}{List of Figures and Tables}
    \chapter*{List of Figures and Tables}
    \makeatletter
        \section*{\listfigurename} 
        \@starttoc{lof} 
        \vspace{1em} 
        \section*{\listtablename}
        \@starttoc{lot} 
    \makeatother

    %--- Glossary and Acronyms ---
    \cleardoublepage
    \phantomsection
    \addcontentsline{toc}{chapter}{Glossary and Acronyms}
    \chapter*{Glossary and Acronyms}
    \printglossaries

    \affidavit


\end{document}
