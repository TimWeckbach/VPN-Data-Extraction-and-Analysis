\label{chap:introduction}

\section{Background and Context}
The globalization of digital services has created a paradox in the modern digital economy. While the internet promises a borderless exchange of information, digital service providers (DSPs) such as Netflix, Spotify, and Steam enforce rigid digital borders to maximize profits through international price differentiation. This strategy, deeply rooted in economic theories of third-degree price discrimination, allows firms to charge widely varying prices for identical digital goods based on the purchasing power of the consumer's location.

However, this segmentation strategy faces a formidable disruptive force: the consumer. Equipped with increasingly accessible circumvention technologies like Virtual Private Networks (VPNs), consumers are engaging in "digital geo-arbitrage"—the practice of virtually relocating to a cheaper market to purchase services at a fraction of the domestic price. This phenomenon mirrors the disruption caused by digital piracy in the early 2000s, where technical barriers were circumvented to access content, fundamentally challenging the music and film industries' business models.

\section{Problem Statement}
The core problem addressed in this thesis is the strategic conflict between a firm's geographic market segmentation and the technical circumventability of these digital borders. 
Firms are currently trapped in a "cat-and-mouse" game:
\begin{enumerate}
    \item \textbf{Economic Necessity:} They must segment markets to remain affordable in low-income regions while maximizing revenue in high-income nations.
    \item \textbf{Technical Reality:} The same internet architecture that enables global delivery also enables global circumvention.
\end{enumerate}
This tension challenges the economic viability of established business models and creates pressure for Business Model Innovation (BMI). Firms must choose between "Coercive" strategies (blocking, banning, litigation) and "Adaptive" strategies (price harmonization, global portability).

\section{Research Questions (RQs)}
To analyze this strategic conflict, this thesis pursues the following research questions:

\begin{description}
    \item[RQ1 (The Economic Incentive):] To what extent does international price differentiation for digital services deviate from Purchasing Power Parity (PPP), creating a "super-normal" incentive for arbitrage? (Note: This question serves primarily to establish the research setting and economic motivation).
    \item[RQ2 (The Strategic Response):] \textbf{How do digital subscription providers modify their business model in response to regional pricing circumvention and how has the mix of coercive versus adaptive responses reflected in their corporate disclosures changed over time?}
\end{description}

\section{Structure of the Thesis}
The thesis is structured as follows: \textbf{Chapter \ref{chap:theory}} establishes the theoretical foundations, linking price discrimination theory with the behavioral mechanics of circumvention. \textbf{Chapter \ref{chap:methodology}} details the mixed-methods research design, including the novel "Digital Services Price Index" (DSPI) and the LLM-based classification pipeline. \textbf{Chapter \ref{chap:results}} presents the empirical findings. \textbf{Chapter \ref{chap:discussion}} interprets these results within the framework of Business Model Innovation, and \textbf{Chapter \ref{chap:conclusion}} summarizes the contributions and limitations of the study.
