\label{chap:introduction}

\section{Background and Context}
Digital services are global, but their prices are local. Companies like Netflix, Spotify, and Steam split the world into regions to maximize profit. They use third-degree price discrimination to charge different prices for the same product based on where the customer lives.

Customers are responding to this by using Virtual Private Networks (VPNs). They use these tools to fake their location and buy subscriptions from cheaper markets. This practice is called "digital geo-arbitrage". It is similar to digital piracy in the early 2000s: users circumvent technical barriers to access content or prices that were not intended for them.

\section{Problem Statement}
The core problem is the conflict between a firm's need to segment markets and the user's ability to bypass that segmentation. 
Firms face a dilemma:
\begin{enumerate}
    \item \textbf{Economic Necessity:} They need to lower prices in poorer countries to get subscribers, but keep them high in rich countries to make money.
    \item \textbf{Technical Reality:} The internet allows users to easily jump between these markets.
\end{enumerate}
This forces companies to change their business models. They must decide whether to fight these users with "Coercive" strategies (blocking, banning) or accept the reality with "Adaptive" strategies (standardizing prices).

\section{Research Questions (RQs)}
To analyze this strategic conflict, this thesis pursues the following research questions:

\begin{description}
    \item[RQ1 (The Economic Incentive):] To what extent does international price differentiation for digital services deviate from local affordability (Median National Wage), creating a "super-normal" incentive for arbitrage? (Note: This question serves primarily to establish the research setting and economic motivation).
    \item[RQ2 (The Strategic Response):] \textbf{How do digital subscription providers modify their business model in response to regional pricing circumvention and how has the mix of coercive versus adaptive responses reflected in their corporate disclosures changed over time?}
\end{description}

\section{Structure of the Thesis}
The thesis is structured as follows: \textbf{Chapter \ref{chap:theory}} establishes the theoretical foundations, linking price discrimination theory with the behavioral mechanics of circumvention. \textbf{Chapter \ref{chap:methodology}} details the mixed-methods research design, including the novel "Digital Services Price Index" (DSPI) and the LLM-based classification pipeline. \textbf{Chapter \ref{chap:results}} presents the empirical findings. \textbf{Chapter \ref{chap:discussion}} interprets these results within the framework of Business Model Innovation, and \textbf{Chapter \ref{chap:conclusion}} summarizes the contributions and limitations of the study.
