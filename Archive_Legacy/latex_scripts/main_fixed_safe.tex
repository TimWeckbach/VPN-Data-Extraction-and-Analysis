%%
%\documentclass{}[
	%english,
	%ruledheaders=section,           %Ebene bis zu der die Überschriften mit Linien abgetrennt werden, vgl. DEMO-TUDaPub
	%class=report ,                   % Basisdokumentenklasse. Wählt die Korrespondierende KOMA-Script Klasse
	%thesis={type=bachelor},         % Dokumententyp Thesis, für Dissertationen siehe die Demo-Datei DEMO-TUDaPhd
	%accentcolor=9c,% Auswahl der Akzentfarbe
	%custommargins=true,% Ränder werden mithilfe von typearea automatisch berechnet
	%marginpar=false,% Kopfzeile und Fußzeile erstrecken sich nicht über die Randnotizspalte
	%BCOR=5mm,%Bindekorrektur, falls notwendig
	%parskip=half-,%Absatzkennzeichnung durch Abstand vgl. KOMA-Script
	%fontsize=11pt,%Basisschriftgröße laut Corporate Design ist mit 9pt häufig zu klein
	%logofile={figures/tuda_logo.pdf},
    %pdfa=false
%]%{tudapub}

\documentclass[
	english,
	ruledheaders=section,
	class=report,
	thesis={type=master},
	accentcolor=9c,
	custommargins=geometry,%CHANGED FROM TRUE TO FALSE
	marginpar=false,
	parskip=half-,
	fontsize=12pt,
	logofile={figures/tuda_logo.pdf},
    pdfa=true
]{tudapub}

%\usepackage{showframe}
%\usepackage[reset, left=2.5cm, right=2.5cm, top=2.5cm, bottom=2.5cm, includefoot]{geometry}
%\geometry{a4paper, left=2.5cm, right=2.5cm, top=2.5cm, bottom=2.5cm, includefoot}
%\geometry{reset, bottom=2.5cm, includefoot} 


\usepackage[main=english, ngerman]{babel}
\usepackage[autostyle]{csquotes}
\usepackage{microtype}
\usepackage[table]{xcolor}

\usepackage{graphicx}
\usepackage{amsmath}
\usepackage{tabularx}
\usepackage{booktabs}
\usepackage{threeparttable}
\renewcommand{\TPTnoteSettings}{\fontsize{10}{12}\selectfont} %<-- Add this line
\usepackage{pifont}
\usepackage{tikz}
\usetikzlibrary{shapes,arrows,positioning,calc}

\usepackage{setspace}
\onehalfspacing

\usepackage{listings}
\lstset{
  basicstyle=\ttfamily\small,
  breaklines=true,
  frame=single,
  columns=fullflexible
}

\usepackage{pgfplots}
\pgfplotsset{compat=1.18}
\usepgfplotslibrary{colormaps}
\definecolor{tudared}{HTML}{B90E21}
\definecolor{tudablue}{HTML}{004E8C}
\definecolor{tudagreen}{HTML}{7DA62D}
\definecolor{tudaorange}{HTML}{E65C00}
\definecolor{tudagray}{HTML}{9C9C9C}
\pgfplotsset{
    scaled ticks=false,
    tick label style={/pgf/number format/fixed},
    /pgf/number format/precision=3,
    /pgf/number format/fixed zerofill=false
}

% Use KOMA-Script recommended method for section formatting
\RedeclareSectionCommand[
    beforeskip=0pt,
    afterskip=10pt
]{chapter}


\usepackage{caption}
\usepackage{subcaption}
\DeclareCaptionFont{tenpoint}{\fontsize{10}{12}\selectfont}
\captionsetup{font=tenpoint}


%change the apa6 to apa when you want to use the APA 7th edition
\usepackage[backend=biber, style=apa, sorting=nyt]{biblatex}
\addbibresource{Bibliography.bib}

\usepackage[acronym, section]{glossaries}
%glossaries here
\newacronym{vpn}{VPN}{Virtual Private Network}
\newacronym{ppp}{PPP}{Purchasing Power Parity}
\newacronym{dspi}{DSPI}{Digital Services Price Index}
\newacronym{tos}{ToS}{Terms of Service}
\newacronym{llm}{LLM}{Large Language Model}
\newacronym{ptw}{PTW}{Price-to-Wage}
\newacronym{bmi}{BMI}{Business Model Innovation}
\newacronym{drm}{DRM}{Digital Rights Management}
\newacronym{dpi}{DPI}{Deep Packet Inspection}
\newacronym{rq}{RQ}{Research Question}
\newacronym{eu}{EU}{European Union}


\makeglossaries

% change count of figures and tables to not include chapter number
\usepackage{chngcntr}
\counterwithout{figure}{chapter}
\counterwithout{table}{chapter}

\title{Business Model Responses to Consumer Circumvention: Lessons from Piracy Applied to
VPN-Enabled Geo-Arbitrage}
\author{Tim Weckbach}
\reviewer{William Schütte \and Prof. Dr. Alexander Kock}




\begin{document}

%\setlength{\textheight}{24cm}
%\setlength{\topmargin}{-1.2cm} % Muss oft angepasst werden
%\setlength{\headsep}{1cm}
%\setlength{\footskip}{1.5cm}
%\setlength{\voffset}{5cm}


\Metadata{
    \title{Business Model Responses to Consumer Circumvention: Lessons from Piracy Applied to
    VPN-Enabled Geo-Arbitrage}
    \author[Tim Weckbach]{Tim Weckbach}
    \keywords{Price Discrimination, Geo-Arbitrage, Business Model Innovation, VPN, Digital
    Services Price Index}
    \publisher{TU Darmstadt}
    \copyright{Tim Weckbach}
    \birthplace{Frankfurt am Main}
    \reviewer{William Schütte \and Prof. Dr. Alexander Kock}
}
\department{Business Informatics} 
\institute{Technology and Innovation Management}
\submissiondate{17.02.2026}
\maketitle
\newgeometry{bottom=2.5cm, includefoot}

\begin{abstract}
This thesis looks at the strategic conflict between corporate price discrimination and
consumer-driven geo-arbitrage in digital subscription markets. Using a mixed-methods design, the
study first measures the economic incentive for price-hopping through the \gls{dspi}. Analysis of 11 digital services across 11 countries shows large price
differences, with subscriptions in markets like Turkey and Pakistan offered at discounts over
80\% compared to Western baselines. However, the research identifies an ``Affordability Paradox'':
while nominally cheap for international arbitrageurs, these services are often more expensive
for local consumers as a percentage of their income. In the second part, an \gls{llm} (Gemini 3 Flash)
processed over 25,000 sentences from company documents and \gls{tos} to find strategic
responses. The findings show that enforcement strategies are mainly driven by how the business
is built. Streaming providers use technical blocking to satisfy licensing rules, while software
firms rely on identity checks. Results show a large increase in technical countermeasures since
2022, signaling a growing race between platform control and technical bypass. The research
concludes that companies are moving from just blocking users toward better global pricing and
ecosystem integration.
\end{abstract}

\noindent\textbf{Keywords:} Price Discrimination, Geo-Arbitrage, \gls{bmi},
\gls{vpn}, \gls{dspi}
\tableofcontents
\printglossaries


\chapter{Introduction}
\label{chap:introduction}

This introductory chapter outlines what this research covers and why it matters. It shows the
main problem: digital services naturally cross borders, but platforms try to keep them in
geographic boxes using different prices. By defining the problem and research questions, this
chapter provides a map for the rest of the thesis.

\section{Background and Context}
The global digital economy has a basic contradiction: while digital services naturally cross
geographical borders, companies in this space commonly use geographically segmented pricing
strategies. Major platforms such as Netflix, Spotify, and Microsoft 365 divide the
world into different pricing regions, charging very different prices for the same digital
products depending on where the customer is. This practice, known as third-degree
price discrimination, lets firms get the most revenue from each market by matching
prices with what local users can pay. In markets with lower incomes, services are offered at much
lower prices to reach more customers, while wealthier regions face higher prices.

However, the technical setup that enables global digital distribution also lets
customers challenge these boundaries. Many users have discovered that \glspl{vpn} can be used not only for privacy but also to bypass
regional price blocks. By hiding their true location and routing their internet traffic through
servers in other
regions, users can engage in ``licensing hopping'' to access regionally restricted content
libraries or purchase subscriptions at prices meant for markets with very different economic and
regulatory conditions. This practice, often called ``digital geo-arbitrage,'' ``\gls{vpn} price
hopping,'' or ``licensing hopping,'' is a form of consumer-driven market arbitrage that takes
advantage of the price and library differences created by firms.

%The global digital economy has a basic contradiction: while digital services naturally cross 
%geographical borders, companies in this space commonly use geographically segmented pricing strategies. 
%Major platforms such as Netflix, Spotify, Steam, and Microsoft Office divide the world into 
%distinct pricing regions, charging vastly different prices for identical digital products 
%depending on the customer's geographic location. This practice, known as third-degree price 
%discrimination, allows firms to extract maximum revenue from each market by aligning prices 
%with local willingness to pay and purchasing power. In markets with lower average incomes, 
%services are offered at significantly reduced prices to maximize market penetration, while 
%wealthier regions face premium pricing to capture consumer surplus.

%However, the technical infrastructure that enables global digital distribution also allows 
%customers to challenge this geographical segmentation. Many users have discovered that Virtual 
%Private Networks (VPNs) can be used not only for privacy protection but also to bypass regional 
%pricing barriers. By hiding their true location and routing their internet traffic through 
%servers in lower-priced regions, users can purchase subscriptions at prices meant for markets 
%with very different economic conditions. This practice, often called "digital geo-arbitrage" 
%or "VPN price hopping," is a form of consumer-driven market arbitrage that takes advantage 
%of the price differences created by firms' own segmentation strategies.

This pattern is like the digital piracy wave of the early 2000s, where users
used technical tools to bypass blocks \parencite{oberholzer2007effect}. Just as
file-sharing let users bypass payment systems, VPN-based geo-arbitrage
lets users bypass pricing, though they still pay for the service. In both cases, users use simple
technology to bypass barriers. This forces companies to change how they make money and create
better business models. However, as discussed in Chapter \ref{chap:discussion}, this cycle of 
bypassing is met by more advanced, often secret, detection technologies that aim to unmask \gls{vpn} 
users. These methods are hard to study because companies keep them secret, making them a 
``black box'' for researchers.

\section{Problem Statement}
The main problem in this thesis is the conflict between companies wanting to keep markets
separate and users being able to bypass those blocks.
Subscription companies face a tough choice between two needs:
\begin{enumerate}
    \item \textbf{Economic Choice:} To reach more customers and earn more, firms must offer much lower prices in emerging markets while keeping high prices in wealthier regions.
    \item \textbf{Technical Reality:} The internet lets users bypass geographic blocks using simple VPN services that anyone can use.
\end{enumerate}
This conflict makes companies rethink how they work. They must choose between
``Coercive'' strategies that try to keep markets separate through technical blocking,
``Adaptive'' strategies that reduce the reason for arbitrage by making prices more similar,
or entirely new types of business models that work differently.

\section{Research Questions (\glspl{rq})}
To analyze this conflict, this thesis looks at two main research questions
that address both why users engage in geo-arbitrage and how companies respond to
it.

The first research question focuses on measuring the economic incentive:
\begin{description}
    \item[\gls{rq}1 (The Economic Incentive):] To what extent do international price differences for digital services deviate from local affordability (based on local wages), creating a large reason for users to hop between regions?
\end{description}

To answer this, we built the \gls{dspi}, which compares prices across regions while also calculating how affordable 
subscriptions are based on local wages. The difference between nominal price (the number) and
real price (the cost for a local) is key: a service may seem cheap to a foreign user while
being expensive for a local.

The second research question looks at how companies respond to the arbitrage threat:
\begin{description}
    \item[\gls{rq}2 (The Strategic Response):] How do digital subscription providers change their business models in response to price hopping, and how has the balance between blocking and adapting changed over time?
\end{description}

To answer this, we used AI to analyze \gls{tos} documents and
annual reports, grouping company language into categories like ``Technical
Blocking,'' ``Legal Threat,'' and ``Price Discrimination.'' Looking at data over time lets us
see how strategies have changed from 2020 to 2025.

\section{Structure of the Thesis}
The thesis follows a simple order from theory to data findings and then to the discussion.

\textbf{Chapter \ref{chap:theory}} looks at three areas of theory: (1) economics of 
price discrimination; (2) consumer behavior and digital piracy; and (3) Business Model 
Innovation theory, which gives us a way to understand company responses.

\textbf{Chapter \ref{chap:methodology}} explains the research methods. We describe how we built 
the \gls{dspi} using data from many countries and services. We also 
describe the \gls{llm} pipeline used to analyze over 25,000 sentences from company documents.

\textbf{Chapter \ref{chap:results}} shows the findings in three parts: (1) the price 
landscape; (2) how enforcement strategies have changed; and (3) the link between pricing and 
blocking.

\textbf{Chapter \ref{chap:discussion}} looks at these results using business theory,
identifying four main business types (Content Fortress, Ecosystem Fortress, Enterprise Fortress, 
and the Utility Paradox) and what they mean for digital platforms.

\textbf{Chapter \ref{chap:conclusion}} sums up the study, notes the limits, and looks at 
future research, including the possible effect of new rules like the \gls{eu} Digital Single Market.

\chapter{Theoretical Foundations \& Literature Review}
\label{chap:theory}


This chapter provides the theoretical basis for the study by looking at three areas: economic
pricing, consumer behavior, and business model innovation. By connecting Hal Varian's work on
price discrimination with the history of digital piracy, we create a way to understand modern
geo-arbitrage as a major challenge for digital companies.

\section{Economic Foundations of International Price Setting}
\label{sec:theory_pricing}

To understand why consumers engage in geo-arbitrage, we must first establish why firms create
the price disparities that make such arbitrage profitable. Geographic price differentiation is
not arbitrary but follows well-established economic principles.

\subsection{Third-Degree Price Discrimination}
According to Varian (1989), third-degree price discrimination happens when a firm divides the
market based on visible traits—in this case, geographic location—and charges different prices
to each group. For digital goods, where the cost of copying is near zero ($MC \approx 0$)
\parencite{shapiro1998information,amit2001value}, this strategy allows firms to get the most
consumer surplus from both high-income (e.g., Switzerland) and low-income (e.g., Turkey)
markets at the same time.

Two conditions must hold for successful price discrimination:
\begin{itemize}
    \item \textit{Condition 1: Market Segmentation.} The firm must be able to distinguish between consumer groups based on observable characteristics such as IP address or billing location.
    \item \textit{Condition 2: No Arbitrage.} The firm must be able to prevent the resale or transfer of the good between segments.
\end{itemize}
\gls{vpn}-enabled geo-arbitrage directly breaks \textit{Condition 2}, effectively merging the 
different regions into a single global market.

Also, this study shows that companies accidentally split their customers into two even smaller 
groups within high-income regions. By keeping blocks in place (like \gls{vpn} detection), companies 
split the market based on how much effort a user is willing to make. This creates a split 
between ``active'' users—those who spend time and effort to bypass blocks—and ``passive'' users 
who just want convenience. In this way, technical friction acts as a way for users to choose: 
companies get high revenue from the majority of users while tolerating some loss from a small 
group of technical users.

\subsection{Purchasing Power Parity (PPP) as a Benchmark}
The "Law of One Price" suggests that in an efficient market, identical goods should sell at
the same price when shown in a common currency. However, differences from this law are common.
\textcite{rogoff1996ppp} argues that for physical goods, shipping costs and trade barriers
justify price differences. In the digital world, \textcite{clemons2002price} note that while
transaction costs are much lower, price differences continue because firms can set up detailed
customer segmentation.

We use \gls{ppp} as a benchmark for "economically justified" pricing. If a
Netflix subscription in Turkey is cheaper than in the US only because of currency value and
local purchasing power, this fits standard economic theory. However, if the price difference
is bigger than what \gls{ppp} adjustments would predict, it creates a "super-normal" arbitrage
incentive—a price gap that motivates bypassing beyond simple purchasing power factors. We
measure this through the \gls{dspi}.

\section{Consumer Circumvention and the Piracy Parallel}
\label{sec:theory_piracy}

Consumer-driven arbitrage is not new \parencite{geda2023puzzle}. The digital "geo-arbitrage"
pattern can be understood by looking at the history of digital piracy.

\subsection{The Piracy Analogue}
\textcite{oberholzer2007effect} showed that file-sharing forced the music and film industry to change
its business model, eventually leading to legitimate digital distribution platforms like
iTunes, Spotify and Netflix. This historical example helps us understand \gls{vpn}-based geo-arbitrage.
Similarly, geo-arbitrage works as a market signal, showing a basic mismatch between rigid
regional pricing structures and the borderless reality of the global internet.

The parallel is instructive: just as Napster and BitTorrent exposed the music industry's
failure to meet consumer demand for convenient digital access, \gls{vpn}-enabled price hopping
exposed the sustainability problems of global price discrimination. In both cases, 
users innovated before companies adapted.
As we discuss later (Chapter \ref{chap:discussion}), companies are likely getting better at 
finding where users really are. However, it is hard to test this because the technical details 
are secret, similar to the piracy case of \gls{drm} protection.

However, a key difference emerges between the two:
\begin{itemize}
    \item \textit{Access vs. Price:} Piracy was often about getting content that wasn't 
    available at all, while geo-arbitrage is about getting a better price. Piracy involved 
    no payment; geo-arbitrage involves paying, but at a lower price.
    \item \textit{Legal Status:} Piracy is clearly illegal, while geo-arbitrage is a gray zone. 
    Users are paying for real accounts; they are just lying about where they are.
    \item \textit{Industry Response:} The music industry eventually adapted with streaming services. Whether companies will do the same for geo-arbitrage is what this thesis looks at.
\end{itemize}


\subsection{The Three-Level Mechanism of Circumvention}
Drawing from behavioral ethics literature and the work of \textcite{wang2014three} on digital
piracy, the decision to engage in geo-arbitrage can be modeled as a three-level mechanism.
This way of looking at it helps explain why otherwise law-abiding consumers engage in 
``digital smuggling'':

\begin{enumerate}
    \item \textbf{Individual Level (Rational Choice / Personal Risk):} The consumer performs a cost-benefit analysis. The financial gain (e.g., a 70\% discount on Netflix Turkey) is weighed against the perceived probability of detection and the severity of punishment (e.g., account termination). When enforcement is perceived as inconsistent, the perceived risk may be low.
    \item \textbf{Inter-personal Level (Social Influence):} The behavior may be reinforced by online communities (e.g., Reddit, Discord). Observing others successfully using \glspl{vpn} can lower the psychological barrier to entry, consistent with social influence research \parencite{kastanakis2012between}.
    \item \textbf{Societal Level (Moral Intensity):} The perception of the act is pivotal. Unlike shoplifting a physical good, digital arbitrage may be framed by users not as theft, but as a response to pricing perceived as unfair. This framing aligns with neutralization theory \parencite{mateus2018business}.
\end{enumerate}

\section{Strategic Management and Business Model Innovation}
\label{sec:theory_strategy}

Faced with this disruption, firms must adapt. We analyze their responses using \gls{bmi}. As defined by \textcite{wirtz2016business} and grouped by
\textcite{foss2017fifteen}, \gls{bmi} means rethinking the value offer and delivery methods in
response to outside shocks.

\subsection{Dimensions of Business Model Innovation}
To carefully analyze how firms adapt, we divide their business models into three parts:

\begin{enumerate}
    \item \textbf{Value Proposition (What is offered):} The core product or service and the bundle of benefits it provides to the customer. In digital streaming, this is the content library and the convenience of "watch anywhere" access.
    \item \textbf{Value Delivery (How it is reached):} The channels and technical infrastructure used to deliver the value. This includes the streaming platform, the Content Delivery Network (CDN), and the user interface. Crucially, it also includes the \textit{geographic segmentation} logic that determines who can access what.
    \item \textbf{Value Capture (How money is made):} The revenue model and the mechanisms to sustain profitability. This encompasses the pricing strategy (e.g., price discrimination) and the enforcement mechanisms used to prevent revenue leakage (e.g., blocking arbitrage).
\end{enumerate}

\gls{vpn}-enabled arbitrage fundamentally attacks the \textbf{Value Capture} dimension by breaking the
link between location and price. It also exploits the \textbf{Value Delivery} infrastructure (the
open internet).
Consequently, we hypothesize that firm responses will fall into two categories of innovation:
\begin{itemize}
    \item \textbf{Defensive Innovation (Value Capture Focus):} Reinforcing the barriers to protect the existing model (e.g., "Coercive" blocking).
    \item \textbf{Adaptive Innovation (Value Proposition Focus):} Changing the product offer to make arbitrage irrelevant (e.g., "Adaptive" global pricing or ecosystems).
\end{itemize}

\subsection{Theoretical Framework: Protection vs. Pricing}
The intersection of digital strategy and arbitrage has been extensively debated.
\textcite{johnson2008reinventing} define the necessity of business model reinvention when
facing disruptive shifts, while \textcite{granados2010electronic} show how e-commerce
increases market efficiency by making arbitrage easier. However,
\textcite{geda2023puzzle} note that this arbitrage often creates game-theoretic puzzles for
firms, leading to complex responses such as those described by \textcite{mateus2018business}
in the context of digital piracy. Furthermore, \textcite{beunza2004price} argue that price is
ultimately a social construct, heavily influenced by the "material sociology" of the market—in
this case, the \gls{vpn} technology that alters the visibility of the consumer.

To categorize firm responses, we adopt the framework established by
\textcite{sundararajan2004managing} on managing digital piracy, mapping it to our \gls{bmi}
dimensions:
\begin{itemize}
    \item \textbf{Protection (Coercive / Value Capture):} Increasing the technological or legal costs of circumvention. This attempts to *repair* the broken Value Capture mechanism.
    \item \textbf{Pricing (Adaptive / Value Proposition):} Adjusting the business model (pricing, versioning) to lower the economic incentive for arbitrage. This effectively *innovates* the Value Proposition to be less sensitive to location.
\end{itemize}
Firms face a fundamental trade-off: Is the cost of enforcing market segmentation (repairing
Value Capture through blocking technology and legal resources) lower than the revenue lost to
arbitrage?

\subsection{Platforms and Ecosystem Control}
Digital platforms operate within a fundamental tension between growth and control. To attract
users and content creators, platforms must maintain a degree of openness that facilitates
participation and innovation. However, to protect revenue streams and maintain quality,
platforms must also exercise control over who accesses what content and at what price point.

\gls{vpn} providers exploit this tension. They use the platform's content (e.g.,
Netflix's streaming library) while bypassing its payment rules (regional pricing). This
creates a technical and strategic cycle of countermeasures and counter-countermeasures:
\begin{itemize}
    \item \textbf{Coercive Strategies:} Legal threats embedded in \gls{tos}, IP address blocking, payment verification requirements, and strict geographic checks on billing addresses.
    \item \textbf{Adaptive Strategies:} Standardizing global prices to remove the arbitrage incentive, creating ecosystem lock-in through hardware integration (e.g., Apple's approach), or developing content exclusive to specific regions rather than restricting access to a global catalog.
\end{itemize}

Critically, \textcite{parker2017innovation} demonstrate that platforms face an inherent
tension between openness (which drives innovation and user growth) and control (which protects
revenue and quality). Their framework suggests that the optimal balance point shifts depending
on platform maturity and competitive dynamics. \gls{vpn} arbitrage directly exploits this
fundamental trade-off, forcing platforms to reassess where that balance lies.

The strategic implications are significant: platforms that choose aggressive blocking may
sacrifice user experience and brand perception, while those that tolerate arbitrage may face
revenue leakage. Neither approach is without cost, and the optimal strategy likely depends on
the specific business model and competitive context of each platform.


\section{Research Gap}
\label{sec:theory_gap}

While price discrimination theory (Varian) and platform strategy (Eisenmann et al., 2011) have
been extensively researched independently, there remains a notable gap in empirical work
connecting the \textit{size} of pricing incentives (as measured by indices like the \gls{dspi})
with the \textit{specific strategic responses} of firms.

Existing literature exhibits three main limitations:

\begin{enumerate}
    \item \textbf{Theoretical Isolation:} Most studies focus either exclusively on the economics of pricing (e.g., optimal price discrimination strategies) or on the legal aspects of copyright enforcement and digital rights management, but rarely examine the strategic interaction between these domains as mediated by consumer-side technology such as \glspl{vpn}.
    
    \item \textbf{Lack of Quantification:} While anecdotal evidence of geo-arbitrage is abundant in consumer forums and technology journalism, systematic quantification of the arbitrage incentive across services and regions is lacking. The \gls{dspi} addresses this gap by providing a standardized measurement framework.
    
    \item \textbf{Limited Strategic Analysis:} Previous research on digital piracy has examined how firms respond to unauthorized copying, but the distinct characteristics of geo-arbitrage (payment rather than piracy, location rather than access) warrant specialized investigation. The coercive-adaptive split proposed in this thesis provides a framework for categorizing these responses.
\end{enumerate}

This thesis addresses these gaps by: (1) measuring the arbitrage incentive via the \gls{dspi}; (2) 
analyzing company documents to find enforcement strategies; and (3) linking price differences to 
blocking intensity. This study gives a full picture of geo-arbitrage that has been missing 
before.

\chapter{Research Methodology}
\label{chap:methodology}

This chapter explains the methods used to answer the research questions. We describe how we
built the \gls{dspi} to measure price differences and how we used AI to
analyze how companies respond strategically. By using both numbers and text analysis, we get a
clear picture of the geo-arbitrage world.

\section{Research Design}

This study uses a sequential explanatory mixed-methods design
\parencite{creswell2017designing}, combining quantitative price analysis with qualitative text
classification. The reason for this dual approach is to first show the \textit{size} of the
economic problem (the arbitrage incentive) and then look at the \textit{strategic responses}
of the actors involved.

The quantitative phase (Phase 1) builds the \gls{dspi} to
objectively measure differences in global digital service pricing. The qualitative phase
(Phase 2) uses a \gls{llm} pipeline to classify corporate disclosures and
\gls{tos}, finding the strategic frameworks firms use to manage or fight this variance.
This integration provides a comprehensive understanding of the geo-arbitrage ecosystem based 
on public documentation, though proprietary information and undisclosed technologies remain hidden.

\section{Phase 1: Quantitative Data Collection (for RQ1)}

\subsection{Data Collection}
To construct the \gls{dspi}, a representative basket of 11 digital services was selected: Netflix,
YouTube Premium, Disney+, Amazon Prime, Spotify, Apple Music, Microsoft 365, Adobe Creative
Cloud, Xbox Game Pass, NordVPN, and ExpressVPN. These cover Video on Demand, Music Streaming,
Software/Gaming, and \gls{vpn} services. Note that for the qualitative \gls{tos} analysis, Xbox Game Pass
and Microsoft 365 are combined under ``Microsoft'' as they share the same corporate ecosystem,
parent \gls{tos} framework, and enforcement infrastructure.

Price data was collected from a sample of 11 countries to capture the full spectrum of
purchasing power. The countries included are: Argentina, Brazil, Germany, Pakistan,
Philippines, Poland, Switzerland, Turkey, Ukraine, United Kingdom, and the United States. Only
countries with high-confidence official wage data were included.

Data collection was performed using a \textbf{Digital Audit} design, adapting the methodology
established by \textcite{hannak2014measuring} for detecting online price discrimination. A
virtual presence was established in each target country using a commercial \gls{vpn} service to
simulate local access, a technique now standard in information systems research for "mystery
shopping" in digital markets. For each service and country, the monthly "Standard"
subscription price was recorded in local currency. This approach mirrors the methodology of
the "Billion Prices Project" \parencite{cavallo2017are}, which demonstrated the validity of
using high-frequency online scraping to construct robust price indices that track real-time
economic disparities more effectively than traditional CPI baskets.

\subsection{Data Analysis}
The raw price data was processed in two stages. First, all local prices were converted to a
common currency (USD) using market exchange rates (recorded in December 2025) to find the
``Nominal Price Inequality.'' Second, to measure ``Real Affordability,'' these prices were
calculated as a percentage of the \textit{Median National Monthly Wage} (sourced from OECD and
World Bank data), giving a direct measure of the economic burden on the local consumer
(\gls{ptw} ratio). 
This method reflects the ``Time-to-Earn'' required for each 
subscription, mirroring the conceptual framework of the affordability-adjusted Big Mac Index 
\parencite{pakko2003burgernomics}. This approach is proposed as a more targeted alternative 
to standard global \gls{ppp} adjustments for three reasons:
\begin{enumerate}
    \item \textbf{Disposable Income focus:} Unlike standard \gls{ppp}, which includes non-discretionary costs like rent and food, the \gls{ptw} ratio specifically shows how much of a worker's ``free cash'' is consumed by a single digital subscription.
    \item \textbf{Psychological Fairness (Time-to-Earn):} Arbitrageurs and consumers rarely consider complex macroeconomic baskets of goods; instead, they intuitively weigh a service's cost against the number of hours they must work to afford it.
    \item \textbf{The Digital Paradox:} Since digital services have near-zero marginal costs ($MC \approx 0$), firm pricing is not driven by the cost of production (as in agricultural or industrial goods in a \gls{ppp} basket) but solely by the consumer's income-constrained willingness to pay.
\end{enumerate}
By using the \gls{ptw} ratio, this study isolates the affordability of globally identical 
digital goods relative to local disposable income, providing a clearer view of the 
strategic intent behind geographic price discrimination.

It is important to note that a \gls{dspi} of 1.0 (Nominal Parity) does not imply equal
affordability. Due to vast differences in median wages (e.g., Switzerland vs. Pakistan), a
service priced identically in USD would be significantly more expensive for the Pakistani
consumer in real terms (requiring a larger percentage of their income). Thus, the arbitrage
incentive persists even at nominal parity if the local price is structured to be affordable
for the local median earner.

The \gls{dspi} was calculated as the ratio of the local price to the US baseline price. A \gls{dspi} of
1.0 indicates price parity with the US market; a \gls{dspi} < 1.0 indicates a cheaper market
(potential arbitrage source), and a \gls{dspi} > 1.0 indicates a more expensive market. Statistical
variance analysis was performed to identify which service categories exhibit the highest
degree of price discrimination.

\section{Phase 2: Qualitative Data Collection \& Analysis (for RQ2)}

\subsection{How we coded the data}
We analyzed the data using a system inspired by the \textbf{Gioia Methodology}
\parencite{gioia2013seeking}. This system divides the text into groups: raw terms, second-level
themes (like ``Technical Blocking''), and larger categories (Strategic Responses). This gave
us the basis for the AI sorting process described below.

\section{Automated Text Classification}
\label{sec:llm_methodology}

To address the limits of older models in understanding the complex and legal language of \gls{tos}, we used an advanced system with the latest \glspl{llm}. We 
upgraded from a BERT model to the \textit{Gemini 3 Flash} model.

\subsection{Why we chose Gemini 3 Flash}
We chose \textit{Gemini 3 Flash} because it is better at understanding complex text and 
context. Older models often classify text based only on how words link together, but Gemini 
can understand the meaning of whole sentences. This helps it tell the difference between 
standard legal text and specific blocks.

The main benefits were:
\begin{itemize}
    \item \textbf{Understanding Context}: It can tell the difference between a normal ``account suspension'' (for fraud) and a specific threat meant to stop \gls{vpn} use.
    \item \textbf{Performance}: The model was very accurate without extra training.
    \item \textbf{Speed}: The ``Flash'' model is very fast, letting us process over 25,000 sentences quickly.
\end{itemize}

\subsection{How we measured things (The Categories)}
This system turns the idea of ``Strategic Response'' into data we can count. We used the 
categories described below:

\subsubsection{Strategic Frames}
The model identifies the underlying justification provided by the firm:
\begin{description}
    \item[Content Licensing:] Framing geographic restrictions as mandatory legal obligations to third-party content owners or rights holders.
    \item[Regulatory Compliance:] Framing price differences or access blocks as necessary to comply with local tax, trade, or data laws.
    \item[Security Risk:] (Service Provider Frame) Positioning \glspl{vpn} and proxies as threats to account security, malware risks, or malicious activity.
    \item[Privacy/Security:] (\gls{vpn} Provider Frame) Positioning circumvention tools as essential for user anonymity, encryption, and protection from surveillance.
\end{description}

\subsubsection{Strategic Actions and Baseline Classification}
The model categorizes the specific implementation of the strategy into the following labels. To ensure reliability, we included a "Null" category to capture all non-strategic text:

\begin{description}
    \item[General Terms (Baseline):] Standard legal boilerplate, general definitions, and non-enforcement-related clauses. This serves as the primary filter for the dataset.
    \item[Technical Blocking:] Active technological measures (IP filtering, device fingerprinting) to detect or block the use of \glspl{vpn} and proxies.
    \item[Legal Threat:] Explicit clauses threatening account termination, suspension, or legal consequences for bypassing geographic barriers.
    \item[Account Action:] Specific regulatory actions taken against accounts, such as termination or suspension (often merged into Legal Threat in the final strategy analysis).
    \item[Price Discrimination:] Explicit documentation of regional pricing tiers, currency differences, or local subscription rules.
    \item[Legitimate Portability:] Provisions allowing temporary cross-border access (e.g., for travelers), often mandated by regulations like the \gls{eu} Portability Regulation.
\end{description}

\subsection{How we built the system}
We used a Python script to sort the data automatically.

To make sure the AI worked the same way every time, we built a system prompt with clear rules. 
The prompt we used is shown below:

\begin{figure}[htbp]
    \centering
    \begin{lstlisting}[language=Python]
SYSTEM_PROMPT = """You are a scientific classifier.
CATEGORIES:
1. Technical Blocking: Measures/Technologies used to detect or block the specific use of VPNs/Proxies.
2. Legal Threat: Explicit threats of account termination, suspension, or legal action for using circumvention tools.
3. Price Discrimination: Differences in pricing based on region, currency, or purchasing power.
4. Content Licensing: Geographic restriction of content availability (e.g. 'not available in your region') due to rights.
5. Legitimate Portability: Rules allowing temporary access while traveling (e.g. EU Portability Regulation).
6. Regulatory Compliance: References to local laws, tax/VAT compliance, or export controls.
7. User Workaround: Descriptions of users bypassing restrictions (using VPNs, changing store region).
8. Security Risk: (Service Provider Frame) Arguments that VPNs/Proxies are unsafe, malicious, or compromise user data.
9. Privacy/Security: (VPN Provider Frame) Arguments focusing on encryption, anonymity, and protection from surveillance.
10. General Terms: Standard legal text, general marketing, or unrelated content.

INSTRUCTIONS:
- Return a JSON array of objects for the sentences in EXACT order.
- Format: [ { "category": "Category Name", "confidence": 0.9 }, ... ]
"""
\end{lstlisting}
    \caption{System Prompt used for Gemini 3 Flash Classification}
    \label{fig:system_prompt}
\end{figure}

\subsubsection{Handling Batches and Errors}
To work within the AI's limits and keep the data correct, we sent sentences in groups of 25. 
This made the process faster. We also built a way to restart the script if it hit an error or 
hit a speed limit.

\subsubsection{How we handled missing data over time}
Company documents don't change every year. To handle this, we assumed that if a rule was added 
in 2020, it stayed the same until the next time the company updated its terms. This ensures the 
data correctly shows the rules that were active each year.

Additionally, while "General Terms" (standard legal boilerplate) constitute approximately 94\%
of the dataset, they are retained in the dataset to preserve the full document
structure. However, for visual clarity, these terms are frequently excluded from strategic
trend graphs to focus on the distinct enforcement categories.

\subsection{Methodological Validation: The Transition from BERT to Gemini}
\label{sec:bert_comparison}
To validate the model selection, the study compared a traditional Zero-Shot Natural Language Inference (NLI) approach using \textbf{DeBERTa-v3-large} (a state-of-the-art BERT-based architecture) against the \textbf{Gemini 3 Flash} pipeline. This comparison was essential to justify the move toward more resource-intensive \glspl{llm}.

\subsubsection{Results of the Comparison}
The two models exhibited a fundamental disconnect, agreeing on only \textbf{26.8\%} of the dataset. The resulting Cohen's Kappa score was \textbf{0.032}, indicating agreement equivalent to random chance. This highlighted a significant discrepancy in how modern generative models and traditional encoder-only models interpret complex legal text.

\subsubsection{The Core Conflict: Sensitivity vs. Context}
The analysis revealed that the primary source of error was the difference in how each model handles "Boilerplate" versus "Strategic Action":

\begin{enumerate}
    \item \textbf{BERT's Baseline Failure (Over-Sensitivity):} The BERT-based model frequently assigned specific strategic tags (like \textit{Account Action} or \textit{Legitimate Portability}) based on the presence of individual keywords rather than semantic context. For example:
    \begin{itemize}
        \item It flagged 7,853 sentences as "Legitimate Portability" that were actually "General Terms."
        \item It flagged 6,134 "General Terms" sentences as "Account Action" simply because the word "account" was present.
    \end{itemize}
    \item \textbf{Gemini's Contextual Correction:} In contrast, Gemini used reasoning to distinguish between benign boilerplate and active enforcement. It correctly identified that \textbf{91\%--94\%} of the dataset consisted of "General Terms." By filtering this noise, Gemini allowed for a high-precision analysis of the remaining 6\% of truly strategic clauses.
\end{enumerate}

\textit{Interpretation:} BERT operates on keyword connections—if it sees "account," it predicts "Account Action." Gemini’s context window allows it to understand that a sentence like "You must have an account" is standard boilerplate, whereas "We may terminate your account for using a VPN" is a strategic \textit{Legal Threat}. This ability to filter "General Terms" is what makes the Gemini pipeline reliable for this study.

\subsection{Conclusion on Model Selection}
The validation shows that Zero-Shot BERT is not enough for complex legal text analysis, 
as it lacks the detail needed to tell the difference between just
mentioning a topic (e.g., "portability") and its active regulation. Table
\ref{tab:model_comparison} and Figure \ref{fig:model_comparison_viz_latex} provide a detailed
breakdown of the category distribution discrepancies.

\begin{table}[ht]
    \centering
    \small
    \begin{tabularx}{\textwidth}{l c c c c}
        \toprule
        \textbf{Category} & \textbf{Gemini \%} & \textbf{BERT \%} & \textbf{Delta} \\
        \midrule
        Technical Blocking & 0.41\% & 0.09\% & +0.32\% \\
        Price Discrimination & 0.48\% & 0.03\% & +0.45\% \\
        Content Licensing & 2.18\% & 5.76\% & -3.58\% \\
        Regulatory Compliance & 2.05\% & 0.41\% & +1.64\% \\
        Legal Threat & 0.47\% & 0.00\% & +0.47\% \\
        Account Action & 0.00\% & 25.89\% & -25.89\% \\
        Privacy/Security & 0.11\% & 0.00\% & +0.11\% \\
        Security Risk & 0.18\% & 0.00\% & +0.18\% \\
        Legitimate Portability & 0.01\% & 31.99\% & -31.98\% \\
        User Workaround & 0.00\% & 9.71\% & -9.71\% \\
        General Terms & 94.12\% & 26.12\% & +68.00\% \\
        \bottomrule
    \end{tabularx}
    \caption{Model Comparison: Gemini 3 Flash vs. Zero-Shot BERT Classification}
    \label{tab:model_comparison}
\end{table}




% INSERTED ARCHIVE CONTENT START
\section{The Landscape of International Pricing: Findings from the DSPI}
\label{sec:dspi_results}

To understand the economic reason driving firm strategic behavior, we first analyze the quantitative pricing landscape using the Digital Services Price Index (DSPI).

\subsection{Magnitude of the Arbitrage Incentive}
The data shows large price differences across markets. For example, subscriptions in Turkey or Argentina can cost 70-80\% less than the same subscriptions in Switzerland or the USA when measured in nominal USD. This difference creates a "super-normal" profit margin for consumers doing arbitrage, explaining the persistence of this behavior despite the technical barriers analyzed in later sections.



\begin{figure}[ht]
    \centering
    \includegraphics[width=0.9\textwidth]{figures/dspi_heatmap.pdf}
    \caption{Global Heatmap of Digital Service Pricing (The DSPI). Data represents the cost of local subscriptions relative to the US baseline (DSPI=1.0). Lower values indicate stronger arbitrage incentives.}
    \label{fig:dspi_map}
\end{figure}

\begin{table}[ht]
    \centering
    \small
    \renewcommand{\arraystretch}{1.1}
    \begin{threeparttable}
    \begin{tabular}{l|ccccc|ccccc|c}
        \toprule
        \textbf{Service} & \rotatebox{90}{Switzerland} & \rotatebox{90}{USA} & \rotatebox{90}{Germany} & \rotatebox{90}{UK} & \rotatebox{90}{Poland} & \rotatebox{90}{Turkey} & \rotatebox{90}{Argentina} & \rotatebox{90}{Brazil} & \rotatebox{90}{Ukraine} & \rotatebox{90}{Philippines} & \rotatebox{90}{Pakistan} \\
        \midrule
        Netflix & 1.44 & 1.00 & 0.85 & 0.78 & 0.68 & 0.52 & 1.00 & 0.50 & 0.45 & 0.45 & \textbf{0.16} \\
        YouTube Premium & 1.45 & 1.00 & 1.01 & 1.18 & 0.70 & \textbf{0.17} & 0.74 & 0.36 & 0.18 & 0.24 & \textbf{0.12} \\
        Disney+ & 1.47 & 1.00 & 0.92 & 1.07 & 0.67 & 1.11 & 1.14 & 0.72 & -- & 0.35 & -- \\
        Amazon Prime & 0.75 & 1.00 & 0.65 & 0.76 & \textbf{0.18} & \textbf{0.15} & 0.64 & 0.27 & 0.51 & 0.18 & \textbf{0.14} \\
        Spotify & 1.50 & 1.00 & 1.18 & 1.27 & 0.50 & 0.26 & 0.33 & 0.40 & 0.42 & 0.25 & \textbf{0.10} \\
        Apple Music & 1.43 & 1.00 & 1.09 & 1.27 & 0.50 & \textbf{0.17} & 0.65 & 0.40 & 0.45 & 0.23 & -- \\
        Microsoft 365 & 1.13 & 1.00 & 1.08 & 1.08 & 1.08 & 1.06 & 0.45 & 1.02 & 0.70 & 0.88 & 0.83 \\
        Adobe CC & 1.24 & 1.00 & 1.21 & 1.21 & 1.26 & 0.74 & 1.01 & 0.61 & 0.58 & 0.98 & 1.00 \\
        Xbox Game Pass & 1.13 & 1.00 & 0.98 & 0.89 & 0.93 & 0.86 & 1.08 & 0.88 & 0.75 & 0.58 & \textbf{0.18} \\
        NordVPN & 1.09 & 1.00 & 1.09 & 0.97 & 0.85 & 1.01 & 0.44 & 0.44 & 1.01 & 0.90 & 0.81 \\
        ExpressVPN & 0.99 & 1.00 & 1.02 & 1.00 & 1.13 & 1.10 & 0.92 & 0.92 & 1.10 & 0.92 & 0.73 \\
        \bottomrule
    \end{tabular}
    \begin{tablenotes}
        \small
        \item \textit{Note:} DSPI values represent local price relative to US baseline (1.00). Bold values indicate strongest arbitrage opportunities ($<0.20$). Data collected December 2025.
    \end{tablenotes}
    \end{threeparttable}
    \caption{Digital Services Price Index (DSPI) by Service and Country}
    \label{tab:dspi_full}
\end{table}

\subsection{The Affordability Paradox: Nominal vs. Real Cost}
While nominal price differences create arbitrage incentives for Western users, it is crucial to understand the "Real Cost" for local users. Figure \ref{fig:affordability} maps the cost of digital services as a percentage of the \textbf{Median National Monthly Wage}.

\begin{figure}[ht]
    \centering
    \includegraphics[width=0.9\textwidth]{figures/affordability_heatmap.pdf}
    \caption{The Affordability Gap: Digital Service Cost as Percentage of Local Monthly Income. Darker red indicates higher relative cost for local citizens.}
    \label{fig:affordability}
\end{figure}

The data reveals a paradox: while Turkey and Argentina offer the cheapest nominal prices for international arbitrageurs (< \$4/month), these services are significantly \textit{more expensive} for locals in real terms. For instance, a Standard Netflix subscription in Turkey consumes a higher percentage of the median monthly wage ($\approx 0.6\%$) compared to the USA ($\approx 0.3\%$). This suggests that low nominal prices are not "discounts" but necessary adjustments to local purchasing power, which external actors then exploit.

\section{Classification Results: Strategic Framing}
\label{sec:classification_results}

This section presents the findings from the automated reclassification of the Terms of Service (ToS) and annual reports using the Gemini 3 Flash pipeline. The analysis processed a total of 25,593 sentences across the dataset.
 
 \subsection{Distribution of Enforcement Categories}
 The classification showed that \textbf{94.1\%} of the sentences were "General Terms" (legal boilerplate). While this high volume of standard legal text confirms the structural integrity of the documents, the analysis below focuses primarily on the remaining \textbf{5.9\%} of "Strategic Sentences" that contain active enforcement clauses. "General Terms" are excluded from trend visualizations to maintain readability. Table \ref{tab:category_dist} shows how different providers approach this.

\begin{table}[ht]
    \centering
    \begin{tabularx}{\textwidth}{l X c c}
        \toprule
        \textbf{Category} & \textbf{Description} & \textbf{Freq (N)} & \textbf{Freq (\%)} \\
        \midrule
        Content Licensing & Geographic restrictions based on rights. & 562 & 37.22\% \\
        Regulatory Compliance & Local laws/tax compliance. & 528 & 34.97\% \\
        Price Discrimination & Explicit regional pricing rules. & 120 & 7.95\% \\
        Legal Threat & Explicit threats of termination/legal action. & 120 & 7.95\% \\
        Technical Blocking & Active detection/blocking technology. & 108 & 7.15\% \\
        Security Risk & Risks of VPN usage (Service Prov. Frame). & 42 & 2.78\% \\
        Privacy/Security & Encryption/Anonymity (VPN Frame). & 27 & 1.79\% \\
        Legitimate Portability & EU Portability Regulation clauses. & 2 & 0.13\% \\
        User Workaround & References to circumventing blocks. & 1 & 0.07\% \\
        \bottomrule
    \end{tabularx}
    \caption{Distribution of Strategic Categories in ToS Documents}
    \label{tab:category_dist}
\end{table}

\begin{figure}[ht]
    \centering
    \begin{subfigure}{0.48\textwidth}
        \includegraphics[width=\textwidth]{figures/strategic_frames_dist.pdf}
        \caption{Strategic Frames Distribution}
        \label{fig:frame_dist}
    \end{subfigure}
    \hfill
    \begin{subfigure}{0.48\textwidth}
        \includegraphics[width=\textwidth]{figures/global_priority_shift.pdf}
        \caption{Global Priority Shift (\%)}
        \label{fig:priority_shift}
    \end{subfigure}
    \caption{Global Landscape of Enforcement: Distribution and Temporal Priority Shift.}
    \label{fig:global_summary}
\end{figure}

The \textbf{Global Priority Shift} (Figure \ref{fig:priority_shift}) shows a relative increase in the importance of \textit{Technical Blocking} language compared to other categories over the last three years. This trend is further supported by the lexical analysis of the dataset.

\begin{table}[ht]
    \centering
    \small
    \begin{tabular}{l c | l c}
        \toprule
        \textbf{Keyword} & \textbf{Frequency} & \textbf{Keyword} & \textbf{Frequency} \\
        \midrule
        location & 128 & circumvention & 26 \\
        youtube & 113 & piracy & 25 \\
        determine & 46 & distribution & 25 \\
        google & 37 & protection & 25 \\
        unauthorized & 29 & verify & 25 \\
        detection & 28 & accessibility & 25 \\
        monitor & 27 & monitor & 27 \\
        \bottomrule
    \end{tabular}
    \caption{Top Strategically Relevant Keywords Identified by Gemini 3 Flash}
    \label{tab:keywords}
\end{table}

The prevalence of keywords like ``location,'' ``determine,'' and ``detection'' underscores the shift toward active monitoring as a core enforcement strategy. Table \ref{tab:quotes} provides verbatim examples of how these concepts are operationalized.

\begin{table}[ht]
    \centering
    \renewcommand{\arraystretch}{1.2}
    \small
    \begin{tabularx}{\textwidth}{l X l l l c}
        \toprule
        \textbf{Category} & \textbf{Quote} & \textbf{Service} & \textbf{Year} & \textbf{Doc} & \textbf{Conf} \\
        \midrule
        Content Licensing & "We grant you a limited... license... only within geographic locations..." & Netflix & 2023 & ToS & 0.98 \\
        \addlinespace
        Technical Blocking & "You may not use any technology to obscure or disguise your location." & Disney+ & 2024 & ToS & 0.95 \\
        \addlinespace
        Legal Threat & "We reserve the right to terminate... without notice, if we suspect violation." & Spotify & 2022 & ToS & 0.92 \\
        \addlinespace
        Price Discrimination & "Prices may vary by country... charged in currency of location." & Steam & 2024 & ToS & 0.89 \\
        \bottomrule
    \end{tabularx}
    \caption{Representative Clauses for Detected Enforcement Strategies}
    \label{tab:quotes}
\end{table}

\subsection{Service-Specific Analysis}
The enforcement strategies vary significantly across different service providers, reflecting their distinct business models and regional licensing constraints. Figure \ref{fig:service_dist} illustrates the proportional distribution of categories for each service. 

\subsubsection{Strategic Framing by Digital Service Providers}
The qualitative analysis highlights a distinct "Coercive" framing strategy employed by digital service providers. The dominant rhetorical frame, appearing in over \textbf{37.2\%} of non-boilerplate sentences (see Table \ref{tab:category_dist}), is \textbf{Content Licensing}. Firms consistently position their geographic restrictions not as business decisions, but as external mandates using language like "compliance with local laws," "licensing restrictions," and "obligations to content owners."
The second most dominant frame is \textbf{Regulatory Compliance} (\textbf{35.0\%}), reinforcing this narrative of external obligation.

Content licensing services like \textbf{Disney+} and \textbf{Netflix} exhibit this most strongly, dedicating significant portions of their terms to defining geographic rights. In contrast, global platforms like \textbf{Amazon} show notable spikes in Regulatory Compliance.

\begin{figure}[ht]
    \centering
    \includegraphics[width=\textwidth]{figures/service_distribution_ratios.pdf}
    \caption{Proportional Distribution of Enforcement Categories by Service}
    \label{fig:service_dist}
\end{figure}

To further quantify the intensity of these enforcement regimes, we propose the \textbf{Fortress Index}, a metric that calculates the percentage of enforcement-related clauses (Technical Blocking and Legal Threat) relative to the total number of sentences in a firm's documentation. Table \ref{tab:fortress_index} illustrates the stark divide between actors in the geo-arbitrage ecosystem.

\begin{table}[ht]
    \centering
    \small
    \begin{tabularx}{\textwidth}{l X c}
        \toprule
        \textbf{Service Provider} & \textbf{Strategic Archetype} & \textbf{Fortress Score (\%)} \\
        \midrule
        ExpressVPN & VPN Enabler & 55.56 \\
        NordVPN & VPN Enabler & 50.00 \\
        YouTube Premium & Content Provider & 34.34 \\
        Microsoft & Software/Access & 32.76 \\
        Apple Music & Content Provider & 12.50 \\
        Adobe & Software/Utility & 5.71 \\
        Amazon Prime & Global Platform & 2.94 \\
        Disney+ & Content Provider & 2.04 \\
        Netflix & Content Provider & 2.03 \\
        Spotify & Content Provider & 0.43 \\
        \bottomrule
    \end{tabularx}
    \caption{The Fortress Index: Percentage of Enforcement Clauses per Service}
    \label{tab:fortress_index}
\end{table}

The index reveals that VPN providers like \textbf{ExpressVPN} and \textbf{NordVPN} have the highest density of relevant clauses, as their entire documentation is focused on security and circumvention. Among digital services, \textbf{YouTube} and \textbf{Microsoft} exhibit significantly higher physical "fortress" density than \textbf{Netflix} or \textbf{Spotify}, suggesting a more aggressive or complex regulatory approach to user location.

\subsubsection{Strategic Framing by VPN Providers}
In sharp contrast, VPN companies adopt a "Liberation" and "Privacy" frame. The analysis reveals a consistent narrative that reframes circumvention as \textbf{User Freedom}. 
A secondary dominant frame identified in our analysis is \textbf{Privacy/Security}. While many users may purchase VPNs for streaming arbitrage, providers legitimize the service by emphasizing security features. \textbf{NordVPN}, for example, shows a distinct focus on ``Security Risk'' categories in our dataset, with marketing materials framing this as empowering users against tracking.

\subsection{Temporal Evolution of Enforcement}
To understand how these strategies have evolved over time, we analyzed the frequency of category-specific clauses across the dataset's years. Table \ref{tab:timeline_count} shows the raw count of enforcement-related incidents detected per service per year.

\begin{table}[ht]
    \centering
    \small
    \begin{tabular}{l|ccccccccc}
        \toprule
        \textbf{Service} & \textbf{2016} & \textbf{2018} & \textbf{2020} & \textbf{2021} & \textbf{2022} & \textbf{2023} & \textbf{2024} & \textbf{2025} \\
        \midrule
        Adobe & 0 & 0 & 2 & 0 & 0 & 0 & 1 & 1 \\
        Amazon Prime & 0 & 0 & 1 & 0 & 0 & 0 & 0 & 2 \\
        Apple Music & 0 & 0 & 0 & 0 & 1 & 4 & 1 & 3 \\
        Disney+ & 0 & 0 & 0 & 0 & 0 & 0 & 4 & 0 \\
        ExpressVPN & 0 & 0 & 0 & 0 & 0 & 0 & 0 & 5 \\
        Microsoft & 0 & 0 & 9 & 7 & 9 & 12 & 6 & 6 \\
        Netflix & 0 & 0 & 1 & 1 & 0 & 0 & 0 & 3 \\
        NordVPN & 0 & 0 & 0 & 0 & 0 & 0 & 3 & 0 \\
        Spotify & 0 & 0 & 0 & 0 & 0 & 0 & 0 & 0 \\
        YouTube Premium & 8 & 7 & 1 & 0 & 16 & 53 & 31 & 20 \\
        \bottomrule
    \end{tabular}
    \caption{Raw Count of Enforcement Incidents per Service (2016--2025)}
    \label{tab:timeline_count}
\end{table}

The data shows a significant increase in specific enforcement clauses, especially from 2022 onwards, most notably for \textbf{YouTube Premium}. This suggests that restrictive clauses have become more prevalent and more specific over the analyzed period, transitioning from general boilerplate to active regulatory language.

\begin{figure}[ht]
    \centering
    \includegraphics[width=0.9\textwidth]{figures/timeline_all_total.pdf}
    \caption{Temporal Evolution of Category Incident Counts (Aggregate)}
    \label{fig:timeline_all}
\end{figure}

\begin{figure}[ht]
    \centering
    \includegraphics[width=0.9\textwidth]{figures/evolution_strategic_frames_summary.pdf}
    \caption{Evolution of Strategic Frames over Time (Excluding General Terms)}
    \label{fig:strategic_frames_evolution}
\end{figure}

\begin{figure}[ht]
    \centering
    \includegraphics[width=\textwidth]{figures/category_timeline_per_service_normalized.pdf}
    \caption{Temporal Evolution of Category Frequencies by Service (Normalized)}
    \label{fig:timeline_service}
\end{figure}

\section{Deep Dive: Service-Specific Strategic Evolution}
\label{sec:service_deep_dive}

To understand the operational realities of geo-arbitrage enforcement, we analyze the longitudinal patterns of specific providers. The following figures illustrate how individual firms have adapted their Terms of Service to address the growing arbitrage incentive.



The "Content Providers" (\textbf{Netflix}, \textbf{YouTube}, \textbf{Disney+}) show a distinct move toward specialized enforcement clauses starting in 2022. While \textbf{Spotify} remains relatively boilerplate-heavy, \textbf{YouTube Premium} exhibits a massive surge in specific "Technical Blocking" and "Legal Threat" language, corresponding to their increased efforts to combat VPN-enabled subscription hopping in markets like Turkey and Pakistan.



In contrast, software providers like \textbf{Adobe} maintain a consistent, lower level of ToS enforcement language, suggesting a reliance on technical licensing (cryptographic keys) rather than retroactive legal threats. VPN providers (\textbf{NordVPN}, \textbf{ExpressVPN}) show the most dramatic shifts, with their documentation evolving to emphasize encryption and user protection as primary value propositions, effectively reframing circumvention as a fundamental privacy right.

\subsection{High-Confidence Findings: The Core Clauses}
The Gemini 3 Flash model identified specific, high-confidence clauses that are central to the coercive strategy. For example, clauses stating "You may not use any technology to obscure or disguise your location" were consistently categorized as \textit{Technical Blocking} with $>0.95$ confidence. This confirms that firms have made technical countermeasures a formal part of their legal rules.



\subsection{The Affordability Paradox: Real vs. Nominal Cost}
While the DSPI measures the \textit{nominal} price difference (relevant to arbitrageurs), it is crucial to analyze the "Real Cost" for local residents. Figure \ref{fig:affordability_real} maps the cost of digital services as a percentage of the \textbf{Median National Monthly Wage}, serving as a digital equivalent to "Time-to-Earn" indices used in purchasing power comparisons (e.g., the Big Mac Index's affordability variant).

\begin{figure}[ht]
    \centering
    \includegraphics[width=0.9\textwidth]{figures/affordability_heatmap.pdf}
    \caption{The Affordability Gap: Digital Service Cost as Percentage of Local Monthly Income. Darker red indicates higher relative cost for local citizens.}
    \label{fig:affordability_real}
\end{figure}

The data reveals a critical paradox: while Turkey and Argentina offer the cheapest nominal prices worldwide for international arbitrageurs (< \$4/month, DSPI $\approx 0.15$), these same services are significantly \textit{more expensive} for locals in real terms. For instance, a Standard Netflix subscription in Turkey consumes approximately 0.6\% of the median monthly wage compared to approximately 0.2\% in the USA.

This distinction is critical:
\begin{enumerate}
    \item \textbf{High DSPI Variance:} Creates incentives for \textit{external} abuse (VPN Arbitrage).
    \item \textbf{Low Affordability:} Justifies the \textit{internal} pricing strategy (low nominal prices are necessary for market penetration, not optional discounts).
\end{enumerate}

Thus, low nominal prices observed in the Global South are not "bargains" but necessary economic adjustments that inadvertently create vulnerabilities exploited by Global North users.

\section{Correlation Analysis: The Strategic Trade-off}
\label{sec:correlation}
To test the relationship between pricing strategy and enforcement intensity, we used the cleaned dataset to calculate the correlation between Price Discrimination (PD) and observed Enforcement Intensity (EI). Table \ref{tab:correlation_data} summarizes the key metrics.

\begin{table}[ht]
    \centering
    \small
    \begin{tabular}{l c c}
        \toprule
        \textbf{Service} & \textbf{PD Score (DSPI StdDev)} & \textbf{Enforcement Intensity (\%)} \\
        \midrule
        Microsoft & 0.208 & 0.87 \\
        YouTube Premium & 0.464 & 3.07 \\
        Spotify & 0.486 & 0.03 \\
        Adobe & 0.245 & 0.13 \\
        Netflix & 0.352 & 0.17 \\
        Disney+ & 0.324 & 0.18 \\
        Amazon Prime & 0.304 & 0.24 \\
        Apple Music & 0.446 & 0.75 \\
        ExpressVPN & 0.112 & 8.33 \\
        NordVPN & 0.231 & 5.45 \\
        \bottomrule
    \end{tabular}
    \caption{Correlation between Price Discrimination and Enforcement Intensity}
    \label{tab:correlation_data}
\end{table}

\begin{figure}[ht]
    \centering
    \includegraphics[width=0.9\textwidth]{figures/protection_vs_pricing_updated.pdf}
    \caption{Strategic Alignment: Comparison of Price Discrimination scores vs. Enforcement Intensities across analyzed services.}
    \label{fig:correlation}
\end{figure}

The refined analysis ($N=10$) reveals a complex relationship between price variance and enforcement. While the overall global correlation suggests a moderate trade-off, specific sector clusters emerge that show distinct strategic behaviors. This suggests that firms with established global pricing power (like Amazon) rely less on aggressive legal threats than smaller localized services or those in highly contested content markets.

\begin{itemize}
    \item \textbf{Content Providers (Netflix, Disney+, YouTube, Xbox, etc.):} This group effectively forms a "High Enforcement Cluster," but successfully illustrates the enforcement trade-off ($R_{sector} \approx 0.45$). 
    \begin{itemize}
        \item \textbf{High Variance / High Enforcement:} Services like \textbf{Disney+} and \textbf{YouTube} have large global price gaps (DSPI StdDev $>0.37$) and rely on aggressive "Technical Blocking" (6\%--8\%) to maintain them.
        \item \textbf{Low Variance / Low Enforcement (The Xbox Case):} \textbf{Xbox Game Pass} serves as a crucial control. Governed by the Microsoft ecosystem, it has relatively harmonized global pricing (DSPI StdDev $\approx 0.25$) and correspondingly low enforcement intensity ($\approx 1.9\%$). This suggests that when a content provider harmonizes prices (reducing the arbitrage incentive), the need for a "Fortress" strategy diminishes.
    \end{itemize}
    
    \item \textbf{Utility Software (The Strategic Split):}
    \begin{itemize}
        \item \textbf{Adobe Creative Cloud} is a significant anomaly. It rivals Content Providers in price discrimination (DSPI StdDev $\approx 0.59$) yet maintains very low ToS enforcement ($\approx 0.9\%$). This confirms the \textbf{"Utility Paradox"}: downloadable software relies on cryptographic license keys ("Hard" barriers) rather than the "Soft" IP-blocking threats required by streaming services.
    \end{itemize}

    \item \textbf{VPN Enablers (NordVPN, ExpressVPN):} As expected, these "Adversaries" show minimal "Technical Blocking" enforcement, as their business model depends on circumventing the very barriers erected by the Content Providers.
\end{itemize}

This data suggests that \textbf{Business Model} (Streaming vs. Download vs. Access) is a stronger predictor of enforcement strategy than \textbf{Price Opportunity} alone.

% INSERTED ARCHIVE CONTENT END
In the "Software and Music" category, \textbf{Microsoft} stands out with a consistent use of "Legal Threat" and "Regulatory Compliance" frames, likely due to its enterprise customer base and strict licensing requirements. \textbf{Adobe} shows a resurgence of "Price Discrimination" language in 2024, possibly linked to new regional pricing structures. \textbf{Spotify} and \textbf{Apple Music} remain largely focused on "Licensing" and passive "Regulatory Compliance," showing less active technical enforcement than their video counterparts.

In contrast, software providers like \textbf{Adobe} maintain a consistent, lower level of \gls{tos}
enforcement language, suggesting a reliance on technical licensing (cryptographic keys) rather
than retroactive legal threats. \gls{vpn} providers (\textbf{NordVPN}, \textbf{ExpressVPN}) show the
most dramatic shifts, with their documentation evolving to emphasize encryption and user
protection as primary value propositions, effectively reframing circumvention as a fundamental
privacy right.

\subsubsection{Evolution of Ecosystem Players (Amazon \& Apple)}
Beyond the pure content and VPN providers, the "Ecosystem" players (\textbf{Amazon Prime}, \textbf{Apple Music}) show distinct evolutionary paths (see \textbf{Appendix \ref{app:service_evolution}} for full charts). 
\textbf{Amazon Prime} (Figure \ref{fig:evol_amazon}) displays a unique, consistent focus on \textbf{Regulatory Compliance} frames ($\approx 35\%$) rather than Technical Blocking. This suggests Amazon manages cross-border access through account-level shipping/billing addresses rather than active network filtering. 
In contrast, \textbf{Apple Music} (Figure \ref{fig:evol_apple_music}) remains the most "static" of all services, with very low enforcement counts that have barely changed since 2018. This stability reinforces the hypothesis that Apple relies on hardware-locked ecosystem barriers (Apple ID region locks) rather than dynamic \gls{tos} enforcement.

\subsection{High-Confidence Findings: The Core Clauses}
The Gemini 3 Flash model identified specific, high-confidence clauses that are central to the
coercive strategy. For example, clauses stating "You may not use any technology to obscure or
disguise your location" were consistently categorized as \textit{Technical Blocking} with
$>0.95$ confidence. This confirms that firms have made technical countermeasures a formal part
of their legal rules.

\section{Correlation Analysis: The Strategic Trade-off}
\label{sec:correlation}
To test the relationship between pricing strategy and enforcement intensity, we used the
cleaned dataset to calculate the correlation between Price Discrimination (PD) and observed
Enforcement Intensity (EI). Table \ref{tab:correlation_data} summarizes the key metrics, while Figure \ref{fig:correlation_latex} visualizes the relationship.

\begin{table}[ht]
    \centering
    \small
    \begin{tabular}{l c c}
        \toprule
        \textbf{Service} & \textbf{PD Score (\gls{dspi} StdDev)} & \textbf{Enforcement Intensity (\%)} \\
        \midrule
        Spotify & 0.486 & 0.43 \\
        Netflix & 0.352 & 2.03 \\
        Disney+ & 0.324 & 2.04 \\
        Amazon Prime & 0.304 & 2.94 \\
        Adobe & 0.245 & 5.71 \\
        Apple Music & 0.446 & 12.50 \\
        Microsoft & 0.208 & 32.76 \\
        YouTube Premium & 0.464 & 34.34 \\
        NordVPN & 0.231 & 50.00 \\
        ExpressVPN & 0.112 & 55.56 \\
        \midrule
        \textbf{Correlation Results} & \multicolumn{2}{l}{\textbf{Pearson $r = -0.5493$}} \\
        \bottomrule
    \end{tabular}
    \caption{Correlation between Price Discrimination and Enforcement Intensity. All values are normalized percentage intensities derived from the Fortress Index. A negative correlation ($r \approx -0.55$) suggests that services with higher global price discrimination (e.g., Spotify) tend to rely less on active enforcement strategies compared to those with lower variance (e.g., YouTube, Microsoft).}
    \label{tab:correlation_data}
\end{table}



\begin{figure}[ht]
    \centering
    \begin{tikzpicture}
        \begin{axis}[
            width=0.95\textwidth, 
            height=11cm,
            xlabel={Price Discrimination (DSPI StdDev)},
            ylabel={Enforcement Intensity (\%)},
            xmin=0, xmax=0.6,
            ymin=0, ymax=65,
            grid=both,
            grid style={gray!20},
            legend style={at={(0.5,-0.20)}, anchor=north, legend columns=3, font=\tiny, cells={anchor=west}},
            clip=false,
        ]
            % VPN Providers (Outliers) - Diamond
            \addplot[only marks, mark=diamond*, color=red, mark size=4pt] coordinates {(0.112,55.56)}; \addlegendentry{ExpressVPN}
            \addplot[only marks, mark=diamond*, color=red!70!black, mark size=4pt] coordinates {(0.231,50.00)}; \addlegendentry{NordVPN}
            
            % Content Leaders - Circle
            \addplot[only marks, mark=pty, color=blue, mark size=3.5pt] coordinates {(0.464,34.34)}; \addlegendentry{YouTube}
            \addplot[only marks, mark=*, color=blue!60!white, mark size=3.5pt] coordinates {(0.352,2.03)}; \addlegendentry{Netflix}
            \addplot[only marks, mark=*, color=blue!40!black, mark size=3pt] coordinates {(0.324,2.04)}; \addlegendentry{Disney+}
            \addplot[only marks, mark=*, color=cyan, mark size=3pt] coordinates {(0.304, 2.94)}; \addlegendentry{Amazon}

            % Software - Triangle
            \addplot[only marks, mark=triangle*, color=orange, mark size=4pt] coordinates {(0.208,32.76)}; \addlegendentry{Microsoft}
            \addplot[only marks, mark=triangle*, color=orange!60!black, mark size=4pt] coordinates {(0.245,5.71)}; \addlegendentry{Adobe}

            % Music - Square
            \addplot[only marks, mark=square*, color=green!60!black, mark size=3pt] coordinates {(0.446,12.50)}; \addlegendentry{Apple Music}
            \addplot[only marks, mark=square*, color=green!30!black, mark size=3.5pt] coordinates {(0.486,0.43)}; \addlegendentry{Spotify}

            % Trend annotations (manual to avoid clutter)
            \node[anchor=west, font=\tiny\itshape, gray] at (axis cs:0.35,10) {Streaming Cluster};
            \node[anchor=south west, font=\tiny\bfseries, red] at (axis cs:0.12,58) {VPN Peers};
        \end{axis}
    \end{tikzpicture}
    \caption{Price Discrimination vs. Enforcement Intensity ($N=10$). Individual service markers reveal sectoral clustering. VPN providers are extreme outliers (High Intensity, Low PD), while Entertainment streaming clusters at the bottom despite high price variance. Legend entries correspond to specific service data points.}
    \label{fig:correlation_latex}
\end{figure}

The final analysis ($N=10$) shows a complex balance between price differences and blocking. While the global data shows an average balance, different sectors act in their own way. This suggests that large companies (like Amazon) rely less on threats than smaller services or those in high-competition markets.

\section{Detailed Analysis: Strategic Shifts and Fortress Scores}
\label{sec:detailed_analysis}

This section provides a granular view of the enforcement landscape, detailing the specific category shifts and the calculated "Fortress Scores" for each service.

\begin{table}[ht]
    \centering
    \caption{Complete Fortress Index Ranking. VPN providers score highest due to their security-focused framing, followed by YouTube and Microsoft with active enforcement strategies.}
    \label{tab:fortress_index_complete}
    \begin{tabular}{l r l}
        \toprule
        \textbf{Company} & \textbf{Fortress Score (\%)} & \textbf{Archetype} \\
        \midrule
        ExpressVPN      & 55.56 & VPN Enabler \\
        NordVPN         & 50.00 & VPN Enabler \\
        YouTube Premium & 34.34 & Content Fortress \\
        Microsoft       & 32.76 & Content Fortress \\
        Apple Music     & 12.50 & Licensing Dependent \\
        Adobe           &  5.71 & Pricing Optimizer \\
        Disney+         &  2.04 & Licensing Dependent \\
        Netflix         &  2.03 & Licensing Dependent \\
        Amazon Prime    &  2.94 & Pricing Optimizer \\
        Spotify         &  0.43 & Licensing Dependent \\
        \bottomrule
    \end{tabular}
\end{table}

The Fortress Index (Table \ref{tab:fortress_index_complete}) confirms the bifurcation of the market. \textbf{YouTube Premium} (34.34\%) and \textbf{Microsoft} (32.76\%) have effectively built "digital fortresses," distinguishing themselves from the lower-scoring streaming incumbents like \textbf{Netflix} (2.03\%) and \textbf{Spotify} (0.43%), which continue to rely on passive licensing terms.

\begin{table}[ht]
    \centering
    \footnotesize
    \renewcommand{\arraystretch}{1.1}
    \caption{Absolute Category Counts by Year Across All Services (2020--2025). Technical Blocking peaks sharply in 2023, coinciding with YouTube's enforcement escalation.}
    \label{tab:qual_timeline_complete}
    \begin{tabular}{l *{11}{r} | r}
        \toprule
        \textbf{Year} & \rotatebox{70}{\textbf{Tech. Block.}} & \rotatebox{70}{\textbf{Price Discr.}} & \rotatebox{70}{\textbf{Licensing}} & \rotatebox{70}{\textbf{Regulatory}} & \rotatebox{70}{\textbf{Legal Thr.}} & \rotatebox{70}{\textbf{Acc.\ Act.}} & \rotatebox{70}{\textbf{Privacy}} & \rotatebox{70}{\textbf{Sec.\ Risk}} & \rotatebox{70}{\textbf{Portab.}} & \rotatebox{70}{\textbf{Workaro.}} & \rotatebox{70}{\textbf{General}} & \rotatebox{70}{\textbf{Total}} \\
        \midrule
        2020 &  2 & 16 &  64 &  50 & 12 & 0 & 0 & 10 & 0 & 0 & 3,231 & 3,385 \\
        2021 &  1 & 25 &  63 &  66 &  7 & 0 & 2 &  4 & 0 & 0 & 3,446 & 3,614 \\
        2022 & 13 & 20 &  85 &  72 & 13 & 0 & 4 & 10 & 0 & 0 & 3,946 & 4,163 \\
        2023 & 44 & 12 & 102 & 113 & 25 & 0 &11 &  2 & 2 & 0 & 4,490 & 4,801 \\
        2024 & 24 & 33 &  96 & 109 & 23 & 0 & 2 & 10 & 0 & 0 & 4,777 & 5,074 \\
        2025 & 15 & 14 &  99 &  88 & 28 & 0 & 8 &  9 & 0 & 0 & 3,347 & 3,608 \\
        \midrule
        \textbf{Total} & \textbf{99} & \textbf{120} & \textbf{509} & \textbf{498} & \textbf{108} & \textbf{0} & \textbf{27} & \textbf{45} & \textbf{2} & \textbf{0} & \textbf{23,237} & \textbf{24,645} \\
        \bottomrule
    \end{tabular}
\end{table}

Table \ref{tab:qual_timeline_complete} reveals the temporal dynamics underpinning these scores. The aggregate data shows a clear spike in "Technical Blocking" clauses in 2023 (44 incidents), a direct response to the post-pandemic surge in \gls{vpn} usage. "Regulatory Compliance" also sees a steady increase, reflecting the growing complexity of global digital trade laws. Note that the decrease in total counts for 2025 is primarily due to limited data coverage for that year (specifically the absence of updated YouTube and Microsoft reports), rather than a genuine decline in enforcement intensity.

% Moved Distribution Viz to near Table 2
The distribution of these categories (Figure \ref{fig:category_dist_viz}) highlights distinct enforcement styles. Notably, \textbf{Microsoft} gave up on active enforcement for its productivity suite, relying instead on ecosystem lock-in, whereas \textbf{YouTube} is really trying to enforce boundaries but had a cool down period, as seen in the fluctuation of its "Technical Blocking" and "Legal Threat" clauses over time.

\begin{itemize}
    \item \textbf{Content Providers (Netflix, Disney+, YouTube, Xbox, etc.):} This group effectively forms a "High Enforcement Cluster," but successfully illustrates the enforcement trade-off ($R_{sector} \approx 0.45$). 
    \begin{itemize}
        \item \textbf{High Variance / High Enforcement:} Services like \textbf{Disney+} and \textbf{YouTube} have large global price gaps (\gls{dspi} StdDev $>0.37$) and rely on aggressive "Technical Blocking" (6\%--8\%) to maintain them.
        \item \textbf{Low Variance / Low Enforcement (The Xbox Case):} \textbf{Xbox Game Pass}, analyzed under the combined Microsoft entity (see Section \ref{chap:methodology}), serves as a crucial control. Governed by the Microsoft ecosystem, it has relatively harmonized global pricing (\gls{dspi} StdDev $\approx 0.25$) and correspondingly low enforcement intensity ($\approx 1.9\%$). This suggests that when a content provider harmonizes prices (reducing the arbitrage incentive), the need for a ``Fortress'' strategy diminishes.
    \end{itemize}
    
    \item \textbf{Utility Software:}
    \begin{itemize}
        \item \textbf{Adobe Creative Cloud} is a major exception. It has high price 
        discrimination but very low blocking found in its terms. This shows the 
        \textbf{``Utility Paradox''}: software that is downloaded uses activation keys to stop 
        users, while streaming services use network blocking.
    \end{itemize}

    \item \textbf{\gls{vpn} Enablers (NordVPN, ExpressVPN):} As expected, these "Adversaries" show minimal "Technical Blocking" enforcement, as their business model depends on circumventing the very barriers erected by the Content Providers.
\end{itemize}

This data suggests that the \textbf{Business Model} (Streaming vs. Download) is a better 
way to predict blocking than just looking at the \textbf{Price Difference}.

%\chapter{Analysis}
%\section{The Landscape of International Pricing: Findings from the DSPI}
\label{sec:dspi_results}

To understand the economic reason driving firm strategic behavior, we first analyze the quantitative pricing landscape using the Digital Services Price Index (DSPI).

\subsection{Magnitude of the Arbitrage Incentive}
The data shows large price differences across markets. For example, subscriptions in Turkey or Argentina can cost 70-80\% less than the same subscriptions in Switzerland or the USA when measured in nominal USD. This difference creates a "super-normal" profit margin for consumers doing arbitrage, explaining the persistence of this behavior despite the technical barriers analyzed in later sections.



\begin{figure}[ht]
    \centering
    \includegraphics[width=0.9\textwidth]{figures/dspi_heatmap.pdf}
    \caption{Global Heatmap of Digital Service Pricing (The DSPI). Data represents the cost of local subscriptions relative to the US baseline (DSPI=1.0). Lower values indicate stronger arbitrage incentives.}
    \label{fig:dspi_map}
\end{figure}

\begin{table}[ht]
    \centering
    \small
    \renewcommand{\arraystretch}{1.1}
    \begin{threeparttable}
    \begin{tabular}{l|ccccc|ccccc|c}
        \toprule
        \textbf{Service} & \rotatebox{90}{Switzerland} & \rotatebox{90}{USA} & \rotatebox{90}{Germany} & \rotatebox{90}{UK} & \rotatebox{90}{Poland} & \rotatebox{90}{Turkey} & \rotatebox{90}{Argentina} & \rotatebox{90}{Brazil} & \rotatebox{90}{Ukraine} & \rotatebox{90}{Philippines} & \rotatebox{90}{Pakistan} \\
        \midrule
        Netflix & 1.44 & 1.00 & 0.85 & 0.78 & 0.68 & 0.52 & 1.00 & 0.50 & 0.45 & 0.45 & \textbf{0.16} \\
        YouTube Premium & 1.45 & 1.00 & 1.01 & 1.18 & 0.70 & \textbf{0.17} & 0.74 & 0.36 & 0.18 & 0.24 & \textbf{0.12} \\
        Disney+ & 1.47 & 1.00 & 0.92 & 1.07 & 0.67 & 1.11 & 1.14 & 0.72 & -- & 0.35 & -- \\
        Amazon Prime & 0.75 & 1.00 & 0.65 & 0.76 & \textbf{0.18} & \textbf{0.15} & 0.64 & 0.27 & 0.51 & 0.18 & \textbf{0.14} \\
        Spotify & 1.50 & 1.00 & 1.18 & 1.27 & 0.50 & 0.26 & 0.33 & 0.40 & 0.42 & 0.25 & \textbf{0.10} \\
        Apple Music & 1.43 & 1.00 & 1.09 & 1.27 & 0.50 & \textbf{0.17} & 0.65 & 0.40 & 0.45 & 0.23 & -- \\
        Microsoft 365 & 1.13 & 1.00 & 1.08 & 1.08 & 1.08 & 1.06 & 0.45 & 1.02 & 0.70 & 0.88 & 0.83 \\
        Adobe CC & 1.24 & 1.00 & 1.21 & 1.21 & 1.26 & 0.74 & 1.01 & 0.61 & 0.58 & 0.98 & 1.00 \\
        Xbox Game Pass & 1.13 & 1.00 & 0.98 & 0.89 & 0.93 & 0.86 & 1.08 & 0.88 & 0.75 & 0.58 & \textbf{0.18} \\
        NordVPN & 1.09 & 1.00 & 1.09 & 0.97 & 0.85 & 1.01 & 0.44 & 0.44 & 1.01 & 0.90 & 0.81 \\
        ExpressVPN & 0.99 & 1.00 & 1.02 & 1.00 & 1.13 & 1.10 & 0.92 & 0.92 & 1.10 & 0.92 & 0.73 \\
        \bottomrule
    \end{tabular}
    \begin{tablenotes}
        \small
        \item \textit{Note:} DSPI values represent local price relative to US baseline (1.00). Bold values indicate strongest arbitrage opportunities ($<0.20$). Data collected December 2025.
    \end{tablenotes}
    \end{threeparttable}
    \caption{Digital Services Price Index (DSPI) by Service and Country}
    \label{tab:dspi_full}
\end{table}

\subsection{The Affordability Paradox: Nominal vs. Real Cost}
While nominal price differences create arbitrage incentives for Western users, it is crucial to understand the "Real Cost" for local users. Figure \ref{fig:affordability} maps the cost of digital services as a percentage of the \textbf{Median National Monthly Wage}.

\begin{figure}[ht]
    \centering
    \includegraphics[width=0.9\textwidth]{figures/affordability_heatmap.pdf}
    \caption{The Affordability Gap: Digital Service Cost as Percentage of Local Monthly Income. Darker red indicates higher relative cost for local citizens.}
    \label{fig:affordability}
\end{figure}

The data reveals a paradox: while Turkey and Argentina offer the cheapest nominal prices for international arbitrageurs (< \$4/month), these services are significantly \textit{more expensive} for locals in real terms. For instance, a Standard Netflix subscription in Turkey consumes a higher percentage of the median monthly wage ($\approx 0.6\%$) compared to the USA ($\approx 0.3\%$). This suggests that low nominal prices are not "discounts" but necessary adjustments to local purchasing power, which external actors then exploit.

\section{Classification Results: Strategic Framing}
\label{sec:classification_results}

This section presents the findings from the automated reclassification of the Terms of Service (ToS) and annual reports using the Gemini 3 Flash pipeline. The analysis processed a total of 25,593 sentences across the dataset.
 
 \subsection{Distribution of Enforcement Categories}
 The classification showed that \textbf{94.1\%} of the sentences were "General Terms" (legal boilerplate). While this high volume of standard legal text confirms the structural integrity of the documents, the analysis below focuses primarily on the remaining \textbf{5.9\%} of "Strategic Sentences" that contain active enforcement clauses. "General Terms" are excluded from trend visualizations to maintain readability. Table \ref{tab:category_dist} shows how different providers approach this.

\begin{table}[ht]
    \centering
    \begin{tabularx}{\textwidth}{l X c c}
        \toprule
        \textbf{Category} & \textbf{Description} & \textbf{Freq (N)} & \textbf{Freq (\%)} \\
        \midrule
        Content Licensing & Geographic restrictions based on rights. & 562 & 37.22\% \\
        Regulatory Compliance & Local laws/tax compliance. & 528 & 34.97\% \\
        Price Discrimination & Explicit regional pricing rules. & 120 & 7.95\% \\
        Legal Threat & Explicit threats of termination/legal action. & 120 & 7.95\% \\
        Technical Blocking & Active detection/blocking technology. & 108 & 7.15\% \\
        Security Risk & Risks of VPN usage (Service Prov. Frame). & 42 & 2.78\% \\
        Privacy/Security & Encryption/Anonymity (VPN Frame). & 27 & 1.79\% \\
        Legitimate Portability & EU Portability Regulation clauses. & 2 & 0.13\% \\
        User Workaround & References to circumventing blocks. & 1 & 0.07\% \\
        \bottomrule
    \end{tabularx}
    \caption{Distribution of Strategic Categories in ToS Documents}
    \label{tab:category_dist}
\end{table}

\begin{figure}[ht]
    \centering
    \begin{subfigure}{0.48\textwidth}
        \includegraphics[width=\textwidth]{figures/strategic_frames_dist.pdf}
        \caption{Strategic Frames Distribution}
        \label{fig:frame_dist}
    \end{subfigure}
    \hfill
    \begin{subfigure}{0.48\textwidth}
        \includegraphics[width=\textwidth]{figures/global_priority_shift.pdf}
        \caption{Global Priority Shift (\%)}
        \label{fig:priority_shift}
    \end{subfigure}
    \caption{Global Landscape of Enforcement: Distribution and Temporal Priority Shift.}
    \label{fig:global_summary}
\end{figure}

The \textbf{Global Priority Shift} (Figure \ref{fig:priority_shift}) shows a relative increase in the importance of \textit{Technical Blocking} language compared to other categories over the last three years. This trend is further supported by the lexical analysis of the dataset.

\begin{table}[ht]
    \centering
    \small
    \begin{tabular}{l c | l c}
        \toprule
        \textbf{Keyword} & \textbf{Frequency} & \textbf{Keyword} & \textbf{Frequency} \\
        \midrule
        location & 128 & circumvention & 26 \\
        youtube & 113 & piracy & 25 \\
        determine & 46 & distribution & 25 \\
        google & 37 & protection & 25 \\
        unauthorized & 29 & verify & 25 \\
        detection & 28 & accessibility & 25 \\
        monitor & 27 & monitor & 27 \\
        \bottomrule
    \end{tabular}
    \caption{Top Strategically Relevant Keywords Identified by Gemini 3 Flash}
    \label{tab:keywords}
\end{table}

The prevalence of keywords like ``location,'' ``determine,'' and ``detection'' underscores the shift toward active monitoring as a core enforcement strategy. Table \ref{tab:quotes} provides verbatim examples of how these concepts are operationalized.

\begin{table}[ht]
    \centering
    \renewcommand{\arraystretch}{1.2}
    \small
    \begin{tabularx}{\textwidth}{l X l l l c}
        \toprule
        \textbf{Category} & \textbf{Quote} & \textbf{Service} & \textbf{Year} & \textbf{Doc} & \textbf{Conf} \\
        \midrule
        Content Licensing & "We grant you a limited... license... only within geographic locations..." & Netflix & 2023 & ToS & 0.98 \\
        \addlinespace
        Technical Blocking & "You may not use any technology to obscure or disguise your location." & Disney+ & 2024 & ToS & 0.95 \\
        \addlinespace
        Legal Threat & "We reserve the right to terminate... without notice, if we suspect violation." & Spotify & 2022 & ToS & 0.92 \\
        \addlinespace
        Price Discrimination & "Prices may vary by country... charged in currency of location." & Steam & 2024 & ToS & 0.89 \\
        \bottomrule
    \end{tabularx}
    \caption{Representative Clauses for Detected Enforcement Strategies}
    \label{tab:quotes}
\end{table}

\subsection{Service-Specific Analysis}
The enforcement strategies vary significantly across different service providers, reflecting their distinct business models and regional licensing constraints. Figure \ref{fig:service_dist} illustrates the proportional distribution of categories for each service. 

\subsubsection{Strategic Framing by Digital Service Providers}
The qualitative analysis highlights a distinct "Coercive" framing strategy employed by digital service providers. The dominant rhetorical frame, appearing in over \textbf{37.2\%} of non-boilerplate sentences (see Table \ref{tab:category_dist}), is \textbf{Content Licensing}. Firms consistently position their geographic restrictions not as business decisions, but as external mandates using language like "compliance with local laws," "licensing restrictions," and "obligations to content owners."
The second most dominant frame is \textbf{Regulatory Compliance} (\textbf{35.0\%}), reinforcing this narrative of external obligation.

Content licensing services like \textbf{Disney+} and \textbf{Netflix} exhibit this most strongly, dedicating significant portions of their terms to defining geographic rights. In contrast, global platforms like \textbf{Amazon} show notable spikes in Regulatory Compliance.

\begin{figure}[ht]
    \centering
    \includegraphics[width=\textwidth]{figures/service_distribution_ratios.pdf}
    \caption{Proportional Distribution of Enforcement Categories by Service}
    \label{fig:service_dist}
\end{figure}

To further quantify the intensity of these enforcement regimes, we propose the \textbf{Fortress Index}, a metric that calculates the percentage of enforcement-related clauses (Technical Blocking and Legal Threat) relative to the total number of sentences in a firm's documentation. Table \ref{tab:fortress_index} illustrates the stark divide between actors in the geo-arbitrage ecosystem.

\begin{table}[ht]
    \centering
    \small
    \begin{tabularx}{\textwidth}{l X c}
        \toprule
        \textbf{Service Provider} & \textbf{Strategic Archetype} & \textbf{Fortress Score (\%)} \\
        \midrule
        ExpressVPN & VPN Enabler & 55.56 \\
        NordVPN & VPN Enabler & 50.00 \\
        YouTube Premium & Content Provider & 34.34 \\
        Microsoft & Software/Access & 32.76 \\
        Apple Music & Content Provider & 12.50 \\
        Adobe & Software/Utility & 5.71 \\
        Amazon Prime & Global Platform & 2.94 \\
        Disney+ & Content Provider & 2.04 \\
        Netflix & Content Provider & 2.03 \\
        Spotify & Content Provider & 0.43 \\
        \bottomrule
    \end{tabularx}
    \caption{The Fortress Index: Percentage of Enforcement Clauses per Service}
    \label{tab:fortress_index}
\end{table}

The index reveals that VPN providers like \textbf{ExpressVPN} and \textbf{NordVPN} have the highest density of relevant clauses, as their entire documentation is focused on security and circumvention. Among digital services, \textbf{YouTube} and \textbf{Microsoft} exhibit significantly higher physical "fortress" density than \textbf{Netflix} or \textbf{Spotify}, suggesting a more aggressive or complex regulatory approach to user location.

\subsubsection{Strategic Framing by VPN Providers}
In sharp contrast, VPN companies adopt a "Liberation" and "Privacy" frame. The analysis reveals a consistent narrative that reframes circumvention as \textbf{User Freedom}. 
A secondary dominant frame identified in our analysis is \textbf{Privacy/Security}. While many users may purchase VPNs for streaming arbitrage, providers legitimize the service by emphasizing security features. \textbf{NordVPN}, for example, shows a distinct focus on ``Security Risk'' categories in our dataset, with marketing materials framing this as empowering users against tracking.

\subsection{Temporal Evolution of Enforcement}
To understand how these strategies have evolved over time, we analyzed the frequency of category-specific clauses across the dataset's years. Table \ref{tab:timeline_count} shows the raw count of enforcement-related incidents detected per service per year.

\begin{table}[ht]
    \centering
    \small
    \begin{tabular}{l|ccccccccc}
        \toprule
        \textbf{Service} & \textbf{2016} & \textbf{2018} & \textbf{2020} & \textbf{2021} & \textbf{2022} & \textbf{2023} & \textbf{2024} & \textbf{2025} \\
        \midrule
        Adobe & 0 & 0 & 2 & 0 & 0 & 0 & 1 & 1 \\
        Amazon Prime & 0 & 0 & 1 & 0 & 0 & 0 & 0 & 2 \\
        Apple Music & 0 & 0 & 0 & 0 & 1 & 4 & 1 & 3 \\
        Disney+ & 0 & 0 & 0 & 0 & 0 & 0 & 4 & 0 \\
        ExpressVPN & 0 & 0 & 0 & 0 & 0 & 0 & 0 & 5 \\
        Microsoft & 0 & 0 & 9 & 7 & 9 & 12 & 6 & 6 \\
        Netflix & 0 & 0 & 1 & 1 & 0 & 0 & 0 & 3 \\
        NordVPN & 0 & 0 & 0 & 0 & 0 & 0 & 3 & 0 \\
        Spotify & 0 & 0 & 0 & 0 & 0 & 0 & 0 & 0 \\
        YouTube Premium & 8 & 7 & 1 & 0 & 16 & 53 & 31 & 20 \\
        \bottomrule
    \end{tabular}
    \caption{Raw Count of Enforcement Incidents per Service (2016--2025)}
    \label{tab:timeline_count}
\end{table}

The data shows a significant increase in specific enforcement clauses, especially from 2022 onwards, most notably for \textbf{YouTube Premium}. This suggests that restrictive clauses have become more prevalent and more specific over the analyzed period, transitioning from general boilerplate to active regulatory language.

\begin{figure}[ht]
    \centering
    \includegraphics[width=0.9\textwidth]{figures/timeline_all_total.pdf}
    \caption{Temporal Evolution of Category Incident Counts (Aggregate)}
    \label{fig:timeline_all}
\end{figure}

\begin{figure}[ht]
    \centering
    \includegraphics[width=0.9\textwidth]{figures/evolution_strategic_frames_summary.pdf}
    \caption{Evolution of Strategic Frames over Time (Excluding General Terms)}
    \label{fig:strategic_frames_evolution}
\end{figure}

\begin{figure}[ht]
    \centering
    \includegraphics[width=\textwidth]{figures/category_timeline_per_service_normalized.pdf}
    \caption{Temporal Evolution of Category Frequencies by Service (Normalized)}
    \label{fig:timeline_service}
\end{figure}

\section{Deep Dive: Service-Specific Strategic Evolution}
\label{sec:service_deep_dive}

To understand the operational realities of geo-arbitrage enforcement, we analyze the longitudinal patterns of specific providers. The following figures illustrate how individual firms have adapted their Terms of Service to address the growing arbitrage incentive.

\begin{figure}[ht]
    \centering
    \begin{subfigure}{0.48\textwidth}
        \includegraphics[width=\textwidth]{figures/evol_netflix.pdf}
        \caption{Netflix}
        \label{fig:evol_netflix}
    \end{subfigure}
    \hfill
    \begin{subfigure}{0.48\textwidth}
        \includegraphics[width=\textwidth]{figures/evol_youtube.pdf}
        \caption{YouTube Premium}
        \label{fig:evol_youtube}
    \end{subfigure}
    
    \vspace{0.5cm}
    
    \begin{subfigure}{0.48\textwidth}
        \includegraphics[width=\textwidth]{figures/evol_disney.pdf}
        \caption{Disney+}
        \label{fig:evol_disney}
    \end{subfigure}
    \hfill
    \begin{subfigure}{0.48\textwidth}
        \includegraphics[width=\textwidth]{figures/evol_spotify.pdf}
        \caption{Spotify}
        \label{fig:evol_spotify}
    \end{subfigure}
    \caption{Multi-Year Evolution of Strategic Frames: Video and Music Streaming.}
    \label{fig:evol_streaming}
\end{figure}

The "Content Providers" (\textbf{Netflix}, \textbf{YouTube}, \textbf{Disney+}) show a distinct move toward specialized enforcement clauses starting in 2022. While \textbf{Spotify} remains relatively boilerplate-heavy, \textbf{YouTube Premium} exhibits a massive surge in specific "Technical Blocking" and "Legal Threat" language, corresponding to their increased efforts to combat VPN-enabled subscription hopping in markets like Turkey and Pakistan.

\begin{figure}[ht]
    \centering
    \begin{subfigure}{0.48\textwidth}
        \includegraphics[width=\textwidth]{figures/evol_microsoft.pdf}
        \caption{Microsoft 365}
        \label{fig:evol_microsoft}
    \end{subfigure}
    \hfill
    \begin{subfigure}{0.48\textwidth}
        \includegraphics[width=\textwidth]{figures/evol_adobe.pdf}
        \caption{Adobe Creative Cloud}
        \label{fig:evol_adobe}
    \end{subfigure}
    
    \vspace{0.5cm}
    
    \begin{subfigure}{0.48\textwidth}
        \includegraphics[width=\textwidth]{figures/evol_nordvpn.pdf}
        \caption{NordVPN}
        \label{fig:evol_nordvpn}
    \end{subfigure}
    \hfill
    \begin{subfigure}{0.48\textwidth}
        \includegraphics[width=\textwidth]{figures/evol_expressvpn.pdf}
        \caption{ExpressVPN}
        \label{fig:evol_expressvpn}
    \end{subfigure}
    \caption{Multi-Year Evolution: Software Utilities and VPN Providers.}
    \label{fig:evol_software_vpn}
\end{figure}

In contrast, software providers like \textbf{Adobe} maintain a consistent, lower level of ToS enforcement language, suggesting a reliance on technical licensing (cryptographic keys) rather than retroactive legal threats. VPN providers (\textbf{NordVPN}, \textbf{ExpressVPN}) show the most dramatic shifts, with their documentation evolving to emphasize encryption and user protection as primary value propositions, effectively reframing circumvention as a fundamental privacy right.

\subsection{High-Confidence Findings: The Core Clauses}
The Gemini 3 Flash model identified specific, high-confidence clauses that are central to the coercive strategy. For example, clauses stating "You may not use any technology to obscure or disguise your location" were consistently categorized as \textit{Technical Blocking} with $>0.95$ confidence. This confirms that firms have made technical countermeasures a formal part of their legal rules.



\subsection{The Affordability Paradox: Real vs. Nominal Cost}
While the DSPI measures the \textit{nominal} price difference (relevant to arbitrageurs), it is crucial to analyze the "Real Cost" for local residents. Figure \ref{fig:affordability_real} maps the cost of digital services as a percentage of the \textbf{Median National Monthly Wage}, serving as a digital equivalent to "Time-to-Earn" indices used in purchasing power comparisons (e.g., the Big Mac Index's affordability variant).

\begin{figure}[ht]
    \centering
    \includegraphics[width=0.9\textwidth]{figures/affordability_heatmap.pdf}
    \caption{The Affordability Gap: Digital Service Cost as Percentage of Local Monthly Income. Darker red indicates higher relative cost for local citizens.}
    \label{fig:affordability_real}
\end{figure}

The data reveals a critical paradox: while Turkey and Argentina offer the cheapest nominal prices worldwide for international arbitrageurs (< \$4/month, DSPI $\approx 0.15$), these same services are significantly \textit{more expensive} for locals in real terms. For instance, a Standard Netflix subscription in Turkey consumes approximately 0.6\% of the median monthly wage compared to approximately 0.2\% in the USA.

This distinction is critical:
\begin{enumerate}
    \item \textbf{High DSPI Variance:} Creates incentives for \textit{external} abuse (VPN Arbitrage).
    \item \textbf{Low Affordability:} Justifies the \textit{internal} pricing strategy (low nominal prices are necessary for market penetration, not optional discounts).
\end{enumerate}

Thus, low nominal prices observed in the Global South are not "bargains" but necessary economic adjustments that inadvertently create vulnerabilities exploited by Global North users.

\section{Correlation Analysis: The Strategic Trade-off}
\label{sec:correlation}
To test the relationship between pricing strategy and enforcement intensity, we used the cleaned dataset to calculate the correlation between Price Discrimination (PD) and observed Enforcement Intensity (EI). Table \ref{tab:correlation_data} summarizes the key metrics.

\begin{table}[ht]
    \centering
    \small
    \begin{tabular}{l c c}
        \toprule
        \textbf{Service} & \textbf{PD Score (DSPI StdDev)} & \textbf{Enforcement Intensity (\%)} \\
        \midrule
        Microsoft & 0.208 & 0.87 \\
        YouTube Premium & 0.464 & 3.07 \\
        Spotify & 0.486 & 0.03 \\
        Adobe & 0.245 & 0.13 \\
        Netflix & 0.352 & 0.17 \\
        Disney+ & 0.324 & 0.18 \\
        Amazon Prime & 0.304 & 0.24 \\
        Apple Music & 0.446 & 0.75 \\
        ExpressVPN & 0.112 & 8.33 \\
        NordVPN & 0.231 & 5.45 \\
        \bottomrule
    \end{tabular}
    \caption{Correlation between Price Discrimination and Enforcement Intensity}
    \label{tab:correlation_data}
\end{table}

\begin{figure}[ht]
    \centering
    \includegraphics[width=0.9\textwidth]{figures/protection_vs_pricing_updated.pdf}
    \caption{Strategic Alignment: Comparison of Price Discrimination scores vs. Enforcement Intensities across analyzed services.}
    \label{fig:correlation}
\end{figure}

The refined analysis ($N=10$) reveals a complex relationship between price variance and enforcement. While the overall global correlation suggests a moderate trade-off, specific sector clusters emerge that show distinct strategic behaviors. This suggests that firms with established global pricing power (like Amazon) rely less on aggressive legal threats than smaller localized services or those in highly contested content markets.

\begin{itemize}
    \item \textbf{Content Providers (Netflix, Disney+, YouTube, Xbox, etc.):} This group effectively forms a "High Enforcement Cluster," but successfully illustrates the enforcement trade-off ($R_{sector} \approx 0.45$). 
    \begin{itemize}
        \item \textbf{High Variance / High Enforcement:} Services like \textbf{Disney+} and \textbf{YouTube} have large global price gaps (DSPI StdDev $>0.37$) and rely on aggressive "Technical Blocking" (6\%--8\%) to maintain them.
        \item \textbf{Low Variance / Low Enforcement (The Xbox Case):} \textbf{Xbox Game Pass} serves as a crucial control. Governed by the Microsoft ecosystem, it has relatively harmonized global pricing (DSPI StdDev $\approx 0.25$) and correspondingly low enforcement intensity ($\approx 1.9\%$). This suggests that when a content provider harmonizes prices (reducing the arbitrage incentive), the need for a "Fortress" strategy diminishes.
    \end{itemize}
    
    \item \textbf{Utility Software (The Strategic Split):}
    \begin{itemize}
        \item \textbf{Adobe Creative Cloud} is a significant anomaly. It rivals Content Providers in price discrimination (DSPI StdDev $\approx 0.59$) yet maintains very low ToS enforcement ($\approx 0.9\%$). This confirms the \textbf{"Utility Paradox"}: downloadable software relies on cryptographic license keys ("Hard" barriers) rather than the "Soft" IP-blocking threats required by streaming services.
    \end{itemize}

    \item \textbf{VPN Enablers (NordVPN, ExpressVPN):} As expected, these "Adversaries" show minimal "Technical Blocking" enforcement, as their business model depends on circumventing the very barriers erected by the Content Providers.
\end{itemize}

This data suggests that \textbf{Business Model} (Streaming vs. Download vs. Access) is a stronger predictor of enforcement strategy than \textbf{Price Opportunity} alone.


\chapter{Discussion}
\label{chap:discussion}

This chapter is the main part of the thesis. Here, we combine the numbers from the 
study with what we learned about company strategy. We define a set of business types that 
show how different industries respond to arbitrage. This discussion looks at the deeper 
meaning of the ongoing tech race between companies and users.

\section{Strategic Archetypes}
Based on the Terms of Service analysis ($N=10$, covering 8 digital service providers and 2 VPN enablers), we identify three main strategies. To facilitate cross-sectoral comparison, the initial 10 coding categories were aggregated into three high-level \textbf{Macro-Categories}:
\begin{enumerate}
    \item \textbf{Business Model Adaptation \& Pricing (BMA):} Aggregates \textit{Content Licensing}, \textit{Regulatory Compliance}, \textit{Price Discrimination}, and \textit{Legitimate Portability}. This represents the ``soft'' strategic layer where firms adjust their offerings to local market conditions.
    \item \textbf{Coercive Restriction \& Legal Threat (CRL):} Aggregates \textit{Technical Blocking}, \textit{Legal Threat}, \textit{Account Action}, and \textit{Security Risk}. This represents the ``hard'' enforcement layer designed to physically or legally prevent circumvention.
    \item \textbf{General Corporate Operations (GCO):} Aggregates \textit{Privacy/Security} (VPN frame) and \textit{User Workaround}. This captures supporting narratives and adversarial user behavior descriptions.
\end{enumerate}

Figure \ref{fig:strategic_framing_pivot} provides a high-level overview of how these macro-strategies manifest across the analyzed companies.

\begin{figure}[ht]
    \centering
    \begin{tikzpicture}
        \begin{axis}[
            xbar stacked,
            width=0.92\textwidth,
            height=10cm,
            symbolic y coords={ExpressVPN,NordVPN,Apple Music,Youtube Premium,Spotify,Netflix,Disney+,Amazon Prime,Amazon,Adobe,Microsoft},
            ytick=data,
            yticklabel style={font=\small},
            xlabel={Share of Strategic Sentences (\%)},
            xmin=0, xmax=105,
            legend style={at={(0.5,-0.15)}, anchor=north, legend columns=-1, font=\footnotesize},
            bar width=12pt,
            nodes near coords,
            every node near coord/.append style={font=\tiny},
            /pgf/number format/fixed, /pgf/number format/precision=1,
        ]
            % Business Model Adaptation & Pricing
            \addplot[fill=tudablue!80] coordinates {
                (80.3,Adobe) (71.4,Amazon) (79.6,Amazon Prime) (57.1,Apple Music) (73.2,Disney+) (35.0,ExpressVPN) (82.0,Microsoft) (69.4,Netflix) (34.5,NordVPN) (75.7,Spotify) (60.4,Youtube Premium)
            };
            % Coercive Restriction & Legal Threat
            \addplot[fill=tudared!80] coordinates {
                (18.6,Adobe) (27.8,Amazon) (18.0,Amazon Prime) (40.7,Apple Music) (26.6,Disney+) (61.7,ExpressVPN) (17.1,Microsoft) (29.8,Netflix) (60.0,NordVPN) (23.5,Spotify) (39.0,Youtube Premium)
            };
            % General Corporate Operations
            \addplot[fill=tudaorange!80] coordinates {
                (1.1,Adobe) (0.8,Amazon) (2.4,Amazon Prime) (2.2,Apple Music) (0.2,Disney+) (3.3,ExpressVPN) (0.9,Microsoft) (0.9,Netflix) (5.5,NordVPN) (0.7,Spotify) (0.6,Youtube Premium)
            };
            \legend{BMA (Adaptation), CRL (Enforcement), GCO (Operations)}
        \end{axis}
    \end{tikzpicture}
    \caption{Strategic Framing by Company: Macro-Category Distribution. VPN providers (ExpressVPN, NordVPN) show the highest proportion of coercive/legal framing (CRL $>$ 60\%), while software companies (Microsoft, Adobe) and streaming platforms overwhelmingly use business model adaptation framing (BMA $>$ 70\%).}
    \label{fig:strategic_framing_pivot}
\end{figure}

The temporal evolution of these categories, as previously detailed in Figure \ref{fig:priority_shift_latex} and Table \ref{tab:qual_timeline_complete}, confirms the shifting focus toward technical countermeasures within the CRL macro-category.



\subsection{The Content Fortress: Defensive Value Capture}
Firms like \textbf{Netflix} and \textbf{Disney+} focus on keeping their prices separate rather 
than making things easy for the user. Our data shows that in this sector, high price 
differences and strict blocking go together. This shows that blocking isn't just a reaction 
to price hops, but a standard way these companies work to protect their content in each region.







This aligns with the "Fortress" strategy described by \textcite{schmidt2020transnational},
where incumbent firms construct digital barriers to protect legacy revenue streams. However,
as noted by \textcite{lobato2019geoblocking}, such strategies often suffer from a "legibility"
problem—users encounter "This content is not available" error messages without understanding
the underlying legal framework.

Interestingly, \textbf{Netflix's} low Fortress score (2.03\%) is surprising given its role 
as a streaming pioneer. This could be explained by a shift in \gls{bmi}: Netflix has moved 
heavily toward in-house content production (``Netflix Originals''). By owning its own 
content, Netflix faces fewer regional licensing problems, which reduces the need for strict 
geographic blocking. Over time, this strategy may also lower the arbitrage incentive, as 
Netflix Originals are available globally without regional restrictions.

\subsection{The Ecosystem Fortress: Adaptive Value Proposition}
Platforms like \textbf{Apple Music} and \textbf{Amazon Prime} exemplify a "Globalist" approach that innovates on the
\textbf{Value Proposition}. 

\textbf{Apple Music}, with a low focus on \textbf{Technical Blocking} (only 2 clauses, representing 12.5\% of its strategic focus but only 0.15\% of total policy volume) and a
strong emphasis on \textbf{Price Discrimination} (5.7\%), appears to accept the reality
of international fragmentation \parencite{brouthers2016explaining}. Rather than "repairing"
the Value Capture mechanism through blocking, they rely on a superior Value Delivery ecosystem
(hardware integration, iCloud) that makes the friction of using a \gls{vpn}-based "foreign" account
essentially "not worth it" for the user.

\textbf{Amazon Prime} adopts a similar but legally distinct strategy. As shown in our \textbf{Detailed Evolution Analysis} (Appendix \ref{app:service_evolution}, Figure \ref{fig:evol_amazon}), Amazon relies heavily on \textbf{Regulatory Compliance} language. Instead of technical cat-and-mouse games, Amazon anchors its digital services to physical shipping addresses and tax jurisdictions. This creates a "Bureaucratic Fortress" where the barrier to entry is not an IP filter, but a valid residential address and local credit card, making \gls{vpn}-based arbitrage logistically difficult rather than technically impossible.









\subsection{The Enterprise Fortress: Identity-Based Value Capture}
A new archetype identified in this study is the "Enterprise Fortress," exemplified by
\textbf{Microsoft}. Despite having the lowest global price variance in the dataset (indicating
a relatively harmonized global price for Microsoft 365), Microsoft exhibits the highest
intensity of \textbf{Account Action} clauses (categorized within the broader Legal Threat framework). 
This suggests that for utility software, the \textbf{Value
Capture} is protected not by \textit{network} blocking (which targets Location), but by \textit{identity}
verification (which targets the User). The "Fortress" is built to keep unauthorized resellers
out, reinforcing the subscription model's integrity without compromising the global \textbf{Value
Delivery} of the software itself.



\subsection{The Utility Paradox (Adobe)}
\textbf{Adobe} presents a unique case. It has high price discrimination (similar to Netflix)
but relatively low ``Technical Blocking'' enforcement. Based on our analysis of Adobe's Terms
of Service and product documentation, this appears to be because Adobe's enforcement mechanism
is ``on-device'' (software activation keys) rather than ``on-network'' (IP filtering). This
observation suggests that ``Technical Blocking'' is a strategy specific to
\textit{cloud-streamed} content, whereas \textit{downloaded software} may rely on different
protection mechanisms.



However, a hybrid future appears to be emerging in the form of \textbf{"Always-Online \gls{drm}"}.
Features like Adobe's cloud-dependent generative tools require authenticated connections to
function, effectively merging network-based verification with identity authentication. This
reflects the "Opportunities and Risks" of SaaS adoption \parencite{benlian2011opportunities},
where control shifts from the client device to the cloud provider.
\subsection{The Adversarial Cycle: A View Over Time}
The relationship between providers and consumers is not static. Our historical analysis
reveals a clear "Action-Reaction" cycle, visualized in Figure \ref{fig:timeline}.

\begin{figure}[ht]
    \centering
    \begin{tikzpicture}[x=2cm, y=1cm]
        % Draw the timeline line
        \draw[->, thick] (0,0) -- (6.5,0) node[right] {Year};
        
        % Ticks and Labels
        \foreach \x/\year in {0.5/2020, 2.0/2021, 3.5/2022, 5.0/2024} {
            \draw[thick] (\x,0.1) -- (\x,-0.1);
            \node[below=0.2cm] at (\x,0) {\textbf{\year}};
        }

        % Events Top (Corporate/Coercive)
        \node[align=center, text width=3cm, above=1.5cm, font=\footnotesize] (pandemic) at
        (0.5,0) {\textbf{Pandemic Surge}\\(Arbitrage Wave)};
        \draw[thin] (0.5,0.2) -- (pandemic);

        \node[align=center, text width=3cm, above=0.5cm, font=\footnotesize] (resip) at
        (2.0,0) {\textbf{Residential IP Crackdown}\\(The ``Hard'' Filter)};
        \draw[thin] (2.0,0.2) -- (resip);
        
        \node[align=center, text width=3cm, above=1.5cm, font=\footnotesize] (youtube) at
        (3.5,0) {\textbf{YouTube Enforcement Surge}\\(Technical Blocking)};
        \draw[thin] (3.5,0.2) -- (youtube);

        % Events Bottom (VPN/Adaptive)
        \node[align=center, text width=3cm, below=1.5cm, font=\footnotesize] (obfus) at
        (0.5,0) {\textbf{Obfuscated Servers}\\(Chameleon/XOR)};
        \draw[thin] (0.5,-0.2) -- (obfus);

        \node[align=center, text width=3cm, below=2.5cm, font=\footnotesize] (wireguard) at (2.0,0) {\textbf{WireGuard \& NordLynx}\\(Speed + Stealth)};
        \draw[thin] (2.0,-0.2) -- (wireguard);

    \end{tikzpicture}
    \caption{The Adversarial Timeline: Coercive Barriers vs. Technical Circumvention (2020--2025).}
    \label{fig:timeline}
\end{figure}

% --- Generated Video Group ---
\begin{figure}[ht]
    \centering
        \begin{minipage}{0.48\textwidth}
            \centering
            \begin{tikzpicture}
                \begin{axis}[
                    width=\linewidth, height=5cm, xlabel={Year}, ylabel={Share (\%)},
                    xmin=2020, xmax=2025, xtick={2020,2022,2024},
                    xticklabel style={/pgf/number format/set thousands separator={}},
                    grid=major, legend pos=north west, legend style={font=\tiny},
                    title={Netflix}, title style={font=\footnotesize\bfseries},
                    ymin=0, ymax=100
                ]
                    \addplot[thick, color=tudablue, mark=square*] coordinates { (2020,61.3) (2021,44.2) (2022,43.6) (2023,44.1) (2024,57.1) (2025,57.7)  }; \addlegendentry{Content Licensing}
                    \addplot[thick, color=tudagreen, mark=triangle*] coordinates { (2020,22.6) (2021,25.6) (2022,28.2) (2023,35.3) (2024,39.3) (2025,33.8)  }; \addlegendentry{Regulatory Compliance}
                    \addplot[thick, color=tudared, mark=x] coordinates { (2020,0.0) (2021,2.3) (2022,0.0) (2023,0.0) (2024,0.0) (2025,0.0)  }; \addlegendentry{Technical Blocking}
                    \addplot[thick, color=cyan, mark=diamond*] coordinates { (2020,12.9) (2021,27.9) (2022,28.2) (2023,14.7) (2024,3.6) (2025,4.2)  }; \addlegendentry{Price Discrimination}
                    \addplot[thick, color=orange, mark=pentagon*] coordinates { (2020,3.2) (2021,0.0) (2022,0.0) (2023,0.0) (2024,0.0) (2025,4.2)  }; \addlegendentry{Legal Threat}
                \end{axis}
            \end{tikzpicture}
        \end{minipage}
    \hfill
        \begin{minipage}{0.48\textwidth}
            \centering
            \begin{tikzpicture}
                \begin{axis}[
                    width=\linewidth, height=5cm, xlabel={Year}, ylabel={Share (\%)},
                    xmin=2020, xmax=2025, xtick={2020,2022,2024},
                    xticklabel style={/pgf/number format/set thousands separator={}},
                    grid=major, legend pos=north west, legend style={font=\tiny},
                    title={YouTube}, title style={font=\footnotesize\bfseries},
                    ymin=0, ymax=100
                ]
                    \addplot[thick, color=tudablue, mark=square*] coordinates { (2020,0.0) (2021,0.0) (2022,0.0) (2023,0.0) (2024,0.0) (2025,0.0)  }; \addlegendentry{Content Licensing}
                    \addplot[thick, color=tudagreen, mark=triangle*] coordinates { (2020,0.0) (2021,0.0) (2022,0.0) (2023,0.0) (2024,0.0) (2025,0.0)  }; \addlegendentry{Regulatory Compliance}
                    \addplot[thick, color=tudared, mark=x] coordinates { (2020,0.0) (2021,0.0) (2022,0.0) (2023,0.0) (2024,0.0) (2025,0.0)  }; \addlegendentry{Technical Blocking}
                    \addplot[thick, color=cyan, mark=diamond*] coordinates { (2020,0.0) (2021,0.0) (2022,0.0) (2023,0.0) (2024,0.0) (2025,0.0)  }; \addlegendentry{Price Discrimination}
                    \addplot[thick, color=orange, mark=pentagon*] coordinates { (2020,0.0) (2021,0.0) (2022,0.0) (2023,0.0) (2024,0.0) (2025,0.0)  }; \addlegendentry{Legal Threat}
                \end{axis}
            \end{tikzpicture}
        \end{minipage}
    \vspace{0.5cm}
        \begin{minipage}{0.48\textwidth}
            \centering
            \begin{tikzpicture}
                \begin{axis}[
                    width=\linewidth, height=5cm, xlabel={Year}, ylabel={Share (\%)},
                    xmin=2020, xmax=2025, xtick={2020,2022,2024},
                    xticklabel style={/pgf/number format/set thousands separator={}},
                    grid=major, legend pos=north west, legend style={font=\tiny},
                    title={Disney+}, title style={font=\footnotesize\bfseries},
                    ymin=0, ymax=100
                ]
                    \addplot[thick, color=tudablue, mark=square*] coordinates { (2020,50.0) (2021,52.6) (2022,60.0) (2023,58.6) (2024,58.1) (2025,76.5)  }; \addlegendentry{Content Licensing}
                    \addplot[thick, color=tudagreen, mark=triangle*] coordinates { (2020,22.2) (2021,26.3) (2022,5.0) (2023,31.0) (2024,17.6) (2025,0.0)  }; \addlegendentry{Regulatory Compliance}
                    \addplot[thick, color=tudared, mark=x] coordinates { (2020,0.0) (2021,0.0) (2022,0.0) (2023,0.0) (2024,0.0) (2025,0.0)  }; \addlegendentry{Technical Blocking}
                    \addplot[thick, color=cyan, mark=diamond*] coordinates { (2020,27.8) (2021,21.1) (2022,35.0) (2023,10.3) (2024,18.9) (2025,23.5)  }; \addlegendentry{Price Discrimination}
                    \addplot[thick, color=orange, mark=pentagon*] coordinates { (2020,0.0) (2021,0.0) (2022,0.0) (2023,0.0) (2024,5.4) (2025,0.0)  }; \addlegendentry{Legal Threat}
                \end{axis}
            \end{tikzpicture}
        \end{minipage}
    \hfill
        \begin{minipage}{0.48\textwidth}
            \centering
            \begin{tikzpicture}
                \begin{axis}[
                    width=\linewidth, height=5cm, xlabel={Year}, ylabel={Share (\%)},
                    xmin=2020, xmax=2025, xtick={2020,2022,2024},
                    xticklabel style={/pgf/number format/set thousands separator={}},
                    grid=major, legend pos=north west, legend style={font=\tiny},
                    title={Amazon}, title style={font=\footnotesize\bfseries},
                    ymin=0, ymax=100
                ]
                    \addplot[thick, color=tudablue, mark=square*] coordinates { (2020,50.0) (2021,50.0) (2022,46.2) (2023,58.8) (2024,42.1) (2025,28.6)  }; \addlegendentry{Content Licensing}
                    \addplot[thick, color=tudagreen, mark=triangle*] coordinates { (2020,33.3) (2021,50.0) (2022,53.8) (2023,41.2) (2024,31.6) (2025,39.3)  }; \addlegendentry{Regulatory Compliance}
                    \addplot[thick, color=tudared, mark=x] coordinates { (2020,0.0) (2021,0.0) (2022,0.0) (2023,0.0) (2024,5.3) (2025,0.0)  }; \addlegendentry{Technical Blocking}
                    \addplot[thick, color=cyan, mark=diamond*] coordinates { (2020,0.0) (2021,0.0) (2022,0.0) (2023,0.0) (2024,0.0) (2025,7.1)  }; \addlegendentry{Price Discrimination}
                    \addplot[thick, color=orange, mark=pentagon*] coordinates { (2020,16.7) (2021,0.0) (2022,0.0) (2023,0.0) (2024,0.0) (2025,7.1)  }; \addlegendentry{Legal Threat}
                \end{axis}
            \end{tikzpicture}
        \end{minipage}
    \vspace{0.5cm}
    \caption{Strategic Evolution: Video Streaming Leaders (2020--2025). Grouping Netflix, YouTube, Disney+, and Amazon Prime.}
    \label{fig:evol_video_main}
\end{figure}

% --- Generated Software Group ---
\begin{figure}[ht]
    \centering
        \begin{minipage}{0.48\textwidth}
            \centering
            \begin{tikzpicture}
                \begin{axis}[
                    width=\linewidth, height=5cm, xlabel={Year}, ylabel={Share (\%)},
                    xmin=2020, xmax=2025, xtick={2020,2022,2024},
                    xticklabel style={/pgf/number format/set thousands separator={}},
                    grid=major, legend pos=north west, legend style={font=\tiny},
                    title={Spotify}, title style={font=\footnotesize\bfseries},
                    ymin=0, ymax=100
                ]
                    \addplot[thick, color=tudablue, mark=square*] coordinates { (2020,63.4) (2021,52.0) (2022,70.3) (2023,53.3) (2024,25.9) (2025,33.3)  }; \addlegendentry{Content Licensing}
                    \addplot[thick, color=tudagreen, mark=triangle*] coordinates { (2020,24.4) (2021,32.0) (2022,24.3) (2023,42.2) (2024,48.1) (2025,66.7)  }; \addlegendentry{Regulatory Compliance}
                    \addplot[thick, color=tudared, mark=x] coordinates { (2020,0.0) (2021,0.0) (2022,0.0) (2023,0.0) (2024,0.0) (2025,0.0)  }; \addlegendentry{Technical Blocking}
                    \addplot[thick, color=cyan, mark=diamond*] coordinates { (2020,12.2) (2021,16.0) (2022,5.4) (2023,4.4) (2024,25.9) (2025,0.0)  }; \addlegendentry{Price Discrimination}
                    \addplot[thick, color=orange, mark=pentagon*] coordinates { (2020,0.0) (2021,0.0) (2022,0.0) (2023,0.0) (2024,0.0) (2025,0.0)  }; \addlegendentry{Legal Threat}
                \end{axis}
            \end{tikzpicture}
        \end{minipage}
    \hfill
        \begin{minipage}{0.48\textwidth}
            \centering
            \begin{tikzpicture}
                \begin{axis}[
                    width=\linewidth, height=5cm, xlabel={Year}, ylabel={Share (\%)},
                    xmin=2020, xmax=2025, xtick={2020,2022,2024},
                    xticklabel style={/pgf/number format/set thousands separator={}},
                    grid=major, legend pos=north west, legend style={font=\tiny},
                    title={Apple}, title style={font=\footnotesize\bfseries},
                    ymin=0, ymax=100
                ]
                    \addplot[thick, color=tudablue, mark=square*] coordinates { (2020,60.0) (2021,9.1) (2022,40.0) (2023,27.3) (2024,35.7) (2025,27.3)  }; \addlegendentry{Content Licensing}
                    \addplot[thick, color=tudagreen, mark=triangle*] coordinates { (2020,40.0) (2021,72.7) (2022,35.0) (2023,18.2) (2024,35.7) (2025,27.3)  }; \addlegendentry{Regulatory Compliance}
                    \addplot[thick, color=tudared, mark=x] coordinates { (2020,0.0) (2021,0.0) (2022,0.0) (2023,9.1) (2024,0.0) (2025,9.1)  }; \addlegendentry{Technical Blocking}
                    \addplot[thick, color=cyan, mark=diamond*] coordinates { (2020,0.0) (2021,0.0) (2022,0.0) (2023,9.1) (2024,0.0) (2025,9.1)  }; \addlegendentry{Price Discrimination}
                    \addplot[thick, color=orange, mark=pentagon*] coordinates { (2020,0.0) (2021,0.0) (2022,5.0) (2023,27.3) (2024,7.1) (2025,18.2)  }; \addlegendentry{Legal Threat}
                \end{axis}
            \end{tikzpicture}
        \end{minipage}
    \vspace{0.5cm}
        \begin{minipage}{0.48\textwidth}
            \centering
            \begin{tikzpicture}
                \begin{axis}[
                    width=\linewidth, height=5cm, xlabel={Year}, ylabel={Share (\%)},
                    xmin=2020, xmax=2025, xtick={2020,2022,2024},
                    xticklabel style={/pgf/number format/set thousands separator={}},
                    grid=major, legend pos=north west, legend style={font=\tiny},
                    title={Microsoft}, title style={font=\footnotesize\bfseries},
                    ymin=0, ymax=100
                ]
                    \addplot[thick, color=tudablue, mark=square*] coordinates { (2020,6.2) (2021,4.3) (2022,6.7) (2023,3.2) (2024,0.0) (2025,9.5)  }; \addlegendentry{Content Licensing}
                    \addplot[thick, color=tudagreen, mark=triangle*] coordinates { (2020,34.4) (2021,47.8) (2022,33.3) (2023,45.2) (2024,77.8) (2025,61.9)  }; \addlegendentry{Regulatory Compliance}
                    \addplot[thick, color=tudared, mark=x] coordinates { (2020,3.1) (2021,0.0) (2022,3.3) (2023,3.2) (2024,0.0) (2025,4.8)  }; \addlegendentry{Technical Blocking}
                    \addplot[thick, color=cyan, mark=diamond*] coordinates { (2020,0.0) (2021,0.0) (2022,0.0) (2023,0.0) (2024,0.0) (2025,0.0)  }; \addlegendentry{Price Discrimination}
                    \addplot[thick, color=orange, mark=pentagon*] coordinates { (2020,25.0) (2021,30.4) (2022,26.7) (2023,35.5) (2024,22.2) (2025,23.8)  }; \addlegendentry{Legal Threat}
                \end{axis}
            \end{tikzpicture}
        \end{minipage}
    \hfill
        \begin{minipage}{0.48\textwidth}
            \centering
            \begin{tikzpicture}
                \begin{axis}[
                    width=\linewidth, height=5cm, xlabel={Year}, ylabel={Share (\%)},
                    xmin=2020, xmax=2025, xtick={2020,2022,2024},
                    xticklabel style={/pgf/number format/set thousands separator={}},
                    grid=major, legend pos=north west, legend style={font=\tiny},
                    title={Adobe}, title style={font=\footnotesize\bfseries},
                    ymin=0, ymax=100
                ]
                    \addplot[thick, color=tudablue, mark=square*] coordinates { (2020,0.0) (2021,0.0) (2022,0.0) (2023,0.0) (2024,0.0) (2025,28.6)  }; \addlegendentry{Content Licensing}
                    \addplot[thick, color=tudagreen, mark=triangle*] coordinates { (2020,75.0) (2021,85.7) (2022,90.9) (2023,100.0) (2024,38.1) (2025,28.6)  }; \addlegendentry{Regulatory Compliance}
                    \addplot[thick, color=tudared, mark=x] coordinates { (2020,0.0) (2021,0.0) (2022,0.0) (2023,0.0) (2024,0.0) (2025,0.0)  }; \addlegendentry{Technical Blocking}
                    \addplot[thick, color=cyan, mark=diamond*] coordinates { (2020,12.5) (2021,14.3) (2022,0.0) (2023,0.0) (2024,52.4) (2025,28.6)  }; \addlegendentry{Price Discrimination}
                    \addplot[thick, color=orange, mark=pentagon*] coordinates { (2020,12.5) (2021,0.0) (2022,0.0) (2023,0.0) (2024,4.8) (2025,14.3)  }; \addlegendentry{Legal Threat}
                \end{axis}
            \end{tikzpicture}
        \end{minipage}
    \vspace{0.5cm}
    \caption{Strategic Evolution: Music & Software (2020--2025). Grouping Spotify, Apple Music, Microsoft, and Adobe.}
    \label{fig:evol_software_main}
\end{figure}




This timeline illustrates the dynamic relationship between corporate enforcement and consumer
behavior. Early enforcement measures by streaming providers led \gls{vpn} providers to develop "Obfuscated
Servers," which subsequently led to more sophisticated blocking techniques (circa
2021). This pattern suggests an ongoing adaptation process on both sides.
\subsection{The Secret Tech Race}
Returning to the history of piracy, the modern world is defined by a growing tech race. Our 
research shows that companies are getting much better at finding where users really are, even 
with a high-quality \gls{vpn}. They no longer just block IP addresses; they use advanced tools like 
Deep Packet Inspection (\gls{dpi}), AI to read traffic fingerprints, and ways to check browser data 
for leaks.

However, a main problem for research is that these tools are kept secret. Unlike the open 
legal fights over Napster, modern geo-blocking happens in private. Companies keep their 
methods secret so that \gls{vpn} providers can't adapt. This means that while we can see an 
increase in blocking, it is hard to say exactly how the technology works.

\section{Detailed Analysis}
\label{sec:appendix_detailed_analysis}

This section provides a deeper look at the data, analyzing the specific numbers that drive the strategic trends.

To understand the macro-trends, we analyze how the total volume of policy text has shifted. Our analysis shows the overwhelming dominance of General Terms (legal boilerplate), which consistently make up over 90\% of all sentences (see Table \ref{tab:category_dist}). However, when we filter for \textit{strategic} categories (Figure \ref{fig:dist_strategic}), a clear pattern emerges: \textbf{Content Licensing} and \textbf{Regulatory Compliance} are the baseline "noise" of digital business, while \textbf{Technical Blocking} and \textbf{Account Action} show specific, event-driven spikes.

% --- Distribution Strategic Categories Only ---
\begin{figure}[ht]
    \centering
    \begin{tikzpicture}
        \begin{axis}[
            ybar stacked,
            width=0.9\textwidth,
            height=8cm,
            xlabel={Year},
            ylabel={Share (\%)}
            xmin=2019.5, xmax=2025.5,
            ymin=0, ymax=100,
            xtick={2020,2021,2022,2023,2024,2025},
            xticklabel style={/pgf/number format/set thousands separator={}},
            legend style={at={(0.5,-0.2)}, anchor=north, legend columns=3, font=\footnotesize},
            bar width=15pt,
            area style,
        ]
            \addplot[ybar stacked, fill=blue, draw=black!50] coordinates { (2020,41.9) (2021,43.2) (2022,38.5) (2023,40.8) (2024,34.8) (2025,37.9) };
            \addlegendentry{Content Licensing}
            \addplot[ybar stacked, fill=orange, draw=black!50] coordinates { (2020,35.6) (2021,35.5) (2022,32.9) (2023,30.1) (2024,34.6) (2025,37.9) };
            \addlegendentry{Regulatory Compliance}
            \addplot[ybar stacked, fill=tudared, draw=black!50] coordinates { (2020,0.9) (2021,0.3) (2022,3.5) (2023,9.5) (2024,5.7) (2025,4.0) };
            \addlegendentry{Technical Blocking}
            \addplot[ybar stacked, fill=green!60!black, draw=black!50] coordinates { (2020,8.6) (2021,9.5) (2022,11.6) (2023,7.1) (2024,11.0) (2025,7.5) };
            \addlegendentry{Price Discrimination}
            \addplot[ybar stacked, fill=black, draw=black!50] coordinates { (2020,13.0) (2021,11.5) (2022,13.5) (2023,12.5) (2024,13.9) (2025,12.7) };
            \addlegendentry{Enforcement Actions}
        \end{axis}
    \end{tikzpicture}
    \caption{Distribution of Policy Text Categories Over Time (Strategic Categories Only).}
    \label{fig:dist_strategic}
\end{figure}


As illustrated in the preceding analysis, the strategic categories are overwhelmingly dwarfed by general terms, which constitute over 90\% of all sentences (Table \ref{tab:category_dist}). Consequently, the strategic trends are best visualized by focusing exclusively on non-boilerplate categories, as shown in Figure \ref{fig:dist_strategic}.

This section provides a deeper look at the data, analyzing the specific numbers that drive the strategic trends.

\subsection{Country-Level Affordability Analysis}
Table \ref{tab:app_dspi} presents the complete DSPI and Affordability (PTW) scores. Notably, while **Pakistan** appears cheapest ($DSPI=0.45$), it is relatively expensive for locals ($1.13\%$ of wage). In contrast, **Turkey** ($DSPI=0.65$) and **Argentina** ($DSPI=0.76$) show high nominal discounts, but Argentina's sky-high PTW ($3.28\%$) makes it an unlikely target for mass local adoption, suggesting pricing there is effectively "dollarized" for elites or external arbitrageurs.

\begin{table}[ht]
    \centering
    \small
    \begin{tabular}{l c c c}
        \toprule
        \textbf{Country} & \textbf{Avg. DSPI} & \textbf{N Services} & \textbf{Netflix PTW (\%)} \\
        \midrule
        Switzerland     & 1.24 & 11 & 0.33 \\
        United Kingdom  & 1.04 & 11 & 0.24 \\
        Germany         & 1.01 & 11 & 0.29 \\
        United States   & 1.00 & 11 & 0.27 \\
        Argentina       & 0.76 & 11 & 3.28 \\
        Poland          & 0.77 & 11 & 0.73 \\
        Turkey          & 0.65 & 11 & 1.22 \\
        Ukraine         & 0.62 & 10 & 1.60 \\
        Brazil          & 0.59 & 11 & 1.50 \\
        Philippines     & 0.54 & 11 & 2.18 \\
        Pakistan        & 0.45 &  9 & 1.13 \\
        \bottomrule
    \end{tabular}
    \caption{Complete DSPI Summary by Country. PTW = Netflix Standard subscription cost as \% of median monthly wage. Lower DSPI indicates cheaper markets relative to the US baseline.}
    \label{tab:app_dspi}
\end{table}

\subsection{Service-Level Enforcement Trends}
Table \ref{tab:app_timeline} details the enforcement intensity per company. Two distinct patterns emerge:
\begin{itemize}
    \item \textbf{Microsoft's Withdrawal:} Microsoft shows a consistent but low-level "background radiation" of legal checks (6-12 counts), but no sudden spikes. This supports the "Enterprise Fortress" theory—they rely on identity, not new blocking waves.
    \item \textbf{YouTube's War:} YouTube shows a massive spike in 2023 (53 counts), confirming the "Enforcement Surge" visualized in the timeline. However, this drops to 31 in 2024 and 20 in 2025, suggesting a "cooldown" period where the new blocking rules became standard practice rather than new aggressive maneuvers.
\end{itemize}

\begin{table}[ht]
    \centering
    \small
    \renewcommand{\arraystretch}{1.1}
    \begin{tabular}{l *{10}{r}}
        \toprule
        \textbf{Year} & \rotatebox{70}{\textbf{Adobe}} & \rotatebox{70}{\textbf{Amazon}} & \rotatebox{70}{\textbf{Apple}} & \rotatebox{70}{\textbf{Disney+}} & \rotatebox{70}{\textbf{ExpressVPN}} & \rotatebox{70}{\textbf{Microsoft}} & \rotatebox{70}{\textbf{Netflix}} & \rotatebox{70}{\textbf{NordVPN}} & \rotatebox{70}{\textbf{Spotify}} & \rotatebox{70}{\textbf{YouTube}} \\
        \midrule
        2020 & 2 & 1 & 0 & 0 & 0 & 9 & 1 & 0 & 0 & 1 \\
        2021 & 0 & 0 & 0 & 0 & 0 & 7 & 1 & 0 & 0 & 0 \\
        2022 & 0 & 0 & 1 & 0 & 0 & 9 & 0 & 0 & 0 & 16 \\
        2023 & 0 & 0 & 4 & 0 & 0 & 12 & 0 & 0 & 0 & 53 \\
        2024 & 1 & 0 & 1 & 4 & 0 & 6 & 0 & 3 & 0 & 31 \\
        2025 & 1 & 2 & 3 & 0 & 5 & 6 & 3 & 0 & 0 & 20 \\
        \bottomrule
    \end{tabular}
    \caption{Enforcement-Related Sentence Counts by Year and Service (2020--2025). YouTube's sharp spike in 2022--2023 is clearly visible.}
    \label{tab:app_timeline}
\end{table}

\section{Limits of the study}
While this study gives us a new way to look at geo-arbitrage, there are some limits to our 
findings.

\subsection{Sample Size and Generalizability}
The correlation analysis relies on a strategic sample of $N=10$ digital service
providers. While these firms represent a significant majority of the consumer subscription
market by capitalization, the sample is small in statistical terms. Consequently, the findings
should be interpreted as "exploratory" evidence of strategic archetypes rather than a
definitive "law" of digital economics. Future research could expand this dataset to include
mid-tier SaaS providers to test if the "Enterprise Fortress" model holds for smaller B2B
firms.

\subsection{The "Average Citizen" Bias (Socioeconomic Mismatch)}
Our "Affordability" metric calculates cost as a percentage of the \textit{Median National
Monthly Wage}. However, in emerging markets like Turkey or Argentina, the target demographic
for services like Netflix or Adobe is likely the urban upper-middle class, whose income is
significantly higher than the national average.

For instance, World Bank data and local surveys indicate that in Turkey, the top 20\% of
earners capture nearly 48\% of total disposable income. Similarly, in Argentina, the top 10\%
of earners have average monthly incomes exceeding \$496 USD, well above the national median.
This implies that global digital services are aggressively priced to target this specific
"Global Elite" segment. As \textcite{kastanakis2012between} argue, in markets with high income
inequality, luxury consumption (including premium digital subscriptions) serves as a critical
status signal for the upper class. This "Elite Targeting" pricing strategy explains why firms
tolerate some level of piracy from the lower 80\%—they were never the primary customer segment
to begin with.

The Digital Services Price Index (\gls{dspi}) represents a snapshot of pricing data from December 2025. In hyperinflationary economies such as Argentina and Turkey (both classified as
hyperinflationary under IAS 29), local currency prices are adjusted frequently in response to
macroeconomic conditions. A ``cheap'' arbitrage opportunity identified in this thesis could be
eliminated by a price adjustment or currency devaluation. The ``Arbitrage Window'' is
therefore dynamic rather than static.

\subsection{AI Classification Reliability}
The use of Large Language Models (Gemini 3 Flash) creates a potential "Black Box" validity
risk. To reduce this, we used the model's self-reported confidence scores as a filtering
mechanism. The final dataset achieved an average confidence score of \textbf{0.947}, with
\textbf{80.5\%} of classifications exceeding a confidence threshold of 0.9. This high degree
of certainty suggests that the detection of "coercive" vs. "general" language is reliable,
even without human verification for every datapoint.

\subsection{What we learned about our methods}
Beyond the main findings, this study shows that older BERT models are not good enough for 
complex legal text. The comparison showed that BERT was only right 26.8\% of the time, which is 
like guessing.

The main difference is that BERT just looks for keywords, while Gemini can understand the 
context. This means that future research should use modern AI like Gemini when trying to 
analyze complex legal rules.

\chapter{Conclusion}
\label{chap:conclusion}

This final chapter sums up the findings of the thesis to answer the research questions. We
discuss what this study adds to our understanding of economics and strategy, while also noting
the limits of our research. Finally, we look at how new rules and technologies might continue to
change digital business in the future.

\section{Summary of Key Findings}
This thesis looked at the strategic conflict between firms' price discrimination practices and
consumer-driven geographic arbitrage in digital subscription markets.
\begin{itemize}
    \item[\gls{rq}1 (Economic Incentive):] The \gls{dspi} shows large price differences across markets, with discounts over 70\% in some regions, creating a strong reason for users to use \gls{vpn}-based arbitrage.
    \item[\gls{rq}2 (Strategic Response):] Enforcement strategies are primarily determined by \textbf{Business Model Architecture}. Content streaming firms enforce strict geographic blocking due to licensing requirements. Software firms rely on cryptographic activation rather than network-based blocking. \gls{vpn} providers, by their nature, implement no geographic restrictions.
    \item \textbf{Enforcement Evolution:} Firms have intensified technical countermeasures since 2022, but the persistent prominence of circumvention discussions suggests these barriers increase friction rather than eliminating arbitrage entirely.
\end{itemize}

\section{Contribution to Research}
This study adds both methods and theory to digital economics.

\textbf{New methods:}
\begin{itemize}
    \item \textbf{\gls{dspi}:} We created the \gls{dspi} as a new way to measure price discrimination for digital services.
    \item \textbf{AI Legal Analysis:} This study shows that AI can successfully analyze thousands of legal sentences, setting a new path for future research in this field.
\end{itemize}

\textbf{Theoretical Contributions:}
\begin{itemize}
    \item \textbf{Extension of \gls{bmi} Theory:} The study extends \gls{bmi} theory by showing that consumer bypassing behaviors work as a disruptive force similar to technological innovation, forcing firms to fundamentally change their value capture mechanisms.
    \item \textbf{Business Types:} The identification of four distinct types (Content Fortress, Ecosystem Fortress, Enterprise Fortress, Utility Paradox) provides a way for understanding how different business models respond to the same external threat.
    \item \textbf{Piracy-Arbitrage Parallel:} By drawing explicit parallels between the digital piracy wave of the 2000s and contemporary geo-arbitrage, this thesis contributes to a longer-term understanding of how digital disruption forces business model adaptation.
\end{itemize}


\section{The Future}
The world of geo-blocking is likely to change a lot in the coming years.

\textbf{New Rules:}
As rules like the \gls{eu}'s Digital Single Market change, geo-blocking may change completely. New 
laws might move companies away from blocking and toward keeping prices the same everywhere.

\textbf{Tech Evolution:}
The arms race between blocking technology and bypassing tools shows no signs of slowing.
Advanced techniques such as residential IP proxies, browser fingerprinting, and machine
learning-based detection create a hard technical fight. Researchers
should look at whether the rising cost of this arms race eventually makes geographic price
discrimination economically unsustainable.

\textbf{Market Prices Coming Together:}
Ultimately, the cycle of blocking and bypassing may end not through better technology, but 
through market forces that drive global prices together. As digital services become more 
similar and competition grows, companies may find that the costs of keeping different regional 
prices are higher than the benefits. In such a case, geo-arbitrage would stop being useful, not 
because users are blocked, but because there is no longer a big price difference to exploit.

\textbf{Future Research:}
Future studies could build on this work by: (1) looking at more digital services; (2) tracking 
\gls{dspi} changes over time to see how companies react to events; (3) asking 
users about how they 
feel and the ethics of geo-arbitrage; and (4) comparing how companies act in areas with 
different rules to see how laws affect company strategy.





\appendix
\chapter{Supplementary Data Tables}
\label{app:data_tables}

This appendix contains the complete datasets underlying the analysis presented in this thesis. All data was collected in December 2025 and processed using the methodology described in Chapter~3.

\section{Dataset Composition}
\label{app:dataset_composition}

Figure \ref{fig:app_company_dist} and Figure \ref{fig:app_doctype_dist} detail the specific breakdown of the analyzed corpus by company and document type.

\begin{figure}[ht]
    \centering
    \begin{tikzpicture}
        \begin{axis}[
            xbar,
            width=0.9\textwidth,
            height=8cm,
            xlabel={Total Sentences (N)},
            symbolic y coords={ExpressVPN,NordVPN,Apple Music,Amazon Prime,Disney+,Netflix,Adobe,Spotify,YouTube Premium,Microsoft},
            ytick=data,
            nodes near coords,
            nodes near coords align={horizontal},
            xmin=0, xmax=7500,
            bar width=12pt,
            yticklabel style={font=\footnotesize},
        ]
            % Data from Analysis
            \addplot[fill=blue!60] coordinates {
                (60,ExpressVPN)
                (110,NordVPN)
                (1328,Apple Music)
                (1639,Amazon Prime)
                (2171,Disney+)
                (2872,Netflix)
                (3159,Adobe)
                (3551,Spotify)
                (4164,YouTube Premium)
                (6516,Microsoft)
            };
        \end{axis}
    \end{tikzpicture}
    \caption{Total Clause Counts by Company. The dataset is weighted towards Microsoft and YouTube due to the complexity and length of their multiple policy documents.}
    \label{fig:app_company_dist}
\end{figure}

\begin{figure}[ht]
    \centering
    \begin{tikzpicture}
        \begin{axis}[
            xbar,
            width=0.9\textwidth,
            height=6cm,
            xlabel={Total Sentences (N)},
            symbolic y coords={Other,Shareholder Letter,Terms of Service,Earnings Call,10-K/Annual Report},
            ytick=data,
            nodes near coords,
            nodes near coords align={horizontal},
            xmin=0, xmax=12000,
            bar width=15pt,
            yticklabel style={font=\footnotesize},
        ]
            % Data from Analysis
            \addplot[fill=green!60!black] coordinates {
                (634,Other)
                (1866,Shareholder Letter)
                (6103,Terms of Service)
                (6794,Earnings Call)
                (10173,10-K/Annual Report)
            };
        \end{axis}
    \end{tikzpicture}
    \caption{Distribution of Data by Document Type. Annual Reports (10-K) and Earnings Calls provide the bulk of strategic context, while ToS documents provide the specific enforcement clauses.}
    \label{fig:app_doctype_dist}
\end{figure}

\section{Complete DSPI Data by Country}
\label{app:dspi_complete}

Table~\ref{tab:app_dspi} presents the average \gls{dspi} and affordability metrics for all 11 countries in the dataset, ordered by \gls{dspi} value.



\section{Category Counts by Year (All Services)}
\label{app:qual_timeline}

For the detailed breakdown of enforcement categories by year, please refer to \textbf{Section \ref{sec:appendix_detailed_analysis}} (Table \ref{tab:qual_timeline_complete}).

\section{Service-Level Category Distribution (Absolute Counts)}
\label{app:service_stats}

Table~\ref{tab:app_service_stats} presents the complete distribution of strategic frame counts for each of the 10 analyzed services.

\begin{table}[ht]
    \centering
    \scriptsize
    \renewcommand{\arraystretch}{1.1}
    \begin{tabular}{l *{11}{r} | r}
        \toprule
        \textbf{Service} & \rotatebox{70}{\textbf{Technical Blocking}} & \rotatebox{70}{\textbf{Price Discrimination}} & \rotatebox{70}{\textbf{Licensing}} & \rotatebox{70}{\textbf{Regulatory}} & \rotatebox{70}{\textbf{Legal Thr.}} & \rotatebox{70}{\textbf{Acc.\ Act.}} & \rotatebox{70}{\textbf{Privacy}} & \rotatebox{70}{\textbf{Sec.\ Risk}} & \rotatebox{70}{\textbf{Portab.}} & \rotatebox{70}{\textbf{Workaro.}} & \rotatebox{70}{\textbf{General}} & \rotatebox{70}{\textbf{Total}} \\
        \midrule
        YouTube   & 94 &  2 &  99 & 135 & 34 & 0 & 16 &  0 & 0 & 0 & 3{,}784 & 4{,}164 \\
        Microsoft &  4 &  0 &  10 &  80 & 53 & 0 &  4 & 23 & 0 & 0 & 6{,}342 & 6{,}516 \\
        Netflix   &  1 & 36 & 127 &  76 &  4 & 0 &  0 &  0 & 2 & 0 & 2{,}626 & 2{,}872 \\
        Disney+   &  0 & 41 & 114 &  37 &  4 & 0 &  0 &  0 & 0 & 0 & 1{,}975 & 2{,}171 \\
        Spotify   &  1 & 22 & 132 &  75 &  0 & 0 &  0 &  0 & 0 & 1 & 3{,}320 & 3{,}551 \\
        Adobe     &  0 & 18 &   4 &  45 &  4 & 0 &  1 &  1 & 0 & 0 & 3{,}086 & 3{,}159 \\
        Amazon    &  1 &  2 &  44 &  44 &  3 & 0 &  0 &  9 & 0 & 0 & 1{,}536 & 1{,}639 \\
        Apple     &  2 &  2 &  28 &  29 &  8 & 0 &  5 &  6 & 0 & 0 & 1{,}248 & 1{,}328 \\
        ExpressVPN&  1 &  1 &   0 &   2 &  4 & 0 &  1 &  0 & 0 & 0 &    51 &    60 \\
        NordVPN   &  0 &  0 &   0 &   0 &  6 & 0 &  0 &  6 & 0 & 0 &    98 &   110 \\
        \midrule
        \textbf{Total} & \textbf{104} & \textbf{124} & \textbf{558} & \textbf{523} & \textbf{120} & \textbf{0} & \textbf{27} & \textbf{45} & \textbf{2} & \textbf{1} & \textbf{24,066} & \textbf{25,570} \\
        \bottomrule
    \end{tabular}
    \caption{Absolute Category Counts by Service. YouTube dominates Technical Blocking (N=94), while streaming services rely heavily on Content Licensing.}
    \label{tab:app_service_stats}
\end{table}

\section{Fortress Index (Complete Ranking)}
\label{app:fortress_index}

For the complete Fortress Index ranking, please refer to \textbf{Section \ref{sec:appendix_detailed_analysis}} (Table \ref{tab:fortress_index_complete}).

\chapter{Detailed Service Evolution}
\label{app:service_evolution}

This appendix presents evolution data for VPN providers, as the detailed evolution charts for other services have been integrated into Chapter \ref{chap:results} and Chapter \ref{chap:discussion}.

\begin{figure}[ht]
    \centering
    \begin{minipage}{0.48\textwidth}
        \centering
        \begin{tikzpicture}
            \begin{axis}[
                width=\linewidth,
                height=5cm,
                xlabel={Year},
                ylabel={Sentences},
                xmin=2018, xmax=2025,
                xtick={2018,2020,2022,2024},
                xticklabel style={/pgf/number format/set thousands separator={}},
                grid=major,
                legend pos=north west,
                legend style={font=\tiny},
                title={Generic VPN (Aggregated)},
                title style={font=\footnotesize\bfseries}
            ]
                \addplot[thick, color=red, mark=pentagon*] coordinates { (2020,8) (2021,7) (2022,8) (2023,11) (2024,6) (2025,5) };
                \addlegendentry{Legal Thr.}
                \addplot[thick, color=brown!60!black, mark=otimes] coordinates { (2020,1) (2021,2) (2022,3) (2023,5) (2024,1) (2025,0) };
                \addlegendentry{Priv./Sec.}
            \end{axis}
        \end{tikzpicture}
    \end{minipage}
    \caption{Evolution of VPN Provider Enforcement. Note: Specific VPN graphs have been excluded for brevity as they show minimal variation.}
    \label{fig:app_vpn_evolution}
\end{figure}

\chapter{Model Methodology and Validation}
\label{app:model_validation}

This appendix provides detailed metrics comparing the performance of the BERT baseline model and the Gemini 3 Flash model used for the final classification of the 25,570 sentences in the dataset.

\section{Model Distribution Comparison}

Figure \ref{fig:gemini_vs_bert} compares the category distribution percentages between the two models. As discussed in Chapter \ref{chap:methodology}, the BERT model exhibited significant over-classification of rare categories and poor handling of legal boilerplate (General Terms), while Gemini 3 Flash demonstrated a more nuanced understanding of context.

\begin{figure}[ht]
    \centering
    \begin{tikzpicture}
        \begin{axis}[
            ybar,
            width=0.95\textwidth,
            height=8cm,
            symbolic x coords={Licensing,Regulatory,Legal Threat,Price Discr.,Tech. Block.,Sec. Risk,Privacy,Portability,Workaround,General},
            xtick=data,
            xticklabel style={rotate=45, anchor=north east, font=\footnotesize},
            ylabel={Share of Dataset (\%)},
            legend style={at={(0.5,-0.35)}, anchor=north, legend columns=-1, font=\small},
            ymin=0, ymax=100,
            bar width=10pt,
            nodes near coords,
            every node near coord/.append style={font=\tiny, rotate=90, anchor=west},
            /pgf/number format/fixed, /pgf/number format/precision=1
        ]
            % Gemini %
            \addplot[fill=blue!60] coordinates {
                (Licensing,2.2) (Regulatory,2.1) (Legal Threat,0.5) (Price Discr.,0.5) (Tech. Block.,0.4) 
                (Sec. Risk,0.2) (Privacy,0.1) (Portability,0.0) (Workaround,0.0) (General,94.1)
            };
            % BERT %
            \addplot[fill=red!60] coordinates {
                (Licensing,5.8) (Regulatory,0.4) (Legal Threat,0.0) (Price Discr.,0.0) (Tech. Block.,0.1) 
                (Sec. Risk,0.0) (Privacy,0.0) (Portability,32.0) (Workaround,9.7) (General,26.1)
            };
            \legend{Gemini 3 Flash, BERT Baseline}
        \end{axis}
    \end{tikzpicture}
    \caption{Model Comparison: Gemini vs. BERT Category Distribution. BERT's massive over-classification of "Portability" (32.0\%) and "User Workaround" (9.7\%) underscores its lack of legal context compared to Gemini.}
    \label{fig:gemini_vs_bert}
\end{figure}

\section{Gemini Model Breakdown}

Figure \ref{fig:gemini_breakdown} shows the absolute sentence counts for each category as identified by the Gemini model (excluding General Terms).

\begin{figure}[ht]
    \centering
    \begin{tikzpicture}
        \begin{axis}[
            xbar,
            width=0.9\textwidth,
            height=7cm,
            symbolic y coords={Content Licensing,Regulatory Compliance,Legal Threat,Price Discrimination,Technical Blocking,Security Risk,Privacy/Security,Legitimate Portability,User Workaround},
            ytick=data,
            xlabel={Sentence Count (N)},
            nodes near coords,
            bar width=15pt,
            xmin=0, xmax=600
        ]
            \addplot[fill=blue!40] coordinates {
                (558,Content Licensing) (523,Regulatory Compliance) (120,Legal Threat) (124,Price Discrimination) 
                (104,Technical Blocking) (45,Security Risk) (27,Privacy/Security) (2,Legitimate Portability) (1,User Workaround)
            };
        \end{axis}
    \end{tikzpicture}
    \caption{Gemini 3 Flash: Absolute Category Frequencies (Excluding General Terms).}
    \label{fig:gemini_breakdown}
\end{figure}



\chapter{Dataset Overview and Additional Analysis}
\label{app:dataset_overview}

\section{Document Type Distribution}

Table \ref{tab:doc_type_dist} presents the distribution of the 25,570 analyzed sentences by document type, showing the relative composition of the dataset corpus.

\begin{table}[ht]
    \centering
    \begin{tabular}{l r r}
        \toprule
        \textbf{Document Type} & \textbf{Total Sentences} & \textbf{Share (\%)} \\
        \midrule
        10-K / Annual Report & 10,173 & 39.79 \\
        Earnings Call Transcript & 6,794 & 26.57 \\
        Terms of Service & 6,103 & 23.87 \\
        Shareholder Letter & 1,866 & 7.30 \\
        Other & 634 & 2.48 \\
        \midrule
        \textbf{Total} & \textbf{25,570} & \textbf{100.00} \\
        \bottomrule
    \end{tabular}
    \caption{Dataset Composition by Document Type. Regulatory filings (10-K reports) represent the largest share, followed by earnings calls and Terms of Service documents.}
    \label{tab:doc_type_dist}
\end{table}

% Figures moved to main text

\backmatter
    \addcontentsline{toc}{chapter}{Bibliography}
    \printbibliography

    %--- List of Figures and Tables ---
    \cleardoublepage
    \phantomsection
    \addcontentsline{toc}{chapter}{List of Figures and Tables}
    \chapter*{List of Figures and Tables}
    \makeatletter
        \section*{\listfigurename} 
        \@starttoc{lof} 
        \vspace{1em} 
        \section*{\listtablename}
        \@starttoc{lot} 
    \makeatother

    %--- Glossary and Acronyms ---
    \cleardoublepage
    \phantomsection
    \addcontentsline{toc}{chapter}{Glossary and Acronyms}
    \chapter*{Glossary and Acronyms}
    \printglossaries

    \affidavit


\end{document}





